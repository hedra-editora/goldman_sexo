\textbf{Emma Goldman} (1869–1940) foi uma revolucionária anarquista de origem russa, que migrou para Rochester (\textsc{eua}) em 1886. Em 1899, mudou-se para Nova York e conheceu Alexander Berkman, destacado anarquista que além de grande amigo e companheiro político foi também seu amante durante determinada época. Como grande parte dos emigrantes do leste europeu, Goldman trabalhou em uma fábrica --- de roupas ---, onde tomou contato com as doutrinas socialista e anarquista. Ativista dos direitos da mulher, uniu-se a Margaret Sanger na luta pelo controle de natalidade e deu palestras por todo os Estados Unidos, um dos motivos que levaram à sua perseguição constante pelos agentes do \textsc{fbi}. Foi presa inúmeras vezes entre 1893 e 1921, acusada de incitar rebeliões e opor-se, entre outras ações, à Primeira Guerra Mundial e ao alistamento militar. Em 1931, publica sua autobiografia e mantém intensa atividade como palestrante, além de residir nos principais países da Europa. Durante a Guerra Civil Espanhola, em 1936, apoiou ativamente os anarquistas na luta contra o fascismo. Faleceu em Toronto, Canadá, em 1940.

\textbf{Sobre anarquismo, sexo e casamento} trata de temas como o controle de natalidade, o puritanismo norte-americano, a repressão sexual, o amor livre, o ciúme, a prostituição, a homossexualidade, a desigualdade entre os sexos, a maternidade, a emancipação feminina, o movimento sufragista na Inglaterra e Estados Unidos e a trajetória de uma série de mulheres extraordinárias, dentre elas heroínas e mártires do movimento revolucionário russo. O contexto no qual esses textos foram escritos passou pela Primeira Guerra Mundial, a Revolução Russa e a ascensão do fascismo italiano e do nacional-socialismo na Alemanha. A condição de Goldman como russa, judia, anarquista e crítica implacável do puritanismo estadunidense à autocracia soviética, tornaram-na ainda mais vulnerável em relação ao ativismo pela condição da mulher --- dos Estados Unidos à Rússia, e nos mais diferentes círculos.

\textbf{Mariana Lins} é bolsista de pós-doutorado na Universidade Federal de Sergipe, onde atua também como professora colaboradora nos cursos de graduação e pós-graduação em Filosofia. Autora do livro \textit{O herói niilista e o impossível além do homem}, trabalha na interface entre filosofia, literatura, política e crítica literária, com destaque para a filosofia de Nietzsche, a literatura de Dostoiévski e o movimento populista russo da segunda metade do \textsc{xix} --- em especial, a crítica literária esopiano-revolucionária de dois dos seus maiores expoentes, Nikolai Tchernichévski e Nikolai Dobroliúbov.




