\part[Sobre anarquismo, sexo e casamento]{Sobre anarquismo, sexo\break e casamento}

\pagestyle{baruch}

\chapter{Anarquia e a questão do sexo\footnote{Texto originalmente publicado no
  jornal anarquista \textit{The Alarm}, da cidade de Chicago, em 1896.}}
\markboth{a questão do sexo}{}\label{ref1}

O trabalhador, cuja força e músculos são tão admirados pela prole pálida
e frágil dos ricos --- muito embora o seu trabalho quase não lhe
possibilite manter longe da porta de casa o lobo da fome ---, casa"-se tão
somente para ter uma esposa e uma empregada doméstica, a quem deve
escravizar da manhã até a noite, de modo que ela faça todos os esforços
possíveis para diminuir as despesas. Os seus nervos ficam tão esgotados
pelo esforço contínuo de fazer com que o miserável salário do marido
possa sustentar a ambos que ela se torna cada vez mais irritável, a ponto de 
já não ter mais sucesso em esconder a sua indiferença para com o seu
senhor e mestre, e, infelizmente, logo chega à conclusão de que as suas 
esperanças e planos foram por água abaixo. Assim, começa a pensar que o
casamento é um fracasso.

\section{as correntes ficam cada vez mais pesadas}

Como as despesas aumentam ao invés de diminuir, a esposa, ao perder
completamente a pouca força que tinha com o casamento, sente"-se também
traída, e a preocupação constante e o pavor da fome consomem a sua
beleza pouco tempo depois do casamento. Ela se sente cada vez
mais desanimada, passa a negligenciar as suas tarefas domésticas, e como
não há laços de amor e simpatia entre ela e seu marido, de modo a
dar"-lhes forças para encarar a miséria e a pobreza de suas vidas, ao
invés de se apoiarem um no outro, tornam"-se cada vez mais estranhos,
cada vez mais impacientes com suas falhas.

O homem não pode, como o milionário, ir ao clube, assim, vai para algum
bar e tenta afogar a sua miséria em um copo qualquer de cerveja ou
uísque. A desafortunada parceira de sua miséria, muito honesta
para buscar o esquecimento nos braços de um amante e muito pobre para se
permitir alguma recreação ou distração que sejam legítimas, permanece em
meio ao cenário desmilinguido e mal"-ajambrado que ela chama de lar,
lamentando amargamente a loucura que fez dela a esposa de um homem
pobre.

E, no entanto, não há caminho que lhes permita separarem"-se um do outro.

\section{mas eles devem usá"-las}

Por maiores que sejam os tormentos causados pelas correntes
enroladas nos pescoços dos esposos pela lei e pela Igreja, elas não podem ser quebradas
a não ser que aquelas duas pessoas optem por que sejam serradas.

Caso a lei seja misericordiosa o suficiente para restituir"-lhes a liberdade,
todos os detalhes da vida privada de ambos devem ser
trazidos à luz. A mulher é condenada pela opinião pública e toda a sua
vida é arruinada. O medo dessa desgraça causa, com frequência, o
seu colapso sob o peso esmagador da vida de casada, já que não se atreve
a ensejar um único protesto contra o sistema ultrajante que tritura não
só a ela, mas muitas das suas irmãs.

Os ricos suportam o casamento para evitar o escândalo --- os pobres, para
o bem de seus filhos e pelo medo da opinião pública. Suas vidas são uma
longa prorrogação da hipocrisia e do embuste.

As mulheres que vendem seus favores têm a liberdade de deixar o homem
que as compra a qualquer momento, ao passo que \textit{a respeitável esposa}
não pode libertar"-se de uma união que a degrada.

Todas as uniões não naturais, não santificadas pelo amor, são
prostituição, quer sejam ou não sancionadas pela Igreja e pelo Estado.
Tais uniões não podem ter outra consequência que não seja a de degradar
tanto a moral, quanto a saúde da sociedade.

\section{o sistema é culpado}

O sistema que força a mulher a vender a sua feminilidade e independência\label{sistema}
ao melhor candidato é apenas um ramo do mesmo sistema malévolo que dá a poucos
o direito de viver da riqueza produzida por seus companheiros. A maioria dos indivíduos tem de 
trabalhar e ser escravizados para
conseguir manter a alma e o corpo minimamente unidos, já que os frutos
do seu trabalho são absorvidos por uns poucos vampiros ociosos cercados
de todo o luxo e riqueza que se pode comprar.

Imagine, por um momento, dois cenários do nosso sistema social do século \textsc{xix}.

Olhe para a casa dos ricos, aqueles palácios magníficos cuja mobília
cara colocaria milhares de homens e mulheres necessitados em
circunstâncias confortáveis. Olhe para os jantares desses filhos e
filhas da riqueza, um único prato alimentaria centenas de famintos para
quem uma porção de pão com água é luxo. Olhe para esses adeptos da moda
enquanto passam seus dias inventando novos meios de diversão egoísta --- teatros, bailes, concertos, iatismo, correndo de uma parte do globo para
outra na sua busca louca por alegria e prazer. Depois, então, vire"-se,
por um momento, e olhe para aqueles que produzem a riqueza que paga por
essas diversões excessivas e não naturais.

\section{a outra imagem}

Olhe para aqueles reunidos em porões escuros e úmidos, onde nunca se respira
ar fresco: vestidos com trapos, carregam os seus fardos de miséria do
berço ao túmulo; suas crianças correm pelas ruas nuas e famintas, 
sem ter ninguém para ofertar"-lhes uma palavra amorosa ou cuidado terno, 
crescem na ignorância e superstição, amaldiçoam o dia do seu nascimento.

Olhe para esse contraste surpreendente, vocês moralistas e filantropos,
e me digam quem deve ser culpado por isso! Aquelas que são levadas à
prostituição, seja legal ou não, ou aqueles que conduzem suas vítimas a
tamanha desmoralização?

A causa reside não na prostituição, mas na sociedade em si mesma; no
sistema de desigualdade da propriedade privada, no Estado e na Igreja. No
sistema de roubo legalizado, de assassinato e violação de mulheres
inocentes e crianças indefesas.

\section{a cura para o mal}

Só depois de este monstro ser destruído, estaremos livres da doença que
existe no Senado e em todos os departamentos públicos; nas casas dos
ricos e nos barracões miseráveis dos pobres. A humanidade deve
se tornar consciente da sua força e das suas capacidades, deve ser livre
para iniciar uma nova vida, uma vida melhor e mais nobre.

A prostituição nunca será suprimida pelos meios empregados pelo
reverendo doutor Parkhurst\footnote{Pastor da Igreja Presbiteriana da cidade de Nova York e presidente da Sociedade de Prevenção ao Crime, Charles Henry Parkhurst (1842--1933) deu um sermão, em 1892, cujos trechos publicados num jornal geraram grande repercussão na opinão pública. Nesse, Parkhurst atacou a Tammany Hall (tendência [\textit{political machine}] do Partido Democrata do Estados Unidos) e o Departamento de Polícia da cidade de Nova York pela corrupção estrutural.} e outros reformistas. Existirá enquanto
existir o sistema que lhe dá origem. Quando todos os reformistas
unirem os seus esforços com os daqueles que estão lutando pela abolição desse sistema
que gera o crime e pela construção de um novo que seja baseado na
equidade perfeita --- um sistema que garantirá a cada membro, homem,
mulher ou criança, todos os frutos do seu trabalho e o direito
perfeitamente igual de desfrutar os dons da natureza e de atingir o
mais alto conhecimento ---, a mulher será autossuficiente e independente.
Sua saúde já não será triturada pelo trabalho sem fim e pela escravidão,
não mais será ela a vítima do homem, do mesmo modo que o homem não mais
será possuído por paixões e vícios doentios e não naturais.

\section{sonho de anarquista}

Cada um entrará na vida de casado com força física e confiança moral
mútua. Cada um amará e estimará o outro, e trabalhará não apenas para o
seu próprio bem"-estar: por serem felizes, desejarão também a felicidade
universal da humanidade. Os filhos de tais uniões serão fortes e
saudáveis, tanto no que diz respeito ao corpo, quanto à mente, e
honrarão e respeitarão os seus pais, não porque é o seu dever fazê"-lo,
mas porque os seus pais merecem. Eles serão instruídos e cuidados por
toda a comunidade e serão livres para seguir as suas inclinações, e não
haverá necessidade de ensiná"-los a bajulação e a arte chula de rapinar
seus semelhantes. Seu objetivo na vida será não o de obter poder sobre
os seus irmãos, mas o de ganhar o respeito e a estima de cada membro da
comunidade.

\section{divórcio anarquista}

Para o caso de a união entre um homem e uma mulher tornar"-se insatisfatória
e desagradável, eles irão de modo tranquilo e amigável separar"-se, não
rebaixarão as várias relações matrimoniais através da continuação de uma
união incompatível.

Se ao invés de perseguirem as vítimas, os reformistas da vez unirem seus
esforços para erradicar as causas, a prostituição já não mais
desgraçará a humanidade.

Reprimir uma classe e proteger outra é pior do que loucura. É um crime.
Não virem as costas, vocês, homens e mulheres morais.

Não permitam que o seu preconceito os influencie: olhem para a questão
de um ponto de vista imparcial.

Ao invés de exercer a sua força inutilmente, deem"-se as mãos e ajudem a
abolir esse sistema corrupto e doente.

Se a vida de casado não lhe roubou a honra e o amor"-próprio, se você tem
amor por aqueles a quem chama de filhos, você deve, para o seu próprio
bem como para o bem deles, buscar a emancipação e instituir a liberdade.
Aí então, e não antes, irão os males do matrimônio, finalmente, cessar.

\chapter{Casamento\footnote{Originalmente publicado no então mais afamado jornal
  anarquista dos Estados Unidos, com sede em Portland, Oregon, \textit{The
  Firebrand}, em 1897. No ano seguinte, o jornal viria a mudar o nome para \textit{Free
    Society}. É o primeiro texto sobre a temática do casamento publicado por
  Goldman.}}
  \markboth{casamento}{}

Casamento. Quanta tristeza, miséria, humilhação; quantas lágrimas e maldições; que
agonia e sofrimento essa palavra tem trazido à humanidade. Do seu
nascimento até a nossa atualidade, homens e mulheres crescem sob o jugo
ferrenho da instituição casamento, de modo que parece não haver nenhum
alívio, nenhuma maneira de escapar dela.\label{casamento}

Em todos os tempos, em todas as eras, os oprimidos lutaram para quebrar
as correntes dessa escravidão mental e física. Após milhares de vidas
nobres terem sido sacrificadas nas fogueiras e nas forcas, e outras
terem perecido nas prisões ou nas mãos impiedosas das inquisições, as
ideias desses destemidos heróis se realizaram. Ainda que os
dogmas religiosos, o feudalismo e a escravidão negra tenham sido abolidos e
novas ideias, mais avançadas, amplas e claras tenham vindo à tona, o que vemos
sempre e mais uma vez é a humanidade pobre e pisoteada lutando por seus
direitos e independência. O ponto é que a mais rude e tirânica de todas
as instituições, o casamento, permanece tão firme quanto antes, e \textit{ai}
daqueles que ousam duvidar da sua sacralidade. Colocar o casamento sob
discussão enfurece não apenas cristãos e conservadores, mas também
liberais, livres pensadores e radicais. O que é isto que faz com que
todas essas pessoas defendam o casamento? O que faz com que se agarrem a
este preconceito (porque tal atitude não é nada mais do que
preconceito)? Isto se dá porque são as relações conjugais o fundamento
da propriedade privada e, portanto, o fundamento do nosso sistema cruel\label{cruel}
e desumano. Com a riqueza e a abundância, de um lado, e a inatividade,
exploração, pobreza, fome e crime de outro, abolir o casamento significa
abolir tudo o que foi acima mencionado. Alguns progressistas têm optado
por reformas ou melhorias nas nossas leis matrimoniais, já não permitem
que a Igreja interfira nas suas relações maritais. Outros vão além,
casam"-se livremente, isto é, sem o consentimento da lei. No entanto,
mesmo esta forma de casamento é tão coercitiva, tão \textit{sagrada} quanto
a antiga, porque o problema não é a forma ou o tipo de casamento que
podemos estabelecer, mas a coisa, a própria coisa em si mesma que é
questionável, nociva e degradante. Não importa o quanto mude, sempre dá
direito e poder ao homem sobre a sua esposa, não apenas sobre o seu corpo,
mas também sobre as suas ações, seus desejos; na verdade, sobre a sua vida
como um todo. E como poderia ser diferente?

Por detrás da relação entre um determinado homem e uma determinada
mulher, há a história da evolução da relação entre os dois sexos em
geral, a história que conduziu à diferença de posição e privilégios
entre os sexos existente ainda hoje.

Dois jovens podem ficar juntos, mas a relação que estabelecem é em
grande medida determinada por causas sobre as quais não têm controle.
Eles sabem pouco um do outro, já que a sociedade mantém os dois sexos
separados, o menino e a menina são criados a partir de diretrizes bem
diferentes. Tal como Olive Schreiner escreve em sua \textit{História de
uma fazenda africana}:\footnote{O livro original, em inglês, chama-se \textit{Story of an African Farm}.} ``Ao menino
foi ensinado \textit{a ser}, à menina, \textit{a parecer}.''\footnote{Com essa citação,
  Goldman faz referência às palavras da heroína do romance. Publicado em
  1883, sob o pseudônimo masculino de Ralph Iron, esse livro da
  escritora sul"-africana Olive Schreiner obteve, para o bem e para o
  mal, reconhecimento internacional imediato, sendo atualmente
  considerado como o primeiro romance propriamente feminista na língua
  inglesa.} Dito mais precisamente: ao menino é ensinado a ser
inteligente, brilhante, esperto, forte, atlético, independente e
autossuficiente; a desenvolver as suas faculdades naturais, a seguir
suas paixões e desejos. À menina é ensinado a se vestir, a ficar
diante de um espelho e admirar a si mesma, a controlar as suas emoções, as
suas paixões e os seus desejos; é ensinado a esconder as suas dificuldades
intelectuais e a se valer da sua inteligência e habilidade pouco desenvolvidas com vistas a um único fim: sendo este o modo mais rápido e eficiente de
arranjar um marido, de arranjar um casamento vantajoso. E o resultado é
que os dois sexos dificilmente entendem a natureza um do outro,
dificilmente entendem que seus interesses intelectuais e ocupações são
diferentes. A opinião pública separa de maneira rígida os seus direitos
e deveres, sua honra e desonra. O tema do sexo é um livro selado para as
meninas, porque lhes foi dado a entender que é impuro, imoral e
indecente até mesmo mencionar algo que esteja relacionado à questão do
sexo. Ao garoto, é um livro cujas páginas lhe trouxeram doença e vícios
secretos e, em alguns casos, ruína e morte.

Dentre as classes ricas, há tempos está fora de moda se apaixonar.
Homens da sociedade se casam, depois de uma vida de devassidão e luxúria,
para reestabelecer a sua saúde arruinada. Há também aqueles que, por terem
perdido o seu capital em jogos de azar ou em negócios, chegam à decisão
de que uma herdeira é justamente o que precisam, sabem muito bem que o
laço matrimonial não os impedirá, de modo algum, de esbanjar o cabedal
da sua noiva endinheirada. A garota rica, criada para ser pragmática e
sensível, e acostumada a viver, respirar, andar e vestir"-se apenas de
acordo com a moda, oferece os seus milhões por algum título ou por um
homem de boa posição social. Ela tem apenas um consolo, o de
que a sociedade permite mais liberdade de ação a uma mulher casada e, no
caso de ela vir a se desapontar no casamento, estará em posição de
gratificar os seus desejos de uma outra maneira. Como sabemos, as
paredes dos \textit{boudoirs}\footnote{Nome então dado ao aposento da casa
  exclusivo à mulher.} e dos salões são surdas e mudas, de modo que um
pouco de prazer na interioridade delas não é nenhum crime.

No que diz respeito aos homens e mulheres da classe trabalhadora, o
casamento é algo bem diferente. O amor não é raro como na classe alta e,
muito frequentemente, ajuda ambos a suportar os desapontamentos e
tristezas da vida, mas mesmo nesse caso a maioria dos casamentos dura
apenas um período curto até que seja engolido pela monotonia da vida
cotidiana e da luta pela existência. Aqui também o homem trabalhador se
casa, porque ele se cansa de uma vida em pensionatos e nutre o desejo de
construir uma casa que seja sua, onde possa encontrar algum conforto.
Seu objetivo principal, portanto, é encontrar uma garota que irá
preparar uma boa comida e cuidar dos afazeres domésticos; alguém que se
preocupará exclusivamente com a sua felicidade, com os seus prazeres;
alguém que olhará para ele como o seu senhor, seu mestre, seu defensor,
seu suporte, como o único ideal pelo qual vale a pena viver. Uma outra
possibilidade é a expectativa de que a jovem com quem se case esteja
disposta a trabalhar e, assim, o ajudar a poupar alguns centavos a mais
para os dias de chuva. Não obstante, após alguns meses dessa chamada
felicidade, ele acorde para a realidade amarga de que sua esposa em
breve se tornará uma mãe, que ela não poderá trabalhar, que as despesas
irão aumentar e que, se antes ele conseguia administrar o pequeno salário
autorizado pelo seu \textit{gentil} senhor, agora este já não será suficiente
para sustentar a sua família.

A jovem que passa a infância e parte da vida adulta na fábrica, ao
sentir a sua força a abandonando, pinta para si mesma a condição
terrível de ter de permanecer para sempre uma vendedora, nunca certa do
seu trabalho, de modo que é compelida a buscar por um homem, um bom
marido, o que significa buscar por aquele que possa sustentá"-la e lhe
dar uma boa casa. Ambos, o homem e a mulher, casam pelo mesmo propósito,
com a única exceção de que, no que diz respeito ao homem, não é esperado
que ele abra mão da sua individualidade, do seu nome, da sua
independência, ao passo que a garota tem de vender a si mesma, seu corpo
e alma, pelo prazer de ser a esposa de alguém; portanto, eles não se
encontram em pé de igualdade, e onde não há igualdade não pode haver
harmonia. A consequência é que pouco depois dos primeiros meses, ou
sendo mais flexível, após o primeiro ano, ambos chegam à conclusão de
que o casamento é um fracasso.

Como as suas condições se tornam cada vez piores, o que se soma ao
aumento do número de filhos, a mulher, cada vez mais, sente"-se
desanimada, infeliz, insatisfeita e fraca. Sua beleza logo a abandona;
por conta do trabalho pesado, das noites sem dormir, da preocupação com
os seus pequenos e das discordâncias e brigas com o marido, ela se
arruína fisicamente e amaldiçoa o momento que a tornou esposa de um
homem pobre. Uma vida tão sombria e miserável certamente não favorece a
manutenção do amor e respeito mútuos. O homem pode ao menos esquecer
suas desgraças na companhia dos amigos, pode se deixar absorver pela
política, ou afogar sua desgraça num copo de cerveja. A mulher está
acorrentada à casa por milhares de obrigações; ela não pode desfrutar,
tal como o seu marido, de alguma diversão, afinal nem tem recursos para
isso, como também lhe é recusado, pela opinião pública, os mesmos
direitos dele. Ela tem de carregar a cruz até a morte, porque as
leis matrimoniais não conhecem a piedade, a não ser que ela deseje expor
a sua vida de casada ante os olhos de alguma senhora Grundy,\footnote{Personagem
  da peça de Thomas Morton \textit{Acelere o
  arado}, no original inglês \textit{Speed the Plough}, de 1798. A expressão \textit{Mrs.\,Grundy} tornou"-se na língua
  inglesa indicativa de extrema rigidez para com a observância das
  convenções sociais e de forte censura moral para qualquer desvio
  mínimo dessas convenções.} e mesmo assim ela apenas pode quebrar as
correntes que a amarram ao homem que odeia se tomar toda a culpa
sobre os próprios ombros, e se tiver energia suficiente para se inserir
no mundo dos desonrados pelo resto da vida. Quantas têm coragem para
fazer isso? Pouquíssimas. Aqui e ali, acontece, tal como a luz de um
relâmpago, que uma mulher, como a Princesa de Chimay,\footnote{Aqui, Goldman se refere à estadunidense Clara Ward (1873--1916), muito famosa na época do presente texto. Ward, que se casou em 1890 com o príncipe de Chimay, da Bélgica, fez a alegria dos norte"-americanos ao se tornar uma princesa. Não obstante, seis anos após o casamento real, e já mãe de dois filhos, ela abandonou tudo para fugir com o húngaro Rigó Jancsi, que ganhava a vida tocando música cigana. Ao lado de Jancsi, com quem se casou, Clara se tornou dançarina de teatro e cabarés, e musa de fotógrafos e artistas. Embora tenha se divorciado de Jancsi pouco tempo após o casamento, casou"-se ainda mais duas vezes.} tenha coragem
suficiente para romper as barreiras convencionais e seguir os desejos do
seu coração. Mas essa exceção é uma mulher rica, que não depende de
ninguém. Uma mulher pobre tem de levar em conta os seus pequenos; ela é
menos afortunada do que a sua irmã rica. Além disso, a mulher que
permanece na servidão é chamada de respeitável: não importa que toda a
sua vida seja uma longa cadeia de mentiras, engano e traição, ela olha
de cima e com nojo para as suas irmãs que foram forçadas pela sociedade
a vender os seus encantos e afetos nas ruas. Não importa quão pobre,
quão miserável uma mulher casada possa ser, ela ainda se considerará
acima da outra, a quem chama prostituta --- que é marginalizada, odiada e
desprezada por todos, inclusive por aqueles que não hesitam em comprar
seus abraços, considerando a pobre coitada como um mal necessário, e
mesmo algumas pessoas muitíssimo bondosas sugerem o confinamento desse
mal num distrito específico de Nova York, de modo a \textit{purificar} todos
os outros distritos da cidade. Que farsa! Os reformistas poderiam muito
bem exigir, a partir disso, que também todas as mulheres casadas
residentes em Nova York fossem expulsas, já que certamente não são
moralmente superiores a qualquer mulher da noite. A única diferença
entre essa última e a mulher casada é que uma vendeu a si mesma como
escrava privada durante toda a sua vida, por uma casa ou um título, e a
outra vende a si mesma pelo período de tempo que deseja; a prostituta tem o
direito de escolher o homem a quem presenteará com as suas afeições, ao
passo que a mulher casada não tem esse direito em absoluto; ela deve se
submeter ao seu senhor, não importa o quanto esse domínio seja
abominável para ela, deve obedecer aos seus comandos; ela tem de
dar à luz os filhos dele, mesmo que custe a sua força e saúde; em uma palavra,
ela se prostitui a cada hora, a cada dia da sua vida. Eu não
consigo encontrar outro nome para esta condição hedionda, humilhante e
degradante das minhas irmãs casadas que não seja o de prostituição da pior
espécie, com a única exceção de que esta é legal, ao passo que a outra é ilegal.

Não irei lidar aqui com os poucos casos excepcionais de casamentos que
são baseados no amor, na estima e no respeito; essas exceções apenas
confirmam a regra. Quer seja legal ou ilegal, a prostituição é de
qualquer forma antinatural, dolorosa e desprezível, e eu sei muito bem
que as condições não podem ser modificadas até que esse sistema infernal
seja abolido, como sei que não é apenas a dependência econômica das
mulheres que tem causado a sua escravidão, mas também a sua ignorância e
preconceito. Sei ainda que muitas das minhas irmãs poderiam ser
libertadas agora mesmo, não fosse pela nossa instituição casamento que
as mantém na ignorância, estupidez e preconceito. Por conseguinte,
considero como o meu maior dever denunciar o casamento, não apenas na
sua forma antiga, mas também o chamado casamento moderno, a ideia de ter
uma esposa e doméstica, a ideia da posse privada de um sexo pelo outro.
Reivindico a independência da mulher; seu direito de sustentar a si
mesma; de viver para si mesma; de amar quem quer que deseje ou quantos
deseje. Reivindico a liberdade para ambos os sexos, liberdade de ação,
liberdade no amor e liberdade na maternidade.

Não venham me dizer que um tal estado só pode ser alcançado com a
implementação da anarquia; isto está completamente errado. Se nós
quisermos concretizar a anarquia precisamos, em primeiro lugar, de mulheres livres, mulheres que sejam economicamente tão
independentes quanto os seus irmãos, pois se não tivermos mulheres
livres, não poderemos ter mães livres, e se as mães não forem livres,
não podemos esperar que a nova geração nos ajude a atingir o nosso
objetivo, que é o estabelecimento de uma sociedade anarquista.

Para vocês livres pensadores e liberais, vocês que aboliram apenas um deus, ao mesmo tempo que criaram muitos outros para adorar; para vocês, radicais e
socialistas que ainda mandam seus filhos a escolas cristãs; e para
todos aqueles que fazem concessões aos padrões morais do nosso tempo;
para todos vocês, eu digo que é a falta de coragem o que faz com que
se apeguem e defendam a instituição casamento, pois ao mesmo tempo
em que admitem o seu absurdo na teoria, não têm energia para
desafiar a opinião pública e viver a prática em sua própria vida. Vocês
matraqueiam sobre a igualdade dos sexos na sociedade do futuro, mas\label{matraqueiam}
pensam que é um mal necessário que a mulher deva sofrer no presente.
Vocês dizem que a mulher é inferior e mais fraca, e ao invés de
auxiliá"-la a se tornar mais forte, ajudam a mantê"-la numa posição
degradante. Vocês exigem de nós exclusividade, mas amam a variedade de
parceiras e aproveitam essa variedade sempre que têm a chance.

O casamento --- maldição de tantos séculos, causa do ciúme, suicídio e
crime --- tem de ser abolido se nós desejamos que as novas gerações
floresçam com homens e mulheres saudáveis, fortes e livres.

\chapter[O que há na anarquia para as mulheres?]{O que há na anarquia para\break as mulheres?\footnote{Entrevista concedida por Emma Goldman
  ao jornal \textit{St.\,Louis Post"-Dispatch} em 24 de outubro de 1895, por
  ocasião da sua breve estadia na cidade de St.\,Louis, acompanhada de
  perto não só pela imprensa, como também pela polícia. Apesar da
  perseguição política e ameaça de prisão, o objetivo de Goldman foi
  atingido com sucesso, dado que a sua palestra, em 17 de outubro de
  1895, teve a audiência de centenas de pessoas.}}
\markboth{anarquia para as mulheres}{}

\setlength{\epigraphwidth}{.65\textwidth}
\begin{epigraphs} 
\qitem{O que a anarquia oferece a mim, uma mulher? Oferece mais à mulher do que a qualquer outro, tudo o que ela não tem, liberdade e igualdade.}{}
\end{epigraphs} 

\noindent{}\textit{De modo rápido e sincero, Emma Goldman, a sacerdotisa da anarquia,
exilada da Rússia, temida pela polícia}\footnote{Quatro dias antes da publicação desta entrevista, o mesmo jornal, \textit{St.\,Louis Post"-Dispatch}, havia noticiado, em tom jocoso, que rumores falsos acerca do dia e local da primeira conferência de Goldman em St.\,Louis transformaram \textit{um encontro de vermelhos} em \textit{um encontro de policiais}.} \textit{e agora convidada
pelos anarquistas de St.\,Louis, dá esta resposta à minha questão.}

\textit{Eu a encontrei na Avenida Oregon, número 1722, numa casa
de tijolos antiga, com dois andares, residência de um dos seus
simpatizantes --- não de um parente como havia sido divulgado.}

\textit{Fui recebido por uma alemã simpática e corpulenta e conduzido a uma sala de jantar tipicamente alemã, muito arrumada e limpa. Depois de espanar a
cadeira em que eu devia me sentar com o seu avental, ela me anunciou à
corajosa e pequena livre pensadora. Encontrei Emma Goldman tomando café
e comendo pão com geleia no seu desjejum. Ela estava elegantemente
vestida com uma camisa de percal acinturada de gola e punhos brancos e
saia, os pés estavam calçados com um par de chinelos despojados, de
pano. Ela não parecia em nada com uma niilista russa que caso adentre as
fronteiras da sua terra natal, será mandada para a Sibéria.}

\begin{center}\adforn{47}\end{center}

``Você acredita em casamento?'', perguntei.

``Não'', respondeu a miúda anarquista, tão prontamente quanto antes.
``Eu acredito que quando duas pessoas se amam ninguém tem nada a ver com
isso, seja juiz, ministro, um tribunal ou grupo de pessoas.
Apenas os envolvidos podem determinar as relações que desejam estabelecer um com o
outro. Quando essa relação se torna incômoda para ambas as partes ou
apenas para uma delas, deve"-se, então, terminar tão calmamente quanto se
começou.''

A senhorita Goldman fez um leve aceno com a cabeça de modo a enfatizar
as suas palavras, e que bela cabeça era aquela, coroada com um cabelo
castanho macio, penteado de lado. Seus olhos são de um azul intenso,
sua pele, extremamente branca. Seu nariz, embora largo de tipo
teutônico, é bem formado. Seu tipo, como um todo, é mais alemão do que
russo. Ela é de estatura baixa e a sua figura como um todo é bastante harmoniosa.
O único problema físico grave que ela tem está nos seus olhos. Ela é tão míope que,
mesmo com óculos, mal consegue distinguir as letras impressas.

``Uma aliança deve ser formada'', ela continuou, ``não como a de agora,
para dar à mulher um sustento e uma casa, mas, sim, porque o amor
existe, e esse estado das coisas só pode ser suscitado com uma revolução
interna, em suma, com a anarquia.''\label{alianca}

Ela disse isso tão calmamente como se tivesse acabado de expressar um
fato cotidiano comum, mas o brilho nos seus olhos mostrava as
``revoluções internas'' que já estavam em curso no seu cérebro ocupado.

``O que a anarquia promete à mulher?''

``Oferece tudo à mulher --- liberdade e igualdade --- tudo o que ela não
tem no momento.''

``A mulher não é livre?''

``Livre! Ela é uma escrava do marido e dos filhos. Ela deve tomar parte
nos assuntos do mundo tanto quanto o homem; ela deve ser sua igual
perante o mundo, o que ela de fato é na realidade. A mulher é tão capaz
quanto o homem, mas quando ela trabalha, ela recebe um salário menor.
Por quê? Porque ela usa saia, ao invés de calças.''

``Mas o que vai ser da vida doméstica ideal e de tudo aquilo que envolve
a maternidade, de acordo com a perspectiva do homem?''

``Vida doméstica ideal! A mulher, ao contrário da \textit{rainha do lar} cantada
nos livros de histórias, é a criada, a amante e a escrava do marido e dos
filhos. Ela perde a sua individualidade inteiramente, mesmo o seu nome
não lhe é permitido manter. Ela é a esposa de John Brown ou a esposa de
Tom Jones, ela é isto e nada mais. Esse é o peso da mulher.'' 

A senhorita Goldman tem um sotaque agradável. Ela enrola os seus \textit{erres}
transformando"-os em \textit{vês}, numa pronúncia verdadeiramente russa. Ela
gesticula bastante. Quando ela se anima, suas mãos, pés e ombros todos
ajudam a ilustrar os significados do que diz.

``O que você proporia para as crianças da era anarquista?''

``As crianças morariam em espaços comuns, como grandes colégios
internos, onde seriam cuidadas e educadas de modo apropriado que, em
todos os aspectos, seriam bons, e, na maioria dos casos, ainda
melhores do que os cuidados e a educação que recebem em suas próprias
casas. Pouquíssimas mães sabem como cuidar adequadamente dos filhos. É
uma ciência que pouquíssimos aprenderam.''

``Mas e as mulheres que desejassem uma vida doméstica e cuidar dos seus
próprios filhos, o que seria delas?''

``Ah, sim, claro, as mulheres que desejassem poderiam criar seus filhos
em casa e confinar a si mesmas com atividades domésticas, com tantas
limitações quanto desejassem. Mas o ponto é que isto daria às mulheres
que desejam algo maior, a chance de atingir a altura que almejam. Sem
pobres ou capitalistas, e um propósito em comum, esta Terra
proporcionará o Céu que os cristãos procuram no outro mundo.''

Ela mirou fixamente o fundo da xícara vazia como se visse na sua
imaginação o Estado ideal, já presente na realidade.

``Quem irá cuidar das crianças?'', perguntei, interrompendo o seu
devaneio.

``Todo mundo'', ela respondeu, ``tem preferências e habilidades
adequadas a uma determinada ocupação. Eu sou enfermeira de formação.
Gosto de cuidar de doentes. A mesma coisa irá acontecer com algumas
mulheres. Elas vão querer cuidar e ensinar às crianças.'' 
%cf. transitividade

``As crianças não perderiam com isso o amor por seus pais e não
sentiriam falta da sua presença?'' Um pensamento sobre os pequeninos
serem relegados a um tipo de orfanato cruzou a minha mente.

``Os pais terão as mesmas oportunidades de dar segurança e afeto que têm
agora. Eles passarão com os filhos o tempo que desejarem, estarão ao seu
lado tão frequentemente quanto desejarem. Essas serão crianças do amor
--- saudáveis e determinadas ---, e não como as de agora que, na maioria
dos casos, são nascidas em meio ao ódio e brigas domésticas.''

``O que você chama de amor?''

``Quando um homem ou uma mulher encontra no outro uma ou mais qualidades
que admira e sente, com isso, um desejo irresistível de fazer bem à
pessoa, até diante do sacrifício de desejos pessoais; quando há aquele
algo sutil que une ambos, aquilo que aqueles que amam identificam ao
senti"-lo no mais íntimo do seu ser --- a isto chamo amor.'' Ela terminou
de falar e a sua face foi banhada de rubor.

``É possível amar mais de uma pessoa de uma vez?''

``Eu não vejo por que não --- se ela encontrar as mesmas características
que lhe fascinam em diferentes pessoas. O que poderia impedir de amar as
mesmas características em todas as diferentes pessoas?

Se, como disse antes, nós deixamos de amar um homem ou uma
mulher e encontramos outra pessoa, simplesmente discutimos o assunto e mudamos
com tranquilidade nosso modo de vida. Os assuntos privados de uma família
não precisam ser discutidos nos tribunais, não precisam se tornar
domínio público. Ninguém pode controlar os sentimentos, por causa disso
é que não deve haver ciúmes.''

``Mágoas? Oh, sim, com certeza'', ela disse com tristeza, ``mas não
ódio, porque ele ou ela se cansou da relação. A humanidade sempre terá
mágoas enquanto o coração bater no peito.''

``Minha religião'', ela riu repetidamente. ``Eu tinha fé no judaísmo
quando criança --- como você sabe, eu sou judia ---, mas agora sou ateia.
Ninguém foi capaz de provar a inspiração divina da \textit{Bíblia} ou a
existência de Deus de um modo que me satisfizesse. Eu não acredito numa
vida futura que não seja a vida futura da matéria que compõe o corpo
humano. Eu acredito que a matéria ganha vida novamente de alguma outra
forma, não acho que algo uma vez criado seja para sempre perdido --- continua primeiro numa forma, depois em outra. Não existe essa
coisa de alma --- tudo é matéria.''

A bela senhorita Goldman terminou de falar e um rubor se fixou nas suas
bochechas quando lhe perguntei se tinha intenção de casar.

``Não; eu não acredito em casamento no que diz respeito aos outros e,
certamente, não sou de pregar uma coisa e fazer outra.''

Ela estava sentada de modo natural com uma perna cruzada sobre a outra.
Ela é, em todos os sentidos, uma mulher de aparência bem feminina, ainda
que dotada de uma mente masculina e de grande coragem.

Ela riu ao dizer que havia cinquenta policiais na sua palestra na quarta
à noite e acrescentou: ``Se tivessem jogado uma bomba, eu certamente
seria responsabilizada por isso.''

\chapter{A nova mulher\footnote{Conferência transcrita e publicada no jornal
  anarquista \textit{Free Society}, em 1898. A temática da \textit{nova mulher} foi
  desenvolvida por Goldman em algumas conferências daquele ano.}}
\markboth{A nova mulher}{}

A história bíblica da desigualdade e inferioridade da mulher é baseada
na declaração de que ela foi criada da costela do homem. A mulher não
pode, sem as mesmas oportunidades, chegar a uma condição de igualdade
para com o homem e a consequência disso é que as mulheres são escravas
da sociedade, condição intensificada pelo código do casamento. O
despotismo gera revolta no povo e sempre gerará revolta, é uma
necessidade. A mulher é criada para ser vista e exposta e isso é uma
vergonha para a sociedade. Sua única missão é casar e ser esposa e mãe e
atender às demandas de um marido que por esse motivo irá sustentá"-la.
Assim, ela degrada a si mesma. As mães atuais não são culpadas por
sua condição, isso ocorre porque elas imitam as suas próprias mães. Uma
mãe assim educada não pode ter qualquer ideia do verdadeiro conhecimento
sobre como educar os próprios filhos, isto é, sobre a profissão de
educar os filhos e, desse modo, sob este sistema, ela nunca educa os
filhos como deveria. As mães são dominadas pelos filhos, o que é
incompatível com ser uma boa mãe.

O dever de uma esposa é considerado um assunto impuro para uma jovem
mulher que ainda não se casou, assim a jovem ignorante é forçada a
enfrentar despreparada a batalha cujas consequências dizem respeito a toda a sua vida.
Outro grande erro na nova mulher ideal, um erro que deve ser condenado,
é o de imitar o homem, o de tornar"-se masculina por considerar o homem
superior à mulher. Nenhuma mulher decente deve emulá"-los. Precisamos,
primeiro, do Novo Homem. Em todas as coisas as mulheres são iguais aos
homens, mesmo no campo da produção. Os mais radicais não diferem, neste
ponto, dos cristãos; eles não desejam que suas esposas se tornem
radicais; também eles se julgam imprescindíveis à sua proteção. Enquanto
a mulher necessitar de proteção, ela não estará em pé de igualdade, pois
precisamos proteger apenas aqueles que são fracos. Um dos aspectos do
caráter do homem que lhe torna invasivo é que ele é muito autoritário
ao forçar o progresso da mulher; embora ele próprio venha
evoluindo lentamente, comete o erro fatal de buscar assegurar mais
liberdade para a mulher por meio do mesmo mecanismo que conduziu à sua própria
escravização, a saber, a autoridade. Somente a oposição a isso pode
corrigir esse mal.

Leis matrimoniais desprezíveis e a adesão a elas tendem a aumentar ainda
mais a degradação. A afirmação de que a liberdade nas relações sexuais é
uma lei natural é interpretada como luxúria gratuita. A lei do amor é
que governa as relações em todos os aspectos, somente o amor é capaz de
realizar a lei. A beleza da maternidade, sobre a qual os poetas cantaram
e escreveram, é uma farsa, e não pode ocorrer enquanto não tivermos
liberdade econômica.\label{maternidade}

Os homens são todos heróis em casa, mas covardes lá fora. As mulheres,
se pudessem, seriam tão injustas ao votar quanto os homens. Elas são tiranas assim como são os homens. A mulher, para ser livre, deve ser amiga e companheira do homem e de modo
recíproco. O indivíduo é o ideal de liberdade. Não temos deveres para com ninguém que não seja nós mesmos.
Quando a mulher universal tiver finalmente compreendido esse ideal,
então todas as leis protecionistas, todas as leis que tenham por objetivo
proteger, e indiretamente denotar sua suposta fraqueza, irão
desaparecer, assim como esse sistema adúltero e, com ele, a caridade e
todos os demais males que o acompanham. Em suma, o movimento da nova
mulher exige um avanço igual da parte do homem moderno.


\chapter{A hipocrisia do puritanismo\footnote{Sétimo capítulo da coletânea
  \textit{Anarquismo e outros ensaios}, em inglês \textit{Anarchism and Other Essays}, publicada por Goldman na revista \textit{Mother Earth} em 1910.}}\label{hipocrisia}
\markboth{A hipocrisia do puritanismo}{}

Ao relacionar puritanismo e arte americana, o sr.\,Gutzon Borglum\footnote{Escultor norte"-americano (1867--1941) que ficou conhecido sobretudo pela obra colossal, ainda hoje bastante afamada, no Monte Rushmore, no qual esculpiu os rostos de quatro dos presidentes dos Estados Unidos: George Washington, Thomas Jefferson, Theodore Roosevelt e Abraham Lincoln.} disse:

\begin{quote}
O puritanismo nos tornou autocentrados e hipócritas por tanto tempo,
que a sinceridade e reverência pelo que é natural em nossos impulsos
foram, como não poderia ser diferente, degeneradas em nós, com o
resultado de que não pode haver nem verdade, nem individualidade em
nossa arte.
\end{quote}

O sr.\,Borglum acrescenta ainda que o puritanismo transformou a própria vida
em algo impossível. Para além da arte, para além do esteticismo, a vida
é representação da beleza em milhares de variações; é, de fato, o
panorama gigante da eterna mudança. O puritanismo, por outro lado,
repousa sobre uma concepção de vida fixa e imóvel; baseia"-se na ideia
calvinista de que a vida é uma maldição imposta ao homem pela ira de
Deus. De modo a redimir"-se, um homem deve fazer penitências
constantemente, deve repudiar todo impulso natural e saudável, e virar
as costas para a alegria e a beleza.

O puritanismo celebrou o seu reino de terror na Inglaterra, ao longo dos
séculos \textsc{xvi} e \textsc{xvii}, destruindo e esmagando toda manifestação da arte e
da cultura. Foi o espírito do puritanismo que roubou Shelley dos seus
filhos, porque não se curvou ante os ditos da religião. Foi o mesmo
espírito tacanho que transformou Byron num estranho na sua terra natal, porque
o grande gênio se rebelou contra a monotonia, apatia e mesquinhez do seu
país. Foi o puritanismo também que forçou algumas das mulheres mais
livres da Inglaterra à mentira do casamento convencional: Mary Wollstonecraft e,
mais tarde, George Eliot. E recentemente o puritanismo exigiu mais um
tributo: a vida de Oscar Wilde. De fato, o puritanismo nunca deixou de
ser o fator mais pernicioso nos domínios de John Bull,\footnote{Semelhante
  ao Tio Sam dos Estados Unidos, John Bull é o símbolo que personifica a
  Inglaterra.} agindo como o censor da expressão artística das pessoas e
estampando sua aprovação apenas na estupidez que constitui a
respeitabilidade da classe média.

Portanto, é puro chauvinismo britânico apontar para a América como o
país do provincialismo puritano. É verdade que aqui nossas vidas são
atrofiadas pelo puritanismo, que mata tudo o que é natural e saudável em
nossos impulsos. Mas é igualmente verdade que é com a Inglaterra
que estamos em débito por ter transplantado esse espírito para o solo
americano. Foi legado a nós pelos peregrinos. Fugindo da perseguição e
da opressão, a fama dos peregrinos de Mayflower se estabeleceu no Novo
Mundo através do seu reino de tirania, puritanismo e crime. A história
da Nova Inglaterra e, especialmente, a de Massachusetts, é tão cheia de
horrores que transformou a vida em angústia, alegria em desespero, a
naturalidade em doença e a honestidade e a verdade em mentiras e
hipocrisias hediondas. A cadeira do castigo\footnote{Em inglês, \textit{ducking"-stool}. literalmente \textit{mergulho nas fezes}, consistia numa cadeira de madeira ou ferro, presa num dos polos de uma
  alavanca que era baixada repetidas vezes na água, com o culpado
  amarrado. Método muito utilizado para punir mulheres, especialmente
  pelos \textit{crimes} de bruxaria, prostituição ou de dar à luz a filhos
  bastardos.} e o pelourinho, dentre muitos outros instrumentos de
tortura, eram os métodos ingleses favoritos na tarefa de purificação da
América.

Boston, a cidade da cultura, entrou para os anais do puritanismo como a
\textit{cidade sangrenta}. Até rivalizava com Salem, em sua perseguição cruel
às opiniões religiosas não autorizadas. No agora famoso Common Park, em
Boston, uma mulher seminua, com um bebê em seus braços, foi publicamente
açoitada pelo crime de liberdade de expressão; e no mesmo local, Mary
Dyer, outra mulher quaker, foi enforcada em 1659. De fato, Boston foi
mais de uma vez palco de crimes arbitrários perpetrados pelo
puritanismo. Salem, no verão de 1692, assassinou 18 pessoas por
bruxaria. A cidade de Massachusetts não foi a única a expulsar o diabo
com fogo e enxofre. Como Canning disse tão precisamente: ``Os peregrinos
infestaram o Novo Mundo para equilibrar a balança com o
Velho.''\footnote{Referência a uma passagem do livro \textit{The Pilgrim
  of Scandinavia} {[}O peregrino da Escandinávia{]} de Charles John
  Spencer George Canning.} Os horrores daquele período encontraram a sua
mais suprema expressão no clássico americano \textit{A letra
escarlate}.\footnote{Nathaniel Hawthorne,~\textit{A letra escarlate}
  (1850). O livro, ambientado na
  rígida comunidade puritana de Boston em meados do século \textsc{xvii}, narra a história de Hester Prynne, que por ter cometido adultério é obrigada a usar um \textsc{a} de adúltera, em cor escarlate, bordado na sua roupa.}

O puritanismo já não utiliza o parafuso de polegar\footnote{Instrumento
  de tortura utilizado, geralmente, em interrogatórios. Consistia em
  esmagar as falanges dos polegares gradativamente tendo de, para isso,
  apenas girar o parafuso.} e o chicote; não obstante ainda possua o
mais pernicioso domínio sobre as mentes e sentimentos do povo americano.
Nada mais é capaz de explicar o poder de Comstock.\footnote{Antony
  Comstock (1844--1915) foi um reformista americano, responsável pela lei
  de 1873 (\textit{Comstock Law}) que tornava ilegal qualquer material
  considerado imoral e obsceno, o que incluía livros, panfletos e gravuras,
  além de textos científicos ou de divulgação de métodos contraceptivos
  e abortivos, quer fossem publicações ou meras informações trocadas em
  correspondência privada.} Como os Torquemadas dos tempos de
pré"-guerra,\footnote{No original \textit{ante"-bellum days}, o que
  literalmente significa \textit{dias do pré"-guerra}, não obstante, no
  contexto americano, a expressão \textit{ante"-bellum} costuma ser
  utilizada para indicar o período imediatamente anterior à Guerra Civil
  Americana.} Anthony Comstock é o autocrata da moral americana; ele
dita os padrões do bem e do mal, da pureza e do vício. Como um ladrão à
noite, ele se infiltra na vida privada das pessoas, nas suas relações
mais íntimas. O sistema de espionagem montado por esse homem, Comstock,
é de dar vergonha à infame Terceira Divisão da polícia secreta russa.
Por que as pessoas toleram tal ultraje à sua liberdade? Simplesmente
porque Comstock é a mais alta expressão do puritanismo gerado pelo
sangue anglo"-saxônico, de cuja escravização nem mesmo os mais liberais
conseguem se emancipar completamente. Com Anthony Comstock como santo
padroeiro, os quadros sem visão nem liderança de organizações antigas 
como \textit{Young Men's and Women's Christian Temperance Unions, Purity
Leagues},\footnote{A \textit{Women's Christian Temperance Union} teve papel
  ativo na promulgação da Lei Seca que tornava
  crime a produção e o comércio de bebidas alcoólicas nos Estados
  Unidos.} \textit{American Sabbath Unions} e do Partido da Proibição tornam"-se
os coveiros da arte e da cultura americanas.
%{[}\textit{Prohibition}{]}

A Europa pode, ao menos, gabar"-se de possuir uma arte e literatura audaciosas
o suficiente para examinar a profundidade dos problemas sociais e sexuais do
nosso tempo, exercitando a crítica severa sobre todas as nossas
vergonhas. Como se se tratasse de um bisturi cirúrgico, toda carcaça
puritana é dissecada, e o caminho é assim aberto para a libertação do
homem do peso morto do passado. Mas com o puritanismo a inspecionar, a
todo momento, a vida americana, nem a verdade, nem a sinceridade são
possíveis. Nada que não seja tristeza e mediocridade para ditar a
conduta humana, cercear a expressão natural e abafar os nossos melhores
impulsos. Neste século \textsc{xx}, o puritanismo é tão inimigo da liberdade e da
beleza quanto o era no tempo em que atracou em Plymouth Rock. Ele
repudia, como algo vil e pecaminoso, nossos mais profundos sentimentos;
e é por ser absolutamente ignorante em relação às funções reais das
emoções humanas que o puritanismo termina por criar os mais indizíveis
vícios.

A história inteira do ascetismo prova que essa é a mais pura verdade. A
Igreja, assim como o puritanismo, combate a carne como algo terrível;
a carne tem de ser subjugada e escondida a qualquer custo. O resultado
dessa atitude viciosa, somente agora, foi identificado pelos pensadores
modernos e educadores. Eles perceberam que ``a nudez tem um valor

higiênico assim como um significado espiritual, o que está muito além da sua influência em apaziguar a curiosidade natural dos jovens ou prevenir
emoções mórbidas. Inspira adultos que já há muito tempo superaram sua
curiosidade juvenil. A visão da forma humana essencial e eterna, a que
está mais próxima a nós do que tudo no mundo, com o seu vigor, a sua
beleza e a sua graça, é um dos tônicos principais da vida.''\footnote{Havelock
  Ellis, \textit{Studies in the Psychology of Sex: Sex in Relation to
  Society}, vol. 6 (1910).} O problema é que o espírito do puritanismo
perverteu a mente humana a tal ponto que ela perdeu a capacidade de
apreciar a beleza da nudez, forçando"-nos a esconder a nossa forma
natural sob a justificativa da castidade. No entanto, a castidade
nada mais é do que uma imposição artificial sobre a natureza, a
expressão de uma vergonha falsa da forma humana. A ideia moderna de
castidade, especialmente no que se refere à mulher --- a sua maior vítima
---, é tão somente o exagero da sensualidade dos nossos impulsos
naturais. ``A castidade varia de acordo com a quantidade de roupa'' e,
consequentemente, os cristãos e os puritanos sempre se apressam em
cobrir os \textit{pagãos} com os seus farrapos, para que assim possam convertê"-los à pureza e castidade.

O puritanismo com a sua perversão do significado do corpo humano,
especialmente no que diz respeito à mulher, condenou"-a ao celibato, ou à
procriação indiscriminada de uma raça doente, ou ainda à prostituição. A
escala desse crime contra a humanidade se torna clara quando observamos
atentamente os resultados. Mulheres solteiras devem abster"-se
completamente do sexo, sob a pena de serem consideradas imorais ou
decaídas, o que tem como resultado a neurastenia, impotência, depressão
e grande variedade de doenças nervosas que envolvem perda de capacidade
para o trabalho, limitação da capacidade de ter prazer com a vida,
insônia e preocupação excessiva com desejos e fantasias sexuais. A
máxima arbitrária e perniciosa da total abstinência sexual provavelmente
também explica a desigualdade intelectual entre os sexos. Freud acredita
que a inferioridade intelectual de tantas mulheres está relacionada com
a inibição do pensamento imposta sobre elas com o fim da repressão
sexual. Enquanto, de um lado, o puritanismo suprime os desejos sexuais
naturais da mulher solteira, de outro, abençoa sua irmã casada com a
fertilidade desenfreada do casamento. De fato, não apenas a abençoa,
como força a mulher, obcecada por sexo por conta da repressão anterior,
a ter filhos, independentemente de a sua constituição física estar fraca
ou da viabilidade econômica para manter uma família numerosa. A
contracepção é estritamente proibida, inclusive por métodos que tenham
sido cientificamente comprovados seguros; até falar sobre o assunto é
considerado criminoso.

Graças a essa tirania puritana, a maioria das mulheres logo vê a sua
força física diminuir rapidamente. Doentes e cansadas, elas ficam
incapacitadas de cuidar dos filhos até mesmo no sentido mais elementar.
Isso e a pressão econômica obrigam muitas mulheres a se arriscarem no
maior dos perigos, para que não continuem a parir. O hábito de fazer
abortos atingiu uma proporção tão avassaladora na América que é quase
inacreditável. Pesquisas recentes mostram que existem 17 abortos por cem
gestações. Esse percentual terrível representa apenas os casos que
chegaram ao conhecimento dos médicos. Se pensarmos ainda no sigilo que
esse tipo de prática necessariamente envolve, e em sua implicada %Repensar
ineficiência e negligência profissional, veremos que o puritanismo exige continuamente
milhares de vítimas à sua estupidez e hipocrisia.

Embora a prostituição seja perseguida, aprisionada e acorrentada, é, não
obstante, o maior triunfo do puritanismo. Ela é dos filhos a mais
querida, apesar de toda santimônia hipócrita. A prostituição é a
fúria do nosso século, varre os países \textit{civilizados} como um furacão,
deixando um rastro de doença e desastre. O único remédio que o
puritanismo oferece para essa sua filha má concebida é a repressão cada
vez maior e a perseguição cada vez mais implacável. A atrocidade mais
recente é a \textit{Page Law},\footnote{Em tradução para o português, \textit{Lei de Page}.} que impõe ao estado de Nova York
a falha terrível, o crime da Europa de registrar e identificar as
infelizes vítimas do puritanismo.\footnote{Lei federal de 1875 com
  objetivo de impedir a imigração, especialmente de mulheres e homens orientais considerados suspeitos de trabalhar no ramo da prostituição.
  Os imigrantes orientais que não fossem
  considerados suspeitos recebiam um certificado que lhes permitia
  entrar nos Estados Unidos, os que fossem, eram deportados.} De modo
igualmente estúpido, o puritanismo visa controlar o flagelo terrível da
sua criação --- as doenças venéreas. O mais desanimador é que esse
espírito de mentalidade estreita e obtusa tenha envenenado até os nossos
chamados liberais, cegando"-os a ponto de fazê"-los aderir à cruzada
contra aquilo que nasceu da própria hipocrisia do puritanismo: a
prostituição e suas consequências. 

Nessa cegueira deliberada, o
puritanismo se recusa a ver que o verdadeiro método de prevenção é
aquele que deixa claro para todos que ``doenças venéreas não são uma\label{std}
coisa misteriosa ou terrível, não são um castigo pelos pecados da carne,
alguma espécie de mal do qual se deva ter vergonha, tal como são
rotuladas pela maledicência puritana; mas, sim, que são doenças comuns
que podem ser tratadas e curadas.'' Dados os seus métodos de
obscurecimento, disfarce e ocultação, o puritanismo criou as condições
favoráveis para o crescimento e disseminação dessas doenças. Sua
intolerância é, mais uma vez, claramente demonstrada, e do modo mais
absurdo, pela atitude disparatada em relação à grande descoberta do
professor Ehrlich, ao esconder hipocritamente um remédio importante para a
sífilis com alusões vagas a um remédio para \textit{certo veneno}.

A capacidade quase ilimitada do puritanismo para o mal se deve ao fato
de que ele está consolidado nos bastidores do Estado e da lei. Fingindo
proteger o povo contra a \textit{imoralidade}, impregnou a maquinaria do
governo e, além de usurpar o papel de tutor moral, atribui"-se a função de
censor legal dos nossos pontos de vista, sentimentos e até mesmo da
nossa conduta.

A arte, a literatura, o teatro, as nossas correspondências privadas, em
suma, tudo o que diz respeito aos nossos gostos mais íntimos estão à
mercê desse tirano implacável. Anthony~Comstock, ou qualquer outro
policial da moral igualmente ignorante, recebeu o poder de profanar o
gênio, de contaminar e mutilar a criação mais sublime da natureza --- o
corpo humano. Os livros que lidam com as questões mais importantes de
nossas vidas e que buscam lançar luz sobre os problemas mais
perigosamente obscurecidos são considerados crimes pela lei, e seus
autores indefesos são jogados na prisão ou conduzidos à destruição e à
morte.

Nem mesmo sob o domínio do czar a liberdade pessoal é atacada
diariamente na mesma medida que é nos Estados Unidos, o reduto dos
eunucos puritanos. Aqui, o único dia de lazer permitido às massas, o
domingo, foi desfigurado e tornado totalmente impossível. Todos os
estudiosos dos costumes primitivos e das primeiras civilizações
concordam que o sábado sempre foi um dia de festa, livre de trabalho e
obrigações, um dia de alegria e diversão em geral. Em todos os países
europeus essa tradição ainda traz algum alívio à monotonia e estupidez
de nossa era cristã. Todos os lugares, salas de concerto, teatros, 
museus e jardins ficam cheios de homens, mulheres e crianças,
especialmente, indivíduos da classe trabalhadora com suas famílias, que
cheias de entusiasmo pela vida podem se esquecer das regras e convenções
que restringem a sua existência cotidiana. É nesse dia que as massas
mostram como seria a vida em uma sociedade saudável, se do trabalho
fosse extraído o propósito de lucrar e destruir almas.

O puritanismo roubou das pessoas até mesmo esse único dia. Obviamente,
isso afeta apenas os trabalhadores: nossos milionários têm suas casas
luxuosas e clubes selecionados. Os pobres, no entanto, estão condenados
à monotonia e ao tédio do domingo americano. O convívio e a diversão da
vida ao ar livre na Europa são aqui substituídos pela melancolia da
igreja, pelo abafamento dos salões saturados de germes das cidades do
interior ou pela atmosfera brutal dos bares. Nos estados em que o
consumo de álcool é proibido, as pessoas
não têm nem mesmo isso, a menos que possam investir seus parcos ganhos
em bebidas adulteradas.\footnote{Antes de valer para todo o território
  nacional em 1920, a Lei Seca dos Estados Unidos já havia sido adotada por alguns estados.}
Quanto à proibição, todo mundo sabe que farsa efetivamente representa.
Como todas as outras realizações do puritanismo, também empurrou o
\textit{diabo} ainda mais para o fundo do sistema humano. Em nenhum outro
lugar se encontra tantos bêbados quanto nas cidades em que ela vigora. 
Mas desde que se possa usar fragrâncias doces para suavizar o
hálito sujo da hipocrisia, o puritanismo triunfará. Diz"-se que a
proibição do álcool foi promulgada por razões econômicas e de saúde, mas
como o próprio espírito da proibição é anormal, ele só pode criar uma
vida igualmente anormal.
%{[}\textit{Prohibition States}{]}
%{[}\textit{Prohibition}{]}

Todo estímulo que desperta a imaginação e eleva o espírito é tão
necessário à vida quanto o ar. Fortalece o corpo e aprimora nossa visão
da comunidade humana. Sem estímulos, seja qual for a forma, o trabalho
criativo é impossível, como é também impossível o espírito de bondade e
generosidade. O fato de alguns gênios notáveis terem buscado inspiração
no cálice com frequência excessiva não justifica que o puritanismo
tente agrilhoar todo o leque das emoções humanas. Um Byron, um Poe
comoveram a humanidade muito mais profundamente do que todos os
puritanos juntos sequer podem sonhar. Esses dois escritores deram à vida
significado e cor; já os puritanos estão transformando sangue vermelho
em água, beleza em feiura, diversidade em uniformidade e decadência. O
puritanismo, em qualquer uma das suas expressões, é um germe venenoso.
Na superfície, tudo pode parecer forte e vigoroso; no entanto, o veneno
vai exercendo a sua função continuamente, até que todo o tecido esteja
condenado. Com Hippolyte Taine, todo espírito verdadeiramente livre já
começou a perceber que ``o puritanismo é a morte da cultura, da
filosofia, do humor e do companheirismo; suas características são o
tédio, a monotonia e a melancolia.''

\chapter{Tráfico de mulheres\footnote{Uma primeira versão deste texto foi
  publicada na revista anarquista \textit{Mother Earth}, então editada
  pela própria Goldman, sob o título ``O tráfico de escravas
  brancas.'' Já a publicação da versão que o leitor tem em mãos nesta edição aconteceu no mesmo ano em que a
  primeira, na coletânea de ensaios de 1910 \textit{Anarquismo e outros
  ensaios}.}}\label{trafico}
\markboth{Tráfico de mulheres}{}

De repente, os nossos reformistas fizeram uma grande descoberta --- o
tráfico de escravas brancas. Os seus artigos estão cheios dessas
\textit{condições sem precedentes}, e os legisladores já estão planejando um
novo conjunto de leis para controlar esse horror.\footnote{Em 25 de
  junho de 1910 foi aprovada a Lei Mann, que na época ficou conhecida
  como Lei do Tráfico de Escravas Brancas, que proibia o deslocamento
  interestadual de mulheres que tivesse como fim a exploração sexual. Na
  prática, a lei foi utilizada não só para criminalizar a prostituição e
  o tráfico de pessoas, mas também quaisquer relações que fossem
  consideradas \textit{imorais}, como, por exemplo, relações sexuais entre
  homens negros e mulheres brancas ou relações sexuais consensuais fora
  do casamento.}

É significativo que toda vez que se pretende desviar a opinião pública
de um problema social, uma cruzada seja inaugurada contra a indecência,
jogos de azar, bares etc. E qual é o resultado de tais cruzadas? Os
jogos de azar continuam aumentando, os bares estão realizando negócios
às escondidas, a prostituição está no auge, e o sistema coordenado por
cafetões e cadetes\footnote{Termo cunhado pelos moralistas para se
  referir ao novo tipo de cafetão, em geral estrangeiro, que, segundo
  eles, seriam os principais responsáveis pela exploração sexual de
  jovens imigrantes que pouco conheceriam da cultura americana --- constatação à qual Goldman justamente se opõe no presente
  texto. Os cadetes foram descritos pelos moralistas como rapazes jovens
  que, em geral, teriam tido certo treinamento como vigias.} se
intensificou.

Como pode acontecer de uma instituição conhecida por quase toda criança
ter sido descoberta assim, de uma hora para outra? Como pode acontecer
de este mal, conhecido por todos os sociólogos, ter se tornado somente agora um
assunto tão importante?

Supor que a investigação recente sobre o tráfico de escravas brancas (que, a
propósito, trata"-se de uma investigação extremamente superficial)\footnote{Na
  primeira versão do presente ensaio, Goldman nomeia expressamente o principal agitador dessa investigação: George Kibbe Turner, que teria sido o responsável por cunhar a
  expressão \textit{escravidão branca} no referido texto publicado, em 1909, na
  revista reformista \textit{McClure's}. Neste, acusou os judeus
  (seguidos pelos italianos) de comporem majoritariamente a então nova
  classe de cafetões, os cadetes, como também de serem os principais
  responsáveis pelo esquema de tráfico de mulheres europeias para fins
  de exploração sexual. Tais acusações inflamaram tanto o
  antissemitismo quanto a rejeição aos emigrantes, presentes na
  população estadunidense de então.} descobriu algo de novo é, no
mínimo, uma tolice. A prostituição foi e é um mal generalizado, não
obstante a humanidade siga com os seus negócios, perfeitamente
indiferente aos sofrimentos e angústia das vítimas da prostituição. Tão
indiferente, inclusive, quanto a humanidade tem sido ao nosso sistema
industrial, à prostituição econômica.

Apenas quando o sofrimento humano se transformar num brinquedo de cores
gritantes irá o povo infantilizado se tornar interessado nele --- por um
tempo, ao menos. As pessoas são como bebês caprichosos que necessitam de
brinquedos novos todos os dias. O grito por \textit{justiça} contra o tráfico
de escravas brancas é um brinquedinho desses. Serve para entreter as
pessoas por um tempo e irá ajudar a criar alguns cargos políticos
generosos a mais --- parasitas que se movimentam pelo mundo na condição
de inspetores, investigadores, detetives etc.

O que realmente é a causa do comércio de mulheres? Não apenas de
mulheres brancas, mas amarelas e negras também. A exploração,
obviamente: o Moloch\footnote{De acordo com o Antigo Testamento, 
Moloch era uma das divindades dos amonitas, cuja característica mais marcante seria a exigência do sacrifício de crianças pelos seus próprios pais.} impiedoso do capitalismo que engorda com o trabalho
mal remunerado, levando milhares de mulheres e meninas à prostituição.
Com a senhora Warren,\footnote{Referência à personagem Kitty Warren do
  livro \textit{Mrs.\,Warren's Profession} {[}A profissão da senhora Warren{]} de Bernard Shaw (1902). Essa personagem é uma
  ex"-prostituta que conseguiu criar uma rede de bordéis, passando da
  miséria a uma vida de luxo e riqueza.} essas garotas passaram a se
questionar: ``Por que desperdiçar a vida trabalhando 18 horas por
dia, numa copa, para ganhar alguns \textit{xelins} por semana?''

Naturalmente, os nossos reformistas não dizem nada sobre esta causa.
Eles a conhecem bem o suficiente, mas não ganham nada em dizer uma coisa
como essa. É muito mais lucrativo bancar o fariseu, fingir uma
moralidade ofendida, do que descer ao fundo das coisas.

No entanto, há uma exceção digna de louvor entre os escritores jovens:
Reginald Wright Kauffman, cujo trabalho \textit{The house of bondage},\footnote{Em tradução para o português, \textit{A casa da servidão}.} é a primeira tentativa séria de tratar
desse mal social de um ponto de vista que não seja sentimental ou
filisteu. Jornalista de longa data, o sr.\,Kauffman comprova que o
sistema industrial não possibilita à maioria das mulheres outra
alternativa que não seja a da prostituição. As mulheres retratadas em
\textit{The house of bondage} pertencem à classe trabalhadora. Tivesse o
autor retratado a vida de mulheres de outras esferas, ele teria se
confrontado com o mesmo estado das coisas.

Em nenhum lugar a mulher é tratada de acordo com o mérito do seu
trabalho, mas, ao contrário, é tratada como um sexo. É, portanto,
praticamente inevitável que ela deva pagar por seu direito de existir,
de manter a sua posição em qualquer que seja o âmbito, com favores
sexuais. Desse modo, é somente uma diferença de grau se ela vende a si
mesma a um único homem, seja dentro ou fora do casamento, ou a muitos
homens. Quer os nossos reformistas admitam ou não, a inferioridade
econômica e social da mulher é responsável pela prostituição.

Neste exato momento, nossas boas pessoas estão chocadas com a divulgação
de que na cidade de Nova York uma a cada dez mulheres trabalha na
indústria, que o salário médio recebido por uma mulher é de \textsc{us}\$ 6 por
semana, ou seja, de \textsc{us}\$ 6 por 48 a 60 horas de trabalho semanais, e que a
maioria das mulheres trabalhadoras enfrenta muitos meses de falta de
trabalho, o que deixa o salário médio anual em torno de \textsc{us}\$ 280. Ao
considerar esses horrores econômicos, é realmente de se admirar que a
prostituição e o comércio de escravas brancas tenham se tornado fatores
tão dominantes?

Para que os números citados não sejam acusados de exagero, é bom
examinarmos o que algumas autoridades no assunto da prostituição têm a
dizer:

\begin{quote}
Uma causa prolífica da depravação feminina pode ser encontrada nas
muitas tabelas que descrevem o emprego que as mulheres procuravam e o
salário que recebiam no momento antes da queda. Será uma questão, para
os economistas políticos, decidir se considerações que levem em conta
exclusivamente os negócios podem ser uma desculpa da parte dos
empregadores para a redução de salários; e se a economia de uma pequena
porcentagem dos seus rendimentos não será contrabalanceada pela enorme
quantidade de impostos aplicados sobre o povo para cobrir as despesas
decorrentes de um sistema de vícios que é, em muitos casos, a
consequência direta de uma remuneração desproporcional ao trabalho
honesto.\footnote{William W.\,Sanger, \textit{The history of
  prostitution: its extent, causes, and effects throughout the world}
  (1859).}
  \end{quote}

Nossos reformistas atuais fariam bem em dar uma olhada no livro do dr.\,Sanger. Lá, eles irão encontrar que dos 2000 casos que observou,
pouquíssimos eram oriundos da classe média, de condições bem
estruturadas e casas confortáveis. Incomparavelmente, a maioria era
formada por mulheres e garotas da classe trabalhadora; algumas
conduzidas à prostituição por pura necessidade, outras por uma vida
familiar cruel e miserável e outras ainda devido a uma constituição
física comprometida, deficiente (sobre a qual falarei adiante). Também
será de grande valia aos mantenedores da pureza e da moralidade saber
que dos dois mil casos, 490 eram mulheres casadas, mulheres que viviam com
os seus maridos. O que evidencia que não havia toda essa garantia à
\textit{segurança e pureza da mulher}, quando no interior da santidade do
casamento.\footnote{É significativo que o livro do dr.\,Sanger tenha sido
  excluído do sistema de distribuição dos \textsc{eua}. Evidentemente, as
  autoridades não estão ansiosas para que o público seja informado da
  verdadeira causa da prostituição. [\textsc{n.\,a.}]}

O dr.\,Alfred Blaschko, na obra \textit{Prostitution in the Nineteenth Century},\footnote{Em tradução para o português,  \textit{A prostituição no século \textsc{xix}}.} é ainda mais
enfático na compreensão da condição econômica como um dos fatores mais
centrais da prostituição.

\begin{quote}
Embora a prostituição tenha existido em todas as eras, coube ao século
\textsc{xix} transformá"-la numa instituição social gigantesca. O desenvolvimento
da indústria com massas imensas de pessoas no mercado competitivo, o
crescimento e congestionamento das cidades grandes, a insegurança e a
incerteza do emprego deram à prostituição um ímpeto nunca sonhado em
nenhum período da história humana.
\end{quote}

E mesmo Havelock Ellis, embora não trate a causa econômica de modo tão
absoluto, é obrigado a admitir que ela é direta ou indiretamente a causa
principal. Ele constata que uma grande porcentagem de prostitutas é
recrutada da classe das empregadas domésticas, embora este ofício
exija menos cuidados e ofereça uma maior segurança. Por outro lado, o
senhor Ellis não nega que a rotina diária, a estafa e a monotonia que
cabem a uma empregada doméstica, e especialmente o fato de que ela nunca
pode participar da companhia e alegria de um lar, sejam fatores
importantes o suficiente para forçá"-la a buscar alguma diversão e
esquecimento na alegria e brilho da prostituição. Em outras palavras, a
empregada doméstica, ao ser tratada como burro de carga, nunca exercendo
os seus direitos e desgastada pelos caprichos da patroa, pode vislumbrar
a prostituição, caso também da trabalhadora da indústria e da balconista, como a sua única escapatória.

O lado mais engraçado dessa questão posta agora diante do público é a
indignação das nossas \textit{pessoas boas e respeitáveis}, especialmente, a
indignação dos vários cavalheiros cristãos, que sempre são encontrados na
linha de frente de cada uma dessas cruzadas. Será que isso ocorre porque são absolutamente ignorantes em história da religião e especialmente
na história da religião cristã? Ou isto se dá porque eles esperam cegar
a geração atual para o papel desempenhado no passado pela Igreja em
relação à prostituição? Qualquer que seja a razão, eles deveriam ser os
últimos a bradar em nome das vítimas desafortunadas de hoje, dado ser
conhecido a qualquer estudante inteligente que a prostituição tem origem
religiosa, que foi mantida e promovida por muitos séculos não como uma
vergonha, mas como uma virtude, aclamada pelos próprios deuses.

\begin{quote}
Ao que parece a origem da prostituição é encontrada
primeiramente dentre os costumes religiosos --- na religião, a grande
conservadora da tradição social, com o intuito de preservar, numa forma transformada, a
liberdade primitiva que estava se esvaindo da vida social em geral. O
exemplo típico é registrado por Heródoto, no século \textsc{v} antes de Cristo,
quando se refere ao Templo de Mylitta, a Vênus babilônica, para o qual toda mulher,
uma vez na vida, teria de ir com o objetivo de entregar"-se ao primeiro estranho que jogasse uma
moeda no seu colo em adoração à deusa. Costumes muito similares
existiram em outras partes da Ásia ocidental, no norte da África, em
Chipre e outras ilhas do Mediterrâneo, como também na Grécia, onde no
templo de Afrodite, localizado em Corinto, havia mais de mil
hierodulos\footnote{Nome dado aos escravos que estavam a serviços dos
  deuses. Quando pertencentes ao sexo feminino, costumavam
  ter por função a prostituição.}
dedicados ao serviço da deusa.

A teoria de que a prostituição religiosa se desenvolveu, via de regra,
a partir da crença de que o ato de gerar seres humanos possui uma
influência misteriosa e sagrada na promoção da fertilidade da natureza
é compartilhada por todas as autoridades do assunto. Não obstante, à
medida em que a prostituição religiosa foi se tornando uma instituição
organizada sob a influência sacerdotal, desenvolveu características
utilitárias, ajudando a aumentar a receita pública.

A ascensão do cristianismo ao poder quase não
alterou essa política. Os mais proeminentes pais da Igreja toleravam a
prostituição. Bordéis sob proteção municipal são encontrados no século
\textsc{xiii}. Eles constituíam uma espécie de serviço público, de modo que os
seus diretores eram praticamente considerados servidores
públicos.\footnote{Havelock Ellis,
  \textit{Studies in the Psychology of Sex: Sex in Relation to Society},
  vol. 6.}
  \end{quote}

A isto deve ser adicionado o seguinte trecho oriundo do trabalho do
doutor Sanger:

\begin{quote}
O papa Clemente \textsc{ii} emitiu uma bula papal na qual estava decretado que
as prostitutas seriam toleradas desde que pagassem certa quantia dos
seus ganhos à Igreja.

O papa Sisto \textsc{iv} foi mais prático; de um único bordel, que ele mesmo
construiu, recebeu vinte mil ducados.
\end{quote}

Nos tempos modernos, a Igreja é um pouco mais cuidadosa nesse sentido.
Ao menos, ela não exige abertamente os tributos das prostitutas. Considera
muito mais lucrativo investir no mercado imobiliário, como, por
exemplo, a Igreja da Trindade, que aluga verdadeiras armadilhas mortais
a preços exorbitantes para os que vivem da prostituição.\footnote{A
  Igreja da Trindade era proprietária dos imóveis de um famoso centro de
  prostituição, em Nova York, que, por conta dessa relação, ficou
  conhecido como \textit{Holy Ground} ou \textit{Solo Sagrado}.}

Por mais que deseje, o tempo não me permite discorrer sobre a
prostituição no Egito, Grécia, Roma e ao longo da Idade Média. As
condições desse último período são especialmente interessantes, uma vez
que a prostituição foi organizada em guildas, que eram presididas pela Rainha do
Bordel. Essas guildas usavam a greve como meio de melhorar suas
condições e manter o preço padronizado. Certamente, um método mais
eficaz do que o utilizado pelo escravo assalariado da nossa sociedade
moderna.\footnote{Na primeira versão, segue"-se a este parágrafo uma
  passagem em que Goldman compara os moralistas anglo"-saxônicos aos
  fariseus, o que inclui uma alusão a Mateus 23. Vide a passagem
  suprimida: ``Nunca, porém, foi atribuída à prostituição a posição
  atual de depravação e criminalidade, porque em épocas passadas a
  prostituição não foi perseguida e reprimida como é hoje em dia,
  especialmente nos países anglo"-saxônicos, onde o farisaísmo está no
  auge, onde cada um encontra"-se ocupado escondendo os esqueletos em sua
  própria casa enquanto aponta para a chaga do companheiro.''}

Seria tanto parcial, quanto extremamente superficial, considerar que o\label{parcial}
fator econômico é a única causa da prostituição. Há outras causas que
são não menos importantes e vitais. Os nossos reformistas também as
conhecem, mas o receio de discuti"-las é ainda maior do que o de discutir
a instituição que suga o âmago da vida dos homens e mulheres. Refiro"-me
agora às questões sexuais cuja simples menção é suficiente para causar,
na maioria das pessoas, espasmos morais.

É um fato amplamente aceito que a mulher é criada como uma mercadoria
sexual e, não obstante, ainda assim, ela é mantida em absoluta
ignorância acerca do significado e da importância do sexo. Tudo o que
seja relacionado a esse assunto é suprimido, e as pessoas que tentam
trazer luz para essa escuridão terrível são perseguidas e jogadas nas
prisões. Assim, torna"-se natural que uma garota não saiba cuidar de si
mesma, que não saiba nada da função mais importante a ser exercida na
sua vida, de modo que não pode haver surpresa que ela se torne uma presa
fácil da prostituição ou de qualquer outra forma de relacionamento que a
degrade à posição de objeto para uma gratificação meramente sexual.\label{objeto}

É devido a essa ignorância que a vida e a natureza de uma mulher são
frustradas e aleijadas. Já há muito tempo aceitamos como uma obviedade
que um rapaz deva seguir o chamado da natureza, o que é o mesmo que
dizer que o garoto pode, tão logo a sua natureza sexual se afirme,
satisfazer essa natureza, não obstante nossos moralistas fiquem
escandalizados com a ideia de que a natureza de uma garota também deva
ser afirmada. Para os moralistas, a prostituição não consiste tanto no\label{vender}
fato de que a mulher venda o seu corpo, mas, ao invés disso, que ela
venda o seu corpo fora do casamento. Que isto não se trata de
palavras jogadas ao vento é comprovado pelo fato de que o casamento por razões
monetárias é considerado como perfeitamente legítimo, santificado pela
lei e pela opinião pública, enquanto qualquer outra união é condenada e
repudiada. Não obstante, se for definida adequadamente, prostituta
não significa nada além do que ``qualquer pessoa para quem os
relacionamentos sexuais estão subordinados ao lucro.''\footnote{Yves
  Guyot, \textit{La Prostitution} (1882).}

\begin{quote}
Essas mulheres são prostitutas, porque vendem seus corpos para o
exercício do ato sexual e fazem disso uma profissão.\footnote{Willem
  Bonger, \textit{Criminalité et conditions économiques} (1905).}
  \end{quote}

De fato, Bonger\footnote{No original, Goldman erra a grafia do nome do
  autor, colocando Banger ao invés de Bonger.} vai além; ele sustenta
que a ação de prostituir"-se é ``intrinsecamente igual à de um homem ou
mulher que se casa por razões econômicas.''

É claro que o casamento é o objetivo de toda garota, mas como milhares
não conseguem se casar, nossos estúpidos costumes sociais as condenam ou a
uma vida de celibato ou à prostituição. A natureza humana afirma a si
mesma independentemente de todas as leis e não há qualquer razão
plausível para que a natureza se adapte a uma concepção perversa de
moralidade.

A sociedade considera as experiências sexuais de um homem como atributos
do seu desenvolvimento geral, enquanto experiências similares por parte
da mulher são vistas como uma calamidade terrível, como perda da honra e
de tudo o mais que seja bom e nobre num ser humano.\footnote{Na primeira
  versão, Goldman inicia esse parágrafo com uma consideração muito
  intrigante e potencialmente polêmica que, aqui, optou por suprimir:
  ``Por ter sido encarada como simples mercadoria sexual, a honra, a
  decência, a moralidade e a utilidade da mulher tornaram"-se apenas uma
  parte de sua vida sexual.''} Esse padrão duplo de moralidade tem
desempenhado um papel que não é pequeno na criação e perpetuação da
prostituição. Envolve manter as jovens numa ignorância absoluta em relação ao sexo, o que, assomado à  \textit{inocência} que lhes é atribuída e à sua natureza sexualmente abundante, ao mesmo tempo em que reprimida, traz à tona justamente o estado das coisas que os nossos puritanos buscam, tão ansiosamente, evitar e prevenir.\footnote{Conforme sugere na versão anterior:
  ``Esse estado das coisas encontra seu retrato magistral na obra de
  Zola, \textit{Fecundidade}.''}

Não que a satisfação sexual leve necessariamente à prostituição; é a
perseguição cruel, impiedosa e criminosa àqueles que ousam se desviar da
trilha batida a responsável por ela.

Meninas, meras crianças, trabalhando em salas lotadas e superaquecidas
de dez a 12 horas por dia, em máquinas, tendem, por conta das
circunstâncias mesmas, a ficar num estado sexual de constante excitação.
Muitas dessas garotas não têm um lar ou conforto de qualquer tipo, de
modo que as ruas ou algum lugar de diversão barata são os únicos meios
de esquecer a rotina diária --- o que, naturalmente, as coloca em
proximidade com o outro sexo. É difícil dizer qual dos dois fatores
conduz o estado de hiperexcitação sexual dessas garotas ao clímax, mas é
por certo natural que este clímax deva se seguir. Este é o primeiro
passo em direção à prostituição. A garota não deve ser responsabilizada
por isso. Pelo contrário, é completamente culpa da sociedade, culpa da
nossa falta de entendimento, da nossa falta de apreço pela vida no que
diz respeito à sua geração; especialmente, trata"-se aí do erro
criminoso dos nossos moralistas, que condenam essas jovens mulheres por
toda eternidade, pelo motivo de elas terem saído do \textit{caminho da
virtude}; isto é, porque a sua primeira experiência sexual ocorreu sem
a sanção da Igreja.

A jovem garota se sente completamente marginalizada, com as portas de
casa e da sociedade fechadas na sua cara. Toda a sua formação e tradição
são tais que ela se sente depravada e decaída e,
portanto, sem nada em que possa se apoiar, sem nada que seja capaz de
elevá"-la, ao invés de jogá"-la ainda mais para baixo. A sociedade cria as
vítimas que, depois, tenta em vão se livrar. Por mais terrível,
depravado e decrépito que seja um homem, ele ainda assim se considera
bom demais para tomar como esposa a mulher cuja graça ele estaria
disposto a comprar, mesmo que, com esse casamento, pudesse salvá"-la de
uma vida de horror. Tampouco a prostituta pode recorrer à ajuda da sua
irmã. Em sua estupidez, essa última se considera pura e casta, sem
perceber que a sua posição é, em muitos aspectos, ainda mais
deplorável do que a da sua irmã da rua.

``A esposa que se casou por dinheiro, comparada à prostituta'',
diz Havelock Ellis, ``é a verdadeira fura"-greve. Ela recebe menos, dá muito
mais em troca em termos de trabalho e cuidados e está absolutamente
atada ao seu mestre. A prostituta nunca abre mão do direito sobre sua
própria pessoa, ela mantém sua liberdade e direitos pessoais, e nem é a
todo momento obrigada a se submeter ao cerco de um homem.''

A mulher que se julga melhor do que as outras também não é capaz de compreender a
declaração apologista de Lecky quando afirma que ``embora ela {[}a
prostituição{]} possa ser o tipo supremo de vício, também é a mais
eficiente guardiã da virtude. Se não fosse por ela, lares felizes seriam
poluídos, práticas antinaturais e prejudiciais abundariam.''\footnote{Neste
  trecho, Goldman está se referindo às palavras de William Edward
  Hartpole Lecky citadas no livro de Ellis.}

Os moralistas estão sempre prontos para sacrificar metade da raça humana
em nome de alguma instituição miserável que eles não conseguem superar.
É verdade que a prostituição não é a salvaguarda da pureza do lar, do
mesmo modo que leis rígidas não são salvaguardas contra a prostituição.
Metade dos homens casados são clientes de bordéis. É
através desse elemento virtuoso que mulheres casadas --- e mesmo crianças
--- são infectadas por doenças venéreas. No entanto, a sociedade não tem
uma palavra de condenação aos homens, ao mesmo tempo em que nenhuma lei
é considerada monstruosa o suficiente ao ponto de não ser posta em ação
contra a vítima desamparada. A prostituta é assediada não apenas por quem a utiliza, está absolutamente à mercê de todos: nas ruas, dos policiais e miseráveis detetives; nas delegacias, dos oficiais; em todas as prisões, encontra"-se à mercê das autoridades.

Num livro recente de uma mulher que, por 12 anos, foi dona de uma
\textit{casa}, são encontrados os seguintes números: ``As autoridades me
obrigavam a pagar todo mês multas que iam de \textsc{us}\$ 14,70 a 29,70, já as
meninas tinham de pagar de \textsc{us}\$ 5,70 a 9,70 para a polícia.''
Considerando que os negócios dessa escritora estavam localizados numa
cidade pequena, e que os valores que ela fornece não incluem subornos e
multas extras, pode"-se ver prontamente a magnitude da receita que o
departamento de polícia obtém com o dinheiro do sangue das suas vítimas,
as quais não irá sequer proteger. Ai daquelas que se recusem a pagar seu
pedágio; elas seriam ajuntadas tal qual gado, ``fosse para causar uma
boa impressão nos bons cidadãos da cidade, fosse porque os poderosos
estivessem necessitando de um dinheiro extra. Para a mente deformada que
acredita que uma mulher decaída é incapaz de emoção humana, é impossível
perceber a tristeza, a desgraça, as lágrimas, o orgulho ferido que nos
tomava todas as vezes que éramos apreendidas.''

Não é estranho que uma mulher que manteve uma \textit{casa} possa se sentir
assim? Mais estranho, porém, é que o mundo cristão, apesar de tão bom, deva fazer
sangrar e extorquir essas mulheres, dando"-lhes nada em troca que não
seja condenação e perseguição. Oh, que caridade a do mundo cristão!

Muita ênfase tem sido colocada nessa questão das escravas brancas que
estariam sendo importadas para a América. Como a América pode manter a
virtude se não tiver a Europa para ajudá"-la? Não vou negar que isso
ocorra em alguns casos, como não negarei que há emissários da Alemanha e
de outros países que estão atraindo escravos econômicos para a América;
não obstante, nego absolutamente que a prostituição esteja sendo
recrutada da Europa numa proporção que seja digna de consideração. Pode
ser verdade que a maioria das prostitutas de Nova York são estrangeiras,
mas isto se dá apenas porque a maioria da população é estrangeira. No
momento em que vamos para qualquer outra cidade americana, como Chicago
ou para a região centro"-oeste, vemos que o número de prostitutas
estrangeiras é de longe a minoria.

Igualmente exagerada é a crença de que a maioria das mulheres da rua
nessa cidade estavam envolvidas neste ramo antes de virem para a
América. A maioria das jovens fala um inglês excelente, elas são
americanizadas em hábitos e aparência --- algo absolutamente impossível,
a não ser que você viva neste país já há muitos anos. Ou seja: elas
foram conduzidas à prostituição pelas condições americanas, pelo
costume estritamente americano de exibir"-se, do modo mais excessivo
possível, em roupas elegantes, para as quais naturalmente se necessita
de dinheiro --- dinheiro que não se pode ganhar com o trabalho nas
fábricas ou nas lojas.\footnote{Na primeira versão, Goldman continua
  esse parágrafo com a comparação, que lhe é usual, entre a prostituta e
  a mulher casada, cunhando para isso o interessante termo 
  \textit{roupafobia} --- em inglês, \textit{clothesophobia} ---, que, pelo contexto, seria mais
  apropriado se fosse designado \textit{roupamania}. Eis a
  passagem em questão: ``A serenidade dos moralistas não é perturbada
  pelas mulheres respeitáveis que satisfazem a sua \textit{roupamania}
  {[}\textit{clothesophobia}{]} ao casar"-se por dinheiro; por que, então, eles
  ficam tão indignados se a pobre garota se vende pela mesma razão? A
  única diferença diz respeito à quantidade recebida, e, é claro, ao
  rótulo que a sociedade lhe dá ou retira.''}

Em outras palavras, não há razão para acreditar que algum grupo de
homens assumiria o risco e as despesas para obter produtos estrangeiros,
quando as condições americanas estão inundando o mercado com milhares de
jovens garotas. Por outro lado, há evidência suficiente para provar
que a exportação de meninas americanas para fins de prostituição não é,
de modo algum, um fator irrelevante.

Clifford G.\,Roe, ex"-procurador assistente do condado de Cook, em
Illinois, fez a denúncia de que jovens da Nova Inglaterra estão sendo
embarcadas para o Panamá com o objetivo expresso de serem
utilizadas por homens empregados pelo Tio Sam. Roe acrescenta que
``parece haver uma ferrovia subterrânea entre Boston e Washington, na
qual muitas garotas viajam.'' Não é significativo que a ferrovia conduza
à sede da autoridade federal? Que o sr.\,Roe tenha falado mais do que o
desejado por certos setores é comprovado pelo fato de que ele perdeu o seu
cargo. Não é aconselhável que homens em seus ofícios contem histórias da
carochinha.

A desculpa dada para o que vem ocorrendo no Panamá foi a de que não há
bordéis na Zona do Canal. Esse é o tipo de desculpa usual em um mundo
hipócrita que não ousa encarar a verdade. Não há prostitutas na Zona do
Canal, nem nos limites da cidade --- logo, a prostituição não existe.

Além do sr.\,Roe, há também James Bronson Reynolds, que fez um estudo
minucioso do tráfico de escravas brancas para a Ásia. Como cidadão
americano fiel e amigo do futuro Napoleão da América, Theodore
Roosevelt, ele seria certamente o último a difamar a decência moral do
seu país. No entanto, somos informados por ele que em Hong Kong, Xangai e
Yokohama estão alocados os estábulos de Aúgias\footnote{Na
  mitologia grega, os estábulos do rei Aúgias eram habitados por um
  grande número de touros, dados de presente pelo deus Hélios, cujos
  estercos jamais haviam sido limpos. Um dos trabalhos de Hércules foi
  limpar os estábulos de Aúgias em um único dia.} do vício americano. Lá,
as prostitutas americanas se fizeram tão visíveis que no Oriente
\textit{garota americana} é sinônimo de prostituta. O sr.\,Reynolds lembra aos
seus compatriotas que, embora os americanos na China estejam sob a
proteção de nossos representantes consulares, os chineses na América não
têm nenhuma proteção. Qualquer um que esteja informado da perseguição
brutal e bárbara que os chineses e japoneses enfrentam na costa do
Pacífico, irão concordar com o sr.\,Reynolds.

Tendo em vista os fatos expostos acima, é excessivamente absurdo apontar
para a Europa como o pântano de onde vêm as doenças sociais da América.
Tão absurdo quanto é corroborar com o mito de que os judeus são os
maiores responsáveis pelo contingente das presas voluntárias. Estou
certa de que ninguém me acusará de tendências nacionalistas. Fico feliz
em dizer que o meu desenvolvimento está além delas, como além de muitos
outros preconceitos. Que, portanto, eu me ressinta da afirmação de que
as prostitutas judias são importadas com este fim, não é por conta de
nenhuma simpatia judaica, mas sim pelo modo de vida característico a
essas pessoas. Ninguém, exceto os mais superficiais, poderá declarar que
jovens judias imigram para terras estranhas sem que tenham algum laço ou
relações que as levem até lá. As jovens judias não são aventureiras. Até
poucos anos atrás, elas não podiam sequer sair de casa, nem mesmo até a
vila ou cidade vizinha, se não fosse para visitar algum parente. Como,
então, pode parecer crível que jovens judias deixem seus pais e
famílias, viajando milhares de quilômetros, por conta de influências e
promessas de algum estranho poder? Dirija"-se a qualquer um desses
grandes navios e veja por si mesmo se essas garotas não vêm acompanhadas
de seus pais, irmãos, tias ou de outros parentes. Pode haver exceções,
obviamente, mas afirmar que um grande número de jovens judias é
importado para o fim da prostituição, ou de qualquer outro propósito, é
simplesmente não conhecer a psicologia judaica.

Aqueles que se sentam numa casa com telhado de vidro fazem mal em atirar
pedras sobre os judeus; até porque o telhado de vidro americano é
bastante fino, quebra com facilidade e o interior da casa é tudo menos
agradável.

O agenciador é, sem dúvida, um pobre espécime da família humana, mas
quais são mesmos os motivos que fazem dele mais desprezível do que o
policial que tira o último centavo da mulher de rua, para depois
trancá"-la na delegacia? Por que o cadete é mais criminoso, ou uma ameaça
maior à sociedade, do que os donos de lojas de departamento e fábricas,
que engordam com o suor de suas vítimas, apenas para conduzi"-las às
ruas? Não estou fazendo nenhum apelo em nome dos cadetes, mas não
consigo ver por que eles devem ser perseguidos sem piedade, enquanto os
reais responsáveis por toda iniquidade social gozam de imunidade e
respeito. Também é bom lembrar que não é o cadete quem cria a
prostituta. É a nossa farsa e hipocrisia que criam a
prostituta e o cadete.

Até 1894, pouco se sabia na América sobre esse agenciador. Só que aí
fomos atacados por uma epidemia de virtude. O vício deveria ser abolido,
o país purificado a todo custo. O câncer social foi, então, desviado da
vista e aprofundado no corpo. Os proprietários dos bordéis, assim como
as suas desafortunadas vítimas, foram entregues à misericórdia da
polícia. As consequências inevitáveis, os subornos exorbitantes e a
penitenciária se seguiram. Quando estavam nos bordéis, essas jovens
desafortunadas encontravam"-se relativamente protegidas, lá tinham certo
valor monetário, mas agora, nas ruas, estão absolutamente à mercê da
corrupção e ganância policiais. Desesperadas, necessitando de proteção e
ansiando por afeição, essas garotas tornam"-se naturalmente presas fáceis
para os cadetes, que nada mais são do que um dos resultados do espírito de nossa
era comercial. Assim, o sistema de cadetes é a consequência direta da
perseguição policial, da corrupção e da tentativa de repressão da
prostituição. É pura tolice confundir uma fase tardia desse mal social
com a sua causa.

Mera repressão e decretos bárbaros servem apenas para afligir e
degradar ainda mais as desafortunadas vítimas da ignorância e da
estupidez. Estupidez que alcançou a sua expressão mais elevada com a
proposta de lei que visava tornar o tratamento humanitário às prostitutas um
crime, cuja pena a qualquer um que abrigasse uma prostituta seria a de cinco anos de
prisão e uma multa de dez mil dólares. Esse tipo de atitude revela a tenebrosa
falta de entendimento das verdadeiras causas da prostituição,
o fator social, a mesma que se manifestou no espírito puritano
dos dias de \textit{A letra escarlate}.

Não existe um único especialista moderno no tema que não concorde
que métodos legislativos são absolutamente inúteis para dar conta do
assunto. O dr.\,Alfred Blaschko compreende que a repressão governamental
e as cruzadas morais não têm resultado algum que não o de desviar o mal
para vias secretas, multiplicando os perigos para a comunidade. Havelock
Ellis, o estudioso mais completo e humano da temática da prostituição, prova a
partir de grande riqueza de dados que, quanto mais rigorosos os métodos
de perseguição, piores as condições se tornam. Em meio a outros dados,
somos informados que na França, ``em 1560, Charles \textsc{ix} aboliu os bordéis
por meio de um decreto, mas o número de prostitutas só aumentou, ao
mesmo tempo em que muitos novos bordéis apareceram de formas inesperadas
e muito mais perigosas. Apesar de toda essa legislação, ou por causa
dela, não houve nenhum país em que a prostituição tenha desempenhado um
papel mais ostensivo.''\footnote{Havelock Ellis, \textit{Studies in the
  Psychology of Sex: Sex in Relation to Society}, vol. 6}.

Uma opinião pública educada, livre da perseguição moral e legal às
prostitutas, é a única coisa capaz de aliviar as condições atuais.
Fechar deliberadamente os olhos e ignorar esse mal, que é um fator
social da vida moderna, pode tão somente agravar o problema. Devemos nos
colocar acima de concepções bobas como a de ser superior ao outro e
aprender a ver a prostituta como um produto de condições sociais e não
morais. Tal percepção varrerá para longe a hipocrisia e garantirá uma
melhor compreensão e um tratamento mais humano. Quanto à erradicação da
prostituição, nada é capaz de garantir isso, exceto uma completa
transvaloração de todos os valores aceitos, especialmente os morais --- transvaloração que deve ser acompanhada da abolição da escravidão
industrial.

\chapter{O sufrágio feminino\footnote{Nono capítulo da coletânea \textit{Anarquismo
  e outros ensaios}, de 1910.}}
\markboth{O sufrágio feminino}{}

Vangloriamo"-nos da nossa era de avanço, ciência e progresso. Não é
estranho, então, que ainda acreditemos na adoração fetichista? É verdade
que nossos fetiches têm agora diferentes formas e substâncias, mas no
que diz respeito ao seu poder sobre a mente humana, eles são tão
desastrosos quanto eram nos tempos arcaicos.

Nosso fetiche moderno é o sufrágio universal. Aqueles que ainda não
alcançaram esse fim travam revoluções sangrentas para conquistá"-lo, e
aqueles que desfrutam do seu reinado trazem pesados ​​sacrifícios ao
altar dessa deidade onipotente. \textit{Ai} dos hereges que se atrevam a
questionar essa divindade!

A mulher, ainda mais do que o homem, é uma adoradora de fetiches e,
embora os seus ídolos possam mudar, ela está sempre de joelhos, sempre
levantando as mãos, sempre cega para o fato de que seu deus tem pés de\label{barro}
barro. A mulher tem sido o suporte de todas as divindades desde os tempos
imemoriais. E, assim, ela tem pagado o preço que somente os deuses podem
cobrar --- sua liberdade, o sangue de seu coração, sua própria vida.

A máxima memorável de Nietzsche, ``Quando você encontrar uma mulher,\label{herege}
leve o chicote'', é considerada muito brutal e, no entanto, Nietzsche
expressou numa frase a atitude da mulher para com os seus deuses.

A religião, especialmente a religião cristã, condena a mulher à vida de
um ser inferior, de um escravo. Isso frustra sua natureza e
acorrenta sua alma, mas a religião cristã não tem maior partidário,
ninguém que seja mais devoto do que a mulher. De fato, é seguro dizer
que, há muito, a religião já teria deixado de ser um fator relevante na vida das
pessoas, não fosse pelo apoio que recebe da mulher. Os mais fervorosos
obreiros de igreja, os mais incansáveis missionários ​​do mundo inteiro,
são mulheres --- sempre a oferecer sacrifícios no altar dos deuses que
acorrentam o seu espírito e escravizam o seu corpo.

O monstro insaciável, a guerra, rouba da mulher tudo o que lhe é caro e
precioso. Extorque"-lhe seus irmãos, amantes e filhos, para dar"-lhe em troca
uma vida de solidão e desespero. No entanto, a maior defensora e
adoradora da guerra é a mulher. É ela quem incute nos seus filhos o amor pela conquista e pelo poder; é ela quem sussurra nos ouvidos das crianças sobre as glórias da guerra; é ela quem embala o sono do bebê ao som das
trombetas e do barulho das armas. Também é a mulher quem coroa o vencedor no
seu retorno dos campos de batalha. Sim, é a mulher quem paga o mais alto
preço a esse monstro insaciável, a guerra.

Em seguida, há o lar. Que fetiche terrível é esse! Como suga a energia
vital da mulher --- essa prisão moderna de barras de ouro. Seu aspecto
brilhante cega a mulher para o preço que ela terá de pagar como esposa,
mãe e empregada doméstica. Ainda assim, ela se agarra tenazmente ao lar,
ao poder que a mantém em cativeiro.

Pode"-se dizer que é porque a mulher reconhece o preço terrível que ela é
obrigada a pagar à Igreja, ao Estado e ao lar, que ela quer que o
sufrágio a liberte. Esta pode ser a verdade de poucas; a maioria das
sufragistas repudiam completamente tal blasfêmia. Ao contrário, elas
insistem que o sufrágio feminino irá torná"-las cristãs e domésticas
ainda melhores, além de cidadãs leais ao Estado. Consequentemente, o
sufrágio é apenas um meio de fortalecer a onipotência dos mesmos deuses
que a mulher serve desde os tempos imemoriais.

Que maravilha, então, que a mulher seja tão devota, tão zelosa, tão
prostrada ante esse novo ídolo, o sufrágio feminino. Como nos tempos
idos, ela suporta a perseguição, a prisão, a tortura e todas as formas
de condenação com um sorriso em seus lábios. Como nos tempos idos, até a
mais esclarecida espera um milagre dessa deidade do século \textsc{xx} --- o
sufrágio. Vida, felicidade, alegria, liberdade, independência: tudo
isso, e muito mais, irá brotar com o sufrágio. Em sua devoção cega, a
mulher não vê o que pessoas dotadas de intelecto perceberam cinquenta
anos atrás: que o sufrágio é um mal, que apenas ajudou a escravizar
pessoas, que não fez mais do que fechar os seus olhos para que não
pudessem ver o quão astutamente elas foram moldadas para se submeter.

A demanda da mulher pelo sufrágio igualitário é baseada, em grande
medida, na alegação de que a mulher deve ter direitos iguais em todos os aspectos referentes à sociedade. Possivelmente, ninguém poderia refutar
tal alegação, se o sufrágio fosse, de fato, um direito. Infelizmente, a
ignorância humana é capaz de ver um direito numa imposição. Ou não se
trata da mais brutal imposição que um grupo de pessoas crie as leis
pelas quais outro grupo será coagido e forçado a obedecer? Ainda assim,
a mulher clama por essa \textit{oportunidade de ouro} que tem trazido tanta
miséria ao mundo e roubado ao homem sua integridade e autoconfiança; uma
imposição que tem corrompido as pessoas de modo absoluto,
transformando"-as em presas indefesas nas mãos de políticos
inescrupulosos.

O pobre, estúpido e livre cidadão americano! Livre para morrer de fome,
livre para vagabundear pelas estradas desse grandioso país, ele desfruta
do sufrágio universal e, em nome desse direito, forja grilhões para os seus
próprios membros. A recompensa que recebe são leis trabalhistas
rigorosas que proíbem o direito ao boicote, ao piquete, precisamente,
que proíbem tudo que não seja o direito de roubar os frutos do seu
trabalho. Todos os resultados desastrosos desse fetiche do século \textsc{xx} não
foram capazes de ensinar algo à mulher. Posto que, ainda assim, é"-nos assegurado que a mulher purificará a política.

Desnecessário dizer que a minha oposição ao sufrágio feminino não se
coaduna com o argumento convencional de que ela não é uma igual. Eu não
vejo qualquer motivo físico, psicológico ou mental para que a mulher não
deva ter o mesmo direito de votar que o homem. O que eu não posso é me
cegar para a concepção absurda de que a mulher será bem"-sucedida naquilo
em que o homem falhou. Se ela não tornar as coisas piores, certamente
não poderá torná"-las melhores. Presumir que a mulher triunfará em
purificar algo que não é passível de purificação é creditar"-lhe poderes\label{purificar}
sobrenaturais. Uma vez que a tragédia da mulher tem sido a de ser olhada
ou como um anjo ou como um demônio, a sua verdadeira salvação reside em ser
colocada sobre a terra; ou seja, em ser considerada um ser humano e,
desse modo, sujeita a todas as loucuras e erros humanos. Como poderemos,
então, acreditar que dois equívocos farão o certo? Devemos presumir que
o veneno já inerente à política diminuirá, se a mulher adentrar a arena
política? Nem mesmo as mais ardentes sufragistas admitiriam uma loucura
como essa.

Na verdade, os maiores estudiosos do sufrágio universal já perceberam
que todos os sistemas de poder político existentes são absurdos, e
completamente inadequados para ir ao encontro das questões mais
prementes da vida. Essa visão é corroborada pela declaração de alguém
que é uma fervorosa crente no sufrágio feminino, a doutora Helen L.
Sumner. No seu competente estudo \textit{Equal Suffrage},\footnote{Em tradução para o português, \textit{Sufrágio igualitário}.} ela diz: ``No Colorado, acreditamos que o
sufrágio igualitário serve para mostrar, do modo mais claro possível, a
podridão essencial e o caráter degradante de todo sistema existente.''
Naturalmente, a dra.\,Sumner tem em mente um sistema particular de
votação, mas o mesmo se aplica, e com força igual, a todo mecanismo do
sistema representativo. Com uma base como essa, é difícil compreender
como a mulher, na condição de fator político, poderia beneficiar a si
mesma ou ao resto da humanidade.

Mas, dizem as nossas sufragistas devotadas, olhem para os países e
estados em que o sufrágio feminino existe. Vejam o que a mulher foi
capaz de atingir --- na Austrália, Nova Zelândia, Finlândia, nos países
escandinavos, e nos nossos quatro estados, Idaho, Colorado, Wyoming e
Utah. A distância empresta certo encantamento --- ou, para citar um
ditado polonês --- \textit{está tudo bem onde nós não estamos}. É de
pressupor que esses países e estados sejam diferentes dos outros países e estados, que desfrutem de maior liberdade, maior igualdade social e
econômica, de uma apreciação mais elevada da vida, de um entendimento
mais profundo do significado da luta social, com todas as questões
vitais que ela traz para a raça humana.

As mulheres da Austrália e da Nova Zelândia podem votar e colaborar na
elaboração das leis. As condições de trabalho lá são melhores do que as
da Inglaterra, onde as sufragistas estão empreendendo uma luta tão
heroica? Existe nesses países uma maternidade mais plena, crianças mais
felizes e livres do que na Inglaterra? É a mulher lá, de fato,
considerada algo mais do que mera mercadoria sexual? Ela se emancipou do
padrão duplo da moralidade puritana para homens e mulheres? Certamente
ninguém, com exceção de uma mulher mediana dotada de objetivos políticos
específicos, se atreveria a dizer que sim. E se assim é, parece ridículo
apontar para a Austrália ou para a Nova Zelândia como a \textit{Meca das
realizações do sufrágio igualitário}.

De outro lado, é um fato, para aqueles que conhecem as condições
políticas reais da Austrália, que a política censurou os trabalhadores
com a aprovação de leis trabalhistas rigorosas, transformando greves
desprovidas da aprovação de algum comitê arbitrário num crime do mesmo
porte que o de traição.

Em nenhum momento tive a pretensão de sugerir que o sufrágio feminino é
responsável por esse estado das coisas. O que quero dizer é que não há
razão em apontar para a Austrália como o milagreiro das realizações das
mulheres, uma vez que sua influência foi incapaz de libertar o
trabalho da subjugação das grandes autoridades políticas.

A Finlândia deu à mulher o sufrágio igualitário; e ainda mais, deu"-lhe o
direito de se sentar no Parlamento. Será que isso ajudou a desenvolver
nelas um heroísmo maior e um fervor mais intenso do que os das
mulheres da Rússia? A Finlândia, como a Rússia, sofre sob o terrível
chicote do czar sanguinário.\footnote{A Finlândia só se tornou
  independente da Rússia em 1917.} Onde encontramos as Perovskaias,
Spiridonovas, Figners, Breshkovskaias\footnote{Referência
  às revolucionárias russas: Sophie Perovskaia (1853--1881), Maria
  Spiridonova (1884--1941), Vera Nikolayevna Figner (1852--1942)
  e Ekaterina Breshko"-Breshkovskaia (1844--1934), respectivamente.
  }
finlandesas? Onde estão as incontáveis jovens finlandesas que vão, com
alegria, para a Sibéria em nome da sua causa? Infelizmente, a
Finlândia precisa de libertadores heroicos. Por que o voto não foi capaz
de criá"-los? O único justiceiro do povo finlandês foi um homem, não uma
mulher, e ele se valeu de uma arma muito mais efetiva do que o
voto.\footnote{É provável que Goldman esteja se referindo a Eugen
  Waldemar Schauman, que assassinou, em 1904, o governador"-geral da
  Finlândia Nikolay Ivanovich Bobrikov, responsável por introduzir, a
  mando do czar Nicolau \textsc{ii}, o programa de russificação da Finlândia --- o
  que marcou o início do período que ficou conhecido como \textit{anos de
  opressão}.}

No que diz respeito aos nossos estados, nos quais as mulheres já podem votar, e
que constantemente são apontados como exemplos de grandes maravilhas, o
que foi mesmo realizado por lá através do voto que as mulheres não podem
desfrutar nos outros estados? Ou dito de outro modo, o que foi realizado
por lá que não lhes seria alcançável caso, sem se valer do
voto, empregassem grandes esforços?

É verdade que, nos estados em que o sufrágio foi instaurado, é garantido
às mulheres direitos iguais à propriedade; mas de que serve esse direito
à massa de mulheres que não possuem propriedade; aos milhares de
trabalhadoras assalariadas que vivem em condições extremamente
precárias? Que o sufrágio igualitário nunca afetou e nem pode afetar a
condição dessas pessoas é admitido pela própria dra.\,Sumner, que
certamente está na posição de saber sobre o que está falando. Na
condição de sufragista fervorosa --- enviada ao Colorado
pelo \textit{Collegiate Equal Suffrage League of New York State},\footnote{Em tradução para o português, \textit{Liga
do Colegiado do Sufrágio Igualitário do Estado de Nova York}.} para
coletar material em favor do sufrágio ---, ela seria a última pessoa a dizer
algo depreciativo; ainda assim, é através dela que somos informados
de que ``o sufrágio igualitário quase não afetou a condição econômica
das mulheres. As mulheres não recebem salários iguais por trabalhos
iguais e, muito embora no Colorado a mulher venha desfrutando do
sufrágio no âmbito escolar desde 1876, lá professoras recebem salários
mais baixos do que as da Califórnia.'' Por outro lado, a srta.\,Sumner falhou em dar conta do fato de que, embora as mulheres [no Colorado] tenham direito ao sufrágio escolar há 34 anos e ao sufrágio igualitário desde 1894, só em Denver, segundo pesquisa recente, há pelo menos 15000 crianças em idade escolar negligenciadas. E isso com as mulheres ocupando a maioria dos postos no Departamento de Educação e,
o que também é notável, após as mulheres no Colorado terem sancionado
``as mais rigorosas leis de proteção às crianças e animais.'' As
mulheres no Colorado, escreve a srta.\,Sumner, ``manifestam grande
interesse nas instituições estatais para cuidado de crianças
dependentes, deficientes e delinquentes.'' Que testemunho terrível
contra os cuidados e interesses típicos à mulher, para o caso de a cidade possuir
quinze mil crianças negligenciadas. Que glória pode ainda haver para o sufrágio
feminino, quando falhou completamente na questão de maior importância para a sociedade,
a criança? Onde está o senso superior de justiça que a mulher deveria
ter trazido para a política? Onde esse senso se encontrava quando, em
1903, os proprietários das minas travaram uma guerra contra a \textit{Western Miners' Union};\footnote{Em tradução para o português, \textit{União dos Mineiros do Oeste}.} quando o general
Bell instaurou um reino de terror, arrancando os homens à noite de suas
camas, sequestrando"-os ao longo de toda a fronteira, para prendê"-los em
celas, enquanto declarava ``a constituição que vá para o inferno, o nosso clube é
a constituição''?\footnote{A \textit{Western Miners' Union}, mais conhecida como \textit{Western Federation of Miners}, era, até 1903, a organização sindical de maior atuação militante nos Estados Unidos. Nesse ano, o general Sherman Bell foi designado, pelo então governador do Colorado, James Peabody, para a missão de não só acabar com a greve então em curso como dissolver a organização --- missão na qual obteve sucesso absoluto. Os irrisórios 225 sindicalistas que se recusaram a renunciar ao sindicato foram deportados da área, sendo este o episódio mencionado aqui, alusivamente, por Goldman.} Onde estavam, nesse caso, as mulheres da política e
por que não exerceram seu poder de voto? Ora, mas elas exerceram e
ajudaram a derrotar o homem mais justo e a mente mais liberal, o
governador Waite.\footnote{Quando governador (1893--1895), Davis H.\,Waite desempenhou papel fundamental na
  promulgação da lei que dava às mulheres o direito ao voto no estado do
  Colorado. Ironicamente, foi derrotado na eleição seguinte.} Isso abriu
caminho para o rei das minas, o governador Peabody, o inimigo do
trabalhador, o \textit{czar do Colorado}. ``Certamente, o sufrágio masculino não
poderia ter feito pior.'' Isso é fato. Em que, então, consiste a
vantagem do sufrágio feminino para a mulher e para a sociedade? A
afirmação constantemente repetida de que a mulher purificará a política
não é nada além de um mito. Por isso, não é corroborada pelas pessoas
que efetivamente conhecem as condições em que se faz política em Idaho,
no Colorado, Wyoming e Utah.

A mulher, em essência tradicionalista, é uma entusiasta natural e
incansável nos seus esforços de transformar os outros em pessoas tão
boas quanto ela acredita que deveriam ser. Por conta disso é que, em
Idaho, ela cassou o direito das suas irmãs da rua sob a alegação de que
todas as mulheres de \textit{caráter obsceno} são inaptas para votar.
Certamente, \textit{obsceno} não foi aí interpretado no sentido da
prostituição no casamento. Não é preciso dizer, que a
prostituição ilegal e os jogos de azar foram proibidos. Nesse sentido, a
lei deve mesmo pertencer ao gênero feminino: ela sempre proíbe. Sob esse
aspecto, todas as leis são maravilhosas. Mesmo que a proibição
implementada pela lei não vá longe demais, ela abre todas as comportas
do inferno. A prostituição e o jogo nunca experimentaram um crescimento
mais promissor do que antes da aprovação de leis contra eles.

No Colorado, o puritanismo da mulher se expressa de forma ainda mais
drástica. ``Tanto os homens notórios pelo seu estilo de vida sujo
quanto os que estão envolvidos com o ramo das tabernas foram removidos
da política, desde que as mulheres receberam o direito de
votar.''\footnote{Helen Sumner. \textit{Equal Suffrage. The results of an
  investigation league in Colorado made for the Collegiate equal
  suffrage league of New York State} (1909).} Poderia o irmão Comstock
ter ido além? Poderiam todos os pais puritanos juntos ter feito ainda mais? Eu
me pergunto se as mulheres compreendem a gravidade desse seu pretenso
feito. Pergunto"-me se elas compreendem que justamente esse tipo de ação, ao
invés de elevar a condição da mulher, simplesmente a transforma numa
espécie de espiã política, numa pessoa desprezível que se intromete nos
assuntos privados das outras pessoas, não tanto pelo bem da causa, mas
porque, como disse uma mulher do Colorado, ``elas gostam de entrar em
casas nas quais nunca estiveram, para descobrir tudo o que podem, seja
relacionado à política ou não.''\footnote{Helen Sumner. \textit{Equal
  Suffrage.} [\textsc{n.\,a.}]} Sim, e uma vez na alma humana, adentram os seus
mais ínfimos recantos e esquinas. Pois nada satisfaz mais o desejo da maioria das
mulheres do que o escândalo. E quando ela pôde aproveitar essas
oportunidades como estão aproveitando, agora, as mulheres da política?

``Homens notórios pelo seu estilo de vida sujo, e os que estão
envolvidos com o ramo das tabernas.'' Certamente, a dama coletora de
votos não pode ser acusada de dispor de muito senso de proporção. Mesmo
que admitamos que essas intrometidas possam decidir quais vidas são
limpas o suficiente para a atmosfera extremamente limpa da política, por
que deve se seguir que os proprietários de taberna pertençam à mesma
categoria? A não ser que estejamos tratando aqui da hipocrisia e
fanatismo americanos tão manifestos no princípio da proibição
--- que sanciona a propagação da embriaguez entre os
homens e mulheres da classe alta, enquanto mantêm o olhar vigilante
sobre o único lugar que resta ao homem pobre. A atitude limitada e
purista da mulher perante a vida a transforma numa grande ameaça à
liberdade onde quer que ela assuma o poder político. O homem há tempos
já superou as superstições que ainda devoram a mulher. No campo
competitivo da economia, o homem foi impelido a exercitar a eficiência,
o julgamento, a habilidade, a competência. Ele, portanto, já não tem nem
tempo, nem inclinação para medir a moralidade de alguém com um
referencial puritano. Nas suas atividades políticas, tampouco ele se encontra de olhos vendados. Ele sabe que quantidade, não qualidade, é o material
que movimenta os moinhos da política, e a menos que se trate de um
reformista sentimental ou de algum fóssil antigo, ele sabe muito bem que
a política nunca poderá ser outra coisa que não um pântano.

As mulheres que estão realmente familiarizadas com o processo da
política conhecem a natureza da besta, mas na sua autossuficiência e
egoísmo, elas se convencem de que têm apenas de afagar a besta para que
ela se torne tão gentil quanto um cordeiro doce e puro. Como se as
mulheres não vendessem os seus votos, como se as mulheres políticas não
pudessem ser compradas! Se o seu corpo pode ser comprado em troca de uma
remuneração material, por que não o seu voto? Que isso tem acontecido no
Colorado e outros estados, não é negado nem mesmo por aquelas que são a
favor do sufrágio feminino.

Como disse antes, a visão limitada da mulher sobre os assuntos humanos
não é o único argumento contra a sua suposta superioridade política ante
o homem. Há outros. Seu parasitismo econômico ao longo da vida
obscureceu completamente seu conceito de igualdade. Ela clama por
direitos iguais, não obstante nós saibamos que ``poucas mulheres buscam
apoio eleitoral em distritos indesejáveis.''\footnote{dra.\,Helen A.
  Sumner. [\textsc{n.\,a.}]} Quão pouco significa a igualdade para elas em
comparação com as mulheres russas que passam pelo inferno em nome do seu
ideal!\label{russas}

A mulher exige os mesmos direitos que os do homem, mas, ainda assim,
fica indignada caso a sua presença não seja capaz de fulminá"-lo: ele
fuma, mantém seu chapéu na cabeça e não pula da sua cadeira como se fosse
um lacaio. Isso pode parecer trivial, no entanto, é a chave para
compreender a natureza das sufragistas americanas. Certamente, suas
irmãs inglesas já superaram essas concepções tolas. Elas já mostraram
que estão à altura das maiores demandas de caráter e poder de
resistência. Toda honra ao heroísmo e força das sufragistas
inglesas.\footnote{Liderado por Emmeline Pankhurst, a quem Goldman se
  referirá logo em seguida, o \textit{Women's Social and Political Union}
  {[}União Social e Política das Mulheres{]} se valeu da ação direta e da
  desobediência civil, dado que suas táticas consistiram não apenas da
  realização de grandes marchas e manifestações ao ar livre e
  interrupção de reuniões políticas, como também em lutas contra a polícia,
  greves de fome, despejamento de ácido em caixas de correio, quebra de
  janelas, incêndios de igrejas, desfiguração de obras de arte na
  Galeria Nacional etc. O objetivo era o de publicizar a luta pelo
  sufrágio universal na Inglaterra.} Graças aos seus métodos enérgicos e
agressivos, elas são uma inspiração para as nossas mulheres sem vida e
sem espírito. Mas, no final das contas, as sufragistas não valorizam a
verdadeira igualdade. Ou então como podemos explicar o esforço enorme,
verdadeiramente gigantesco empreendido por essas bravas lutadoras em nome
de um projeto de lei tão mesquinho que beneficiará exclusivamente um punhado de damas
proprietárias, sem absolutamente nenhum ganho para a vasta massa de
mulheres trabalhadoras?\footnote{Embora se valessem de estratégias de
  ação consideradas extremistas (e até terroristas), o movimento liderado por Emmeline
  Pankhurst era contra o sufrágio para as mulheres da classe
  trabalhadora.} É claro que na condição de políticas elas devem ser
oportunistas, devem tomar meias medidas, se não lhes é possível tomar
medidas completas. O ponto é que, como mulheres inteligentes e liberais, elas
deveriam perceber que se o voto é uma arma, os deserdados precisam dele
mais do que a classe econômica superior, e que esta classe já desfruta
de bastante poder em virtude da sua superioridade econômica mesma.\label{virtude}

A líder brilhante das sufragistas inglesas, a senhora Emmeline
Pankhurst, admitiu, quando em turnê de conferências pela América, que não
pode haver igualdade entre pessoas politicamente superiores e
inferiores. Se assim o é, como as mulheres da classe trabalhadora da
Inglaterra, inferiores economicamente às damas beneficiárias do projeto
de lei de Shackleton,\footnote{{[}Edward Arthur Alexander{]} Schackleton
  era um líder trabalhista. Portanto, é autoevidente que ele só pudesse
  apresentar um projeto de lei que excluísse os seus próprios eleitores.
  O parlamento inglês está cheio desse tipo de Judas. [\textsc{n.\,a.}]} poderão
trabalhar com as suas superiores na política, no caso de o projeto passar? 
O mais provável não é que a classe de Annie Keeney,\footnote{Goldman comete aqui um
  erro de grafia, dado tratar"-se de Annie Kenney (não Keeny), uma das
  figuras mais proeminentes no campo da ação militante do \textit{Women's
  Social and Political Union}, embora fosse pertencente à classe
  trabalhadora.} tão cheia de ardor, devoção e martírio será obrigada a
carregar nas costas os seus novos chefes políticos do sexo feminino, embora já
estejam carregando os seus senhores financeiros? Elas teriam também de
fazer isso, fosse o sufrágio universal sancionado aos homens e mulheres da
Inglaterra. Não importa o que os trabalhadores façam, eles são obrigados
a pagar sempre. Continuamente, aqueles que acreditam no poder do voto
demonstram um senso de justiça tacanho ao desconsiderarem aqueles, para
quem, como eles mesmos bradam, o voto pode servir mais.

Até recentemente, o movimento sufragista americano era um assunto
discutido nos salões e cafés, absolutamente distante das necessidades
econômicas do povo. Por conseguinte, Susan B.\,Anthony,\footnote{Susan B.\,Anthony (1820--1906) é considerada uma das pioneiras do movimento sufragista norte"-americano.} uma mulher sem
sombra de dúvida excepcional, não era apenas indiferente, como
antagônica à classe trabalhadora; e não hesitou em manifestar a sua
hostilidade quando, em 1869, aconselhou as mulheres a tomar o lugar dos
tipógrafos grevistas em Nova York.\footnote{Helen Sumner. \textit{Equal
  Suffrage.} [\textsc{n.\,a.}]} Não sei se a atitude dela mudou antes da sua morte.

É claro que também há sufragistas coligadas às mulheres trabalhadoras
--- como a \textit{Women's Trade Union League},\footnote{Em tradução para o português, Liga Sindical das
Mulheres.} por exemplo; mas elas são minoria, e suas atividades
são essencialmente econômicas. As demais consideram o trabalho como algo
estabelecido pela Providência. O que seria dos ricos, se não fossem
os pobres? O que seria dessas nossas damas preguiçosas e parasitas, que
gastam numa semana mais do que suas vítimas ganham em um ano, se não
fossem os oitenta milhões de trabalhadores assalariados? Igualdade --- quem já ouviu falar numa coisa como essa?

Poucos países produzem tanta arrogância e esnobismo quanto os Estados
Unidos. Isso se aplica, particularmente, à mulher americana da classe
média. Ela não só se considera igual ao homem, como se sente superior a
ele, especialmente no que diz respeito à sua pureza, bondade e
moralidade. Não é de admirar que a sufragista americana reivindique para
seu voto poderes miraculosos. Em sua exaltação e presunção, ela não pode
ver o quão verdadeiramente escravizada é, e não tanto pelo o homem,
quanto por suas concepções tolas e pela tradição. O sufrágio não é capaz
de mudar esse triste fato; só pode acentuá"-lo, como realmente faz.

Uma das maiores líderes americanas reivindica não só que a mulher tenha
o direito de igualdade salarial, como também que a lei lhe dê direito ao
salário do marido. Para o caso de ele falhar em sustentá"-la, deve ser
condenado à prisão, de modo que seus ganhos no presídio sejam recebidos
pela esposa que lhe é uma igual. Não é esse outro exemplo brilhante da
causa aclamada pela mulher de que o seu voto abolirá todo o mal e que tem sido combatida em vão pelos esforços coletivos das mentes mais ilustres ao longo do mundo? É, inclusive, lamentável que o suposto criador do
universo já tenha nos presenteado com o seu maravilhoso esquema de todas
as coisas, pois não fosse isso, o sufrágio feminino por certo
capacitaria a mulher a superá"-lo completamente.

Nada é tão perigoso quanto a dissecação de um fetiche. Se superamos o
tempo em que tal heresia era punida com o empalamento, não superamos a
estreiteza de espírito que condena aqueles que ousam divergir das
ideias aceitas. Desse modo, provavelmente serei considerada como uma
oponente da mulher. Mas isso não pode me impedir de encarar
diretamente a questão. Repito o que disse no começo: não
acredito que a mulher irá transformar a política em algo pior; como tampouco
acredito que irá torná"-la em algo melhor. Se ela não pode
corrigir os erros dos homens, por que perpetuá"-los?

A história pode ser uma compilação de mentiras; mas também contém
algumas verdades e essas são o único guia que temos para o futuro. A
história dos esforços do homem na política prova que não lhe
trouxeram absolutamente nada que não pudesse ter sido alcançado mais
diretamente, de modo menos custoso e de forma mais duradoura. De fato,
todo milímetro a mais de direitos só foi conquistado através de lutas
constantes, uma luta interminável por autoafirmação e jamais pelo
sufrágio. Não há razão para supor que a mulher, na sua escalada para a
emancipação, foi ou será auxiliada pelo voto.

No mais sombrio de todos os países, dado o seu despotismo absoluto, a
Rússia, a mulher se tornou uma igual ao homem, não através do voto, mas
através da sua vontade de ser e de fazer. Não apenas conquistou para
si os caminhos do aprendizado e do desenvolvimento da sua vocação,
como também ganhou a estima do homem, o seu respeito e camaradagem; sim, e
mais do que isso: ela ganhou a admiração e o respeito do mundo inteiro.
E isso, novamente, não através do sufrágio, mas através do seu
maravilhoso heroísmo, da sua coragem, capacidade, força de vontade e
perseverança na luta pela liberdade. Onde estão as mulheres dos países
ou estados sufragistas que possam reivindicar tal vitória? Quando
consideramos as realizações da mulher na América, percebemos também que
algo mais profundo e poderoso do que o sufrágio foi o que efetivamente a
ajudou na marcha para a emancipação.

Faz apenas 62 anos que um punhado de mulheres na Convenção
de Seneca Falls\footnote{Ocorrida em julho de 1848 na cidade de Seneca Falls, no estado de Nova York, a Convenção de Seneca Falls, como ficou conhecida, é considerada a primeira convenção mundial sobre os direitos da mulher a acontecer nos Estados Unidos.} apresentou algumas demandas, como o direito à educação igualitária e o acesso às diversas
profissões, ofícios etc.~Que maravilhosas realizações,~que maravilhosos
triunfos!~Quem, exceto os mais ignorantes, ousa falar da mulher como um
simples burro de carga doméstico?~Quem ousa sugerir que essa ou aquela
profissão não deve lhe ser acessível?~Por mais de sessenta anos, a
mulher tem criado uma nova atmosfera e uma nova vida para si. Ela vem se
tornando uma potência mundial em cada área do pensamento e atividade
humana. E tudo isso sem sufrágio, sem o direito de fazer leis, sem o
\textit{privilégio} de se tornar juiz, carcereiro ou carrasco.

Sim, posso ser considerada uma inimiga das mulheres; mas se eu puder
ajudá"-las a ver a luz, não me cabe reclamar.

O infortúnio da mulher não é o de que ela é incapaz de realizar o
trabalho de um homem, mas o de que ela está desperdiçando a sua
vitalidade na tentativa de superá"-lo, ante uma tradição secular que a\label{secular}
deixou fisicamente incapaz de acompanhar o ritmo dele. Ah, sim, eu bem
sei que muitas conseguiram, mas a que custo, a que custo terrível! O
importante não é o tipo de trabalho que uma mulher faz, mas a qualidade
do trabalho que ela realiza. A mulher não pode dar ao sufrágio ou ao
voto nenhuma qualidade nova, como tampouco pode receber do sufrágio
qualquer melhora à sua própria qualidade. O seu desenvolvimento, a sua
liberdade, a sua independência devem vir dela mesma e através de si
mesma. Primeiro, ao afirmar"-se como uma personalidade, e não como uma
mercadoria sexual. Segundo, ao recusar o direito de outra pessoa sobre o
seu corpo; ao se recusar a ter filhos, a menos que ela mesma os deseje;
ao se recusar ser uma serva de Deus, do Estado, da sociedade, do marido,
da família etc.; ao tornar a sua vida mais simples, não obstante mais
profunda e rica. Isto é, ao tentar compreender o significado e a substância
da vida em toda a sua complexidade; ao se libertar do medo da opinião
pública e da condenação pública. Somente isso, jamais o direito ao voto,
libertará a mulher, fará dela uma força até então desconhecida ao
público, uma força para o amor verdadeiro, para a paz, para a harmonia;
uma força dotada de fogo divino, de vida; uma criadora de homens e
mulheres livres.

\chapter[A tragédia da mulher emancipada]{A tragédia da mulher\break emancipada\footnote{Décimo capítulo da coletânea
  \textit{Anarquismo e outros ensaios}, de 1910.}}
%\markboth{a mulher emancipada}{}

Começo com uma confissão: independentemente de todas as teorias
políticas e econômicas, das diferenças fundamentais entre os vários
grupos da humanidade, independentemente das distinções de classe e raça,
independentemente de todas as linhas que demarcam artificialmente os
respectivos direitos dos homens e das mulheres, defendo que há um ponto
em que todas essas diferenças podem se encontrar, de modo a se
transformar em um todo perfeito.\label{perfeito}

Não estou propondo, com isso, um tratado de paz. O antagonismo social
generalizado que tomou conta da totalidade da nossa vida pública atual,
provocado pelo embate de forças opostas e interesses contraditórios,
esmigalhar"-se"-á em mil pedaços quando a reorganização da nossa vida
social, baseada sobre os princípios da justiça econômica, se tornar uma
realidade.

A paz e harmonia entre os sexos e entre os indivíduos como um todo não
dependem necessariamente de um nivelamento superficial dos seres\label{ref6}
humanos; como tampouco exigem a eliminação dos traços individuais e das
peculiaridades. O problema que nos confronta atualmente, e que o futuro
próximo precisa resolver, é o de como ser si mesmo e, ainda assim, estar
em unidade com os outros, o de como se sentir profundamente ligado a
todos os seres humanos e, ainda assim, reter as próprias características
individuais.

Parece ser essa a base sobre a qual a massa e o indivíduo, o
verdadeiro democrata e a verdadeira individualidade, homem e mulher, podem se
encontrar sem antagonismo e oposição. O lema não deve ser: perdoem"-se
uns aos outros, entendam um ao outro. A sentença frequentemente citada
de Madame Staël, ``Entender tudo significa perdoar tudo'', nunca exerceu
um apelo especial sobre mim; tem o odor do confessionário; perdoar o
semelhante transmite a ideia de superioridade farisaica. Entender o
semelhante é suficiente. Admitir isso representa,
parcialmente, o aspecto fundamental da minha visão sobre a emancipação feminina
e seu efeito sobre o sexo como um todo.

A emancipação deve tornar possível à mulher ser humana no sentido mais
verdadeiro. Tudo o que nela anseia por afirmação e atividade deve
alcançar a mais plena expressão; todas as barreiras artificiais devem
ser quebradas, e a estrada em direção a uma liberdade maior deve ser
purificada de todo e qualquer traço dos séculos de submissão e
escravidão.

Esse era o objetivo original do movimento da mulher pela emancipação.
Mas os resultados alcançados até agora tanto isolaram a mulher quanto
lhe subtraíram a fonte de onde nasce a felicidade que lhe é tão essencial.
Uma emancipação meramente externa fez da mulher moderna um ser
artificial, que lembra os produtos da arboricultura francesa com suas
árvores e galhos em formato de arabesco, pirâmides, círculos e coroas de
flores; de qualquer coisa, exceto das formas que seriam alcançadas
através da expressão de suas qualidades internas. Essas plantas
artificialmente cultivadas do sexo feminino podem ser encontradas em
grande número, especialmente nos chamados círculos intelectuais. 

Liberdade e igualdade para a mulher! Que esperanças e aspirações essas
palavras despertaram quando foram proferidas pela primeira vez por
algumas das almas mais nobres e corajosas daqueles tempos idos. O sol em
toda sua luz e glória parecia nascer sobre um novo mundo; um mundo no
qual a mulher seria livre para dirigir o seu próprio destino --- uma
aspiração certamente digna de grande entusiasmo, coragem, perseverança e
do esforço incansável da parte de muitos homens e mulheres pioneiros,
que apostaram tudo o que tinham para lutar contra um mundo de
preconceito e ignorância.

Minhas esperanças também estão direcionadas para esse objetivo, mas
acredito que a emancipação da mulher, tal como interpretada e vivida
hoje, falhou em alcançar esse fim inaudito. Agora, a mulher é
confrontada com a necessidade de se emancipar da emancipação, se ela
realmente deseja ser livre. Isso pode soar paradoxal, não obstante seja
também uma verdade.

O que ela conseguiu através da emancipação? Sufrágio igualitário em
alguns poucos estados. Será que isso purificou a nossa vida política,
como muitos dos seus defensores bem"-intencionados previram? Certamente
não. Inclusive, já é hora de que pessoas com julgamento claro e
consistente parem de falar da corrupção política em tom escolar.
A corrupção política não tem nada a ver com a\label{suja}
moralidade ou com a frouxidão moral das várias personalidades
políticas. Sua causa é totalmente material. A política é o reflexo do
mundo empresarial e industrial, cujos lemas são: \textit{Tomar é melhor do que
dar}; \textit{compre barato e venda caro}; \textit{uma mão suja lava a outra}.
Não há esperança que a mulher, com o seu direito de votar, possa em
algum momento purificar a política.

A emancipação trouxe à mulher igualdade econômica para com o homem; isto
é, ela pode escolher a sua profissão e ofício; mas como o treinamento
físico no passado e presente não a equipou com a força necessária para
competir com o homem, ela é frequentemente forçada a exaurir toda a sua
energia, a esgotar toda a sua vitalidade e sobrecarregar cada um de seus
nervos de modo a atingir o valor de mercado. Pouquíssimas são
bem"-sucedidas nessa empreitada, pois é um fato que professoras, médicas,
advogadas, arquitetas e engenheiras não são tratadas com a mesma
confiança que os seus colegas homens, como tampouco recebem a mesma\label{igualdade}
remuneração. E aquelas que alcançam essa igualdade sedutora, geralmente,
a atingem à custa do seu bem"-estar físico e mental. No que diz respeito
à massa de garotas e mulheres da classe trabalhadora, quanta
independência pode ser conquistada, quando a limitação e falta de
liberdade do ambiente doméstico são simplesmente trocadas pela limitação e falta de
liberdade nas fábricas, em locais precarizados de trabalho,\footnote{Em inglês, é usado o termo \textit{sweatshop}.} nas lojas de departamento ou nos escritórios?
Some"-se a isso o fardo que pesa sobre muitas mulheres de ter de cuidar
do \textit{lar, doce lar} --- frio, sombrio, desordenado, desagradável --- depois de um dia de trabalho pesado. Que gloriosa independência. Não é
de admirar que centenas de garotas estejam dispostas a aceitar a
primeira oferta de casamento, doentes e cansadas que estão da sua
\textit{independência} atrás do balcão, atrás da máquina de costura ou de
escrever. Elas querem se casar tanto quanto as garotas da classe média
anseiam se livrar do jugo da supremacia parental. A chamada\label{ganho}
independência, que possibilita tão somente o ganho da mera subsistência,
não parece tão atraente, tão ideal, a ponto de ser possível esperar que
a mulher venha a sacrificar tudo por ela. No final das contas, a nossa
independência, excessivamente elogiada, nada mais é do que o processo
lento de embotar e sufocar a natureza da mulher, seu instinto de amor e
seu instinto materno.

Apesar disso, a situação de uma jovem da classe trabalhadora é muito
mais natural e humana do que a da sua irmã, aparentemente mais
afortunada, das esferas profissionais mais cultas --- como professoras, médicas, advogadas, engenheiras etc. ---,
que precisa se dotar de uma aparência digna e apropriada, enquanto a
vida interior se torna vazia e morta.

A limitada concepção existente de independência e emancipação feminina;
o medo de amar um homem que não lhe seja igual socialmente; o medo de
que o amor irá roubar a sua liberdade e independência; o horror de que o
amor e a alegria da maternidade a impedirão de exercer plenamente a
profissão --- tudo isso junto torna a mulher moderna emancipada uma casta
compulsória, diante de quem a vida, com suas tristezas grandiosas e
clarificadoras e com suas alegrias profundas e fascinantes, passa sem
tocar e arrebatar a sua alma.

A emancipação, tal como entendida pela maioria dos seus seguidores e
representantes, é muito estreita para deixar espaço para o amor e o
êxtase sem limites que estão enraizados nas emoções mais profundas de\label{extase}
uma mulher verdadeira, amante e mãe, quando na sua liberdade.

A tragédia da mulher autossuficiente ou economicamente independente não
diz respeito a muitas mulheres, antes a pouquíssimas experiências. É
verdade que ela supera a sua irmã das gerações passadas em termos de
conhecimento sobre o mundo e sobre a natureza humana; mas é justamente
por causa disso que ela sente profundamente a ausência da essência da
vida, que por si só é capaz de enriquecer a alma humana, e sem a qual a
maioria das mulheres se transformou em nada mais do que profissionais
autômatas.

Que tal estado das coisas estava prestes a se estabelecer foi antevisto
por aqueles que puderam perceber que, no domínio da ética, ainda
restavam muitas ruínas dos tempos da superioridade indisputável do
homem; ruínas que ainda hoje são consideradas úteis. E, o que é mais
importante, um grande número de mulheres emancipadas não podem viver sem
elas. Em todo movimento que visa a destruição das instituições
existentes e a sua substituição por algo mais avançado e perfeito, há
seguidores que, em teoria, apoiam as ideias mais radicais, mas que,
apesar disso, na sua prática diária são como qualquer filisteu mediano;
fingem respeitabilidade e clamam pela boa opinião dos seus oponentes.
Há, por exemplo, socialistas e mesmo anarquistas, que defendem a ideia
de que a propriedade é um roubo, muito embora fiquem indignados se
alguém lhes deve algo no valor de meia dúzia de alfinetes.\label{alfinete}

O mesmo filisteu pode ser encontrado no movimento da emancipação
feminina. Jornalistas sensacionalistas e literatos água com açúcar
retratam a mulher emancipada de tal modo que faz os cabelos dos bons
cidadãos e das suas companheiras embotadas ficarem em pé. Toda ativista
do movimento dos direitos da mulher é retratada como uma Georg Sand em
seu absoluto desrespeito pela moralidade. Nada era sagrado para ela. Não
tinha nenhum respeito pela relação ideal entre homem e mulher. Em suma,
a emancipação virou sinônimo de uma vida imprudente de luxúria e pecado;
independentemente da sociedade, religião e moralidade. Grandes figuras
dos direitos da mulher ficaram profundamente indignadas com essa
distorção e, sem qualquer senso de humor, usaram toda a sua energia para
provar que elas não eram tão ruins quanto foram pintadas, mas, antes, o
contrário. É claro que enquanto foi escrava do homem, a mulher não podia
ser boa e pura, mas agora que era livre e independente, provaria quão
boa ela poderia ser e que a sua influência teria um efeito purificador
em todas as instituições sociais. Verdade que o movimento pelos direitos
da mulher quebrou muitos grilhões antigos, mas também forjou novos. O
grande movimento pela \textit{verdadeira} emancipação ainda não encontrou
uma raça de mulheres grandiosa o suficiente para olhar a liberdade de
frente. Sua visão estreita e puritana baniu o homem da sua vida
emocional, como um personagem perturbador e duvidoso. O homem não devia
ser tolerado a preço algum, exceto, talvez, como pai de um filho, uma
vez que um filho dificilmente chegaria à vida sem um pai. Felizmente,
mesmo as mais rígidas puritanas nunca serão fortes o suficiente para
matar o desejo inato de maternidade. A questão é que a liberdade da
mulher está intimamente relacionada à liberdade do homem, e muitas das
nossas irmãs ditas emancipadas parece fazer vista grossa para o fato de
que uma criança nascida na liberdade precisa do amor e devoção de todos
ao seu redor, seja homem ou mulher. Infelizmente, é essa concepção
estreita das relações humanas que provoca essa grande tragédia nas vidas
do homem e da mulher modernos.

Cerca de 15 anos atrás, apareceu o trabalho da brilhante norueguesa
Laura Marholm, intitulado \textit{Woman, a character study}.\footnote{Em tradução para o português, \textit{Mulher, um estudo psicológico}.} Ela foi uma das primeiras a
chamar a atenção para o vazio e a estreiteza da concepção da emancipação
feminina então existente e para o seu efeito trágico sobre a vida
interior da mulher. Em seu trabalho, Laura Marholm relata o destino de
várias mulheres talentosas de fama internacional: a genial Eleonora
Duse; a grande matemática e escritora Sonia Kovalevskaia; a artista e
poetiza inata Marie Bashkirtseff, que morreu tão jovem. Em todas as
biografias dessas mulheres de mentalidade extraordinária, vê"-se um
caminho marcado pelo anseio insatisfeito por uma vida plena, harmoniosa,
completa e bela, e a inquietação e solidão resultantes dessa ausência.
Através desses esboços psicológicos magistrais não se pode deixar de ver
que quão maior seja o desenvolvimento mental de uma mulher, menor a
possibilidade de que ela venha a encontrar um companheiro com quem tenha
afinidades e que veja nela não apenas um sexo, mas também um ser humano,
uma amiga, uma companheira dotada de individualidade forte que não pode
e nem deve perder um único traço do seu caráter.

O homem, com sua autossuficiência e seu ridículo ar superior de
patrono do sexo feminino, é um parceiro impossível para a mulher
retratada no \textit{Character Study} de Laura Marholm. Igualmente impossível, para ela, quanto é o homem que vê apenas a sua intelectualidade e o seu
gênio, mas que falha em despertar a sua natureza de mulher.

Um intelecto rico e uma alma desenvolvida são comumente considerados
atributos necessários a uma personalidade bela e profunda. No caso da
mulher moderna, esses atributos são como um obstáculo à afirmação
completa do seu ser. Há mais de cem anos, a velha forma de casamento,
baseada na Bíblia, \textit{até que a morte vos separe}, é denunciada
como a instituição que representa a soberania do homem sobre a mulher, a
sua completa submissão aos humores e ordens do homem e a dependência
absoluta de seu nome e respaldo. Já foi demonstrado repetida e
conclusivamente que o casamento tradicional reduz a mulher à função
de mera serva e incubadora de filhos. E, no entanto, encontramos muitas
mulheres emancipadas que preferem o casamento, com todas as suas
desvantagens, às limitações de uma vida de solteira: limitada e
insuportável por conta das amarras do preconceito moral e social que
enfaixam e sufocam a sua natureza.

A explicação para essa inconsistência da parte de muitas mulheres
progressistas é que elas nunca compreenderam verdadeiramente o
significado da emancipação. Elas pensaram que a única coisa necessária
seria a independência das tiranias externas; os tiranos internos, muito
mais prejudiciais à vida e ao crescimento --- as convenções éticas e
sociais ---, foram deixados aos seus próprios cuidados; e eles, de fato,
fizeram a sua parte. Ao que parece eles estão tão profundamente
arraigados nas mentes e nos corações das mais ativas representantes da
emancipação, quanto estavam nas mentes e nos corações das nossas avós.

Esses tiranos internos podem assumir a forma da opinião pública ou da
pergunta sobre o que a mãe irá dizer, ou o irmão, o pai, a tia ou algum
outro parente; o que a sra.\,Grundy, o que o sr.\,Comstock, o patrão, o
Conselho de Educação irão dizer? Todos esses enxeridos, todos esses
detetives morais, carcereiros do espírito humano, o que eles irão dizer?
Até que a mulher aprenda a desafiar todos eles, a permanecer firme em
seu próprio terreno, a insistir na sua liberdade irrestrita, a ouvir a
voz da sua natureza, seja esta voz um chamado para o maior tesouro da
vida, o amor por um homem, ou o seu mais glorioso privilégio, o direito
de dar à luz a uma criança, ela não poderá chamar"-se de emancipada.
Quantas mulheres emancipadas foram corajosas o suficiente para admitir
que a voz do amor a está convocando, que está batendo selvagemente
contra o seu peito, demandando ser ouvida, ser satisfeita?

O escritor francês Jean Reibrach, num de seus romances, \textit{La nouvelle beauté},\footnote{Em tradução para o português, \textit{A nova beleza}.} tenta retratar uma mulher
emancipada ideal e bela. Esse ideal é encarnado numa jovem mulher, uma
médica. Ela fala com muita inteligência e sabedoria sobre como alimentar
crianças; ela é gentil e distribui medicamentos gratuitos para mães
pobres. Ela conversa com um jovem conhecido sobre as futuras condições
sanitárias e como vários tipos de bacilos e germes serão exterminados
com a introdução de paredes e piso de pedra que substituirão os tapetes
e cortinas. Ela, obviamente, veste"-se de forma muito simples e prática,
na maioria das vezes de preto. O jovem rapaz que, após o primeiro
encontro, ficou intimidado pela sabedoria da sua amiga emancipada,
gradualmente começa a entendê"-la e, em um belo dia, percebe que a ama.
Os dois são jovens e ela é gentil e bela, e embora esteja sempre em
trajes rígidos, sua aparência é suavizada por uma gola e punhos brancos
sempre impecavelmente limpos. Seria de se esperar que ele declarasse o
seu amor por ela, mas ele não é do tipo que comente absurdidades
românticas. A poesia e o entusiasmo do amor cobrem a sua face corada
quando diante da beleza pura da dama. Ele silencia a voz da natureza e
permanece correto. Ela também está sempre conscienciosa, sempre
racional, sempre muito bem"-comportada. Temo que se eles tivessem formado
uma união, o jovem rapaz teria corrido o risco de morrer de frio. Devo
confessar que não consigo ver beleza nessa nova beleza, que é tão fria
quanto as paredes e chão de pedra com as quais ela tanto sonha.
Preferivelmente, eu ficaria com as músicas de amor das eras românticas,
preferivelmente com Don Juan ou Madame Vênus, preferivelmente com uma
fuga, numa noite de luar, fosse pelas escadas ou por uma corda, e mesmo
se seguida pela maldição do pai, pelos lamentos da mãe e comentários
morais dos vizinhos do que com uma correção e propriedade
milimetricamente mensuradas de acordo com um parâmetro específico. Se o
amor não sabe como dar e receber sem restrições, então não é amor, mas
uma transação comercial na qual mais e menos são constantemente pesados
​​entre si.

A maior deficiência da emancipação de hoje é a sua rigidez artificial e
a sua respeitabilidade estrita, que criam um vazio na alma da mulher que
não lhe permite beber na fonte da vida. Certa vez, observei que parecia
haver uma relação mais profunda entre a mãe à moda antiga, dona de casa ---\label{mae}
sempre atenta à felicidade dos seus filhos e ao conforto dos que
amava --- e a mulher verdadeiramente nova; do que entre essa última e a sua
irmã emancipada com a qual nos deparamos comumente. As discípulas dessa
emancipação pura e simples me sentenciaram à condição de pagã, apta ao
empalhamento. O fervor cego delas não permitiu que elas vissem que a
minha comparação entre o antigo e o novo foi meramente para provar que
boa parte das nossas avós tinham mais sangue nas veias, muito mais humor
e sagacidade, e certamente uma naturalidade, amabilidade e simplicidade
muito maiores do que a mulher profissional em emancipação que preenche
as universidades, salas de estudo e escritórios vários. Isso não
significa que eu queira retornar ao passado, como tampouco estou condenando a mulher de volta à sua antiga órbita, como a cozinha e o berçário.

A salvação reside na marcha energética em direção a um futuro mais
brilhante e mais claro. Precisamos de um caminho que esteja livre das
velhas tradições e hábitos. Até agora, o movimento da emancipação
feminina não deu mais do que o primeiro passo nessa direção. Só podemos
esperar que ganhe força para dar outro. O direito ao voto, ou direitos
civis igualitários podem ser até exigências razoáveis, mas a verdadeira
emancipação não começa nem nas urnas, nem nos tribunais. Começa na alma
da mulher. A história nos ensina que qualquer classe oprimida só
alcançou a verdadeira libertação dos seus senhores através dos seus
próprios esforços. É necessário que a mulher aprenda essa lição, que ela
reconheça que a sua liberdade será alcançada na medida em que o seu
poder de alcançar a liberdade se amplie. É, portanto, muito mais
importante que ela comece pela sua regeneração interna, que se liberte
do peso dos preconceitos, tradições e costumes. A demanda por direitos
iguais em qualquer área da vida é justa e correta; mas, ainda assim, o
direito mais vital é o direito de amar e ser amado. Se for para a
emancipação parcial se tornar, realmente, a emancipação verdadeira e
completa da mulher, ela terá de se livrar da concepção ridícula de que
ser amada, ser amável e ser mãe é sinônimo de ser escrava ou
subordinada. Terá de abolir a ideia absurda do dualismo dos sexos ou a
de que o homem e a mulher são representantes de dois mundos antagônicos.

A mesquinhez separa; a generosidade une. Sejamos grandes e generosos.
Não vamos perder de vista aquilo que é vital por conta das trivialidades
que, a todo momento, nos confrontam. Uma concepção verdadeira da relação\label{concepcao}
entre os sexos não admitirá conquistadores e conquistados; posto que
conhecerá apenas uma grande coisa: doar"-se sem limites, a fim de
encontrar um eu mais rico, profundo, melhor. Somente isso pode preencher
o vazio e transformar a tragédia da emancipação feminina em alegria, em
alegria ilimitada.

\chapter{Casamento e amor\footnote{Décimo primeiro capítulo da coletânea
  \textit{Anarquismo e outros ensaios}, de 1910.}}
\markboth{Casamento e amor}{}

A concepção popular sobre o casamento e o amor é a de que estes dois
termos são sinônimos, que surgem pelas mesmas razões e abarcam as mesmas
necessidades humanas. Tal como ocorre com a maioria das concepções
populares, isso não se baseia em fatos reais, mas tão somente em
superstições.

Casamento e amor não têm nada em comum; eles estão tão longe um do outro
quanto dois polos; são, na verdade, antagônicos entre si. Não há dúvidas
de que alguns casamentos resultam do amor. No entanto, não porque o
amor só pudesse se fazer valer sob o casamento, e sim, muito mais,
porque apenas poucas pessoas podem superar as convenções. Há, hoje em
dia, um grande número de homens e mulheres para quem o casamento nada
mais é do que uma farsa, não obstante se submetam a ele em prol da
opinião pública. Ainda que, para todos os efeitos, seja verdade que
alguns casamentos se baseiam no amor, tal como é igualmente verdadeiro
que, em certos casos, o amor continua na vida de casado, mantenho
a posição de que isso se dá independentemente do casamento, e
não por causa dele.

Por outro lado, é absolutamente falso que o amor resulte do casamento.
Em raras ocasiões, pode"-se ouvir casos miraculosos acerca de casais que
se apaixonam completamente um pelo outro após o casamento, mas com um
exame mais minucioso se descobrirá que se trata apenas de mero ajuste ao
que é inevitável. Certamente, acostumar"-se um ao outro é bem diferente
da espontaneidade, da intensidade e da beleza do amor, elementos sem os
quais a intimidade do casamento demonstrou"-se degradante tanto para a
mulher quanto para o homem.

O casamento é, antes de tudo, um acordo econômico, uma apólice de
seguro. Difere de um seguro de vida propriamente dito apenas porque é
mais compulsório, mais exigente. Seu retorno é insignificante quando
comparado aos investimentos. Ao fazer uma apólice de seguro, paga"-se por
ela em dólares e centavos, sempre tendo a liberdade de suspender os
pagamentos. Se, no entanto, o abono de uma mulher for um marido, ela
paga por ele com o seu nome, sua privacidade, seu amor"-próprio, sua vida
como um \textit{até que a morte vos separe}. Além disso a apólice
casamento a condena a uma dependência por toda a vida, ao parasitismo, a
uma completa inutilidade seja individual ou social. Os homens também
pagam a sua taxa, mas como a sua esfera é mais ampla, o casamento não
limita tanto quanto no caso da mulher. Ele sente as suas correntes mais
no sentido econômico.

Assim, o lema de Dante sobre a porta do Inferno se aplica com igual força
ao casamento: ``Deixai toda esperança, vós que entrais!''

Que o casamento é um fracasso ninguém, exceto os mais
estúpidos, poderá negar. Um breve relance nas estatísticas do divórcio é
suficiente para que se constate quão amargo um casamento fracassado é. Argumentos
estereotipados e típicos de filisteus de que tal aumento se deve à frouxidão
das leis do divórcio e à crescente liberdade das mulheres não dão conta
do fato de que: em primeiro lugar, um em cada 12 casamentos termina em
divórcio; em segundo, que desde 1870, os divórcios aumentaram de 28 para
73 por 100.000 habitantes; em terceiro, que, desde 1867, o adultério,
como motivo para o divórcio, aumentou 270,8\%; e em quarto, que a
deserção do cônjuge aumentou 369,8\%.

Assomado a esses dados alarmantes, há uma vasta quantidade de material
dramático e literário que promove a elucidação do assunto. Robert
Herrick, no seu \textit{Together}; Pinero, em \textit{Mid"-Channel}; Eugene
Walter, em \textit{Paid in full}, e muitos outros escritores discutem a
esterilidade, a monotonia, a sordidez, numa palavra, a inadequação do
casamento para a harmonia e a compreensão.

Qualquer estudioso das questões sociais que seja capaz de pensar não irá
se contentar com a desculpa popular e superficial, em geral, dada a este fenômeno. Ele
terá de se aprofundar na vida real dos sexos para descobrir por que o
casamento é tão desastroso.

Edward Carpenter diz que por trás de todo casamento encontra"-se o
contexto de vida dos dois sexos; um contexto tão diferente um do outro
que homens e mulheres têm de permanecer estranhos um ao outro.
Separados por um muro intransponível de superstição, costumes e hábitos,
o casamento não tem o potencial de desenvolver a intimidade e o
respeito mútuos, sem os quais toda união está fadada ao fracasso.

Henrik Ibsen, que odiava todas as hipocrisias sociais, foi, provavelmente,
o primeiro a entender essa grande verdade. Nora deixa o marido,
não --- como defendem os críticos estúpidos --- porque ela estivesse
cansada das suas responsabilidades ou porque ansiasse pelos direitos da
mulher, mas, sim, porque chegou à compreensão de que viveu com um
estranho por oitos anos, com quem teve filhos. Pode haver algo mais
humilhante, mais degradante do que uma proximidade, ao longo de uma
vida, entre dois estranhos?\footnote{Referência à peça \textit{Casa de bonecas}, cuja publicação e estreia, em 1879, garantiu o primeiro sucesso internacional de Ibsen.} Não há necessidade de a mulher saber
qualquer coisa sobre o marido, exceto a sua renda. E o que o homem
precisa saber sobre a mulher que não seja se ela possui uma aparência
atraente? Ainda não superamos o mito teológico de que a mulher não tem
alma, que ela é um mero apêndice do homem, feita da costela apenas para
a conveniência do cavalheiro em questão, que de tão forte teme a sua
própria sombra.\label{ref7}

Talvez a má qualidade da matéria"-prima com que a mulher foi feita seja a
responsável pela sua inferioridade. De qualquer forma, a mulher não tem
alma --- o que haveria, então, de se saber sobre ela? Além disso, quanto
menos alma uma mulher tiver, mais adequada estará à condição de esposa,
mais prontamente irá se deixar absorver em seu marido. É a aceitação
servil da superioridade do homem que tem mantido a instituição casamento
aparentemente intacta por tanto tempo. Agora que a mulher está se
tornando ela mesma, agora que ela está realmente se tornando consciente
de si como um ser independente das graças do seu mestre, a instituição
sagrada do casamento gradualmente está sendo solapada, e nenhum montante
de lamentação sentimental pode modificar isso.

Desde a infância, é dito, praticamente, a todas as garotas que o
casamento é o seu objetivo final; portanto seu treinamento e educação
devem ser dirigidos com vistas a esse fim. Tal como um animal mudo que é
posto na engorda para o seu conseguinte abate, ela é preparada para o
casamento. No entanto, por mais estranho que pareça, é"-lhe permitido
saber muito menos sobre a sua função como esposa e mãe do que a um
artesão comum sobre o seu ofício. Para uma garota respeitável, é\label{artesao}
indecente e sujo saber qualquer coisa sobre a relação matrimonial. Ora,
ante a inconsistência de tal respeitabilidade, necessita"-se do voto
de casamento para tornar algo que é sujo em um arranjo tão puro e tão
sagrado, que ninguém ousará questionar ou criticar.
Essa é precisamente a atitude de qualquer defensor mediano do
casamento. A futura mãe e esposa é mantida em completa ignorância acerca do seu
único recurso na arena competitiva: o sexo. Assim, ela adentra relações
de longo prazo com um homem apenas para, posteriormente, vir a se sentir
chocada, rechaçada, indignada para além de qualquer medida com o mais
natural e saudável dos instintos, o sexo. Pode"-se dizer com certeza que
uma grande porcentagem da infelicidade, miséria, angústia e sofrimento
físico do casamento se deve à ignorância criminosa em questões sexuais
--- ignorância que é exaltada como uma grande virtude. Não é exagero
dizer que mais do que um lar foi destruído, devido a esse fato
deplorável.

Se, no entanto, a mulher se tornar livre e grande o suficiente para
aprender os mistérios do sexo sem a sanção do Estado e da Igreja, ela
será condenada como absolutamente inadequada para se tornar a esposa de
um \textit{bom homem} --- bondade que consiste numa cabeça vazia e muito
dinheiro. Pode haver algo mais ultrajante do que a ideia de que uma
mulher saudável e adulta, cheia de vida e paixão, deve negar as demandas
da natureza, deve subjugar o seu mais intenso desejo, minar sua saúde,
quebrar seu espírito, atrofiar sua visão, abster"-se da profundidade
e da glória da experiência sexual até que um \textit{bom homem} apareça
para tomá-la como esposa? É exatamente isso o que o casamento significa.
Como tal arranjo poderia terminar de outra forma que não em fracasso?
Este é um dos fatores, certamente não o menos importante, que
diferencia o casamento do amor.

Vivemos numa era prática. Os dias em que Romeu e Julieta arriscaram a
ira de seus pais por amor, em que Gretchen se expôs à maledicência dos
vizinhos por causa do amor, já não existe mais. Nos raros casos em que
os jovens se permitem a luxúria de um romance, eles são postos aos cuidados dos mais velhos, para que de tão perfurados e esmagados tornem-se finalmente \textit{sensatos}.

A lição moral incutida na garota não é se o homem despertou o seu amor,
mas sim: \textit{quanto?} O único Deus que importa na vida prática americana:
esse homem tem condições de ganhar a vida? Ele pode sustentar uma
esposa? Essa é a única coisa que justifica o casamento. Gradualmente
isso satura cada pensamento da garota; seus sonhos não são sobre a luz
do luar e beijos, sobre risadas e lágrimas; ela sonha em fazer compras e
caçar pechinchas. Essa pobreza de alma e sordidez são elementos
inerentes à instituição casamento. O Estado e a Igreja não aprovam
nenhum outro ideal, simplesmente porque é este o ideal que requer o
controle do Estado e da Igreja sobre os homens e as mulheres.

Sem dúvida, ainda existem pessoas para quem o amor é superior a dólares
e centavos. Isto é particularmente verdade para a classe, cuja
necessidade econômica forçou à autossuficiência financeira. A enorme
mudança na situação da mulher provocada por esse fator poderoso e
influente é realmente incrível quando consideramos o pouco tempo que
passou desde que ela adentrou a arena do trabalho industrial. Seis
milhões de mulheres assalariadas; seis milhões de mulheres que têm o
mesmo direito dos homens de serem exploradas, roubadas, de entrar em
greve; e até morrer de fome. Necessita de algo mais, meu caro senhor?
Sim, seis milhões de trabalhadoras em todos os setores da vida, do mais
exigente trabalho cerebral ao mais pesado trabalho braçal nas minas e
nos trilhos das ferrovias; sim, inclusive como detetives e policiais.
Certamente a emancipação está completa.

Apesar de tudo isso, apenas um pequeno número no vasto exército de
mulheres assalariadas considera o trabalho como algo permanente, de modo
semelhante aos homens. Por pior que seja o trabalho, a ele foi ensinado
ser independente e se autossustentar. Ah, eu bem sei que ninguém é
realmente independente em nossa esteira econômica; mas, ainda assim, até
o mais pobre espécime de homem odeia ser um parasita; odeia ser
considerado desta forma, qualquer que seja o preço.

A mulher considera a sua condição de trabalho como temporária, que deve
ser jogada fora pelo primeiro licitador. É por isso que é infinitamente
mais difícil organizar mulheres do que homens. ``Por que devo ingressar
num sindicato? Eu vou me casar, ter uma casa.'' Ela não foi ensinada,
desde a infância, a olhar para o casamento como a sua vocação definitiva?
Logo, porém, ela entenderá que o lar, embora não seja uma prisão tão
grande quanto a fábrica, possui portas e grades mais sólidas. Que possui
um guarda tão fiel que nada pode escapar dele. A parte mais trágica, no entanto, é que o lar não a liberta da escravidão assalariada; ao contrário, apenas aumenta as suas incumbências. 

De acordo com as estatísticas mais recentes apresentadas num comitê
sobre trabalho, salário e superpopulação, 10\% das trabalhadoras
assalariadas da cidade de Nova York são casadas e, ainda assim, têm de
continuar trabalhando na condição de mão de obra mais mal paga do mundo.
Adicione"-se a este aspecto horrendo, o trabalho árduo do ambiente
doméstico: e então o que pode restar da proteção e glória atribuídas ao
lar? O fato é que nem mesmo uma jovem da classe média ao casar pode
falar que possui uma casa, uma vez que é o homem quem cria essa sua
esfera. Não importa aqui se o homem é um bruto ou um queridinho. O que
pretendo provar é que o casamento garante à mulher um lar apenas pelas
graças do seu marido. Lá, ela se movimenta na casa \textit{dele}, ano após
ano, até que cada aspecto da sua vida e dos assuntos humanos em geral se
tornam tão achatados, estreitos e sombrios quanto o seu entorno. Não é de admirar que ela se torne resmungona, mesquinha,
briguenta, fofoqueira e insuportável, expulsando, com isso, o homem de
casa. Ela não poderia ir, caso quisesse; não há lugar para onde ir.
Ademais, mesmo um curto período de vida marital, de completa rendição de
todas as faculdades, faz da mulher comum um ser absolutamente incapaz de
sobreviver no mundo exterior. Ela se torna desleixada com a aparência,
destrambelhada em seus movimentos, dependente na tomada de decisões,
covarde em seus julgamentos, opressiva e maçante, passando a ser odiada
e desprezada pela maioria dos homens. Uma atmosfera maravilhosamente
inspiradora para criar uma nova vida, não?

Mas, e a criança, como pode ser protegida, se não for pelo casamento?
Afinal, não é esta a consideração mais importante? Que farsa, que
hipocrisia tudo isso! Casamento para proteger a criança, enquanto há
milhares de crianças miseráveis e desabrigadas. Casamento para proteger
a criança, não obstante, orfanatos e reformatórios estejam superlotados
e a \textit{Society for the Prevention of Cruelty to Children}\footnote{Em tradução para o português, \textit{Sociedade de Prevenção à Crueldade contra a Crianças}.} esteja, a todo instante,
ocupada em resgatar as pequenas vítimas de seus pais \textit{amorosos}, para
colocá"-las sob um cuidado mais amoroso na \textit{Gerry
Society}.\footnote{Nome pelo qual ficou conhecida a \textit{The New York Society for Prevention of Cruelty to Children}, primeira organização de proteção infantil do mundo, fundada em 1874.} \textit{Oh}, que escárnio tudo isso!

O casamento pode ter o poder de levar Maomé à montanha, mas pode levar a
montanha a Maomé? A lei irá colocar o pai atrás das grades e
vesti"-lo com os trajes da prisão; mas em algum momento fez calar a fome
da criança? Se o pai não tem trabalho ou esconde a sua identidade, o que
o casamento pode fazer então? Ele invoca a lei para colocar o homem perante a
\textit{justiça}, para colocá"-lo em segurança atrás de portas fechadas; mas
então seu trabalho, de todo modo, não se servirá à criança, mas ao
Estado. A criança recebe apenas uma memória arruinada da roupa de
detento do seu pai.

No que diz respeito à proteção da mulher --- neste ponto jaz a maldição
do casamento. Não que o casamento verdadeiramente a proteja, mas a ideia
em si é tão revoltante, um ultraje e um insulto à vida tão degradantes à
dignidade humana, que condena para sempre essa instituição parasitária.

É como aquela outra instituição paternalista --- o capitalismo. Ele rouba
das pessoas o seu direito nato, inibe seu crescimento, envenena os seus
corpos, condena-as à ignorância, pobreza e dependência, para depois
criar instituições de caridade que prosperam sobre os últimos escombros de
amor"-próprio dos seres humanos.

A instituição casamento transforma a mulher num parasita, num ser
absolutamente dependente. Incapacita"-a para a luta pela vida, aniquila a
sua consciência social, paralisa a sua imaginação, e depois impõe a sua
proteção benevolente, que na realidade é uma armadilha, uma paródia do
real caráter humano.\label{parasita}

Se a maternidade é a mais alta realização da natureza da mulher, que
outra proteção ela precisa além do amor e da liberdade? O casamento não
faz mais do que desonrar, ultrajar e corromper essa realização. O
casamento não diz à mulher: ``Apenas quando você me seguir, você poderá
dar à luz''? Não a condena à escravidão, não a degrada e humilha caso
ela recuse comprar o seu direito à maternidade com a venda de si mesma?
O casamento não é o único meio que sanciona a maternidade, ainda que
seja concebida no ódio, sob imposição? No caso de a maternidade ser
fruto de uma escolha livre, do amor, do êxtase, da paixão desafiadora,
não irá colocar uma coroa de espinhos sobre uma cabeça inocente e talhar
com letras de sangue o hediondo epíteto \textit{bastardo}? Ainda que o
casamento contivesse todas as virtudes alegadas, seus crimes contra a
maternidade o baniriam para sempre do reino do amor.

Amor, o mais forte e profundo elemento da vida na sua totalidade, o arauto da
esperança, da alegria, do êxtase; amor, que desafia todas as leis, todas
as convenções; amor, o mais livre, mais poderoso modelador do destino
humano; como poderia essa força tão absolutamente irresistível ser sinônimo
da pobre erva daninha gerada pela Igreja e pelo Estado, o casamento?

Amor livre? Como se o amor pudesse não ser livre! O homem pode comprar
cérebros, mas todos os milhões de dólares do mundo não podem comprar o
amor. O homem subjugou corpos, mas todo o poder na terra foi incapaz de
subjugar o amor. O homem conquistou nações inteiras, mas todos os seus
exércitos juntos não puderam conquistar o amor. O homem acorrenta e
agrilhoa o espírito, mas diante do amor ele está completamente
desamparado. Mesmo que se encontre nas alturas de um trono, com todo o
esplendor e glória que o ouro pode comprar, ainda assim, o homem será
pobre e desolado se o amor passar ao largo. Mas, caso não passe e resolva
ficar, até o casebre mais pobre torna"-se radiante de aconchego, de vida
e de cor. Portanto, o amor tem o poder mágico de fazer de um mendigo, um
rei. Sim, o amor é livre; ele não pode viver em outra atmosfera. Em
liberdade, ele se entrega sem reservas, abundante, completamente.
Todas as leis dos estatutos, todos os tribunais do universo não podem
arrancá"-lo do solo, caso o amor tenha criado raízes. Mas se o solo é
estéril, como o casamento pode dar frutos? É como a luta desesperada e
última entre a vida passageira e a morte.

O amor não precisa de proteção; ele é a sua própria proteção. Contanto
que seja o amor a criar a vida, nenhuma criança será abandonada, ou
ficará com fome, ou faminta de desejo de afeto. Eu sei que isso é
verdade. Conheço mulheres que se tornaram mães em liberdade ao lado dos
homens que amavam. Poucas crianças no casamento desfrutam do cuidado, da
proteção, da devoção que uma maternidade livre é capaz de dar.

Os defensores da autoridade temem o advento de uma maternidade livre,
com receio de que esta lhes roube a presa. Quem lutaria nas guerras?
Quem acumularia a riqueza? Quem seria o policial, o carcereiro, se a
mulher recusasse a geração arbitrária de crianças? A descendência, a
descendência! --- brada o rei, o presidente, o capitalista, o padre. A
descendência deve ser preservada, ainda que a mulher seja degradada a
mera máquina --- e a instituição casamento é nossa única válvula de
segurança contra o pernicioso despertar sexual da mulher. Mas esses
esforços frenéticos para manter o atual estado de servidão são completamente vãos.
Como também são vãos os decretos da Igreja, os ataques loucos dos
governantes, e até mesmo o longo braço da lei. A mulher já não está
mais disposta a fazer parte da produção de seres humanos doentes, fracos,
decrépitos e desprezíveis, que não têm força, nem coragem moral para
se livrar do jugo da pobreza e da escravidão. Ao invés disso, ela deseja
ter menos e melhores filhos, gerados e educados no amor e por livre
escolha; não por obrigação, como o casamento impõe. Nossos
pseudomoralistas ainda não aprenderam o profundo senso de
responsabilidade para com a criança, que o amor em liberdade despertou
no peito da mulher. Certamente, ela renunciaria para todo o sempre a glória
da maternidade, de modo a não ter de dar à luz numa atmosfera em que se
respira apenas destruição e morte. Pois se vier a se tornar uma mãe
será para dar à criança o que de mais profundo e melhor o seu ser pode
produzir. Crescer com a criança é o seu lema; ela sabe que apenas dessa
maneira poderá ajudar a construir uma masculinidade e feminilidade
verdadeiras.

Ibsen deve ter tido em mente a imagem de uma mãe livre, quando, numa
jogada de mestre, retratou a sra.\,Alving. Ela era a mãe ideal porque
havia superado o casamento e todos os seus horrores, porque havia
quebrado as suas correntes e libertado o seu espírito para voar até que
retornasse sob a forma de uma personalidade regenerada e forte.
Infelizmente, era tarde demais para resgatar a alegria da sua vida, o
seu Osvaldo; mas não tarde demais para perceber que o amor em liberdade
é a única condição para uma vida bela. Aqueles que, como a sra.\,Alving,
pagaram por seu despertar espiritual com sangue e lágrimas, repudiam o
casamento como uma imposição, como uma zombaria superficial e vazia.
Eles sabem que quer o amor dure apenas um breve lapso de tempo ou por
toda eternidade, é o único alicerce criativo, inspirador, capaz de fazer
ascender a uma nova raça, a um novo mundo.\footnote{Aqui, Goldman se remete à então polêmica peça de Ibsen, \textit{Espectros}, de 1881.}

No nosso estado atual de pigmeus, o amor é realmente estranho à maioria
das pessoas. Mal"-entendido e deliberadamente evitado, raramente cria
raiz; ou se cria, logo definha e morre. Sua fibra delicada não pode
suportar o estresse e a pressão do nosso esfacelamento diário. Sua alma é complexa
demais para se adaptar à trama viscosa do nosso tecido social. Chora e
geme e sofre junto com aqueles que dele têm necessidade, mas que carecem
da capacidade de ascender aos cumes do amor.

Algum dia, homens e mulheres irão elevar"-se, irão atingir o topo da
montanha, irão se encontrar uns com os outros grandes, livres e fortes,
prontos para receber, compartilhar e aquecer"-se nos raios dourados do
amor. Que fantasia, que imaginação, que gênio poético podem ser antevistos, ainda que aproximadamente, como as potencialidades dessa força na vida
dos homens e mulheres. Se o mundo, algum dia, vir a dar à luz ao verdadeiro
companheirismo e união, não o casamento, mas o amor será o pai.

\chapterspecial{Mary Wollstonecraft}{Vida trágica e luta apaixonada pela
liberdade\footnote{Texto não publicado em vida, a data de 1911 é
  aproximada. Há quem defenda que diga respeito não apenas a um retrato de
  Mary Wollstonecraft, mas também um autorretrato da própria Goldman.}}{}
\markboth{Mary Wollstonecraft}{}

Os pioneiros do progresso humano são como gaivotas: eles contemplam
novas costas, novas esferas de pensamento desafiador, enquanto os seus
companheiros de viagem veem apenas a extensão interminável das águas.
Eles enviam alegres saudações às terras distantes. A fé intensa, ansiosa
e ardente trespassa as nuvens de dúvidas, porque os ouvidos aguçados
desses arautos da vida são capazes de distinguir, em meio aos estrondos
enlouquecedores das ondas, a nova mensagem e o novo símbolo para a
humanidade.

A humanidade mesma não compreende o novo, enfadonha e inerte, coloca"-se
ante o pioneiro da verdade com desconfiança e ressentimento, como se
estivesse diante daquele que veio perturbar sua paz, aniquilar a
estabilidade de todos os seus hábitos e tradição.

Assim, os desbravadores são ouvidos apenas por poucos, porque eles não
seguirão as trilhas já trilhadas, e à massa falta força para seguir
em direção ao desconhecido.

Em conflito com todas as instituições de seu tempo, uma vez que não se
comprometerá com elas, é inevitável que essa guarda avançada se torne
alienígena para aqueles mesmos que desejam servir; que eles sejam
isolados, evitados e repudiados pelos parentes mais próximos e queridos.
Assim, a tragédia que todo pioneiro deve experimentar não é a da falta
de entendimento --- mas sim a que surge do fato de que, tendo visto novas
possibilidades para o progresso humano, não podem criar
raízes no que é antigo, e dado que o novo ainda está distante, eles se
tornam andarilhos e párias pela terra, buscando sem descanso coisas
que nunca encontrarão.

Eles são consumidos pelo fogo da compaixão e simpatia por todo
sofrimento e por todos os seus camaradas, não obstante sejam obrigados a
se afastar do seu entorno. Não é recomendado que esperem em algum
momento receber o amor que suas grandes almas anseiam, pois é esta a
penalidade de um grande espírito: que o que ele receba seja nada
comparado com o que dá.

Foi esse o destino e a tragédia de Mary Wollstonecraft. O que ela deu ao
mundo, àqueles a quem amou, elevou"-se muito acima da possibilidade média
de receber, e não poderia a sua alma desejosa e ardente se contentar com
as migalhas mesquinhas que caem da mesa estéril da vida mediana.

Mary Wollstonecraft veio ao mundo no tempo em que seu sexo estava sob o
regime da escravidão: pertencente ao pai, quando estava em casa, era
passado como mercadoria ao marido na ocasião do casamento. De fato, foi
num mundo estranho que Mary ingressou no dia 27 de abril de
1759, ainda que não muito mais estranho do que o nosso. Desse momento
memorável para cá, a raça humana sem dúvida progrediu, mas, ainda assim,
Mary Wollstonecraft é ainda muito inovadora, está muito à frente do
nosso próprio tempo.

Ela era uma das muitas crianças de uma família de classe média, cujo
chefe exerceu os seus direitos de mestre ao tiranizar sua esposa e
filhos e desperdiçar seu capital numa vida ociosa e festiva. Quem
poderia detê"-lo, o criador do universo? Como em muitas outras coisas,
tais direitos mudaram pouco, da época do pai de Mary para cá. A família
logo se viu em extrema necessidade, mas como poderiam meninas da classe
média ganhar a própria vida com todas as portas fechadas para elas? Elas
tinham apenas uma vocação e esta era o casamento. A irmã de Mary
provavelmente se apercebeu disso. Ela casou com um homem que não amava
para escapar da miséria da casa paterna. No entanto, Mary foi feita
de material diferente, um material tão bem tecido que não poderia caber
em ambientes grosseiros. O seu intelecto viu a degradação do seu sexo, e
a sua alma --- sempre em ardor contra todo o mal --- rebelou"-se contra a
escravidão da metade da raça humana. Ela decidiu se firmar sobre
os próprios pés. Esta decisão foi fortalecida pela sua amizade com Fannie Blood,
uma mulher que já havia dado o primeiro passo em direção à própria
emancipação, ao decidir trabalhar para garantir o próprio sustento.
Mas mesmo sem Fannie Blood que exerceu grande força espiritual na vida
de Mary, e mesmo que não houvesse o fator econômico, ela estava
destinada pela própria natureza a se tornar a iconoclasta dos falsos
deuses cujas normas o mundo lhe exigia que obedecesse. Mary era uma
rebelde nata, alguém que iria criar ao invés de se submeter a uma forma
que lhe fosse estabelecida.

Já foi dito que a natureza utiliza de vasta quantidade de material
humano para criar um único gênio. O mesmo vale para um bom e verdadeiro
rebelde, o verdadeiro pioneiro. Mary nasceu assim, não foi criada
através desse ou daquele incidente específico da sua vida. O tesouro da
sua alma, a sabedoria da sua filosofia de vida, a profundidade do seu
mundo de pensamento, a intensidade da sua batalha pela emancipação
humana e, especialmente, a sua luta incansável pela libertação do seu
próprio sexo estão, ainda hoje, tão à frente do entendimento médio, que
podemos de fato reivindicar para ela a condição de rara exceção que a
natureza criou não mais do que uma vez em um século. Como o falcão que
subiu aos céus para contemplar o sol e depois pagou por isso com sua
vida, Mary tragou o cálice da tragédia, posto que este é o preço da
sabedoria.

Muito se tem escrito e dito sobre esta formidável campeã do século \textsc{xviii}, mas o
assunto é excessivamente vasto e ainda está longe de ser esgotado. O movimento
atual das mulheres e, em especial, o movimento sufragista pode encontrar
na vida e luta de Mary Wollstonecraft muitos elementos capazes de
mostrar a inadequação de ganhos meramente externos para a libertação do
sexo feminino. Sem dúvida, muita coisa foi alcançada desde que Mary
esbravejou contra a escravidão econômica e política da mulher, mas teria
isto tornado a mulher livre? Acrescentou algo à profundidade do seu
ser? Trouxe alegria e satisfação à sua vida? A vida trágica de Mary
prova que os direitos econômicos e sociais das mulheres não podem
sozinhos preencher de modo suficiente a sua vida, como tampouco são
suficientes para preencher qualquer vida profunda, seja de um homem ou
de uma mulher. Não é verdade que um homem profundo e sensível --- não me
refiro aqui a meros machos --- difira de modo considerável de uma mulher
profunda e sensível. Ele também busca a beleza e o amor, a harmonia e o
entendimento. Mary percebeu isto, porque ela não se limitou ao seu
próprio sexo, ela exigiu liberdade para toda raça humana.

Para se tornar economicamente independente, Mary lecionou, em um
primeiro momento, em escolas e depois aceitou a posição de governanta
dos filhos mimados de uma dama mimada. No entanto, logo percebeu a sua
inaptidão para a condição de serva e decidiu se voltar a algo que lhe
permitisse viver, sem que, ao mesmo tempo, fosse arrastada para baixo
por isso. Ela aprendeu sobre a amargura e a humilhação que acompanham a
luta econômica. Não foi tanto a falta de conforto externo o que revoltou
sua alma, mas, sim, a ausência de liberdade interior que resulta da
pobreza e da dependência --- que a fez gritar: ``Como pode alguém
professar ser amigo da liberdade sem que veja que a pobreza é o maior
mal?''

Afortunadamente para Mary e para posteridade existia então, naquele
tempo, um espécime raro da humanidade, ainda em falta no século \textsc{xx},
sendo este o editor Johnson, um liberal ousado. Ele foi o primeiro a
publicar o trabalho de Blake, de Thomas Paine, de Godwin e de todos os
rebeldes do seu tempo sem nenhuma preocupação com o ganho material. Ele
também viu em Mary grandes possibilidades e a contratou como revisora,
tradutora e colaboradora da sua revista \textit{Analytical Review.} E fez
mais. Ele se tornou o seu mais dedicado amigo e conselheiro. De fato,
nenhum outro homem na vida de Mary foi tão leal e compreendeu tanto a
sua natureza difícil, como esse homem raro. Tampouco ela abriu a sua
alma tão sem reservas para outra pessoa como a abriu para ele. Conforme
escreveu num dos seus momentos de análise:

\begin{quote}
A vida é apenas uma piada. Eu sou um composto estranho de fraqueza e
resolução. Há certamente algum defeito grave em minha mente, meu coração
ingovernável cria a sua própria miséria. Porque fui feita assim, eu não
sei e até que possa formar alguma ideia do todo da minha existência,
devo me contentar em chorar e dançar como uma criança que anseia por um
brinquedo para se cansar dele assim que o pega. Devemos cada um de nós
usar um chapéu de bobo da corte, mas o meu, infelizmente, perdeu os seus
gongos e se tornou tão pesado que não posso deixar de achá"-lo
intoleravelmente problemático.
\end{quote}

Que Mary pudesse escrever assim sobre si mesma para Johnson indica que
deve ter havido uma bela camaradagem entre eles. De todo modo, foi
graças ao seu amigo que ela encontrou algum alívio para a sua luta
terrível. E encontrou também alimento intelectual. Os aposentos de
Johnson eram o ponto de encontro da elite intelectual de Londres. Thomas
Paine, Godwin, dr.\,Fordyce, o pintor Fuseli, e muitos outros reuniam"-se
lá para discutir todos os grandes assuntos do seu tempo.

Mary entrou na esfera deles e se tornou o centro dessa agitação
intelectual. Godwin relata uma noite que havia sido organizada para
ouvir Tom Paine, mas ao invés disso ouviu Mary Wollstonecraft, seus
poderes de conversação, como tudo o mais nela, inevitavelmente ficavam
no centro do palco.

Assim, Mary pôde elevar"-se no espaço com seu espírito a alcançar grandes
alturas. A oportunidade logo veio. O antigo campeão do liberalismo
inglês, o grande Edmund Burke, entregou"-se a um sermão sentimental
contra a Revolução Francesa. Ele havia conhecido a honesta Marie
Antoinette e lamentou muitíssimo quando ela caiu nas garras do furioso
povo parisiense. Seu sentimentalismo de classe média viu na maior de
todas as revoltas apenas a superfície e não os erros terríveis que o
povo francês suportou antes que fosse arrastado aos seus atos. Mas Mary
Wollstonecraft viu esses erros e a sua resposta ao poderoso Burke,
\textit{Uma reivindicação dos direitos dos homens}, é uma das mais
poderosas defesas dos oprimidos e deserdados já feita.

Foi escrito com grande fervor, posto que Mary acompanhou a revolução atentamente.
Sua força, seu entusiasmo e acima de tudo, a lógica e a clareza da sua
visão provaram que essa outrora professora primária possuía um cérebro
privilegiado e um coração que palpitava apaixonadamente. Que isso
pudesse emanar de uma mulher era como a explosão de uma bomba, algo que
nunca havia acontecido antes. Chocou o mundo e trouxe para Mary o
respeito e a afeição dos seus contemporâneos do sexo masculino. Eles não
tiveram dúvidas de que ela era não apenas uma igual, mas em muitos
aspectos superior à maioria deles.

\begin{quote}
Apesar de você se considerar um amigo da liberdade, pergunte ao seu
coração se não seria mais consistente intitular"-se campeão da
propriedade, adorador do ídolo do ouro que o poder configurou.

\textit{Segurança da propriedade} captura em poucas palavras a definição da
liberdade inglesa. Mas isto é suavizar, pois é apenas a propriedade dos
ricos que é assegurada, o homem que vive do suor da sua fronte não tem
amparo da opressão.
\end{quote}

Pense em que maravilhoso discernimento para uma mulher há mais de cem
anos atrás. Mesmo hoje há ainda poucos dentre os nossos chamados
reformistas, e certamente ainda menos dentre as mulheres reformistas,
que podem ver de modo tão claro quanto essa gigante do século \textsc{xviii}.
Ela compreendeu muito bem que mudanças meramente políticas não são
suficientes e não atacam em profundidade os males da sociedade.

Mary Wollstonecraft sobre a paixão:

\begin{quote}
Regular a paixão nem sempre é uma atitude sábia. Ao contrário, talvez
seja justamente essa uma das razões pela qual os homens têm um
julgamento superior e maior coragem do que as mulheres: eles dão livre
curso à sua grande paixão e, por perderem a si mesmos com mais
frequência, ampliam as suas ideias.\label{regular}

A embriaguez se deve mais à ausência de uma diversão melhor do que a
alguma tendência inata ao vício, o crime é comumente resultado de uma
vida superabundante.

A mesma energia que torna um homem num vilão ousado poderia tê"-lo
transformado em alguém útil à sociedade fosse a sociedade mesma bem
organizada.
\end{quote}

Mary não foi apenas uma intelectual, posto que, conforme suas próprias
palavras, ela possuía um coração ingovernável. Isto é: ela ansiava por
amor e afeição. Foi, portanto, natural que se deixasse levar pela beleza
e paixão do pintor Fuseli, mas seja porque ele não correspondeu ao seu
amor, ou porque lhe faltou coragem no momento crítico, Mary foi forçada
a passar pela sua primeira experiência de amor e dor. Ela certamente não
era o tipo de mulher que se joga em cima de um homem. Fuseli era uma
pessoa tranquila e facilmente seria levado pela beleza de Mary. Mas ele
tinha uma esposa e a pressão da opinião pública era demais para ele.
Seja como for, Mary sofreu intensamente e fugiu para a França de modo a
escapar dos encantos do artista.

Os biógrafos foram os últimos a entender o seu caso, de outro modo não
teriam feito tanto barulho por esse episódio com Fuseli, já que não
houve nada além disso. Tivesse sido o falastrão Fuseli tão livre quanto
Mary, provavelmente teria vivido tranquila e normalmente ao lado dela.
Mas lhe faltou coragem, e Mary, que era sexualmente faminta, não pôde
facilmente apagar os seus sentidos então despertos. Foi necessário um
interesse intelectual muito forte para trazê"-la de volta a si mesma. E
esse interesse ela encontrou nos agitados eventos da Revolução Francesa.

No entanto, foi antes do incidente com Fuseli que Mary acrescentou ao
seu \textit{Uma reivindicação dos direitos dos homens}, a
\textit{Reivindicação dos direitos da mulher}, um apelo à emancipação do
seu sexo. Não é que ela considerasse o homem responsável pela
escravização da mulher. Mary era muito grande e muito universal para
colocar a culpa em um único sexo. Ela enfatizou o fato de que a própria
mulher é um obstáculo ao progresso humano, porque insiste em ser um
objeto sexual ao invés de uma personalidade, uma força criativa na vida.
Naturalmente, ela reconheceu que o homem tem sido um tirano há tanto
tempo que ele se ressente de qualquer violação do seu domínio, contudo
advogou que era tanto pelo bem do homem quanto pelo da mulher que exigia
a liberdade econômica, política e sexual da mulher como a única solução
ao problema da emancipação humana. ``As leis referentes às mulheres
transformaram a união entre marido e mulher num completo absurdo; ao considerar o homem 
o único responsável, a mulher é reduzida a uma nulidade.''

A natureza foi certamente muito generosa quando criou Mary
Wollstonecraft. Não apenas a dotou de um cérebro prodigioso, mas lhe deu
grande beleza e charme. Também lhe deu uma alma profunda --- e profunda
tanto em alegria, quanto em tristeza. Mary estava, portanto, destinada a
se tornar presa de mais de uma paixão. Seu amor por Fuseli logo abriu
caminho para um amor mais terrível e intenso, o que teve maior força em
sua vida, que a atirou como um brinquedo desamparado e desprovido de
vontade nas mãos do destino.

Vida sem amor para um caráter como o de Mary é inconcebível, e foi a sua
busca e ânsia pelo amor que a lançou contra as rochas da inconsistência
e do desespero.

Enquanto estava em Paris, Mary conheceu, na casa de Thomas Pine, onde
era recebida como sua amiga, o vivaz, belo e americano típico
Gilbert Imlay. Se não fosse o amor de Mary por ele o mundo talvez
nunca tivesse conhecido esse cavalheiro. Não que ele fosse um homem
comum, caso o fosse Mary nunca poderia tê"-lo amado com a louca paixão
que quase destruiu a sua vida. Ele se distinguira na Guerra da
Independência dos Estados Unidos e escreveu uma coisa ou duas, mas no
geral ele nunca teria virado o mundo de cabeça para baixo. No entanto,
ele virou Mary de cabeça para baixo e a manteve em transe por um tempo
considerável.

A própria força da paixão dela por ele excluía a harmonia, mas haveria
alguma culpa no que diz respeito a Imlay? Ele ofereceu a ela tudo
o que lhe era possível, mas dada a sua fome insaciada de amor, ela
jamais poderia se contentar com pouco, daí a tragédia. Além disso, ele
era um viajante, um aventureiro, um desbravador dos corações femininos.
Estava possuído pelo desejo de desbravar, não podia descansar em paz
em qualquer que fosse o lugar, já Mary precisava de paz, ela precisava
do que nunca havia tido em família, um lar tranquilo e aconchegante. Mas
acima de tudo, ela precisava de amor, de um amor sem reservas,
passional. Imlay não podia dar"-lhe nada disso e a luta começou logo após
o sonho louco ter passado.

Imlay sempre estava muito distante de Mary, primeiramente, sob o
pretexto dos seus negócios. Ele não seria um americano se não negligenciasse o
amor pelos negócios. Suas viagens o levaram, como os alemães costumam
dizer, a outras cidades e a outros amores. Como um homem dotado de
direitos, era seu direito enganar Mary. O que ela suportou apenas
aqueles que conhecem por si mesmos a tempestade podem saber.

Ao longo de toda a gravidez do filho de Imlay, Mary ansiou com tristeza
pelo retorno do seu homem, implorou e suplicou, mas ele estava sempre muito ocupado.
O pobre sujeito não sabia que toda riqueza do mundo não poderia
equiparar"-se à riqueza do amor de Mary. O único consolo que ela
encontrou foi no trabalho. Ela escreveu \textit{A revolução francesa} sob
a influência desse drama aterrador. Perspicaz em suas observações,
viu mais profundamente do que Burke: por baixo do terrível número de
mortos, viu o que era ainda mais terrível, a contradição entre
pobres e ricos, todo o derramamento de sangue seria vão enquanto essa
contradição continuasse. Assim, escreveu: ``Se a aristocracia de
nascimento é destruída apenas para dar lugar aos ricos, temo que o estado de ânimo de um povo não poderá ser elevado através desse tipo de mudança. Tudo me diz que
são nomes, e não princípios, o que está sendo transformado.'' Ela
percebeu, enquanto estava em Paris, o que havia previsto no seu ataque a
Burke, que o demônio da propriedade já havia invadido os direitos
sagrados do homem.

Mesmo com todo o seu trabalho, Mary não pôde esquecer o seu amor. Depois
de uma luta inútil e amarga para trazer Imlay de volta, ela tentou o
suicídio. Falhou e, de modo a recuperar a sua força, viajou para Noruega
numa missão dada por Imlay. Ela se recuperou fisicamente, mas sua alma
estava machucada e cheia de cicatrizes. Mary e Imlay se encontraram várias
vezes, mas estavam apenas procrastinando o inevitável. Então veio o
golpe final. Mary soube que Imlay tinha vários casos e que a estava
enganando não tanto por malícia quanto por covardia.

Ela, então, deu o passo mais terrível e desesperado, jogou"-se no Tâmisa
depois de andar por horas, pois, com as roupas molhadas, ela supôs que
certamente se afogaria. Oh, que inconsistência, bradam os críticos
superficiais. Mas teria sido mesmo?

Na luta entre intelecto e paixão, Mary sofreu uma derrota. Ela era muito
orgulhosa e muito forte para sobreviver a um golpe tão terrível. O que
havia mais para ela, além de morrer?

O destino que já pregara tantas peças a Mary Wollstonecraft desejava que
isto acontecesse de uma outra forma. Trouxe"-a de volta para a vida e para a
esperança, mas apenas para matá"-la justo quando estivesse na sua porta.

Ela encontrou em Godwin, o primeiro representante do comunismo
anarquista, uma camaradagem doce e terna, não do tipo selvagem e
primitivo, mas do tipo tranquilo, maduro e aconchegante que acalma a
pessoa tal como uma mão fria sobre uma testa fervendo. Ao seu lado, ela
viveu de modo consistente com as suas ideias, em liberdade, cada um em
separado, compartilhando um com o outro o que podiam.

Novamente Mary se tornou mãe, não em estresse e em dor, mas em paz e
cercada de bondade. No entanto, o destino é tão estranho que Mary teve
de pagar com a sua vida pela vida da sua pequena menina, Mary Godwin.
Ela morreu em 10 de setembro de 1797, com apenas 38 anos. O resguardo da
primeira filha, ainda que sob as mais difíceis circunstâncias, foi mera
atuação ou, conforme escreveu à sua irmã, \textit{uma desculpa para ficar na
cama}. No entanto, aquele tempo trágico também exigiu a sua vítima.
Fannie Imlay morreu da morte que sua mãe falhou em encontrar. Ela
cometeu suicídio por afogamento, enquanto Mary Wollstonecraft Godwin
tornou"-se a esposa da mais doce cotovia da liberdade, Shelley.

Mary Wollstonecraft, intelectual, gênio, lutadora audaz dos séculos
\textsc{xviii}, \textsc{xix} e \textsc{xx}; Mary Wollstonecraft, mulher e amante, fadada à dor por
conta da riqueza mesma do seu ser. Apesar de todos os seus romances, ela
foi bastante solitária, como toda grande alma, teve de permanecer só --- sem
dúvida, é esta a penalidade da grandeza.

Sua coragem indomável em nome dos deserdados da terra a tornou estranha
ao seu próprio tempo e criou discórdia em seu ser, o que, por si só, é
responsável por sua tragédia terrível com Imlay. Mary Wollstonecraft
visava o mais alto cume das possibilidades humanas. Ela era muito sábia
e muito mundana para não ver a discrepância que causou o rompimento do fio da sua alma delicada, complicada --- a discrepância entre o mundo dos seus ideais e o seu universo amoroso.

Talvez tivesse sido melhor para ela morrer na sua primeira tentativa
de suicídio. Aquele que já provou a loucura da vida não poderá nunca mais
ajustar a si mesmo à uniformidade. Mas teríamos perdido muito com isso e
podemos ao menos nos reconciliar com o que ela deixou, e isso já é
muito. Não tivesse Mary Wollstonecraft escrito uma única linha, a sua
vida teria fornecido alimento para o pensamento. Mas ela deu as duas
coisas, portanto está entre os maiores do mundo, uma vida tão profunda,
tão requintada e bela como a dela complementa a humanidade.

\chapter[Causas e possível cura para o ciúme]{Causas e possível cura\break para o ciúme\footnote{Texto não publicado em vida, a data de 1912 é aproximada.}}
\markboth{cura para o ciúme}{}

Ninguém que seja realmente capaz de uma vida interior intensa e
consciente pode esperar escapar da angústia mental e do sofrimento.
Tristeza e, frequentemente, desespero por conta da, por assim dizer, eterna
adequação de todas as coisas são as mais persistentes companhias das
nossas vidas. Mas elas não caem sobre nós como se vindas de fora,
através de maus feitos desempenhados por pessoas particularmente más.
Elas são condicionadas pelo âmago do nosso próprio ser. Na verdade, elas
estão entrelaçadas em mil fios, ásperos e macios, com a nossa
existência.

É absolutamente necessário que atentemos para este fato, posto que as
pessoas que não conseguem se livrar da concepção de que seus infortúnios
se devem à perversidade dos seus companheiros nunca poderão superar o
ódio mesquinho e a malícia com os quais constantemente culpa, condena e
persegue os outros por algo que é inevitavelmente parte delas. Tais
pessoas não alcançarão as elevadas alturas do humanitarismo verdadeiro,
segundo o qual bem e mal, moral e imoral são apenas termos limitados à
encenação interior das emoções humanas sobre o mar humano da vida.

O filósofo \textit{além do bem e do mal}, Nietzsche, é no presente denunciado
como incitador do ódio nacionalista e das armas de destruição em massa;
mas apenas maus leitores e maus pupilos podem interpretá"-lo dessa
maneira. \textit{Além do bem e do mal} significa além da acusação, além do
julgamento, além do assassinato etc. \textit{Além do bem e do mal} coloca, diante dos nossos olhos, uma visão cujo pano de fundo é o da afirmação
individual combinada com a compreensão de que todos os outros são
dessemelhantes a nós, que são diferentes.

Sob esta perspectiva, não estou fazendo coro com as tentativas desajeitadas da
democracia de regular as complexidades do caráter humano por meio do
princípio da igualdade externa. A visão \textit{além do bem e do mal} aponta
para o direito a si mesmo, para o direito à própria personalidade. Tais
possibilidades não excluem a dor pelo caos da vida, mas excluem a
justiça puritana que se senta para julgar todos os outros desde que não
seja o \textit{juiz} em questão.

É autoevidente que o radical completo --- há muitos radicais meia boca, \label{radical}
como vocês devem saber --- deve aplicar este reconhecimento profundo do
humano às relações sexuais e amorosas. As emoções sexuais e o amor estão
entre as expressões mais íntimas, mais intensas e sensíveis do nosso
ser. Elas estão tão profundamente relacionadas com as características
físicas e psíquicas de cada indivíduo em particular que marcam cada caso
de amor de modo independente, diferente de qualquer outro. Em outras
palavras, cada amor é resultado das impressões e características das
duas pessoas envolvidas. Cada relação amorosa deve, dada a sua natureza
mesma, permanecer como um assunto absolutamente privado. Nem o Estado,
nem a Igreja, nem a moralidade ou outras pessoas devem se intrometer nela.

Infelizmente, não é isso o que acontece. A mais íntima relação está
sujeita a proscrições, regulamentações e coerção, ainda que esses
fatores externos sejam completamente alienígenas ao amor --- o que conduz
a contradições e conflitos infindáveis entre o amor e a lei.

O resultado disso é que nossa vida amorosa está imersa em corrupção e em
degradação. O \textit{amor puro}, tão aclamado pelos poetas, em face do
matrimônio, divórcio e disputas alienantes da atualidade, é de fato um
espécime raro. Quando o dinheiro, o status social e a posição são os
critérios do amor, a prostituição se apresenta como inevitável, ainda
que esteja coberta com o manto da legitimidade e da moralidade.

O maior dos males da nossa vida amorosa mutilada é o ciúme, comumente
descrito como o \textit{monstro de olhos verdes}\footnote{Expressão
  idiomática da língua inglesa para se referir ao ciúme.} que mente,
engana, trai e mata. A concepção popular é a de que o ciúme é inato e,
portanto, nunca pode ser erradicado do coração humano. Esta ideia é uma
desculpa conveniente para aqueles a quem falta habilidade e disposição
de se aprofundar nas relações de causa e efeito.

A angústia por um amor perdido, pelo fim de um caso de amor --- é de fato
inerente ao mais próprio do nosso ser. A dor emocional tem inspirado, ao
longo dos séculos, muitos poetas líricos sublimes, muitas visões
profundas e exultações poéticas, como as de um Byron, um Shelley, um
Heine e outros do tipo. Mas irá alguém comparar este tipo de sofrimento
com o que comumente é considerado ciúme? Esses sentimentos são tão
dessemelhantes quanto a sabedoria e a estupidez. Quanto o refinamento e
a vulgaridade. Quanto a dignidade e a coerção brutal. O ciúme é o
próprio reverso do entendimento, da simpatia e dos sentimentos
generosos. Nunca o ciúme acrescentou algo ao caráter, nunca tornou o
indivíduo grande e bom. O que realmente faz é torná"-lo cego com a fúria,
mesquinho com a suspeita e cruel com a inveja.

O ciúme, cujas contorções nós assistimos nas tragédias e comédias
matrimoniais, é invariavelmente um acusador unilateral e fanático,
convencido da sua própria justiça e da maldade, crueldade e culpa da sua
vítima. O ciúme nem mesmo tenta compreender. O seu único desejo é punir
e punir o mais severamente possível. Essa concepção está encarnada numa espécie de
código de honra, tal como era o caso do duelo ou da lei não escrita. Um
código sob o qual a sedução de uma mulher deve ser expiada com a morte
do sedutor. Mesmo quando a sedução não ocorreu, mesmo quando ambos se
renderam voluntariamente à urgência mais profunda, a honra é restaurada
apenas quando o sangue é derramado, seja do homem ou da mulher.

O ciúme é obcecado pelo sentimento de posse e de vingança. Está de
acordo com o espírito daquelas leis punitivas fundamentadas na noção
bárbara de que uma ofensa --- que, em geral, é resultado de alguma
injustiça social --- deve ser adequadamente punida ou vingada.

Um argumento muito forte contra o ciúme pode ser encontrado em dados
levantados por historiados como Morgan, Reclus e outros no que diz
respeito às relações sexuais dos povos primitivos. Qualquer um que
esteja familiarizado com o trabalho desses autores sabe que a monogamia
é um modo muito posterior de se relacionar sexualmente, que veio à tona
como resultado da domesticação e apropriação das mulheres e que, por sua
vez, criou o monopólio sexual e o inevitável sentimento de ciúme.

No passado, quando homens e mulheres misturavam"-se livremente sem
interferência da lei e da moralidade, não podia haver ciúme, uma vez que
este se baseia no pressuposto de que certo homem tem o monopólio sexual
exclusivo sobre certa mulher e \textit{vice"-versa.} No momento em que um
relacionamento amoroso ultrapassa este preceito sagrado, o ciúme declara
guerra. Sob tais circunstancias é ridículo dizer que o ciúme é algo
perfeitamente natural. Na verdade, é apenas o resultado artificial de
uma causa artificial, nada mais.

Infelizmente, não são apenas os casamentos conservadores que estão
saturados com a noção de monopólio sexual; as chamadas relações livres
também são suas vítimas. O que poderia ser tomado como prova para o argumento de que o ciúme é uma característica inata. Mas é
preciso ter em mente que o monopólio sexual tem sido transmitido de
geração a geração como um direito sagrado e a base da perfeição da
família e do lar. E uma vez que tanto a Igreja quanto o Estado
aceitaram o monopólio sexual como a única segurança dos laços
matrimoniais, ambos justificaram o ciúme como a arma de legítima defesa
para a proteção desse direito de propriedade.

Ainda que hoje seja verdade que muitas pessoas superaram a legalidade do
monopólio sexual, é também verdade que elas não superaram suas tradições
e seus hábitos. Assim, o \textit{monstro de olhos verdes} lhes torna tão
cegas quanto aos seus vizinhos conservadores no momento em que seus bens
materiais estão em jogo.

Um homem ou mulher livre e grande o suficiente para não interferir ou
fuçar naquilo de exterior que exerce atração sobre o seu amado,
certamente será desprezado pelos seus amigos conservadores e
ridicularizado pelos radicais. Será ainda depreciado como um degenerado
ou covarde; e, com frequência, condições materiais mesquinhas lhe serão
impostas. Em qualquer caso, estes homens e mulheres serão alvo de
fofocas brutais e piadas indecentes por nenhuma outra razão que a de
concederem à sua esposa, marido ou amantes o direito sobre os seus
próprios corpos e sobre suas expressões emocionais, sem que, em nome
disso, tenham de protagonizar cenas de ciúme ou ameaças violentas de
matar o intruso.

Há outros fatores que envolvem o ciúme: a presunção dos machos e a
inveja das mulheres. O macho em assuntos sexuais é um impostor, um
falastrão que sempre se gaba das suas façanhas e sucesso com as
mulheres. Ele insiste em encenar o papel do conquistador, uma vez que
lhe foi ensinado que as mulheres querem ser conquistadas, que elas amam
ser seduzidas. Sentindo"-se como o único galo do galinheiro, ou como um
touro que deve lutar com os chifres para ganhar a vaca, fica mortalmente
ferido em sua presunção e arrogância no momento em que um rival aparece
em cena --- uma cena que, mesmo entre os chamados homens refinados,
continua a ser a do amor sexual da mulher, que deve pertencer a um único
mestre.

Em outras palavras, a ameaça ao monopólio sexual em conjunto com a
vaidade ofendida dos homens são na vasta maioria dos casos os
antecedentes do ciúme.

No que diz respeito à mulher, as causas do ciúme são, invariavelmente, a
sua insegurança econômica e a das crianças, e a inveja mesquinha de toda
outra mulher que ganhe em graça ante os olhos do seu mantenedor. De modo a
ser justa com as mulheres, é preciso dizer que, ao longo de séculos, os seus
atrativos físicos eram o único atributo que poderia negociar, portanto era
forçoso que ela sentisse inveja do charme e do valor de outras mulheres,
como se se tratasse de uma ameaça à sua preciosa propriedade.

O aspecto grotesco desse assunto todo é que homens e mulheres
normalmente se tornam violentamente ciumentos com pessoas que, na
realidade, não lhes importam muito. Não é, portanto, o seu amor\label{ciume}
ultrajado, mas a sua vaidade ultrajada e inveja que gritam contra esse
\textit{erro terrível}. Muito provavelmente a mulher nunca amou o homem que
ela agora suspeita e espia. Muito provavelmente ela nunca fez nenhum
esforço para manter o seu amor. Mas no momento em que uma concorrente se
aproxima, ela começa a valorizar a sua propriedade sexual e, para a sua
defesa, nenhum meio é desprezível ou cruel em demasia.

Obviamente, portanto, o ciúme não é resultado do amor. Na verdade, se
fosse possível investigar a maioria dos casos de ciúme, provavelmente
seria descoberto que quanto menos as pessoas estão imbuídas de um grande
amor, mais violento e desprezível será o seu ciúme. Duas pessoas que
estão ligadas por uma harmonia interior e um sentimento de unicidade não
têm medo de prejudicar a sua confiança e segurança mútuas se um ou o
outro se sentirem atraídos por algo exterior ao relacionamento, como não
irão terminar a sua relação numa inimizade vil, como ocorre
frequentemente com diversas pessoas. Muitos não são capazes, nem se deve
esperar isso deles, de acolher a escolha do amado no interior da
intimidade de suas vidas, mas isto não dá a nenhum dos dois o direito de
negar a necessidade da atração.

Como discutirei sobre variedade sexual e monogamia daqui a duas semanas,
não vou me debruçar neste momento sobre o assunto, exceto para dizer que
considerar as pessoas que podem amar mais de uma pessoa como perversas
ou anormais é ser, na verdade, muito ignorante. Já discuti algumas
causas do ciúme e devo acrescentar a essas a instituição casamento que a
Igreja e o Estado proclamam como o laço \textit{até que a morte vos separe}.
Isto é aceito como o modo ético de viver corretamente e agir
corretamente.

Com o amor, em toda a sua variedade e mutabilidade, assim agrilhoado e
limitado, não é de admirar que o ciúme surja. O que mais, se não
mesquinharia, maldade, suspeita e rancor poderia surgir quando um homem e
sua esposa são oficialmente unidos sob a fórmula \textit{de agora em diante
vocês são um só corpo e espírito}? Observe-se, apenas, qualquer casal que
esteja amarrado dessa maneira, dependente um do outro para qualquer
pensamento e sentimento, sem qualquer interesse ou desejo exterior, e
pergunte a si mesmo se uma tal relação poderia não se tornar odiosa e
insustentável com o tempo.

De um modo ou de outro os grilhões são quebrados e, como as
circunstâncias que conduziram a eles são geralmente baixas e degradantes, não
é de surpreender que incitem as motivações e características humanas
mais miseráveis e vis.

Em outras palavras, as interferências legais, religiosas e morais são os
pais da desnaturalização da nossa vida sexual e amorosa atual e foi por
conta disto que o ciúme cresceu. É o chicote que açoita e tortura os
pobres mortais por causa de sua estupidez, ignorância e preconceito.

Mas ninguém precisa justificar a si mesmo sob o argumento de ser uma
vítima dessas condições. É uma grande verdade que nós todos sofremos sob
o fardo de arranjos sociais iníquos, sob a coerção e a cegueira moral.
Mas não somos justamente nós, indivíduos conscientes, aqueles que
anseiam trazer a verdade e a justiça aos assuntos humanos? A teoria de
que o homem é um produto das suas condições apenas o levou à indiferença
e a uma aquiescência preguiçosa no que diz respeito a essas condições
mesmas. Ainda que todos saibam que adaptar"-se a um modo de vida injusto
e doentio apenas fortalece a injustiça e a doença, o ser humano,
considerado a coroa de toda criação, dotado da capacidade de pensar, ver
e, acima de tudo, de aplicar seu poder de iniciativa, torna"-se cada vez
mais fraco, mais passivo e mais fatalista.

Não há nada mais terrível e fatal do que alguém revolver os órgãos
vitais daquele que ama e de si mesmo. Esse tipo de atitude pode tão
somente ajudar a partir algum dos delicados fios do afeto ainda inerentes à
relação para finalmente chegar até o último fosso, justo o que o ciúme
tenta impedir, a saber, a aniquilação do amor, da amizade e do respeito.

O ciúme é um meio realmente precário para assegurar o amor, embora seja
um meio muito eficaz para destruir a autoestima. Pois pessoas ciumentas,
tal como os viciados em drogas, descem até o nível mais baixo de modo
que, no final, só podem inspirar nojo e aversão.

A angústia oriunda da perda ou da não reciprocidade no amor entre
pessoas que são capazes de pensamentos elevados e delicados nunca as
transformará em grosseirões. Aqueles que são sensíveis e bons têm apenas
de perguntar a si mesmos se eles podem tolerar alguma relação que seja
obrigatória, e um enfático não será a resposta. Mas a maioria das
pessoas continua a viver uma perto da outra, embora tenham, há muito,
cessado de viver uma com a outra --- uma vida fértil o suficiente para a
atividade do ciúme, cujos métodos vão desde violar a correspondência
privada até o assassinato. Comparado a esses horrores, abrir"-se para o
adultério parece um ato de coragem e libertação.

Um forte escudo contra a vulgaridade do ciúme é considerar que marido e
esposa não são um só corpo e um só espírito. São dois seres humanos, com
temperamentos, sentimentos e emoções diferentes. Cada um é um pequeno
cosmos em si mesmo, absorvido em seus próprios pensamentos e ideias. É
glorioso e poético que esses dois mundos possam se encontrar com
liberdade e igualdade. Mesmo que dure pouco tempo, terá valido a pena.
Mas, a partir do momento que dois mundos são forçados a ficar juntos,
toda a beleza e fragrância cessa, e nada, a não ser ervas daninhas,
prospera. Qualquer um que compreenda esse truísmo considerará o ciúme
como abaixo de si e não permitirá que este, tal como a espada de
Dâmocles,\footnote{A expressão \textit{espada de Dâmocles}, oriunda de uma antiga parábola moral da Antiguidade, popularmente passou a ser utilizada para indicar a ansiedade e medo dos poderosos de terem o seu poder usurpado.} seja pendurado sobre a sua cabeça.

Todos os amantes fazem muito bem ao deixar as portas do seu amor
completamente abertas. Quando o amor pode ir e vir sem medo de se
encontrar com um cão de guarda, o ciúme dificilmente cria raiz, porque
logo aprenderá que onde não há fechaduras e chaves, tampouco há lugar
para suspeita e desconfiança, os dois elementos sobre os quais o ciúme
cresce e prospera.

\chapter{Vítimas da moralidade\footnote{Texto originalmente publicado na revista
  anarquista \textit{Mother Earth}, em 1913.}}
\markboth{Vítimas da moralidade}{}

Não faz muito tempo que participei de um encontro organizado por Anthony
Comstock, que há quarenta anos tem sido o guardião da moral americana.
Nunca ouvi antes, de qualquer palanque, uma divagação mais incoerente e
ignorante.

A questão que se apresentou para mim, ao ouvir a conversa lugar"-comum e
fanática desse homem, foi: ``Como é possível que alguém tão limitado e
sem inteligência exerça o poder de censor e ditador sobre uma nação
supostamente democrática?'' É verdade que Comstock tem a lei para
respaldá"-lo. Quarenta anos atrás, quando o puritanismo era inclusive
mais desenfreado do que hoje, capaz de apagar completamente qualquer luz
de razão e progresso, Comstock conseguiu, através de maquinações
obscuras e conspiração política, fazer passar uma lei que lhe deu
controle absoluto do \textit{Post Office Departament}, o departamento responsável pelas agências dos
correios --- um controle que se
mostrou desastroso à liberdade de imprensa, assim como ao direito de
privacidade do cidadão americano.\footnote{Mesmo antes da promulgação da
  lei de 1873 que ficou conhecida pelo seu nome, Anthony Comstock, já
  atuava na condição de agente da \textit{\textsc{ymca}'s Society for the
  Suppression of Vice} da cidade de Nova York --- cuja tarefa era
  censurar o material \textit{obsceno} enviado pelos correios e,
  eventualmente, prender os seus remetentes e destinatários. Numa
  determinada ocasião, ainda no seu primeiro ano de trabalho em 1872,
  Comstock chegou a prender, de um lance, quarenta homens. Em realidade,
  a lei de 1873 é uma correção às \textit{brechas} supostamente deixadas pela
  lei de 1872 (por sua vez uma revisão de uma lei federal de 1865) que
  justamente tinha por objetivo de interditar o comércio e postagem,
  mesmo se de caráter privado, de material considerado \textit{obsceno}. Para
  Comstock, a lei de 1872 era insuficiente porque não incluía sob a
  insígnia de \textit{obsceno}, propagandas de livros de conteúdo subversivo,
  além de não determinar de modo claro os meios através dos quais
  deveriam ser executadas as penalidades. Com a promulgação da lei,
  Comstock se tornou agente especial do \textit{Post Office Department},
  dotado de todo poder e instrumentos para determinar o que era ou não
  material obsceno e, por conseguinte, para punir os envolvidos.}

Desde então, Comstock invadiu os aposentos privados das pessoas,
confiscou a sua correspondência pessoal, assim como trabalhos
artísticos, e estabeleceu um sistema de espionagem e suborno que
deixaria a Rússia envergonhada. Seja como for, as leis não dão conta de
explicar o poder de Anthony Comstock. Há algo além, ainda mais terrível
do que a lei. É a própria estreiteza do espírito puritano, tal como
representado nas mentes estéreis das \textit{Young"-Men"-and"-Old"-Maid's
Christian Union},\footnote{Aqui Goldman
  faz uma brincadeira maldosa, pois ao invés de se referir à
  \textit{Young Women's Christian Union} --- em tradução livre, \textit{União das Jovens Mulheres Cristãs} ---, ela substitui a expressão por \textit{Young-Men-and-Old-Maid}, isto é, \textit{Jovens-Homens-e-Solteironas}.} \textit{Temperance Union}, \textit{Sabbath Union},
\textit{Purity League} etc. Um espírito que é absolutamente cego às mais
simples manifestações da vida; representante, portanto, de toda
estagnação e decadência. Tal como nos dias que precederam a Guerra Civil
Americana, esses fósseis velhos lamentam a
terrível imoralidade do nosso tempo. Ciência, arte, literatura, drama
estão à mercê de uma censura intolerante munida de recursos legais, o
que tem como resultado que a América, apesar de todos as suas
declarações presunçosas em nome do progresso e da liberdade, ainda está
imersa no provincialismo mais denso.
%{[}\textit{ante"-bellum days}{]}

As menores nações da Europa podem se gabar de possuir uma arte livre dos
grilhões da moralidade, uma arte que tem a coragem de retratar os
grandes problemas sociais de nosso tempo. Com a faca afiada da análise
crítica, corta cada úlcera social, cada erro, exigindo mudanças
fundamentais e a transvaloração dos valores aceitos.

Sátira, sagacidade, humor, assim como os modos de expressão mais sérios
e intensos são empregados para pôr a nu as mentiras das nossas
convenções sociais e morais. Na América, procurar"-se"-ia em vão por tal
meio de expressão, mesmo porque qualquer tentativa nesse sentido está
interditada pelo rígido regime levado a cabo pelo ditador moral e sua
panelinha.

O que de mais próximo disso temos por aqui, são os nossos
\textit{muckrakers},\footnote{\textit{Muckraker} era a expressão utilizada
  para designar membros de um grupo de jornalistas e escritores que
  se concentraram em elaborar matérias jornalísticas (e obras
  literárias) para denunciar a corrupção nas relações entre governo e
  grandes indústrias, além de outras instituições de prestígio. Boa
  parte das suas denúncias exerceu papel relevante (no sentido de
  pressionar) na elaboração e promulgação de projetos de lei, como, por
  exemplo, os que fortaleciam a proteção de trabalhadores e
  consumidores. Em geral, os \textit{muckrakers} são historicamente
  alocados na chamada era progressiva dos Estados Unidos que vai dos
  1890 até a entrada do país na Primeira Guerra Mundial, em 1917.} que,
sem sombra de dúvida, prestaram um ótimo serviço em termos econômicos e
sociais. Quer os \textit{muckrakers} tenham ou não ajudado a mudar
efetivamente as condições, ao menos eles arrancaram a máscara do rosto
mentiroso dessa nossa sociedade arrogante e convencida.

Infelizmente, a mentira da moralidade ainda persiste na perseguição,
dotada de toda a pompa, já que ninguém se atreve a mexer com o santo dos
santos. É certo, porém, que nenhuma outra superstição é tão prejudicial
ao crescimento e tão dotada do poder de debilitar e paralisar as mentes
e corações do povo, quanto a superstição da moralidade.

O aspecto mais patético, e de certo modo desencorajador, dessa situação
toda é certo elemento presente nos liberais e mesmo nos radicais, enfim,
presente nos homens e mulheres aparentemente livres dos fantasmas
religiosos e sociais. Pois, ante o monstro da moralidade, eles estão tão
prostrados quanto os mais devotos --- o que é uma prova adicional da
profundidade em que o verme da moralidade se introduziu no sistema de
suas vítimas e quão amplas e rigorosas devem ser as medidas para
expulsá"-lo mais uma vez.

Não é preciso dizer que a sociedade é obcecada por mais de um tipo de
moral. É um fato que toda instituição da atualidade tem seu próprio
padrão moral. Seja como for, as instituições não poderiam se manter não\label{ref8}
fosse pela religião, que funciona como um escudo, e pela moralidade, que
funciona como uma máscara. Isso explica o interesse que os exploradores
ricos têm tanto na religião quanto na moralidade. Os ricos pregam,
promovem e financiam ambas, como um investimento que gera excelentes
retornos. Por meio da religião, eles paralisaram o entendimento do povo,
assim como a moralidade escraviza o espírito. Em outras palavras, a
religião e a moralidade são um chicote muito mais eficiente, do que o
porrete e o revolver, para manter o povo submisso.

Para ilustrar: a moralidade da propriedade estabelece que essa
instituição é sagrada. Ai de quem se atrever a questionar a santidade da
propriedade ou pecar contra ela! Mesmo que todos saibam que a
propriedade é um roubo, que representa o esforço acumulado de milhões de
pessoas que são desprovidas de propriedades. E, o que é mais terrível,
quanto mais assolada pela pobreza estiver a vítima da \textit{moralidade da
propriedade}, maior respeito e temor ela terá por esse senhor. É, por
isso, que ouvimos de pessoas progressistas, até mesmo dos chamados
trabalhadores com consciência de classe, a condenação moral de métodos
como a sabotagem e a ação direta, justamente porque eles atingem a
propriedade.

Na verdade, se as próprias vítimas estão tão cegas pela \textit{moralidade da
propriedade}, o que se pode esperar dos seus senhores? Ao que parece, já
passou da hora de encarar o fato de que enquanto os trabalhadores não
tiverem perdido o respeito pelo instrumento que lhes escraviza
materialmente, a eles estão vedadas quaisquer esperanças.

%\asterisc

O que, porém, me preocupa mais seriamente é o efeito da moralidade sobre
as mulheres. Esse efeito é tão desastroso e tão paralisante que muitas
das mais progressistas dentre as minhas irmãs nunca o superaram
completamente.

É a moralidade que condena a mulher à posição de celibatária, prostituta
ou ainda de chocadeira irresponsável e imprudente de um sem"-número de
crianças infelizes.

Primeiramente, no que diz respeito ao celibato, ele torna o humano
semelhante a uma planta sedenta e murcha. Quando ainda uma flor jovem e
bela, ela se apaixona por um rapaz respeitável. Mas a moralidade decreta
que, salvo o caso em que ele possa se casar com a jovem, ela nunca deve
conhecer o arrebatamento do amor, o êxtase da paixão, que atinge o seu
cume na relação sexual. O jovem respeitável pode até estar disposto a se
casar, mas a moralidade da propriedade, da família e todas as demais
moralidades sociais decretam que ele primeiro deve fazer fortuna,
economizar o suficiente para providenciar um lar e ser capaz de
sustentar a sua família. Os jovens são, assim, obrigados a esperar, não
raro, por muitos e longos cansativos anos.

Enquanto isso, o jovem respeitável, excitado com o relacionamento e
contato diário com sua namorada, procura uma saída para a sua natureza em
troca de dinheiro. Em 99 por cento dos casos, ele será
infectado, de modo que quando estiver materialmente apto a se casar,
infectará a esposa e os possíveis filhos. E o que dizer da jovem flor,
com todas as suas fibras ardendo com o fogo da vida, com todo o seu ser
demandando amor e paixão? Ela não tem saída --- o que se manifesta em
dores de cabeça, insônia, histeria; torna"-se amargurada, encrenqueira, e
logo se transforma num ser desbotado, murcho e sem alegria, um estorvo
para ela mesma e para todos os outros. Não é de admirar que Stirner
preferisse a \textit{grisette}\footnote{Expressão utilizada para designar
  jovens que complementavam a renda com a prostituição.} à donzela, cuja
virtude a torna amarga e ressequida.

Não há nada mais patético, mais terrível, do que essa vítima embolorada
pela moralidade igualmente embolorada. Isso se aplica com força ainda
maior às jovens da classe média alta do que às do povo. Por conta da
necessidade econômica, as últimas são arremessadas na selva da vida
desde a mais tenra idade; elas crescem, lado a lado, com seus
companheiros homens seja na fábrica, na oficina, ou nas brincadeiras e
festas. O resultado é uma expressão mais normal dos seus instintos
físicos. Além disso, os rapazes e moças do povo não são moldados de modo\label{tradicao}
tão inflexível pelos fatores externos, e frequentemente se lançam ao
chamado do amor e da paixão, independentemente dos costumes e tradição.

Não obstante, a garota de classe média, nervosa e hipersexualizada,
protegida na estreiteza do confinamento das tradições familiares e
sociais, guardada por milhares de olhos, temerosa de sua própria sombra, 
tem de voltar o desejo ardente do seu ser mais profundo, o desejo pelo
homem ou por filhos, para gatos, cães, canários ou estudos
bíblicos. Tal é a cruel máxima da moralidade, que diariamente exclui o
amor, a luz e a alegria da vida de inúmeras vítimas.

Agora, passemos à prostituta. Apesar das leis, decretos, perseguições e
prisões; apesar da segregação, dos registros, das cruzadas viciosas e
outros dispositivos semelhantes, a prostituta é o verdadeiro fantasma da
nossa época. Ela varre as planícies como se fosse um incêndio, atingindo
cada canto da vida, devastando, destruindo.

No final das contas, ela está retribuindo, numa escala infinitamente pequena,
a maldição e os horrores que a sociedade disseminou no seu caminho.
Exaurida pela perambulação ao longo de eras, perseguida e obrigada a
peregrinar de um lado para o outro à mercê de todos, ela é, não
obstante, a Nêmesis dos tempos modernos, o anjo vingador a empunhar
impiedosamente a espada de fogo. Não tem ela o homem em seu poder? E,
através dele, o lar, a criança, a humanidade. É dessa maneira, portanto,
que ela mata, embora seja ela quem é assassinada do modo mais brutal. O
que foi que a criou? De onde ela vem? Da moralidade, da moralidade que é
impiedosa nas suas atitudes para com as mulheres. Uma vez que a mulher
se atreva a ser ela mesma, a ser honesta para com a sua natureza, para
com a vida, não há retorno: ela é ejetada da jurisdição e proteção da
sociedade. A prostituta se torna vítima da moralidade, assim como a
solteirona ressequida e murcha também é sua vítima. Mas a prostituta é
ainda vitimada por outras forças, principalmente pela \textit{moralidade da
propriedade}, que obriga a mulher a se vender como mercadoria sexual por
um dólar a cada vez fora do casamento, ou por \textsc{us}\$ 15 por
semana, quando no solo sagrado do matrimônio. Essa última condição é sem
dúvida mais segura, mais respeitada, mais reconhecida, mas, entre as
duas formas de prostituição, a garota da rua é a menos hipócrita, a
menos degradada, já que o seu comércio não se vale da máscara piedosa da
hipocrisia; e, ainda assim, não obstante, é ela quem é perseguida,
extorquida, ultrajada e banida pelos mesmos poderes que a criaram: pelo
financista, pelo padre, pelo moralista, pelo juiz, pelo carcereiro e
pelo policial e, não podemos esquecer, por sua irmã bem salvaguardada,
respeitosamente virtuosa --- que é a mais implacável e brutal na sua
perseguição à prostituta.

A moralidade e sua vítima, a mãe --- que imagem terrível! Existe de fato
algo mais terrível, mais criminoso do que a glorificada e sagrada função
da maternidade? A mulher, física e mentalmente inapta para ser mãe, mas
condenada a procriar; a mulher, taxada economicamente em cada centelha
da sua energia, mas forçada a procriar; a mulher, amarrada a um homem
que detesta, cuja visão a enche de horror, mas moldada para procriar; a
mulher, cansada e exaurida pelo processo de gestação, mas coagida a
procriar, cada vez mais e mais. Que coisa hedionda, essa tão elogiada
maternidade! Não é de admirar que milhares de mulheres se arrisquem pela
via da mutilação, e prefiram até a morte a essa maldição cruelmente
imposta pelo fantasma da moralidade. Cinco mil mulheres são anualmente
sacrificadas no altar desse monstro, que não tolera prevenção, mas que
supostamente seria uma cura para o aborto. Cinco mil guerreiras na
batalha pela sua liberdade física e espiritual, assim como o são muitos
milhares mais que preferiram se tornar aleijadas e mutiladas, ao invés
de dar à luz numa sociedade erguida sobre a decadência e destruição.

Será que é por não querer assumir responsabilidades, ou por não ter amor
aos seus filhos, que a mulher moderna é levada aos meios mais drásticos
e perigosos para evitar a gravidez? Apenas as pessoas dotadas de uma
mentalidade extremamente superficial e intolerante podem fazer tal
acusação. Não fosse isso, saberiam que a mulher moderna
simplesmente se tornou consciente da raça humana, sensível às
necessidades e direitos da criança, assim como à unidade da raça, e que,
portanto, a mulher moderna tem um senso de responsabilidade e
humanidade, que são completamente diferentes dos da sua avó.

Com a guerra econômica a destruir todo o seu entorno, com os conflitos,\label{entorno}
a miséria, o crime, as doenças e a insanidade postas bem diante dos seus
olhos, com inúmeras crianças pequenas trituradas e destruídas, como
poderia uma mulher autoconsciente e consciente da própria humanidade vir
a se tornar mãe? A moralidade não pode responder a essa pergunta. Só
pode doutrinar, coagir ou condenar --- e quantas mulheres são fortes o
suficiente para enfrentar essa condenação, para desafiar os dogmas
morais? Poucas, de fato. E, assim, terminam por preencher as fábricas,
os reformatórios, as casas de repouso, as prisões, os manicômios, ou
simplesmente morrem na tentativa de impedir uma gravidez indesejada. Oh,
a Maternidade, que crimes são cometidos em vosso nome! Que multidão é
colocada aos seus pés, moralidade, destruidora da vida!

Felizmente, a aurora emerge do caos e da escuridão. A mulher
está despertando, ela está se livrando do pesadelo da moralidade; ela já
não ficará acorrentada. No seu amor pelo homem, ela não está mais
preocupada com o conteúdo de sua carteira, mas com a riqueza da sua
natureza, que por si só é a fonte da vida e da alegria. Tampouco ela
precisa da sanção do Estado. Seu amor é sanção suficiente para ela.
Assim, ela pode se entregar ao homem de sua escolha, como as flores
estão entregues ao orvalho e à luz, em liberdade, beleza e êxtase.

Através da sua consciência renascida como unidade, como personalidade,\label{renascida}
como construtora da raça humana, ela se tornará mãe apenas se desejar o
filho e se puder dar à criança, mesmo antes de seu nascimento, tudo o
que sua natureza e intelecto sejam capazes de alcançar: harmonia, saúde,
conforto, beleza e, acima de tudo, compreensão, reverência e amor, que é
o único solo verdadeiramente fértil para uma nova vida, para um novo
ser.

A moralidade não pode aterrorizar quem se elevou além do bem e do mal. E
embora possa continuar devorando suas vítimas, essa mesma moralidade é
completamente impotente diante do espírito moderno, que brilha em toda a
sua glória na testa do homem e da mulher libertos e sem medo.

\chapter[Os aspectos sociais do controle de natalidade]{Os aspectos sociais do controle\break de natalidade\footnote{Texto
  originalmente publicado na revista \textit{Mother Earth}, em 1916.}}
\markboth{controle de natalidade}{}\label{ref4}

Como já foi sugerido, para que um gênio seja criado, a natureza tem de
usar todos os seus recursos e demora uma centena de anos para levar a
cabo esta sua difícil tarefa. Se isto é verdade, a natureza leva ainda
mais tempo para criar uma grande ideia. Afinal, ao criar um gênio a
natureza se concentra em uma personalidade, enquanto uma ideia deve
eventualmente se tornar a herança de uma raça, devendo, por isso, ser
necessariamente mais difícil de moldar.

Faz apenas cento e cinquenta anos que um grande homem concebeu uma
grande ideia, Robert Thomas Malthus, o pai do controle de natalidade.
Que a raça humana tenha demorado tanto tempo para perceber a
grandiosidade desta ideia é apenas mais uma prova da morosidade da mente
humana. Não é possível entrar numa discussão detalhada acerca do mérito
da controvérsia colocada por Malthus, a saber, que a terra não é fértil
ou rica o suficiente para suprir as necessidades de uma raça
excessivamente abundante. Certamente, se olharmos por sobre as
trincheiras e campos de batalha da Europa, descobriremos que, em
certa medida, a sua premissa estava correta. No entanto, tenho certeza
de que se Malthus vivesse hoje, ele concordaria com todos os estudiosos
da sociedade e revolucionários no sentido de que se as massas continuam
a ser pobres e os ricos tornam"-se cada vez mais ricos, não é porque
esteja faltando fertilidade à terra e riqueza para suprir as
necessidades de uma raça abundante, mas porque a terra é monopolizada
nas mãos de poucos para a exclusão de muitos.

O capitalismo, que ainda estava de fraldas no tempo de Malthus, cresceu
desde então, tornando"-se um enorme e insaciável monstro. Ele ruge
através do apito e da máquina --- ``Mandem suas crianças para mim, vou
torcer os seus ossos; vou extrair o seu sangue; vou roubá"-las do seu
florescimento'' ---, pois o capitalismo tem um apetite insaciável.

É através da sua maquinaria destrutiva que o militarismo e o capitalismo
proclamam: ``Mandem os seus filhos para mim, irei treiná"-los e
discipliná"-los até que toda a humanidade tenha sido programada neles;
até que eles se tornem autômatos prontos para atirar e matar sob o
comando dos seus mestres.'' O capitalismo não pode se realizar sem o
militarismo e uma vez que as massas fornecem o material para ser destruído
nas trincheiras e campos de batalha, o capitalismo necessita de uma raça
numerosa.

Os chamados bons tempos são aqueles em que o capitalismo engole massas
de pessoas para depois, novamente, lançá"-las no tempo da \textit{depressão
industrial}. Essa massa humana supérflua que incha as fileiras de
desempregados e que representa a maior ameaça nos tempos modernos, é
designada pelos nossos políticos e economistas burgueses de margem de
trabalho.\footnote{Em \textit{O capital}, Marx irá criticar essa concepção da relação entre margem de trabalho e capitalismo, isto é, entre o desemprego estrutural e o capitalismo através da elaboração do conceito de \textit{exército industrial de reserva}.} Para eles, sob nenhuma circunstância deve a margem de trabalho
ser diminuída, caso contrário a sagrada instituição conhecida como
civilização capitalista será prejudicada. E assim, os economistas\label{margem}
políticos, em conjunto com todos financiadores do regime capitalista são
a favor de uma raça numerosa e excessiva e, portanto, opõem"-se ao
controle de natalidade.

No entanto, a teoria de Malthus contém muito mais verdade do que ficção.
Dado o seu caráter \textit{moderno}, já não se baseia em especulação, mas em
outros fatores relacionados e entrelaçados às profundas
mudanças sociais que estão acontecendo em todos os lugares.

Primeiro, há o aspecto científico, a contenda entre os mais eminentes
cientistas que nos dizem que um organismo sobrecarregado e subnutrido
não pode reproduzir prole saudável. Além dessa contenda dos cientistas,
estamos sendo confrontados com um fato terrível, agora inclusive
reconhecido pelas pessoas mais ignorantes, a saber: que uma procriação
indiscriminada e incessante da parte das massas sobrecarregadas e
subnutridas resultou no aumento de crianças deficientes, aleijadas e
infelizes. Este fato é tão alarmante que despertou os reformistas
sociais para a necessidade da criação de uma espécie de comissão de inquérito da mente\footnote{Em inglês é usado o termo \textit{mental clearing house}.} que pudesse apurar as causas e efeitos do aumento
no número de crianças aleijadas, surdas, idiotas e cegas. Por sabermos
que os reformistas só aceitam a verdade quando ela se tornou óbvia até
para o indivíduo mais tapado da sociedade, já não há necessidade de
discutir os resultados de uma procriação indiscriminada.

Em segundo lugar, há o despertar intelectual da mulher que desempenha um
papel não pouco significativo em benefício do controle de natalidade. Ao
longo das eras, a mulher vem carregando o seu fardo. Cumpre o seu dever
mil vezes maior do que o de qualquer soldado em campo de batalha.
Afinal, o trabalho do soldado é tirar vidas. Para isso, ele é pago pelo
Estado, elogiado por charlatões políticos e apoiado pela histeria
pública. Já a função da mulher é dar à luz e, em relação a isso, nem os
políticos, nem a opinião pública demonstraram, em qualquer momento, a
mínima disposição de retribuir a vida que a mulher tem dado. Ao longo
das eras, ela se encontra de joelhos ante o altar do dever imposto por
Deus, pelo capitalismo, pelo Estado e pela moralidade. Na atualidade,
porém, ela está acordando desse seu sonho de longa data. Ela está se
libertando do pesadelo do passado; ela se virou em direção à luz que
anuncia, com voz clara, que ela não mais fará parte do crime de trazer
ao mundo crianças infelizes apenas para serem moídas e transformadas em
pó pelas rodas do capitalismo e para serem dilaceradas em pedaços nas
trincheiras e campos de batalha. E quem haverá de dizer a ela que não
seja assim? Afinal de contas, é a mulher quem arrisca a sua saúde e
sacrifica a sua juventude para a reprodução da raça. Certamente, ela
deveria estar em posição de decidir quantas crianças deve trazer ao
mundo, se elas devem ser trazidas ao mundo junto ao homem que ama e
porque ela deseja a criança, ou se devem nascer do ódio e da aversão.

Ademais, é consenso entre os médicos mais sérios que a reprodução\label{saude}
constante da parte das mulheres resulta no que em termos leigos é
chamado \textit{problemas femininos}: uma condição lucrativa para médicos
inescrupulosos. Mas por qual razão a mulher exaure o seu organismo numa
gravidez perpétua?

É precisamente por conta dessa razão que as mulheres devem ter o
conhecimento que as possibilite se recuperar por um período de três a
cinco anos entre cada gestação, o que por si só lhes daria bem"-estar
físico e mental e a oportunidade de cuidar melhor das crianças já
existentes.

Mas não são apenas as mulheres que estão começando a perceber a
importância do controle de natalidade. Homens, especialmente da classe
trabalhadora, estão aprendendo a reconhecer que uma família numerosa é
como uma pedra amarrada em seu pescoço, deliberadamente impostas
pelas forças reacionárias da sociedade; porque uma família numerosa
paralisa o cérebro e embota os músculos das massas de trabalhadores.
Nada amarra mais os trabalhadores à servidão do que uma ninhada de
crianças e é exatamente isso que os oponentes do controle de natalidade
querem.

Como os vencimentos de um homem com uma família numerosa são miseráveis,
ele não pode arriscar nem mesmo um pouco. Permanece preso à
rotina, cedendo e se encolhendo diante de seu mestre, exclusivamente para
ganhar o que sequer dá para alimentar as suas muitas boquinhas. Ele não se
atreve a participar de organizações revolucionárias; ele não se atreve a
fazer greve; ele não se atreve a expressar uma opinião. As massas de
trabalhadores finalmente acordaram para a necessidade do controle de
natalidade como meio de libertar a si mesmas do jugo terrível e ainda
mais como um dos meios efetivamente capaz de fazer algo por aqueles que já existem, ao evitar que mais crianças venham ao mundo.

Por último, embora não menos importante, uma mudança na relação entre os
sexos, ainda que não abranja um número muito grande de pessoas, está se
fazendo sentir numa minoria considerável. Como no passado, para um
grande número de homens da atualidade, a mulher continua a ser mero
objeto, um meio para um fim; basicamente, um meio físico para um fim.
Mas há homens que querem mais do que isso de uma mulher; perceberam que
mesmo que todo homem se emancipe das superstições do passado, isso não
poderia mudar em nada a estrutura social caso a mulher não tome parte
com ele na grandiosa luta social. Lentamente, mas de modo firme, esses
homens têm aprendido que se a mulher desperdiça sua substância em
gestações eternas, resguardos e lavagens de fraldas, ela tem pouco tempo
para o que quer que seja. Muito menos, inclusive, para as questões que
absorvem e agitam o pai de seus filhos. Por exaustão física e estresse
nervoso, a mulher se torna um obstáculo no caminho do homem e
frequentemente seu inimigo mais amargo. É, portanto, para a sua própria
proteção e também em nome da companhia e amizade da mulher amada, que
muitos homens a querem livre da terrível imposição de reprodução
constante da vida, sendo, portanto, favoráveis ao controle de
natalidade.

Seja qual for o ângulo sob o qual a questão do controle de natalidade seja
considerada, trata-se do assunto mais imperioso dos tempos modernos e, em sendo
assim, não pode ser censurado por meio de perseguições, prisões ou de uma
conspiração do silêncio. Aqueles que se opõem ao movimento do controle
de natalidade afirmam que o fazem em nome da maternidade. Todos os
charlatães políticos matraqueiam sobre essa tal maternidade maravilhosa,
que quando examinamos mais de perto, vemos que essa mesma maternidade
tem, ao longo dos séculos, dedicado os seus rebentos a Moloch. Além
disso, uma vez que as mães são obrigadas a trabalhar pesado, ao longo de
horas, para ajudar no sustento das criaturas que trouxeram ao mundo sem
assim o desejar, essa conversa sobre maternidade nada mais é do que
hipocrisia. Dez por cento das mulheres casadas da cidade de Nova York
têm de ajudar a ganhar o pão. A maioria delas ganha o salário bastante
profícuo de 280 dólares por ano. Quem ousa falar das belezas da maternidade
diante de tal crime?

Mas e se considerarmos mães com melhores salários, o que pensaremos
delas? Não faz muito tempo que o nosso Conselho de Educação declarou que
não seria permitido a professoras que fossem mães continuar com o
magistério. Ainda que os antiquados cavalheiros em questão tenham sido
compelidos pela opinião pública a reconsiderar a sua decisão, é
absolutamente certo que quando uma professora se torna mãe, ano após
ano, ela perde a sua posição. Essa é a sina da mãe casada; e o que
acontece com a mãe solteira? Ou alguém tem dúvidas de que há milhares de
mães solteiras? Elas lotam as nossas lojas, fábricas e indústrias, estão
em todos os lugares, não por escolha, mas por necessidade econômica. Em
sua existência tediosa e monótona, a única alegria que resta é
provavelmente a da atração sexual que sem os métodos de prevenção
invariavelmente conduz a abortos. Em segredo e com pressa, milhares de\label{aborto}
mulheres são sacrificadas por causa de abortos realizados por médicos
charlatões e parteiras ignorantes. No entanto, os poetas e os políticos
louvam a maternidade. Não houve crime maior contra a mulher do que esse.

Nossos moralistas sabem disso, ainda que persistam na defesa da
reprodução indiscriminada de crianças. Eles nos dizem que limitar a
procriação é uma inclinação completamente moderna, porque a mulher
moderna está perdida moralmente e deseja fugir da responsabilidade. Em
resposta a isso, é necessário pontuar que a inclinação de controlar a
prole é tão velha quanto a raça humana. Temos como autoridade para essa
contenda o eminente médico alemão dr.\,Theilhaber, que compilou dados
históricos para provar que esta inclinação era comum aos hebreus,
egípcios, persas e muitas tribos dos índios americanos. O medo da
criança era tão grande que muitas mulheres se valiam do mais hediondo
dos métodos para não trazer uma criança indesejada ao mundo. Dr.\,Theilhaber enumera 57 métodos.\footnote{Felix A.
  Theilhaber (1884--1956) foi um médico e sexólogo judeu, militante da
  reforma sexual e da legalização do controle de natalidade e fundador
  de uma das primeiras clínicas em Berlim para o controle de natalidade.
  Nessa passagem, Goldman está se referindo à obra \textit{Das Sterile
  Berlin}, publicada em 1913.} Esses dados são de grande importância na
medida em que dissipam a superstição de que a mulher deseja necessariamente se tornar mãe
de uma família numerosa.

Não, não é que responsabilidade esteja faltando à mulher, mas é porque
ela tem muita responsabilidade que exige saber como prevenir a
concepção. Nunca na história do mundo, a mulher teve tanta consciência
da humanidade como ela tem hoje. Nunca antes ela esteve tão aberta a ver
na criança, não apenas na sua, mas em toda a criança, a unidade da
sociedade, o canal através do qual cada homem e cada mulher tem de
passar; o mais forte fator na construção de um novo mundo. É por esta
razão que o controle de natalidade se fundamenta em terreno sólido.

Foi"-nos dito que se a lei, contida nos estatutos, tiver resultado de
discussões sobre como prevenir um determinado crime, os fatores de
prevenção já não precisam ser discutidos. Em resposta a isso, quero
dizer que não é o movimento do controle de natalidade que deve ser
descartado, mas a própria lei. Afinal, é para isso que as leis servem,
para serem feitas e desfeitas. Como eles ousam exigir que a vida se
submeta às leis? Devemos ficar atados às leis pelo resto das nossas
vidas, apenas porque algum fanático ignorante, na sua própria limitação
de mente e coração, foi bem"-sucedido em aprovar uma lei na época em que
homens e mulheres eram prisioneiros de superstições morais e religiosas?
Eu compreendo perfeitamente por que juízes e carcereiros estão amarrados
às leis. Precisam dela para o seu sustento; em nome da sua função na
sociedade. Mas, às vezes, até juízes progridem. Chamo a atenção de
vocês, aqui, para a decisão a favor do controle de natalidade, do juiz
Gatens de Portland, Oregon. 

\begin{quote}
Parece"-me que o problema com o nosso
pessoal atualmente é que há muito puritanismo. Ignorância e puritanismo
sempre foram a corda amarrada no pescoço do progresso. Todos nós sabemos
que muitas coisas estão erradas na nossa sociedade; que estamos sofrendo
de muitos males, mas nós não temos coragem para levantar e admitir, e
quando uma pessoa chama nossa atenção para algo que já sabemos, fingimos
decência e nos sentimos indignados.'' 
\end{quote}

Este certamente é o problema com a
maioria dos nossos legisladores e com aqueles que se opõem ao controle
de natalidade.

Eu serei julgada nas \textit{Special Sessions} no dia 5 de
abril.\footnote{Como a legislatura e o processo penal dos Estados Unidos
  são específicos ao país, optou"-se por manter aqui o termo original.}
Não sei qual será o resultado, e não me importo.\footnote{Sob as leis de
  Comstock, fornecer informações sobre métodos contraceptivos era
  considerado crime, mesmo no caso de profissionais de saúde. A alegação
  era a de que esse tipo de informação promoveria a depravação sexual.
  Goldman foi presa em 11 de fevereiro daquele ano, poucos dias após
  falar para cerca de 3000 pessoas sobre métodos contraceptivos. No
  julgamento em questão, foi condenada e optou por passar quinze dias na
  prisão, ao invés de pagar a fiança de 100 dólares.} O medo de ir para\label{ref5}
cadeia pelas próprias ideias é tão forte entre os intelectuais
americanos que é o que faz o movimento ser tão pálido e fraco. Eu não
tenho esse medo. Minha tradição revolucionária diz que aqueles que não
estão dispostos a ir para a cadeia em nome de suas ideias nunca deram
muito valor a elas. Além disso, há lugares piores do que a cadeia. Se eu
tiver de pagar pelas minhas atividades em defesa do controle de
natalidade, uma coisa é certa, o movimento de controle de natalidade não
será interrompido, como tampouco serei impedida de levar adiante a
agitação do controle de natalidade. Se eu me abstive de discutir aqui os
métodos,\footnote{Entre os anos de 1915 e 1916, Goldman deu uma série de
  palestras, por todo os Estados Unidos, para ensinar, sobretudo às
  mulheres da classe trabalhadora, técnicas contraceptivas. Vale lembrar
  aqui da sua formação técnica em enfermagem.} não é porque tenha medo
de uma segunda ordem de prisão, mas porque pela primeira vez na história
da América, o tema do controle de natalidade está circulando, de modo
claro, por meio da informação oral e como eu quero que seja combatido
pelos seus próprios méritos, não desejo dar aqui às autoridades a
oportunidade de confundir as minhas palavras com alguma outra coisa.
E justamente no que diz respeito a esse aspecto, destaco
a completa estupidez da lei. Tenho em
mãos o testemunho dado por policiais que, segundo a declaração deles, é
uma transcrição exata do que falei da plataforma. Esses homens são tão
ignorantes que não transcreveram nenhum método contraceptivo
corretamente.\footnote{Goldman está se referindo à audiência
  preliminar ao julgamento mencionado nessa passagem.
  Num artigo publicado em março no
  \textit{Mother Earth}, intitulado ``Minha prisão e audiência
  preliminar'', ela relata que, no testemunho dos policiais que
  presenciaram a fala que lhe conduziu à prisão, todos os métodos
  contraceptivos foram descritos erroneamente, embora alegassem
  fidelidade ao seu discurso.} Está perfeitamente dentro da lei que
policiais deem testemunho, mas não está dentro da lei que eu leia o
testemunho que resultou na minha acusação. Vocês podem me culpar por eu
ser anarquista e não respeitar as leis? Também quero pontuar a estupidez
da corte americana. Supostamente, a justiça é para ser encontrada lá.
Supostamente não há procedimentos do tipo da \textit{câmara estrelada} no interior de uma
democracia,\footnote{\textit{Câmera estrelada}, no original \textit{star chambre}, é
  uma expressão idiomática utilizada para indicar julgamentos injustos
  resultantes de conspirações e perseguições.} não obstante no dia em que
os policiais deram o seu testemunho, isso foi feito entre sussurros ao
juiz como se se tratasse de um confessionário da Igreja Católica e sob
nenhuma circunstância foi permitido às damas então presentes ouvir algo
do que estava ocorrendo.\footnote{Aqui, Goldman está ironizando o fato
  de que quando os policiais testemunharam sobre os métodos
  contraceptivos ensinados na sua palestra, fizeram isso aos sussurros
  para o juiz, de modo que as mulheres presentes na audiência pública
  não pudessem escutá"-los.} Que farsa tudo isso! E ainda esperavam que
nós respeitássemos, que obedecêssemos a isso, que nos submetêssemos.

Não sei quantos de vocês estão dispostos a respeitar e obedecer, eu sei
que eu não estou. Permaneço como uma das responsáveis por um movimento
mundial. Um movimento que almeja tornar as mulheres livres do jugo
terrível que é a servidão pela gravidez forçada; um movimento que exige
o direito de que toda criança seja bem"-nascida; um movimento que
pretende libertar a mão de obra da sua eterna dependência; um movimento
que almeja dar lugar a um novo tipo de maternidade. Eu considero este
movimento importante e vital o suficiente para que desafie todas as leis
contidas nos livros de estatutos. Eu acredito que este movimento abrirá
caminho não apenas para a discussão livre sobre os métodos
contraceptivos, mas para a liberdade de expressão na vida, arte e
trabalho, e para o direito das ciências médicas manipularem os
contraceptivos como o fazem para o tratamento da tuberculose ou de
qualquer outra doença.

Posso ser presa, posso ser julgada e trancada na cadeia, mas nunca
ficarei em silêncio; eu nunca vou concordar em me submeter à autoridade,
nem farei as pazes com um sistema que degrada a mulher à condição de
mera incubadora, engordando"-a com vítimas inocentes. Eu aqui e agora
declaro guerra contra este sistema e não irei descansar até que o
caminho esteja aberto para uma maternidade livre e saudável e uma
infância alegre e feliz.

\chapter[Novamente o movimento do controle de natalidade]{Novamente o movimento do\break controle de natalidade\footnote{Texto publicado
  na \textit{Mother Earth}, sete meses após ``Os aspectos sociais do
  controle de natalidade'', em novembro de 1916.}}
\markboth{novamente o controle\ldots{}}{}

Se alguém tiver dúvida sobre a magnitude do crescimento do movimento do
Controle de Natalidade, dois acontecimentos recentes na cidade de Nova
York devem dissipar essa dúvida.

Um deles é a declaração proferida pelo juiz Wadhams da \textit{General
Sessions}, e o outro é o método desesperado que o Departamento de
Polícia de Nova York tem utilizado para lidar com os ativistas do
Controle de Natalidade. A polícia não apenas prende todos que discutem
sobre o assunto abertamente ou distribuem informações sobre o Controle
de Natalidade, mas também planta acusações contra vítimas inocentes.
Obviamente, o perjúrio não é uma novidade no nosso Departamento de
Polícia; não é de causar surpresa que esse método antiquado esteja sendo
utilizado novamente da maneira mais flagrante possível.

Considere"-se, primeiramente, o juiz Wadhams. Uma mulher foi posta diante
dele pelo crime de roubo. A senhora Schnur declarou que foi obrigada a
roubar para conseguir algum pão para os seus seis filhos, dos quais o
mais novo tinha dez meses de idade.

Até cinco anos atrás, Samuel Schnur sustentava a família com seu
salário de operário num ateliê de casacos para crianças do lado leste da
cidade. O confinamento constante acabou afetando o homem: ele pegou
tuberculose. Apesar da doença, continuou trabalhando, até que ela foi
descoberta por um inspetor do Departamento de Saúde, que o impediu de
trabalhar no ateliê enquanto estiver com a doença. Desde então, ele não
conseguiu arranjar mais nenhum emprego.

O ônus de sustentar a casa recaiu sobre a sra.\,Schnur, que passou a
garantir a sobrevivência da família fazendo todo tipo de trabalho.
Recentemente, ela não foi bem"-sucedida em conseguir um emprego e gastou
rapidamente o pouco dinheiro que tinha. No mês passado, entrou na casa
de Morris Moskowitz, na rua East Seventh, número 203, e roubou uma
pequena quantia em dinheiro e um relógio, o que resultou na sua prisão.

Segundo as palavras do juiz Wadhams, a sra.\,Schnur já havia sido condenada,
numa ocasião anterior por conta da mesma situação, mas
a sentença foi suspensa.~Embora a mulher pudesse ser sentenciada a
uma prisão de longo prazo, o tribunal determinou que as circunstâncias
incomuns eram tais que justificavam mais um pedido de clemência.

Depois de discutir a condição do marido e a sua impossibilidade de cuidar
da família, o juiz Wadhams fez a seguinte declaração:

\begin{quote}
Apesar de tudo, ele continua se tornando pai de filhos que, por conta
das condições, têm pouquíssimas chances de serem outra coisa que não
tuberculosos, repetindo, ao crescer, o mesmo processo na sociedade. Não
há lei contra isso.

O problema não é apenas o de não termos uma regulação da natalidade
para esses casos, mas também o de que pessoas estão sendo levadas aos
tribunais unicamente por fornecer informações sobre o controle dos nascimentos.
Nesse caso, existe uma lei que elas violam.
\end{quote}

O juiz Wadhams destacou que muitas nações da Europa adotaram a regulação
da natalidade com resultados aparentemente excelentes. Ele perguntou se
os americanos estavam sendo razoáveis em relação ao assunto, como seria
o desejado.

``Creio'', continuou a opinião, ``que estamos vivendo numa era de
ignorância que, em algum tempo futuro, será vista com o mesmo horror com
que agora lembrarmos do que antes era perseguido e agora nos orgulhamos
que exista. Eis que então, diante de nós, uma família que não para de
crescer em número, embora o marido seja tuberculoso, a mulher traga um
filho no peito e outros filhos pequenos debaixo da saia e não tenham
dinheiro nenhum.''

Com certeza, valeu a pena ir para a prisão já que, com isso, foi
possível ensinar a um juiz a importância do controle de natalidade para
as massas. No entanto, ir para a cadeia nunca ensina nada à polícia.

Sexta"-feira, 20 de outubro, fui intimada a comparecer como testemunha do
sr.\,Bolton Hall, no seu julgamento perante as \textit{Special Sessions},
departamento seis, por ter distribuído circulares do Controle de
Natalidade num encontro do movimento na Union Square, no dia 20
de maio. Hall foi absolvido. Em conjunto com vários amigos, saí do
tribunal por volta das 17 horas e mal cheguei na calçada fui presa pelo
policial Price. Quando lhe foi pedido para mostrar o mandado, ele disse
que era desnecessário, que eu estava sob sua incumbência e que teria de
o acompanhar. Por experiências passadas, já sabia que um policial, caso
dos Centenas Negras da Rússia, é uma espécie de poder absoluto e, assim,
segui com ele até a delegacia da rua Elizabeth. Lá, me foi arbitrada a
fiança de mil dólares por ter distribuído folhetos de Controle de
Natalidade na mesma reunião na \emph{Union Square} no dia 20 de maio.
Evidentemente, os policiais não levaram em consideração o testemunho
esmagador prestado em nome do sr.\,Bolton Hall, um testemunho que, é
claro, também será prestado em meu favor, de que nem o sr.\,Hall e nem eu
distribuímos folhetos de Controle de Natalidade. Os policiais decidiram
se envolver numa tramoia e começaram, imediatamente, a colocá"-la em
ação.

Vocês devem se recordar que em abril passado, fui presa por \textit{ter
dado informações sobre controle de natalidade}; que eu fui julgada e
considerada culpada e que preferi ir para a cadeia do condado de Queen
ao invés de pagar a fiança de 100 dólares. Com isso em vista, não é
necessário enfatizar o quanto acredito na questão do controle da
natalidade e na necessidade de dar às pessoas esse tipo de informação.
Em outras palavras, estou disposta a assumir as consequências caso seja
culpada perante o que apetece à lei chamar de ofensa. No entanto, o fato
é que eu não distribuí as circulares em questão, e é óbvio que não
pretendo ser presa e ficar trancafiada na cadeia simplesmente porque os
policiais de Nova York decidiram autocoroar-se com os louros do estancamento
da agitação do Controle de Natalidade.

%\asterisc

No dia 27 de outubro, uma sexta"-feira, compareci perante o juiz Barlow,
um tipo que lembra os personagens de Dickens ou Victor Hugo. Severo,
pomposo e estúpido. Renunciei a audiência preliminar e serei levada a
julgamento.

No momento, não posso dizer quando o julgamento ocorrerá, mas espero
conseguir um adiamento e possivelmente um julgamento por júri. Enquanto
isso, foi"-me arbitrada essa fiança de mil dólares.\footnote{Em janeiro de
  1917, Goldman é absolvida da acusação de ter distribuído panfletos de
  controle de natalidade em 20 de maio de 1916.}

Em 30 de outubro, três casos foram julgados no Tribunal Especial:\footnote{Aqui, Goldman se refere ao \textit{Court of Special Sessions}.} Jesse Ashley e dois jovens da
\textsc{iww}\footnote{Sigla da organização \textit{Industrial Workers of the
  World} ou \textit{Sindicato dos Trabalhadores Industriais do Mundo}.}, Kerr
e Marman. Claro que todos foram considerados culpados. Jesse Ashley foi
condenada a pagar a multa de cinquenta dólares ou cumprir dez dias de prisão.
Ela teria preferido a cadeia, mas foi encorajada a abrir um precedente.
Por conta disso, ela pagou a quantia sob protesto e irá recorrer.

Kerr e Marman provavelmente se sairão pior, já que o policial
testemunhou que, enquanto conversavam na Madison Square, os meninos
ofereceram um panfleto de Controle de Natalidade por dez centavos de
dólar e lhe deram --- que horror --- ``A preparação militar nos conduz
ao massacre universal'', de E.\,G.\,Sequer em sonhos poderíamos
imaginar, quando publicamos o panfleto, que papel importante ele iria
desempenhar num tribunal de Nova York.

Os dois meninos insistiram que o que eles fizeram foi vender o panfleto
\textit{A preparação} e dar os folhetos sobre o Controle de Natalidade de
graça. Muito embora os juízes todos saibam que os policiais nunca
hesitam em cometer perjúrio, Kerr e Marman foram considerados culpados.

É evidente que estamos conseguindo educar os juízes. Não por acaso, um
deles foi muito enfático ao afirmar que pretende distinguir entre
aqueles que dão informações gratuitas sobre Controle de Natalidade, por
convicção, e aqueles que as vendem. Nenhuma consideração dessa natureza
foi feita no caso de Bill Sanger, de Ben Reitman ou mesmo no meu, embora
nenhum de nós tenha vendido informações.

Essa novidade na opinião de um juiz prova, portanto, que a ação direta é
a única ação que conta. Mas para todos nós que desafiamos a lei, o
Controle de Natalidade ainda é uma proposta de salão, apesar de todas as
ligas de Controle de Natalidade neste país.\footnote{Goldman organizou
  as ligas também com a intenção de derrubar as leis que proibiam a
  prática e informação sobre o controle de natalidade. A estratégia era
  a de que assim que um dos membros fosse preso, pela distribuição dos panfletos com as informações, outro se encarregaria da distribuição, de modo a também ser preso e, assim, sucessivamente. A ideia era a de que uma série de prisões sem sentido demonstrariam a
  obsolescência da lei.}

\chapter[O camaleônico sufrágio feminino]{O camaleônico sufrágio\break feminino\footnote{Texto originalmente publicado
  na revista anarquista \textit{Mother Earth}, em 1917.}}
\markboth{o camaleônico sufrágio\ldots{}}{}

Há quase meio século, as líderes do sufrágio feminino têm apregoado os
resultados milagrosos que se seguiriam à alforria da mulher. Todos os
males sociais e econômicos dos séculos passados seriam abolidos tão logo
a mulher obtivesse o direito ao voto. Todos os erros e injustiças, todos
os crimes e horrores cometidos ao longo das eras seriam eliminados da
vida através do decreto mágico contido num pedaço de papel.

Quando as líderes do movimento eram alertadas para o fato de que tais
alegações extravagantes não convenciam ninguém, elas diziam: ``Esperem
até que tenhamos a oportunidade; esperem até que estejamos frente a
frente com a grande provação e, então, vocês verão o quão superior é a
mulher em sua atitude para com o progresso social.''

\textls[-10]{Os oponentes realmente inteligentes do sufrágio feminino ---
fundamentados na compreensão de que o sistema representativo serviu
apenas para roubar a independência do homem, e que faria o mesmo com a
mulher --- sabiam muito bem que, em lugar nenhum, o sufrágio feminino
exerceu a mínima influência sobre a vida social e econômica das pessoas.
Ainda assim, eles se dispuseram a dar aos expoentes do sufrágio o
benefício da dúvida. Eles estavam dispostos a acreditar que as
sufragistas eram sinceras na sua afirmação de que a mulher nunca seria
culpada das estupidezes e crueldades cometidas pelo homem. Eles
vislumbraram, especialmente nas sufragistas militantes da Inglaterra, um
tipo superior de feminilidade. Não foi a senhora Emmeline
Pankhurst\footnote{Entre 1909 e 1913, a fundadora do \textit{Women's
  Social and Political Union}, ou simplesmente \textsc{wspu}, Emmeline Pankhurst
  fez uma série de conferências muito bem"-sucedidas nos Estados Unidos,
  daí a alusão de Goldman.} quem fez a afirmação ousada, do alto de uma
plataforma nos Estados Unidos, de que a mulher é mais humana do que o
homem e que nunca seria culpada pelos seus crimes; até porque, para
começar, a mulher não acredita na guerra e nunca apoiará as guerras?}

Mas políticos permanecem políticos. Tão logo a Inglaterra entrou na
guerra, por razões humanitárias, é claro, as damas sufragistas
esqueceram imediatamente toda a bazófia da superioridade e bondade da
mulher e imolaram o seu próprio partido no altar do mesmo governo que
rasgou suas roupas, puxou seus cabelos e as alimentou à força\footnote{A
  greve de fome era uma das formas de protesto mais utilizadas pelas
  sufragistas da \textsc{wspu} --- especialmente, quando presas por atos de vandalismo
  em nome da causa. Muitas
  sufragistas de renome escreveram relatos sobre a experiência de serem
  alimentadas à força nas prisões, denunciando os abusos físicos e
  emocionais que acompanhavam a prática. O uso repetido da sonda
  nasogástrica para a alimentação forçada causava, por si só, danos
  permanentes à saúde. Em alguns relatos, a analogia entre a alimentação
  forçada e o estupro, embora não exposta diretamente, é suficientemente
  clara. Em 1913, foi promulgada a lei que ficou conhecida como Lei do
  Gato e Rato {[}\textit{Cat and Mouse Act}{]}, em que já não mais se
  imporia a alimentação forçada às grevistas, mas ao invés disso,
  deixá-las-ia chegar ao limite das suas forças no presídio, até que sob o
  risco de morte fossem temporariamente liberadas para a sua
  recuperação --- o que, diga"-se de passagem, isentava o governo de
  qualquer responsabilidade para com eventuais mortes decorrentes das
  greves prolongadas. A imagem da sufragista, voluntariamente em greve
  de fome, em sua cela isolada, teve forte ressonância na imaginação
  pública.} por conta das suas atividades militantes. A senhora
Pankhurst e o seu bando tornaram"-se mais apaixonadas na sua mania de
guerra e na sua sede pelo sangue do inimigo do que os militaristas mais
insensíveis.\footnote{\textls[-15]{Com a entrada da Inglaterra na Primeira Guerra
  Mundial, Pankhurst e a sua filha mais velha Christabel, na época a
  líder formal da \textsc{wspu}, comunicaram, em agosto de 1914, a decisão de
  interromper todas as atividades de militância do grupo até o final da
  guerra --- o que, como não poderia ser diferente, conduziu ao
  desligamento e formação de novas células por parte de uma minoria que
  não estava de acordo. Como resposta a este apoio inesperado,
  incialmente visto com desconfiança, o governo britânico libertou todas
  as sufragistas presas e retirou todas as queixas, o que possibilitou a
  ambas as Pankhursts, mãe e filha, retornarem à Inglaterra sem correr o
  risco de serem, mais uma vez, presas. Emmeline e Christabel
  transformaram a \textsc{wspu} em veículo de apoio à guerra e passaram a se
  apresentar como patriotas feministas, defendendo a importância da
  participação das mulheres na guerra em defesa da esclarecida
  democracia britânica, então ameaçada pela autocracia militarista da
  Alemanha. Naquele momento, segundo elas, o dever das feministas havia
  se convertido em inspirar o patriotismo.}} Elas consagraram tudo, até a
própria sensualidade, como meio de atrair os homens ainda relutantes
para o campo militar, para as trincheiras e a morte.\footnote{Dado que,
  na época, o recrutamento militar na Inglaterra era voluntário e não
  obrigatório, Emmeline passou a incentivar nos seus comícios, após
  agosto de 1914, o alistamento dos homens na guerra, tornando"-se uma
  espécie de agente não oficial de recrutamento. Apesar da defesa da
  participação das mulheres na guerra, reforçou estereótipos ao ecoar a
  visão comum acerca da inadequação das mulheres ao trabalho de soldado,
  posto ser o mais alto dever destas a manutenção da
  sociedade e sobretudo a maternidade, o que seria incompatível com sua
  presença efetiva nas trincheiras. Às mulheres caberia impedir o
  colapso da nação, ao assumir os empregos tradicionalmente ocupados por
  homens que, assim, estariam livres para lutar em defesa do seu país e
  das democracias da Europa e, portanto, em defesa das suas mulheres --- conforme sugeria marotamente nas suas conferências, um dos pontos que
  parece estar sendo aqui denunciado por Goldman. Ainda que a defesa da
  assunção das mulheres de trabalhos até então exclusivamente destinados
  aos homens fosse de encontro do lugar tradicionalmente reservado à
  mulher na sociedade britânica, já não se tratava mais de subversão. Em
  realidade, no final da guerra, as Pankhursts, mãe e filha mais velha,
  passaram da condição de \textit{terroristas} a ultrapatriotas defensoras do
  \textit{status quo.} Para a surpresa de muitos, em 1926, Emmeline
  se filiou ao Partido Conservador do Reino Unido, concorrendo como
  candidata do partido, em 1928, cuja campanha teve de ser abortada por
  conta de um escândalo envolvendo outra das suas filhas, dissidente do
  \textsc{wspu} --- que, então recente mãe solteira, passou a dar declarações
  públicas em favor da sua condição de mulher e mãe \textit{livre}.} Por tudo
isso, agora elas serão recompensadas ​​com o sufrágio. Até Asquith, o
antigo inimigo da organização de Pankhurst, está agora convencido de que
a mulher deve ter direito ao voto, já que se provou tão feroz no seu
ódio e tão empenhada na conquista.\footnote{Herbert Henry Asquith ficou
  conhecido pela oposição atuante e agressiva contra o sufrágio feminino
  --- iniciada em 1892 e persistindo ao longo de boa parte do seu mandato
  como primeiro"-ministro do Reino Unido (1908--1916). Embora o seu
  posicionamento contra o sufrágio feminino tenha dado indícios de
  mudança já em 1912, em 1915 declarou publicamente que dado o
  engajamento demonstrado, já não poderia negar a reivindicação das
  mulheres uma vez finda a Guerra --- declaração que, ao que parece, é a
  que Goldman está se referindo aqui. Seja como for, a desconfiança e
  repúdio das duas Pankhursts contra Asquith, tornadas públicas sempre
  que tinham oportunidade, não cessou até a sua renúncia em 1916. O
  sufrágio feminino no Reino Unido foi garantido por lei apenas em 1918,
  não obstante apenas para uma pequena parcela de mulheres cujo projeto
  já havia sido denunciado por Goldman no texto ``O sufrágio feminino''
  de 1910.} Todos saúdam as inglesas que compraram seu direito ao voto
com o sangue de milhões de homens sacrificados ao monstro Guerra. O
preço é de fato alto, mas também altos serão os cargos reservados para
as damas da política.

O Partido Sufragista Americano, desprovido de qualquer ideia original
desde os dias de Elizabeth Cady Stanton, Lucy Stone e Susan
Anthony,\footnote{Elizabeth Cady Stanton (1815--1902), Lucy Stone
  (1818--1893) e Susan Anthony (1820--1906) têm em comum não apenas o fato
  de terem sido pioneiras do movimento sufragista estadunidense, ficando
  conhecidas como o \textit{triunvirato} do movimento sufragista, quanto
  também de serem militantes aguerridas do abolicionismo. As três não só
  se conheceram como trabalharam juntas, influenciando"-se mutuamente. Há
  quem defenda que as primeiras sufragistas possivelmente teriam ligações diretas com
  alguns dos povos nativos norte"-americanos, que as suas ideias
  feministas refletiam, em muitos aspectos, os papéis de liderança e
  poder desempenhados pela mulher nesses povos. Em 1866, o
  \textit{triunvirato} fundou a \textit{American Equal Rights Association}, que
  militava a favor da igualdade de direitos e sufrágio para todos os
  estadunidenses independentemente da raça, cor ou sexo. A associação,
  porém, não teve vida longa: em 1869, Anthony e Stanton fundaram a
  \textit{National Woman Suffrage Association} (\textsc{nwsa}) e, poucos meses
  depois, Stone veio a fundar a \textit{American Woman Suffrage
  Association} (\textsc{awsa}). A gota d'água que culminou na cisão foi a
  discussão em torno do apoio ou não das sufragistas à Décima Quinta
  Emenda que garantia a todos os cidadãos estadunidenses \textit{homens},
  independentemente da cor e raça, o direito ao voto. O motivo do
  desacordo estava relacionado às táticas de ação e não tanto aos fins
  propriamente ditos. Em 1890, os dois grupos, então rivais em muitas
  pautas, fundiram"-se, após longa negociação, na organização
  \textit{National American Woman Suffrage Association} (\textsc{nawsa}).
  Antes da fusão, em 1889, líderes de ambas as associações, o que
  incluía o \textit{triunvirato}, publicaram juntas uma ``Carta aberta para
  as mulheres da América.''} irá necessariamente macaquear, repetir, de
modo estúpido, o exemplo dado por suas irmãs inglesas. Nos dias heroicos
da sua militância, a sra.\,Pankhurst e suas seguidoras foram
completamente repudiadas pelo Partido Sufragista Americano. Uma dama
elegante e respeitável como a senhora Catt\footnote{Carrie Chapman
  Catt~sucedeu Susan Anthony na presidência da \textit{National American}
  \textit{Woman Suffrage Association}, em 1900. Ela exerceu dois mandatos
  como presidente da \textsc{nawsa}, o primeiro de 1900 a 1904 e o segundo de
  1915 a 1920, sendo, portanto, presidente no ano em que o presente
  artigo foi redigido e publicado. Sob a liderança de Catt, a \textsc{nawsa}
  pautava a sua defesa do sufrágio feminino na concepção ampla e
  repetidamente combatida por Goldman: a de que as mulheres,
  essencialmente diferentes dos homens, trariam as suas virtudes
  domésticas de modo a purificar a política. Muitas décadas após a sua
  morte, nos 1990, Carrie Chapman Catt teve o seu nome associado ao
  racismo; foram"-lhe atribuídas uma série de declarações racistas e
  xenófobas, dentre elas a de que o sufrágio feminino fortaleceria a
  causa dos supremacistas brancos. A discussão em torno do racismo de
  Catt culminou num movimento liderado por estudantes da \textit{Iowa
  State University,} ainda na década de 1990, com o objetivo de remover
  o seu nome de um dos edifícios da instituição --- o que não ocorreu. A
  factualidade de boa parte das declarações foi contestada por
  pesquisadores que alegaram ausência de registros históricos que
  confirmassem as fontes secundárias. No que diz respeito à infame
  associação entre o sufrágio feminino e a supremacia branca, deveras
  contida num ensaio de Catt, também datado de 1917 e intitulado ``O
  sufrágio feminino na emenda federal'' {[}\textit{Woman Suffrage by
  Federal Amendment}{]}, seus estudiosos chamam a atenção para a
  necessidade de interpretação do contexto em que tal associação é
  estabelecida --- dentro e fora do texto. Até hoje, a discussão perdura.}
não poderia mesmo ter algo a ver com facínoras como são todos os
militantes. No entanto, quando as sufragistas da Inglaterra, de olho nos
privilégios luxuriosos do Parlamento, deram suas cambalhotas e saltos
mortais, o Partido Sufragista Americano seguiu o exemplo. A verdade é
que a sra.\,Catt sequer esperou que a guerra fosse declarada pelo seu
país. Ela se saiu ainda melhor do que a sra.\,Pankhurst: ofereceu o seu
partido ao militarismo, prometeu apoio a todas as medidas autocráticas
do governo muito antes que houvesse necessidade de tudo
aquilo.\footnote{Ainda antes da entrada formal dos Estados Unidos na
  guerra, Catt, apesar de pacifista já que membro de organizações pacifistas,
  rapidamente, seguindo o exemplo dado pelas Pankhursts (segundo a tese
  de Goldman aqui exposta), resolveu colocar a força da \textsc{nawsa} à
  serviço da guerra, fosse para a mobilização de apoio popular, para a
  arrecadação de fundos e alimentos, preparação e suporte às mulheres
  que viessem a ocupar os empregos deixados pelos homens etc. Muito
  embora o presidente Wilson se opusesse ao sufrágio feminino desde o
  início do seu mandato, em 1913, a editora da \textsc{nwasa} reimprimiu as
  mensagens de guerra do presidente, valendo"-se delas nos seus esforços
  de mobilização da população. O mote legitimador era a de que aqueles
  que já estão treinados em lutar pela democracia no seu país, estão
  prontos para lutar pela democracia mundial. Como as Pankhursts, Catt,
  e muitas outras sufragistas norte"-americanas, passaram a vincular a
  causa sufragista ao patriotismo e às virtudes cívicas.} Por que não?
Por que desperdiçar mais cinquenta anos fazendo lobby pelo direito ao
voto se é possível obtê"-lo por meio da simples traição a um ideal? Que
ideais podem subsistir entre políticos, afinal?

O argumento dos opositores, de que a mulher não precisa do voto porque
ela tem uma arma mais forte --- o seu sexo ---, havia sido respondido com
a declaração de que o voto justamente libertaria a mulher da necessidade
degradante do apelo sexual. Como é possível que esse orgulho outrora tão ostensivo
tenha dado origem à campanha, agora iniciada pelo Partido Sufragista, de
atrair a masculinidade da América para o mar de sangue europeu? As
justíssimas integrantes do Partido Sufragista, além de abordar
descaradamente cada rapaz e cada homem, persuadindo"-os a se alistar,
estão induzindo as esposas e namoradas a jogar com as emoções e
sentimentos dos seus companheiros, a fim de incitar o seu sacrifício ao
Moloch do patriotismo e da guerra.\footnote{Através de uma ampla
  campanha encabeçada pela \textsc{nawsa}, o governo de Wilson inaugurou, em
  1917, o \textit{Woman's Committee of Council Nacional Defense}, que
  subordinado ao \textit{Council Nacional Defense}, por sua vez
  subordinado Departamento de Guerra, teria a responsabilidade de
  organizar a mobilização civil durante a guerra.}

Como essa tarefa poderá ser bem"-sucedida? Certamente, não pela via do
argumento. Se durante os últimos cinquenta anos as mulheres militantes
falharam em convencer a maioria dos homens de que a mulher tem direito
à igualdade política, elas certamente não os convencerão,
repentinamente, de que devem ir para a morte certa, enquanto elas,
mulheres, permanecerão em segurança, enfiadas nas suas casas, a costurar
ataduras. Não, não foi nenhum argumento, razão ou humanitarismo que o\label{argumento}
Partido Sufragista prometeu ao governo; e sim, o poder da atração
sexual, o apelo vulgar, persuasivo e envolvente da mulher liberada a
serviço da glória do seu país. Que homem pode resistir a isso? Se muitos
dos grandes homens perderam a sanidade e capacidade de julgamento,
quando entorpecidos pelo apelo sexual, como a juventude da América
poderia resistir?

O rei está nu. As damas sufragistas já deram provas de que a sua
prerrogativa não é nem a inteligência, nem a sinceridade, e que a sua
ostentação da igualdade está completamente podre; que, inclusive, na
luta pelo direito ao voto, o apelo sexual é o seu único recurso, e a
recompensa da politicagem mais chula, seu único objetivo. Elas agora
estão utilizando ambos, recurso e objetivo, para alimentar o cruel
monstro da guerra, embora devam saber que, por mais terrível que seja o
preço pago pelo homem, não é nada quando comparado à crueldade,
brutalidade e ultraje aos quais a mulher está sujeita durante a guerra.

O crime que as líderes do partido do sufrágio feminino americano estão
cometendo contra o seu eleitorado é análogo à relação do alcoviteiro com
sua vítima. A maioria dessas lideranças já está muito velha para exercer
qualquer efeito sobre o alistamento via o apelo sexual ou mesmo para
prestar qualquer serviço pessoal ao seu país. Ao prometerem ao governo o
apoio do partido, estão vitimando os membros mais jovens. Isso pode soar
excessivamente cruel, mas, no entanto, é verdade. De outro modo, como
poderíamos explicar o compromisso de bater de porta em porta para
trabalhar a histeria patriótica nas mulheres, que, em troca, devem
utilizar a sua sensualidade para fazer seus homens se alistarem? Em
outras palavras, o mesmo atributo que a mulher foi forçada a usar para
garantir o seu \textit{status} econômico e social na sociedade, e que as
damas sufragistas sempre repudiaram, está agora começando a ser
explorado a serviço do Senhor da Guerra.\footnote{Não foram poucas as
  líderes sufragistas que, nos seus esforços no tempo da guerra, foram
  tomadas de grande fervor patriótico.}

Para fazer justiça ao \textit{Woman's Political Congressional Union} e a
alguns poucos membros do Partido Sufragista, é preciso dizer que há
resistência à persuasão das líderes sufragistas. Infelizmente, porém,
para dizer a verdade, o \textit{Woman's Political Congressional Union}
está em cima do muro: não se posiciona nem a favor da guerra, nem da
paz.\footnote{Aqui Goldman está se referindo ao \textit{National Women's
  Party}, conforme foi reorganizado, em 1916, o \textit{Woman's Political
  Congressional Union}. Diferente da \textsc{nawsa}, de onde muitas das suas
  lideranças haviam sido expulsas, as mulheres do \textit{National Women's
  Party} continuaram a protestar contra a oposição do governo de Wilson
  ao sufrágio feminino, independentemente da guerra. Em realidade,
  tomaram como argumento central a contradição de um governo que estaria
  a lutar para assegurar a democracia no exterior, enquanto negava às
  mulheres cidadãs do seu país o direito ao voto. Algumas foram presas
  nesses protestos. Ao se concentrarem, de modo praticamente exclusivo,
  na causa do sufrágio feminino, recusaram qualquer posicionamento
  acerca da atuação dos Estados Unidos na guerra.} Não haveria problema
nisso, desde que o monstro se deslocasse exclusivamente sobre a
Europa. Agora que está se alastrando em casa, o \textit{Congressional
Union} finalmente entenderá que silêncio é sinal de consentimento. A recusa em se
posicionar de modo taxativo contra a guerra praticamente torna os seus
membros parte dela.

Em meio a essa confusão entre as facções do movimento sufragista, é
realmente refrescante encontrar uma mulher decidida e firme. A recusa de
Jannette Rankin em apoiar a guerra faz mais, no sentido de aproximar a
mulher da emancipação, do que todas as medidas políticas juntas. No
momento presente, ela é sem dúvida considerada um anátema, uma traidora
do seu país.\footnote{Primeira mulher eleita para a Câmera dos Deputados
  nos Estados Unidos, em 1916, Jannette Rankin, até então um dos
  expoentes da \textsc{nawsa}, pagou um preço alto ao escolher manter a coerência
  entre o pacifismo pelo qual havia militado ao longo de toda vida, e a
  sua atitude política: votou contra a declaração de guerra à Alemanha
  em 1917.} Mas isso não deveria entristecer a senhorita Rankin. Todos
os homens e mulheres valiosos foram denunciados como tal. No entanto,
eles, e não os patriotas gritalhões e covardes, são efetivamente valorosos
para a posteridade.

\chapter{Louise Michel, uma refutação\footnote{Carta aberta dirigida ao médico e
  se Magnus Hischfeld, escrita em inglês e publicada em alemão na
  \textit{Jahrbuch für sexualle Zwischenstufen}. O texto é uma resposta ao ensaio
  de Karl von Levetzow (1825--1895), no qual defende que a anarquista francesa
  Louise Michel seria lésbica. O artigo de von Levetzow foi publicado na
  mesma revista em 1905, ano da morte de Michel. Uma tradução para o
  inglês da carta foi publicada, em 1976, na coletânea de textos e
  documentos organizada por Jonathan Katz, chamada \textit{Gay American
  History}. Este texto é traduzido do rascunho da carta original em
  inglês, de 1923.}}
\markboth{Louise Michel}{}

\textit{Prezado dr.\,Hirschfeld},

\smallskip 

\noindent Eu tenho conhecimento do seu excelente trabalho sobre psicologia sexual
já há alguns anos. Sou admiradora da sua luta corajosa pelos direitos
das pessoas que, pela sua própria natureza, não se expressam sexualmente
do modo comumente considerado \textit{normal}.\footnote{A \textit{Jahrbuch für
  sexualle Zwischenstufen} era a revista oficial do \textit{Scientic
  Humanitarian Committee}, a primeira organização dedicada a promover os
  direitos dos homossexuais e transgêneros. Dr.\,Hirschfeld, era não só
  editor da revista, como fundador do \textit{Scientic Humanitarian
  Committee}.} E agora que tive a sorte de conhecê"-lo e acompanhar os
seus esforços de perto, estou mais do que nunca impressionada com a sua
personalidade e com o espírito que tem lhe sustentado nessa tarefa tão
difícil. Sua prontidão em publicar a minha refutação à avaliação de
Frhr. von Levetzow de que Louise Michel seria uraniana\footnote{Aqui,
  Goldman utiliza uma expressão então corrente na época, no original
  \textit{urning}. Em 1889, no volume de lançamento do \textit{Jahrbuch
  für sexualle Zwischenstufen}, Hirschfeld publicou uma série de cartas
  do alemão Karl Heirich Ulrichs, escritor de romances homossexuais, e
  considerado o primeiro ativista a lutar politicamente pelos direitos
  dos homossexuais. Embora fosse advogado, por conta das leis
  discriminatórias contra os homossexuais, Ulrich morreu em miséria, na
  Itália. Ele foi responsável por cunhar diversos termos para designar
  diferentes formas de sexualidade, dentre eles, o termo \textit{uraniano}
  que indica uma \textit{alma feminina no corpo de um homem}. Conforme se
  verá, Goldman intercambia livremente as palavras \textit{uraniano} e
  \textit{homossexual}.} prova, se provas são necessárias, que você tem um
senso de justiça elevado cuja meta exclusiva é a averiguação da verdade.
Agradeço por isso e pela postura habilidosa e heroica que você tem
adotado contra a ignorância e a hipocrisia, a favor do esclarecimento e
do humanismo.

Antes de tratar do artigo propriamente dito, permita"-me dizer o
seguinte: não foi qualquer preconceito contra a homossexualidade ou
aversão aos homossexuais o que me impeliu a destacar os erros contidos
na argumentação do autor. Se Louise Michel tivesse demonstrado, em algum
momento, traços homossexuais para aqueles que a conheciam e amavam, eu
seria a última pessoa a negar esse estigma. Na verdade, considero tal
estigma uma tragédia para aqueles que são sexualmente diferenciados num
mundo tão desprovido de compreensão para com os homossexuais e tão
ignorante no que diz respeito ao significado e à importância de toda a
variedade que envolve o sexo. É certo que não penso que essas pessoas
sejam inferiores, menos morais ou menos capazes de sentimentos e ações.
E acima de tudo: não penso que seja necessário algo como \textit{limpar} a reputação da minha
ilustre professora e camarada Louise Michel da acusação de
homossexualidade. Seu valor para a humanidade, sua contribuição para a
emancipação dos escravos são tão grandes que, qualquer que fosse o seu
modo de gratificação sexual, em nada poderia lhe acrescentar ou
depreciar.\footnote{A anarquista francesa Louise Michel (1830--1905), ou
  a \textit{virgem vermelha}, como ficou conhecida, foi a mulher de maior
  destaque na Comuna de Paris. Desempenhou diversos papéis na Comuna,
  inclusive o de soldado, participando ativamente da legendária luta nas
  barricadas. Fosse na prisão, exílio ou quando em liberdade, Louise
  Michel se manteve absolutamente dedicada às atividades revolucionárias
  até a sua morte, em 1905.}

Anos atrás, quando eu não sabia nada sobre psicologia sexual, e o meu
único contato com homossexuais havia sido algumas mulheres que conheci
na prisão, onde estava encarcerada por minhas opiniões políticas, saí em
defesa de Oscar Wilde. Como anarquista, meu lugar sempre foi ao lado dos
perseguidos. E vi na perseguição a Oscar Wilde, e na acusação contra ele,
o reflexo da injustiça cruel e da hipocrisia da mesma sociedade que o
enviou ao seu martírio. Foi por isso que o defendi.\footnote{Oscar Wilde
  foi condenado a dois anos de prisão, por sodomia, em 1895.}

Posteriormente, quando fui à Europa, e lá me deparei com as obras de
Havelock Ellis, Krafft"-Ebbing, Carpenter e muitos outros, o crime contra
Oscar Wilde e demais homossexuais se apresentou sob uma luz ainda mais
gritante. Desse momento em diante, passei a usar a minha caneta e a
minha voz na defesa daqueles a quem a natureza mesma destinou ser
diferente em sua psicologia e necessidades sexuais. Os seus trabalhos,
prezado doutor, ajudaram"-me muitíssimo a iluminar essa questão
extremamente complexa, que é a da psicologia sexual, e a humanizar a
atitude das pessoas que passaram a vir me escutar.

Dito isso, os seus leitores poderão ver que não tenho quaisquer
preconceitos, ou a menor antipatia, pelos homossexuais. Muito pelo
contrário. Tenho entre os meus amigos, homens e mulheres, tanto
uranianos completos, quanto bissexuais. E o que encontro neles é uma
inteligência, habilidade, sensibilidade e charme, em geral, muito acima
da média. Sinto"-me profundamente ligada a eles, porque sei que o seu
sofrimento é maior e mais complexo do que o da maioria das pessoas. No
entanto, há uma tendência predominante entre os homossexuais e à qual eu
me oponho. É a tentativa de reivindicar toda personalidade excepcional
para o seu credo, de atribuir a essas personalidades traços e
características que são inerentes exclusivamente a eles.

Pode ser uma espécie de condicionamento psicológico que as pessoas
perseguidas busquem apoio nos tipos excepcionais das mais diferentes
épocas. A miséria sempre busca companhia. Por exemplo, os judeus
consideram que a maioria dos grandes homens e mulheres do mundo são de
origem judaica ou possuem características judaicas. Os irlandeses fazem
o mesmo. Os hindus proclamarão que a sua civilização é a melhor do mundo
e assim por diante. O mesmo ocorre com os párias políticos. Os
socialistas reivindicam Walt Whitman e Oscar Wilde para as teorias de
Marx, ao passo que muitos anarquistas situam Nietzsche, Wagner, Ibsen e
outros em meio aos seus. Certamente, a grandiosidade sempre acompanha a
versatilidade, mas considero uma imposição reivindicar alguma pessoa
criativa de destaque para minhas ideias, a menos que ele ou ela as tenha
declarado isso por si.

Caso se tome esse tipo de alegação característica a muitos homossexuais
como certa, chegar"-se"-á à conclusão de que não há e nem pode haver
talento e grandiosidade em pessoas que não sejam sexualmente invertidas.
A perseguição gera o sectarismo; o que, em contrapartida, torna as
pessoas limitadas ao entorno imediato e, muitas vezes, injustas na avaliação
de outras pessoas. Prefiro pensar que Frhr.\,Karl von Levetzow sofreu
uma overdose de sectarismo homossexual. Acrescente"-se a isso a sua
visão bastante antiquada da mulher. Ele vê a mulher apenas como uma sedutora
de homens, incubadora de crianças e, num sentido mais vulgar,
cozinheira e faxineira doméstica. Qualquer mulher que não apresente
esses atributos antiquados de feminilidade, será imediatamente
classificada pelo autor como uraniana. À luz das realizações da mulher
moderna nos mais diversos domínios do pensamento humano e da ação
social, esse tipo de visão típica ao homem convencional,
malmente merece ser considerada. Se me propus tratar aqui dessa
compreensão ultrapassada do autor de ``Louise Michel'', foi unicamente
para demonstrar a que conclusões absurdas se pode chegar, quando se
parte de uma premissa absurda.

Minhas críticas a von Levetzow não me impedem de lhe prestar homenagem
como grande artista literário, capaz de compreender, empaticamente, uma
grande alma. Na verdade, sinto"-me inclusive culpada por ter de dissecar
o seu artigo. É como se eu tentasse fatiar um grandioso e radiante
retrato pintado à mão por um mestre, pois a imagem de Louise Michel
traçada pela pena de von Levetzow é uma obra"-prima. Por isso, todos nós
que conhecemos e amamos essa mulher maravilhosa estamos em dívida para com
ele. Seja como for, a verdade exige que eu coloque os meus sentimentos
de lado e lide com os fatos.

De modo a que eu pudesse tratar adequadamente dos pontos levantados no
artigo, seria necessário republicar o texto na íntegra, juntamente com a
minha réplica, ou, no mínimo, fazer uma série de citações extensas,
para, então, abordar cada ponto detalhadamente. O problema é que um tal
feito ocuparia muito espaço do valioso \textit{Jahrbuch}. Desse modo,
contentar"-me"-ei com uma simples síntese dos pontos mais relevantes que
foram levantados como prova da homossexualidade de Louise Michel.

Quais são esses pontos?

Primeiro, Louise Michel foi uma criança excepcional: ávida por
conhecimento e problemas científicos, era uma leitora voraz. Segundo, os
seus brinquedos, diferentemente dos das outras meninas, não eram
bonecas, mas sapos, besouros, camundongos e outros seres vivos.
Terceiro, Louise Michel brincava muito com o seu primo (o que, a
propósito, deveria provar que Louise era uma garota perfeitamente
normal, caso contrário escolheria sobretudo garotas para a sua
companhia), escalava árvores, organizava expedições de caça, corria
livremente, enfim, estava completamente imersa nas brincadeiras e
travessuras dos meninos. Quarto, ela se tornou totalmente indiferente à
sua aparência, odiava e se opunha aos babados das vestimentas femininas,
espartilhos, sapatos de salto alto e todo o resto; era terrivelmente
displicente consigo mesma e desorganizada em seus hábitos e para com
tudo aquilo que a cercava. Quinto, Louise Michel era extraordinariamente
corajosa, desprovida do sentimento de medo, ousava ao ponto da
imprudência. Seu poder de resistência ao sofrimento físico dificilmente
poderia ser alcançado até mesmo por um homem. Sexto, com a exceção dos seus
camaradas, nenhum homem fez parte da sua vida. Por outro lado, ela
estava sempre cercada \textit{de amigas apaixonadas}. Por último, embora não
menos importante, Louise Michel era matemática e compositora, amava
esculturas, era uma entusiasta da música de Wagner e fez muitas coisas
que as mulheres, em geral, nunca fazem. O autor enfatiza bastante o
corpo retangular de Louise Michel, os seus seios pequenos e outras
supostas características masculinas. Em suma, ele utiliza todos os
argumentos imagináveis ​​para provar a masculinidade de Louise Michel,
argumentos que, desde os tempos imemoriais, são usados contra a mulher
sempre que ela busca se elevar da condição de prisioneira do harém e
alcançar a posição de igualdade na vida para com o homem.

Vejamos quão verdadeiros são esses chamados \textit{fatos}.

Primeiro, a inclinação precoce de Louise Michel para os problemas
verdadeiramente profundos da vida, a leitura ávida de livros sérios e o
seu senso matemático são característicos a um número significativo de
mulheres excepcionais; para mencionar apenas algumas, esse é o caso de
Sofia Kovalevskaya, Marie Bashkirtseff e, na atualidade, Madame Curie.
Kovalevskaya resolveu complexos problemas matemáticos aos oito anos de
idade e, quando tinha apenas 25 anos, já estava entre os
maiores matemáticos da época. Bashkirtseff possuía uma compreensão
psicológica do que se passava em seu entorno muito mais profunda do que a de
boa parte dos homens mais célebres; ocupou"-se do estudo da ciência,
sociologia, literatura, arte, música, aos dez anos de idade, e se tornou
uma das figuras mais marcantes do seu tempo. Madame Curie, por sua vez,
é bastante conhecida, e não como mera auxiliar do marido, mas como uma
autoridade científica independente\ldots{} No entanto, essas mulheres, e
muitas outras além delas, não só não eram homossexuais, como eram
extremamente femininas; uma feminilidade que foi, em grande medida, o
motivo da maior tragédia de suas vidas, posto que os homens que
conheceram não conseguiram compreender o espírito dessas mulheres,
sequioso do amor e do companheirismo de um homem. Foi assim que Sofia
Kovalevskaya desperdiçou a sua substância com o seu marido e, num
segundo momento, com uma paixão violenta por um compatriota que nunca
pôde suspeitar da chama que estava a consumir essa grande mulher do
século \textsc{xix}. No que diz respeito a Madame Curie, não desejo entrar na sua
vida privada --- provavelmente, ela consideraria uma invasão. Seja como
for, pelo tanto quanto se sabe sobre a sua vida privada, ela parece
extremamente feminina, sem nenhuma tendência homossexual.

Tenho certeza de que Frhr.\,von\,Levetzow nunca presenciou as brincadeiras de meninas americanas, saudáveis e normais. Ele iria perceber que é
possível brincar, escalar árvores, divertir"-se com sapos, besouros e
cobras e fazer todo tipo de coisas relacionadas a \textit{meninos}, e, ainda
assim, se tornar uma mulher extremamente frívola, tipicamente feminina
e, não raro, absolutamente inútil. Por outro lado, existem inúmeras
mulheres americanas de valor imensurável, em quase todas as esferas da
vida, que apesar de muito traquinas na infância, são amantes apaixonadas
dos seus homens, mães dos seus filhos e ao mesmo tempo em que uma grande
força moral nos mais diferentes movimentos que lutam por valores sociais
mais profundos e elevados para o seu país.

Quanto à coragem de Louise Michel, à sua ousadia que beirava a
imprudência, ausência de medo e resistência física impressionante --- é
impossível que eu não concorde que, de fato, ela possuía tais
qualidades. Não obstante, seria injusto para com um sem"-número de
revolucionárias russas, se eu enfatizasse todos esses traços
maravilhosos de Louise sem lhes dar o devido crédito. É evidente que o
autor do artigo nunca teve notícias dessas mulheres; para mencionar
apenas algumas: Perovskaya, Gelfman, Figner, Breskovskaya, Kovalskaya,
Volkenstein e, atualmente, Spiridonova.\footnote{Sophia Perovskaya (1853--1881),
  Gesya Gelfman (1855--1882), Vera Figner (1852--1942), Catherine Breshkovskya (1844--1934), Yelizaveta
  Kovalskaya (1851--1943), Aleksandrova Volkenstein (1857--1906), and Maria Spiridonova (1844--1941).} Todas essas
mulheres foram heroicas na grande luta revolucionária da Rússia. Elas
realizaram os feitos mais ousados ​​e seguiram para a morte ou para a
Sibéria e a katorga,\footnote{Campos de trabalhos forçados na Sibéria.}
um calvário ainda maior, com um sorriso nos lábios. Perovskaya preferiu
morrer na forca com o seu marido amado do que fugir e garantir a sua
vida, o que ela, facilmente, poderia ter feito. Gelfman e Figner eram
tão profundamente femininas que sofreram mais com a falta de beleza e
delicadeza indispensáveis à vida de uma mulher sensível do que com todos
os horrores físicos da prisão. Kovalevskaya continuou com sua luta e rebeldia
ao longo de todos os anos que passou na prisão --- algo em torno de 22 anos. Volkenstein e Figner eram extremamente belas e femininas na
sua vida amorosa e nas relações que estabeleciam. No que diz respeito a
Spiridonova, ela foi submetida às torturas mais terríveis, o que inclui
ser estuprada por oficiais russos embriagados e ter o seu corpo nu
queimado com charutos acesos --- muito embora, nem assim, qualquer som
de lamento tenha saído da sua boca. Apesar disso, Spiridonova é uma pessoinha delicada e
frágil, profundamente apaixonada pelos seus camaradas e tão sensível
quanto uma flor. Esses poucos exemplos devem ser suficientes para
convencer qualquer pessoa que não esteja submersa no sectarismo ou nas
velhas noções ultrapassadas acerca da natureza da mulher, de que é possível ser
uma mulher absolutamente feminina e, ao mesmo tempo, uma grande rebelde
e combatente. Nesse sentido, eu poderia seguir enumerando mulheres de
todos os países, de todas as idades e de todos os climas, que se puseram
lado a lado com os homens nas grandes lutas pelos direitos humanos e
por sua própria emancipação; mulheres que, certamente, eram tão corajosas e
ousadas quanto os seus companheiros e, ainda assim, não tinham nenhum
traço de masculinidade ou homossexualidade em si.

E isso me leva à conclusão absurda que von Levetzow tirou da
grandiosidade trágica do último encontro entre Dombrovski e
Louise Michel nas barricadas.\footnote{Jaroslaw Dombrowski (1836--1871)
  era polonês, e militar do Exército Imperial Russo. Após participar, em
  1863, de um levante mal sucedido pela independência da Polônia, fugiu
  para Paris e lá estabeleceu conexões com os radicais. Participou
  ativamente da Revolução de 18 de março de 1871, vindo a se tornar o
  comandante militar da Comuna de Paris, pouco antes da invasão do
  exército francês que pôs fim à Comuna. O massacre ficou conhecido como
  \textit{Semana sangrenta}. Dombrowski morreu de complicações decorrentes
  dos ferimentos sofridos nas barricadas.} O autor é tão limitado pela
sua concepção masculina da mulher que simplesmente não é capaz de
entender como duas pessoas, ante a iminência do colapso da causa que
amavam mais do que a própria vida, poderiam se tratar como camaradas.
Ele observa: ``se Dombrovski tivesse visto uma mulher em Louise, ele
teria lhe dado uma tapinha na bochecha; mas, ao invés disto, ele
estendeu ambas as mãos e apertou as dela no seu último adeus.'' Fico
realmente surpresa que um homem com a sensibilidade de von Levetzow seja
capaz de tamanha vulgaridade. Prefiro pensar que isso seja fruto da sua profunda
ignorância sobre o tipo de relação maravilhosa que existia e ainda
existe entre homens e mulheres envolvidos na luta por um ideal ou que
compartilham uma causa comum. Pois a verdade é que, muitas vezes, a
consciência da diferença de sexo é simplesmente obliterada entre eles;
eles são camaradas, capazes do mais alto sacrifício um pelo outro e de
grande devoção um ao outro. Aqui, mais uma vez, eu teria que listar
todos os países e climas que deram ao mundo uma camaradagem tão bonita
entre homens e mulheres, mas o espaço não permite. Só estou levantando
este ponto para enfatizar o absurdo dos argumentos do autor de ``Louise
Michel.''

Louise Michel odiava os babados das roupas femininas e todos os demais
requisitos que compõem a fragilidade da mulher numa sociedade
pervertida, além disso era também descuidada e desorganizada. No que diz
respeito ao suposto primeiro argumento, devo fazer um esclarecimento a
von Levetzow. É verdade que muitas mulheres que se emanciparam da
vergonha do passado desenvolveram uma nova concepção de beleza e
elegância ao se vestir. Mas, na maioria dos casos, tudo isso foi apenas
um protesto contra os trapos, o desperdício e a estupidez de toda a
parafernália que compunha as vestimentas das mulheres comuns. De um
ponto de vista científico, sociológico e moral, essas mulheres
insistiram que a marca da escravidão do seu sexo eram suas roupas e que
elas não poderiam ser realmente livres a menos que transvalorizassem o
valor das coisas que promoviam a sua escravidão. Todas essas mulheres
são homossexuais? Não mais do que Louise o foi. Louise, que dedicou sua
vida à causa da humanidade, que não apenas se engajou na luta pela
existência de sua mãe e de si mesma, mas, acima de tudo, no movimento
que absorveu a maior parte dos seus pensamentos e toda a sua
energia.\footnote{Filha ilegítima e de família humilde, Michel era muita
  próxima da mãe que tinha uma saúde frágil. Mesmo na prisão
  continuou a lhe dar suporte, obtendo, em 1884, autorização para ficar
  ao lado da mãe nos seus últimos dias de vida.} Deveria ela, então, ter
passado horas na frente do espelho, explorando costureiras e torturando
vendedoras em nome da busca vã pelos modelos mais recentes? Deve ela,
portanto, ser considerada uma uraniana, somente porque se vestiu com sensatez e
prestou pouca atenção no que comumente é considerado belo numa mulher?
Na verdade, se o autor não possui provas melhores do que essa sua
alegação, ele deveria ter se abstido de decifrar o seu caso.

Louise Michel possuía um corpo retangular e tinha traços masculinos. Não
é verdade, como diz o autor, que ela tinha uma voz masculina. Eu a ouvi
falar, quando ela tinha 66 anos de idade; sua voz era de contralto,
bela, profunda e melódica, e atingia diretamente o coração dos seus
ouvintes. Havia uma notável simplicidade que explica o grande poder que
ela tinha sobre o seu público. Quanto ao seu rosto, é bem óbvio para mim
que von Levetzow nunca teve a oportunidade de ver Louise sorrir. Caso
tivesse tido não teria visto nela um homem. O efeito iluminador que o
sorriso e os belos olhos de Louise Michel criavam era uma unanimidade
entre todos aqueles próximos a ela. Este argumento, também,
parece muito fraco.

Aqui, chegamos à afirmação mais importante de von Levetzow. Segundo o
relato do autor, Louise recusou dois pretendentes por não se sentir
atraída por homens. É bastante significativo, porém, que isso tenha
acontecido aos 12 e 13 anos de idade e que ambos os homens, seus
pretendentes, tinham idade suficiente para serem seu pai; acrescente"-se
a isso o fato de que na verdade eles foram comprá"-la. Ela mesma expressa
sua indignação e repulsa contra esses homens, na página 330. Também a
sua atitude em relação à instituição casamento é muito significativa
(página 332).\footnote{Goldman está se referindo aqui a trechos da
  autobiografia de Louise Michel, \textit{Mémoires de Louise Michel}
  (1886), citados no ensaio de von Levetzow. A paginação a que ela
  remete o leitor diz respeito ao artigo de von Levetzow
  (\textit{Jahrbuch für sexuelle Zwischenstufen}, vol. \textsc{vii}, 1905 p.\,307--370).} Louise se ressentia do casamento sem amor, assim como
deveria toda mulher que se preze. Não obstante, em nenhuma passagem dos
seus escritos, ela expressou oposição ao amor sem casamento. Tampouco
ela escreveu sobre a necessidade de tornar pública uma experiência tão
profundamente íntima e privada quanto a vida amorosa de duas pessoas.
Abaixo segue a prova disso:

Louise Michel teve uma experiência amorosa com um professor quando era
muito jovem e trabalhava, também ela, como professora. Mais tarde, após
retornar da Nova Caledônia, ela viveu por um tempo com um camarada
belga. E se não teve mais experiências desse tipo foi, provavelmente,
porque, conforme ela mesma declarou: ``eu dei o meu coração à
Revolução.'' Sim, esse era o amor de Louise. Ao longo de toda a sua
vida, ela foi dedicada a esse amor. Tipos como Louise Michel não podem
ter um amor pessoal, que venha a interferir na sua grande
paixão pelo ideal, e a paixão de Louise pelo seu ideal foi o elemento
mais poderoso de sua vida --- que se manteve aceso até o seu último suspiro.

É verdade que ela tinha várias amigas que amava, mas não da maneira que
von Levetzow supôs. Muitas mulheres modernas que não têm muito romance na
sua vida pessoal, dedicam"-se à camaradagem e à amizade com
pessoas do seu sexo, assim como acontece com muitos dos grandes homens.
A razão para isso, no caso das mulheres, é que elas encontram uma
compreensão mais profunda entre os membros do seu sexo do que
com os homens do seu tempo. O problema é que o homem moderno ainda se
assemelha demais ao seu antepassado Adão, não diferindo muito, na sua
atitude em relação à mulher, de um homem mediano qualquer. Por outro
lado, a mulher moderna já não se satisfaz com um homem que seja tão
somente seu amante; ela quer compreensão, camaradagem, quer ser tratada
como um ser humano, e não como um objeto de gratificação sexual. Uma vez
que ela nem sempre pode encontrar isso no homem, ela se volta para suas
irmãs. É precisamente \textit{porque não há o elemento sexual} entre elas
que podem compreender melhor uma a outra. Em outras palavras, ao\label{louise}
invés de se sentir atraída por suas amigas mulheres devido a
tendências homossexuais, Louise se atraía por elas justamente porque era
uma mulher e precisava da companhia de mulheres. Há mais uma coisa que
era muito proeminente em Louise Michel: o seu instinto materno.
Apaixonada por crianças, ela era extremamente maternal com todas aquelas desamparadas
que conseguia ajudar; foi o amor materno que a levou
a adotar Charlotte Vauzelle, para educá"-la e compartilhar com ela os
seus escassos proventos.\footnote{A adoção que Goldman se refere se deu
  no sentido emocional e intelectual (e, segundo o que coloca aqui,
  também financeiro), mas não legal.} Charlotte nunca foi, nem mesmo no
sentido espiritual, a namorada de Louise Michel. O fato é que Louise
pagou caro por sua dedicação a Charlotte. Juntamente com o irmão,
Charlotte tornou os últimos anos de vida de Louise Michel miseráveis. De
certo modo, eles a mantiveram prisioneira, não a deixavam em paz fosse
para receber seus amigos ou para ir morar com eles. Charlotte violava a
correspondência de Louise e a observava atentamente. A razão de tudo
isso é que Charlotte e seu irmão viviam às custas de Louise Michel e
tinham um medo mortal de que mais alguém pudesse se beneficiar da sua
grande generosidade. Qualquer que fosse o caso, é ridículo considerar
que Charlotte Vauzelle era a namorada de Louise Michel.

O amor de Louise Michel pela escultura e sua apreciação pela música de
Wagner são apresentados como evidência de traços homossexuais. Von
Levetzow afirma que nenhuma mulher é capaz de criatividade e gosto
artístico e musical. Generosamente, ele chega a admitir que Francisco
Holmés, uma mulher francesa de origem escandinava, era uma grande
compositora.\footnote{Provavelmente, Goldman está se referindo aqui a
  Augusta Mary Anne Holmès (1847--1903), compositora e pianista francesa
  de ascendência irlandesa.} Seja como for, ele se apressa em
acrescentar que, de acordo com a sua fotografia, também ela parece masculina.
Não acho que valha a pena entrar nesse argumento. O fato de haja poucas
mulheres que sejam grandes compositoras não torna as que são,
homossexuais. Elas são simplesmente pioneiras em um domínio que até
então não foi explorado por muitas mulheres. Quanto ao amor por Wagner,
a verdade é que, em geral, as mulheres ouvem a música wagneriana e a
entendem muito mais e melhor do que os homens. Talvez isso ocorra dessa
maneira porque o espírito ilimitado, fundamento de toda a música de Wagner,
afeta as mulheres como uma força libertadora das emoções reprimidas e
ocultas das suas almas. Não parece ser necessário enfatizar que as
mulheres não são apenas capazes de apreciar a escultura, mas que há um
grande número de mulheres escultoras de mérito não desprezível.

Há, porém, uma coisa em que concordo plenamente com von Levetzow: é
quando ele afirma que Louise Michel estava tão intrinsecamente envolvida
com o anarquismo que, para compreender a sua personalidade e sua natureza
complexa, é preciso também adentrar na discussão da sua filosofia
social. Mas, como ele mesmo diz, o \textit{Jahrbuch} não é lugar para
isso. Ainda que fosse, não creio que o autor estaria em posição de fazer
uma análise do anarquismo, já que ele parece não saber absolutamente
nada sobre o assunto. Afinal, se não for isso, como é possível entender
a sua interpretação contida na página 315 (linhas 4 a 7)? Que insulto à
memória de Louise Michel e à inteligência dos leitores do
\textit{Jahrbuch}!\footnote{Ao especular sobre quem seria o pai de Louise
  Michel, von Levetzow chega à conclusão que ele seria integrante de
  alguma família proprietária de terras de antiga linhagem feudal. Daí
  formula a passagem aqui execrada por Goldman. Segundo ele, essa
  ascendência não é, de modo algum irrelevante, ``porque a história do
  anarquismo parece sugerir que pelo menos uma boa porcentagem de sangue
  tirano é necessária para produzir uma vila anarquista.''}

Conheço Louise Michel já há muitos anos. E muito antes de conhecê"-la,
conhecia as suas ideias e o preço que ela pagou ao lutar por elas. A sua
força, o seu martírio e, mais do que isso, o seu amor ilimitado pela
humanidade eram para mim como uma chama que me purificava e iluminava.
Eu a conheci pela primeira vez em 1896, em Londres; na época,
encontramo"-nos com frequência e foi ela quem me ensinou sobre a história
da luta heroica da Comuna de Paris. Louise nunca falou de si mesma e
sobre o papel que ela desempenhou nessa luta.

Eu estive com ela novamente em 1899, em Londres, e depois, em 1900,
quando pela primeira vez, depois de muitos anos, Louise Michel retornou
a Paris. Foi durante esses dois períodos que eu tive a oportunidade de
conviver bastante com ela e ouvir ela contar alguns episódios da sua
vida, já que era minha intenção escrever uma biografia sua. Mas ela era
tão excessivamente reticente em relação a tudo que dissesse respeito a
si mesma que relutava em discutir a sua vida. Por outro lado, quando
falava dos outros, ela sempre ficava radiante, era como se o seu rosto
se iluminasse com um brilho divino: fosse sobre os seus camaradas, quem
ela auxiliou e cuidou enquanto estava na Nova Caledônia; ou, mesmo sobre
as criaturas irracionais. Dentre as características de Louise Michel
estava a sua grande simpatia pelos animais. A casinha em que ela vivia
em Londres era como que uma coleção perfeita de gatos e cães abandonados
que ela recolhia à noite, quando a caminho de casa. O seu amor por
gatos, em especial, era certamente qualquer coisa, menos masculino. É
verdade que von Levetzow relata o fato de que Louise, quando nas
barricadas e rodeada por balas, resgatou um gato que estava paralisado
pelo medo. A história não mencionou ainda nenhum homem que, em perigo,
tenha feito uma coisa dessas. Não quero dizer que um homem não se poria
em risco para salvar uma criança ou até um cachorro; certamente, porém,
nunca um gato.

A supostamente masculina Louise Michel, que era desorganizada e incapaz
de se arrumar, enfim, que não era uma doméstica, ainda assim aprendeu a
tricotar, costurar, lavar e cozinhar para seus companheiros exilados na
Nova Caledônia, além de cuidar deles quando doentes com toda a ternura
do seu grande coração de mãe, e de dar suporte aos seus espíritos quando
o mais terrível havia acontecido em suas vidas.\footnote{Após a
  repressão sangrenta que deu fim à Comuna de Paris, Louise Michel e
  muitos dos seus companheiros, aproximadamente 4500, foram exilados na
  Nova Caledônia.}

Lembro"-me de uma noite maravilhosa em Paris. Amigos anarquistas
organizaram um pequeno jantar para Louise, para o qual fui convidada.
Louise, vestida de preto como habitual, com apenas uma gola de renda e
punhos da manga brancos, para dar algum alívio, tinha o rosto tão corado
quanto as rosas sobre a mesa, o qual, por sua vez, era como que
emoldurado pelos seus cabelos encaracolados, cor de prata. Ela estava
radiante de alegria por estar de volta à cidade dos seus sonhos e lutas,
cercada por seus amigos íntimos. Ela estava mais falante do que em
qualquer outro momento em que antes eu a ouvira, mais disposta a nos
deixar olhar o interior da sua alma. Em nenhum momento, Louise
demonstrou, sequer remotamente, alguma característica masculina, como
tampouco tendências homossexuais. Estou certa de que tais indícios não
teriam me escapado se, de fato, estivessem presentes em Louise Michel,
posto que, como eu disse no começo do meu artigo, fiz um estudo sobre
homossexualidade a partir do que há de melhor na literatura, conheço muitos
homossexuais e facilmente detecto inclinações homossexuais nas pessoas.
Não havia vestígios de homossexualidade em Louise Michel.

Entre os amigos de Louise Michel estavam os maiores homens do seu tempo:
Piotr Kropotkin, Malatesta, Elisée Reclus, Malato, Rocker. Alguns deles
moravam bem perto dela, em contato quase diário. Houvesse o menor
indício de homossexualidade, eles teriam percebido; certamente isso
seria do conhecimento dos seus camaradas. Recentemente, falei com meu
amigo e camarada, Rudolf Rocker, sobre esse aspecto tratado no artigo de
von~Levetzow. Ele também me garantiu que nunca, em nenhum momento,
nenhum dos amigos íntimos de Louise Michel viu o menor indício de
inclinações homossexuais. Diga"-se de passagem, inclusive, que Rudolf
Rocker, assim como eu, está absolutamente livre de qualquer preconceito
em relação aos homossexuais. Nosso único desejo é retratar Louise Michel
tal como ela realmente era --- uma mulher excepcional, dotada de grande
intelecto e espírito maravilhoso. Ela representava o novo tipo de
feminilidade, ainda que tão antigo quanto a raça, tão sábio quanto o
tempo e dotado de uma alma repleta de amor pela humanidade. Em suma, uma
mulher completa, livre do preconceito e dos costumes que a mantiveram, por séculos, em cativeiro, condenando"-a à posição de escrava
doméstica e sexual. Com Louise Michel surge a nova mulher que é capaz
dos atos mais heroicos, ao mesmo tempo em que continua sendo uma mulher
nas suas paixões e vida amorosa.

Caro dr.\,Hirschfeld, tentei, da maneira mais concisa possível, analisar
criticamente as alegações de Frhr. von Levetzow. Tenho certeza que
concordará comigo que nem a questão da homossexualidade, como tampouco
os homossexuais ganharão qualquer coisa por meio da distorção dos fatos.
Foi por esse motivo que me comprometi a refutar os erros do artigo e
jamais por qualquer outro. Espero que considere minhas críticas
convincentes e que não apenas publique a minha réplica, como já
gentilmente me ofereceu, mas que também retire a fotografia de Louise
Michel da sua galeria de uranianos.

\medskip

Atenciosamente,

\smallskip

\textit{Emma Goldman}



\chapter[Mulheres heroicas da Revolução Russa]{Mulheres heroicas da\break Revolução Russa\footnote{Texto não publicado em
  vida, a data de 1925 é aproximada.}}
\markboth{Mulheres da Revolução}{}

A Rússia pré"-revolucionária tornou"-se única na história mundial pelo
grande número de mulheres excepcionais e heroicas que contribuíram para
o movimento da sua libertação. Teve início com as esposas dos
\textit{decembristas}, os primeiros rebeldes políticos contra o czarismo
autocrático, quase um século atrás, que seguiram voluntariamente os
seus maridos rumo ao exílio na Sibéria, até o último dia da dinastia dos
Romanov.\footnote{Ao optarem por seguir seus maridos no exílio à
  Sibéria, as esposas dos \textit{decembristas} tiveram de abdicar dos títulos de
  nobreza, das propriedades e dos próprios filhos que não poderiam levar consigo. Quanto aos que nascessem no exílio, não poderiam ter o nome do
  pai, sendo, portanto, legalmente considerados ilegítimos. No que diz respeito ao retorno à Rússia, só seria permitido
  após a morte do marido. Na Sibéria, as esposas dos \textit{decembristas} desenvolveram um intenso trabalho de apoio aos presos políticos e na educação dos jovens e crianças das
  aldeias.} As mulheres russas participaram de todas as formas de
atividade revolucionária e foram em direção à morte e à prisão com um
sorriso nos lábios.

No poema vívido e pujante, ``Mulheres russas'', o poeta Nekrassóv pagou
um alto tributo à força e ao valor dessas mulheres que sacrificaram a sua
riqueza, posição social e cultura para seguir o caminho árduo ao longo
das planícies congeladas do norte, a fim de compartilhar o destino cruel
dos seus maridos presos e exilados. Após Nekrassóv, foi a vez de Ivan
Turguêniev que, com sentimentos elevados e admiração empática, pintou a
imagem das mulheres russas revolucionárias do seu tempo. No seu soberbo
poema em prosa, intitulado ``No limiar'', ele imortalizou as idealistas
ardentes, as mulheres russas do tipo de Sofia Perovskaia, cuja fé
apaixonada e devoção altruísta à liberdade iluminaram, tal qual um
farol, o horizonte sombrio da Rússia do início dos anos
1880.\footnote{Sofia Perovskaia (1853--1881) era um dos membros mais
  destacados do relativamente conhecido grupo terrorista russo \textit{Naródnaia Vólia} ou \textit{Vontade do povo}, responsável pelo assassinato do czar
  Alexandre \textsc{ii}, em 1881. Foi enforcada em praça pública, junto a outros
  companheiros no mesmo ano. Primeira mulher a ser executada na
  Rússia moderna.}

A Revolução de Fevereiro de 1917 abriu as portas da prisão para os
sobreviventes da tortura, do calabouço e do exílio siberiano
distribuídos pelo sistema czarista aos seus oponentes políticos. Em
triunfo, dezenas de revolucionários das gerações mais jovens foram trazidos de volta a Moscou e Petrogrado;
dentre eles encontravam"-se
nomes reverenciados como os de Maria Spiridonova, sua amiga de toda vida
Alexandra Izmailovitch, além de Irena Kakhovskaia, Evgenia Ratner e Olga Taratuta
--- representantes de vertentes políticas várias, mas todas inspiradas no
amor ao povo e na devoção à sua causa.

Olga Taratuta, filha de intelectuais, embora dotada de físico delicado e franzino,
possuía uma mente poderosa e foi, em certo sentido, uma pioneira. Quando
tinha apenas vinte anos, organizou, com vários amigos, o primeiro grupo
anarquista do sul da Rússia. Era um empreendimento perigoso, e suas
atividades logo atraíram a atenção da polícia política. Presa no início
da Revolução de 1905, Olga foi condenada a trinta anos de \textit{katorga} --- trabalho
forçado na prisão ---, em Odessa. Engenhosa e ousada, conseguiu escapar, e
retomou o seu trabalho anterior de militância, dessa vez, com um nome falso. Durante
um tempo considerável, todos os esforços da polícia para encontrá"-la não
renderam frutos, mas em 1906 seu disfarce foi descoberto: ela foi presa
novamente e sentenciada, mais uma vez, a trinta anos de prisão. Quando
retornou à liberdade, em 1917, Olga dedicou-se ao trabalho político na
Cruz Vermelha, auxiliando as vítimas do regime imposto pelo comandante militar
Skoropadski,\footnote{Com a revolução de fevereiro de 1917, o general
  conservador Pavlo Skoropadski organizou uma reação na Ucrânia,
  atribuindo"-se o título de Hetman da Ucrânia, o qual ele pretendia que se tornasse
  hereditário. Apesar da pretensão, não se manteve no poder para além do ano de 1918.}
na Ucrânia, e, logo em seguida, aos cuidados dos novos grupos
de presos políticos criados pelo Estado Comunista.

Nos últimos anos dos 1920, uma conferência com anarquistas de toda
Rússia estava sendo organizada na Cracóvia. Embora
a organização desse encontro contasse com o conhecimento e consentimento do
governo soviético, todos os delegados foram presos na véspera da
conferência, sem aviso ou explicação. Em meio às centenas de
prisioneiros estava também Olga Taratuta. Ela foi mandada para a prisão
de Butirka, em Moscou, o mesmo presídio em que tantos dos seus
companheiros haviam sofrido e morrido nos dias do regime Romanov. Lá,
Olga foi obrigada a vivenciar a experiência mais dolorosa de sua vida
cheia de acontecimentos notáveis. Na noite do dia 25 de abril, a ala dos
presos políticos foi invadida pela Tcheka.\footnote{Tcheka é o acrônimo
  da Comissão Especial Pan"-Russa de Luta contra a Contrarrevolução e a
  Sabotagem, primeira forma de organização da polícia política secreta
  soviética.} Atacados enquanto dormiam, após
uma série de maus tratos, os prisioneiros foram arrastados para a estação ferroviária --- com alguns deles vestindo nada além das suas roupas noturnas --- e de lá
transferidos para outros presídios.

Olga se viu, então, encarcerada na temida prisão de Orlov, que, sob o
governo de Nicolau \textsc{ii}, era o polo principal de \textit{distribuição} dos
presos. A administração e o regime daquela prisão eram tais que,
rapidamente, levaram os presos políticos a uma greve de fome como
protesto contra o tratamento que estavam recebendo. Olga foi novamente
removida para outra prisão, e de lá enviada para o exílio na região
sombria de Veliki Ústiug, sendo, por fim, enviada para a prisão de Kiev,
a mesma onde antes havia se dedicado, tão diligentemente, aos
cuidados dos comunistas presos na reação liderada por Skoropadski. Uma
carta recente de Olga a um amigo que vive no exterior contém uma
observação significativa: a de que a perseguição imposta pelo governo
soviético lhe roubou mais vitalidade do que todos os anos de
encarceramento que ela sofreu nas mãos da autocracia Romanov.

Diferentemente de Olga Taratuta, a maioria das outras heroínas da
Revolução Russa tem origem proletária. Dentre elas, Leah Gotman e Fanya
Baron são duas mulheres anarquistas de personalidade excepcional. Na sua
adolescência, ambas partiram da Rússia em direção à América, onde foram
empregadas nas fábricas e participaram ativamente do movimento dos
trabalhadores. Eu conheci bem essas garotas, espécimes esplêndidos da
independência feminina, belas, dotadas de sentimentos elevados e
intelecto forte. À primeira chamada da Revolução de Fevereiro, essas
duas jovens, como muitos refugiados russos, voltaram correndo para a sua terra
natal. Foi assim que ajudaram a realizar a Revolução de Outubro. Leah e
Fanya sentiram que o seu lugar era ao lado do proletariado; trabalharam,
particularmente, com os mujiques do sul, em meio aos agricultores da
Ucrânia, a quem deram todo amor e devoção inerentes às suas naturezas
tão abundantes. Posteriormente, também desenvolveram uma série de
atividades culturais com os camponeses rebeldes liderados pelo famoso
Bat'ka ou \textit{pequeno pai}, Nestor Makhno.\footnote{Nestor Makhno
  (1888--1934) é um dos principais nomes do anarquismo na Ucrânia. Como
  muitos revolucionários presos pelo regime czarista, foi libertado na
  ocasião da Revolução de Fevereiro. Lutou ao lado dos bolcheviques, na guerra civil que se seguiu à Revolução, na condição de comandante de um exército independente, o chamado Exército Negro. Após a bem-sucedida parceria, a tensão crescente entre os bolcheviques e os anarquistas \textit{makhonistas}, como
  ficaram conhecidos os membros do movimento,
  concluiu"-se, em 1920, com a invasão do território \textit{makhonista} pelo
  Exército Vermelho, que estendeu a sua perseguição implacável aos
  anarquistas como um todo.}

As mãos do Kremelin, uma vez erguidas contra Makhno, também caíram
pesadamente sobre Leah Gotman e Fanya Baron. Ambas foram presas na
véspera da conferência na Cracóvia, mencionada acima, e mandadas para a
prisão de Butirka, onde também foram vítimas do ataque da Tcheka, na
noite de 25 de abril de 1920. Arrancada da cama na calada da noite, Leah
foi arrastada ao longo das escadas pelos cabelos e forçada a permanecer por
horas, semivestida, no pátio da prisão junto a outros presos políticos,
enquanto esperava ser transferida a algum destino desconhecido. Ela
permanece presa desde então, sendo agora uma das desafortunadas detentas
do terrível Mosteiro de Solovetsky, situado na zona do Ártico.

Fanya Baron, que sempre me impressionou pela sua coragem ilimitada e
espírito excepcionalmente generoso, pertence ao tipo raro de mulher que
pode realizar as mais difíceis tarefas da prática revolucionária com
calma, graça e altruísmo absolutos. Após o ataque em Butirka, ela foi
transferida para a prisão de Riazã, de onde logo escapou, voltando a pé
para Moscou, sozinha. Chegou lá sem dinheiro e praticamente
sem roupas, sendo obrigada, por conta dessa condição desesperadora, a
buscar refúgio com o irmão do marido, o que a conduziu direto para a Tcheka.
Essa mulher de grande coração, que serviu à causa da Revolução por toda
a vida, foi morta pelo partido que finge ser a guarda avançada da
Revolução. Não satisfeitos com o assassinato de Fanya Baron (em setembro
de 1921), os comunistas mancharam com o estigma de \textit{banditismo} a
memória de sua vítima assassinada.

Não apenas os anarquistas, mas todos os membros de qualquer outro grupo
político, com os social"-revolucionários de direita e de esquerda, os
mencheviques, os maximalistas e mesmo os comunistas, tiveram de pagar
o mais alto tributo ao rolo compressor em que se converteu a autocracia comunista. A partir de agora, nomearei algumas das personalidades mais extraordinárias.

Evgenia Ratner, uma jovem de mente afiada e personalidade forte,
ingressou no Partido Socialista"-Revolucionário, logo após completar seus
estudos em medicina na Suíça. Suas atividades, depois que retornou à
Rússia, envolveram"-na repetidamente em dificuldades para com as
autoridades czaristas, até que foi condenada a muitos anos de prisão.
Libertada pela Revolução de Fevereiro de 1917, sua habilidade e força
excepcionais a levaram a ser eleita como membro do comitê central do seu
partido, ao mesmo tempo em que foi escolhida pelo campesinato como
um dos seus representantes no Soviete de Moscou. Logo após seu partido
ser banido pelos bolcheviques, Evgenia foi presa em 1919 e julgada, em
1922, juntamente com 11 dos seus camaradas. Todos foram condenados à morte.

A intercedência do mundo ocidental, com um protesto internacional de
grande fôlego contra a execução da sentença --- protesto apoiado por homens da estatura de
Anatole France, Romain Rolland e outros ---, salvou as vidas dos 12
social"-revolucionários, Evgenia Ratner entre eles. Ela agora leva uma
vida miserável na prisão de Butirka.

Dos social"-revolucionários de esquerda, Irina Kakhovskaia,
Alexandra Izmailovitch e Maria Spiridonova sofreram o maior martírio.
Kakhovskaia, neta do general Kakhovski, o famoso rebelde \textit{decembrista}
que se opôs à autocracia de Nicolau \textsc{i}, é uma mulher reconhecida pelos
seus dotes literários e idealismo revolucionário. Ela começou seu
trabalho nos movimentos de libertação da Rússia quando era muito jovem,
em 1904. Não demorou muito, foi presa e sentenciada a 20 anos de
katorga, sendo, posteriormente, transferida para a prisão de Akatuy, um dos lugares
mais temidos do exílio czarista. Em 1914, foi autorizada a viver, em
exílio, na região da Transbaikalia,\footnote{A prisão (katorga) de Akatuy ficava, naquele tempo, situada nessa região da Transbaikalia.} do qual foi libertada pela Revolução
de Fevereiro de 1917.

Após o seu retorno do exílio, Irina Kakhovskaia tornou"-se uma das
trabalhadoras mais valiosas do Partido Socialista"-Revolucionário de
Esquerda, era muito respeitada pela sua compreensão de psicologia
camponesa e das necessidades do proletariado. Após o Tratado de
Brest"-Litovsk e da ocupação alemã da Ucrânia, as autoridades alemãs
prenderam Irina como participante da conspiração contra a vida do
general Eichorn, governador militar da Ucrânia, assassinado por um
dos membros do seu partido, B.\,Donskoy. Kakhovskaia foi torturada e
condenada à morte. Felizmente, a eclosão da revolução na Alemanha
impediu a sua execução e ela foi salva.

Irina continuou a trabalhar nas suas convicções políticas, mas, em 1921,
ela foi presa novamente, desta vez pelos bolcheviques, por quem foi
exilada em Kaluga, na Sibéria.

Enquanto estava na prisão, Irina Kakhovskaia escreveu as suas
interessantíssimas memórias, a história incomum de uma personalidade
muito singular. Romain Rolland, depois de examinar o seu trabalho,
disse: ``Sou contrário às ideias de Kakhovskaia, mas sua narrativa tem
uma qualidade humana, ou antes, sobrehumana, muito cativante. É um
documento psicológico do valor mais elevado que pode haver. A
simplicidade absoluta da narradora, sua habilidade, genuinamente russa,
de ver de modo objetivo, sua energia inacreditável inteiramente dedicada
à causa que tem no coração --- tudo isso desperta profunda admiração no
leitor, independentemente de qual seja a sua posição em relação ao valor
da ação realizada ou almejada. Que heroísmo, paciência, completa
abnegação, que tesouros da alma, a humanidade desperdiça com propósitos
terríveis e vergonhosos.''

Alexandra Izmailovitch, filha de um general do exército russo, é outro
exemplo de uma jovem que a autocracia dos Romanov impeliu a atos
individuais de violência como única forma de protesto possível sob um
regime despótico. Em 1906, ela atentou contra a vida do governador
Kurlov da província de Minsk, responsável pela maior parte dos pogroms
diabólicos contra os judeus. Condenada à Sibéria por toda a vida, ela
foi libertada, com os outros ativistas políticos, em 1917. Na condição de
membro do Partido Socialista"-Revolucionário de Esquerda, tornou"-se uma
das lideranças do Soviete de Deputados Camponeses de Toda Rússia.
Quando os bolcheviques decidiram liquidar o seu partido \textit{para
sempre}, em 1919, ela foi presa juntamente com vários de seus
companheiros, permanecendo, quase que ininterruptamente, na prisão desde
então.

O aspecto mais característico dessa mulher extraordinariamente dotada e
enérgica é a devoção, ao longo de toda a sua vida, à sua amiga e
camarada Maria Spiridonova. Elas passaram 11 anos juntas na Sibéria,
retornaram à Rússia para unir seus esforços na libertação do povo, e
foram presas pelo governo bolchevique. Desde então, já há muitos anos,
estão presas. Não é exagero dizer que o cuidado e a devoção de Alexandra
Izmailovitch à amiga são as principais causas de Maria Spiridonova ainda
estar entre os vivos.

Maria Spiridonova é, sem dúvida, uma das figuras mais notáveis ​​e
heroicas do movimento revolucionário russo dos últimos vinte anos. De
família aristocrática, bela e culta, a jovem Maria abandonou o luxo e
sua posição social para se dedicar à causa dos oprimidos.
Bem"-intencionada e empática, ela não pôde suportar sem protestar a
injustiça e tirania que testemunhava de todos os lados. Com 18 anos,
atentou contra a vida do general Lukhanovsky, governador da província de
Tambov, que foi universalmente execrado pela sua selvageria asiática
contra o campesinato.

Os czares russos nunca foram parciais no tratamento às ativistas
políticas: eram igualmente implacáveis ​​com todos os seus oponentes,
fossem homens ou mulheres. Mas no caso de Maria Spiridonova, os capangas
de Nicolau \textsc{ii} superaram até os métodos de Ivan, o Terrível. Depois de
presa, Maria foi espancada até o ponto da insensibilidade, as suas roupas
literalmente arrancadas do corpo, e a jovem foi entregue para que uma guarda
embriagada encontrasse diversão em queimar sua carne nua com cigarros
acesos. Depois de semanas à beira da morte, Maria foi condenada à
execução.

A tortura sofrida por Spiridonova comoveu todo o mundo ocidental, cujos protestos
a salvaram do cadafalso. Ela foi \textit{perdoada}, tendo a sua pena comutada
por prisão perpétua na Sibéria. Os efeitos da sua experiência terrível
provocaram lesões nos pulmões, aleijamento numa das mãos e perda da
visão em um olho. Não obstante, ainda que fisicamente debilitada e
destroçada, seu espírito permaneceu em chamas.

Poucos ativistas políticos receberam uma ovação popular tão unanime, da
Sibéria a Petrogrado e Moscou, como Maria Spiridonova, após sua
libertação da prisão em 1917. Ela, porém, não desperdiçou um único
segundo com o gozo da sua recém"-conquistada liberdade. Dedicou"-se ao
trabalho com todo o ardor característico à sua personalidade intensa,
organizando os camponeses, inspirando e dirigindo as energias então
despertas do povo russo. Tornou"-se a líder adorada dos vários milhões de
camponeses da Rússia, a alma de todas as suas aspirações de longa data e
a porta"-voz das suas necessidades e esperanças. Como a figura de maior
destaque do Partido Socialista"-Revolucionário de Esquerda, Maria exerceu
enorme influência no soviete dos camponeses, elaborou um plano
abrangente para a socialização da terra, o problema mais vital da vida
russa.

Já em 1918, Maria Spiridonova percebeu que a Revolução corria grande
risco, por conta da atuação de alguns de seus supostos amigos, e não dos
inimigos, como seria de se esperar. Ela viu a crescente autocracia do Estado comunista e se
colocou contra. A ruptura final entre o seu partido e os
bolcheviques veio com o Tratado de Brest"-Litovsk, que Spiridonova condenou
tanto por uma questão de princípio, quanto por razões práticas. Logo após
isso, ela foi presa junto a 500 delegados no Congresso dos Camponeses.

Quando cheguei à Rússia, os bolcheviques me disseram que Maria
Spiridonova havia sofrido um colapso nervoso e que, por conta disso,
teria sido alocada num sanatório, para que recebesse os melhores
cuidados. Mas logo descobri que Maria havia conseguido fugir desses \textit{melhores
cuidados} e estava vivendo em Moscou disfarçada de camponesa, como
costumava fazer nos dias de czarismo. Recentemente, a fortuna me
favoreceu e tive a oportunidade de passar vários dias com essa mulher
extraordinária. Não encontrei nela nenhum vestígio de histeria --- na
verdade, o seu equilíbrio, ponderação e objetividade ao narrar os
eventos que marcaram o seu retorno à Rússia são absolutamente
admiráveis.

Alguns meses após isso, no outono de 1920, a Tcheka retornou, mais uma vez,
à sua atividade de caçar conspirações. Numa das suas numerosas incursões
em Moscou, encontraram Maria Spiridonova doente de tifo. Ela foi presa
e removida para Ossoby Otdel --- a seção secreta da Tcheka. Em 1921,
quando estava quase à beira da morte, os esforços dos seus amigos
garantiram a concessão da sua libertação temporária --- sob a condição de
retornar à prisão assim que a sua saúde estivesse melhor. A única
alternativa era deixar Maria morrer na prisão por negligência ou
devolvê"-la --- quando recuperada a saúde --- para os \textit{melhores cuidados}.
De fato, assim que ela começou a se recuperar, a Tcheka se apoderou dela
novamente. Guardas com cães farejadores foram colocados na casa onde
Spiridonova estava sendo assistida pela sua devotada amiga, Alexandra
Izmailovitch. Todos os seus passos eram observados e a existência
tornou"-se tão insuportável que a torturada terminou por exigir ser
levada de volta à prisão. Juntamente com a sua inseparável Izmailovitch,
ela foi enviada para um dos locais mais distantes e inóspitos da
província de Moscou, de onde vêm agora as mais tristes notícias, como a
de que Spiridonova foi obrigada a recorrer ao método desesperado da
greve de fome como forma de protesto contra a implacável perseguição. De fontes
confiáveis, ​​acaba de chegar a informação de que tanto Izmailovitch
quanto Spiridonova estão exiladas nas regiões mais selvagens do
Turquestão.\footnote{Em 11 de setembro de 1941, Spiridonova e
  Izmailovitch, em conjunto com mais de 150 presos políticos, foram
  executadas via ordem pessoal de Stalin, na ocasião da invasão das
  tropas nazistas na \textsc{urss}. Essa execução em massa ficou conhecida como
  \textit{Massacre da Floresta Medvedev}.}

O martírio das mulheres heroicas da Rússia tornou"-se mais doloroso e
intenso sob a tirania da ditadura bolchevique do que quando nos dias do
czarismo. Naqueles tempos, o sofrimento era meramente físico, nada
poderia afetar o espírito. Elas sabiam que enquanto eram odiadas pela
autocracia, desfrutavam do respeito e amor das vastas massas do povo
russo. Na verdade, o \textit{povo simples} as tomava como \textit{santas} que
sofriam por sua causa; no czarismo, a influência moral exercida pelos
ativistas políticos nas prisões, \textit{katorgas} e exílios era muito grande.
Mas tudo isso mudou agora. Os novos autocratas da Rússia desacreditaram
os ideais do socialismo e macularam a honra dos seus grandes expoentes.
Não há opinião pública na Rússia que não seja a do partido governante, e\label{ref2}
os mártires --- homens e mulheres --- da Rússia revolucionária se
transformaram em párias, no sentido mais amplo que pode haver. Eles não
têm nada para os compensar, não podem sequer apelar à consciência do seu
país, pois ela foi paralisada. Inclusive, não apenas a consciência da
Rússia, mas a consciência do mundo como um todo parece silenciada.

O que aconteceu com o senso de justiça e solidariedade que
anteriormente se estendia do mundo ocidental às vítimas políticas do
regime czarista? Naquele tempo, homens e mulheres da Inglaterra, de modo
corajoso, denunciavam em seus protestos as iniquidades cometidas na
Rússia e eram solidários com os que eram perseguidos politicamente por
conta das suas opiniões. Agora, diante das evidências esmagadoras da
opressão e perseguição extremamente cruéis que ocorrem na Rússia, o
mundo está silencioso e insensível. Os mártires heroicos estão
abandonados à misericórdia da Tcheka, para sofrer o calvário do corpo e
do espírito, em nome de um ideal que há muito tempo foi traído pelo
Estado comunista com a sua ditadura do partido.

\chapter[As visões de Emma sobre o amor]{As visões de Emma\break sobre o amor\footnote{Texto originalmente publicado no
  \textit{\textsc{nea} Service}. O título e as subseções mantidas
  aqui foram dados pelo jornal. No rascunho o texto curto, sem
  subseções, é intitulado ``Minha atitude em relação ao casamento'', de 1926.}}
\markboth{visões de Emma sobre o amor}{}

% \setlength{\epigraphwidth}{.65\textwidth}
% \begin{epigraphs} 
% \qitem{}
% \end{epigraphs}
% \footnotemark{}
% \footnotetext{}

\textit{Nos tempos modernos, nenhuma mulher posicionou"-se, de
modo mais abertamente hostil, contra o casamento do que Emma Goldman,
que, há oito anos, estava entre os deportados enviados de volta à
Rússia. Ela fez uma série de palestras na América, em que tratou sobre o
tema do casamento e escreveu panfletos. De repente, veio de Montreal, no Canadá, a notícia de que Emma Goldman havia chegado lá sob o nome de
sra.\,E.\,Colton. Após uma vida de antagonismos, Emma Goldman se casou com
um trabalhador das minas. Seus seguidores ficaram surpresos e ainda
estão. O que isso poderia significar? A \textit{\textsc{nea}} enviou um representante até
ela para saber de primeira mão o que Emma Goldman agora pensa sobre o
casamento. Aqui, apresentamos um artigo exclusivo, escrito pela própria
Goldman, o segundo de uma série de cinco, no qual esboça as suas opiniões
hoje.}\footnote{Em 1887, antes da sua militância como anarquista, Goldman
  conseguiu obter a cidadania estadunidense ao se casar com o seu
  primeiro marido, Jacob Kerstner. Nos Estados Unidos de então, a
  cidadania da mulher era dependente da cidadania do seu pai ou marido.
  Numa jogada para lhe retirar o direito de permanecer ou retornar aos \textsc{eua}
  por parte das autoridades federais, a cidadania de Kerstner foi revogada
  postumamente e, com ela, a de Goldman --- que foi deportada para a
  Rússia em 1919. Seis anos após a sua deportação, em 1925, Goldman se
  casa com o cidadão escocês James Colton, de modo a conseguir um
  passaporte e alguma proteção contra a perseguição aos imigrantes.}

\begin{center}
\adforn{47}
\end{center}

Muitas pessoas se mostraram surpresas de que eu, que por tantos anos
critiquei a instituição casamento, tenha, no final, me submetido a ela. Sem exceções, elas exigem saber se mudei a opinião sustentada no
passado.

Talvez já não soe muito enfática a minha declaração de que agora, como
antes, estou convencida de que a instituição do casamento, do jeito que
é, não acrescenta nada ao que fundamentalmente une homens e mulheres.
Será sempre uma verdade para mim que a união de duas pessoas é um
assunto exclusivamente privado.

\section{bizarro antes, aceito agora}

Há não muitos anos, esse ponto de vista era considerado chocante e
revolucionário. Atualmente, porém, encontra uma aceitação muito mais
generalizada do que alguns se permitem admitir.

Mesmo os conservadores estão começando a perceber que, muito embora o
casamento possa ser conveniente, não exerce qualquer influência sobre os
impulsos emocionais ou a expressão sexual.

É um fato que o ritual do casamento, propriamente dito, não tem qualquer
relação com a vida e os hábitos dos seres humanos. Na melhor das
hipóteses, a instituição consiste na sanção pública de um acordo privado
entre duas pessoas. Nunca o Estado teve o poder de dar mais do que essa
sanção, pois é sempre o amor que domina esse tipo de relação humana.

Costumo dar risada das antigas referências ao chamado amor livre. Como
se o amor pudesse ser algo que não livre!

Atualmente, milhares de pessoas se submetem à cerimônia de casamento,
não porque acreditam nela, mas porque protege as suas vidas particulares
das \textit{bisbilhotices vulgares}.

\section{casamento: a resposta}

E assim, se vocês me veem casada hoje, vocês têm, com isso, a resposta --- se
for realmente importante ter uma resposta! Um casamento não implica
divorciar"-se do próprio ponto de vista.

O maior infrator da sacralidade da privacidade é o Estado. Desde a\label{ref3}
reação desencadeada pela Guerra Mundial, o Estado abandonou a maioria
das suas atividades para se dedicar exclusivamente ao exercício de
controle e sufocamento dos indivíduos.

Já não é mais possível ter liberdade de movimento ou de gosto. Inúmeras
restrições cercam o indivíduo da manhã à noite. O que ele come, bebe,
lê, vê, as relações que estabelece, as opiniões que ouve e com quem
se reúne --- tudo está sob vigilância constante. O resultado é que o
indivíduo se vê constantemente na necessidade de inventar métodos
através dos quais consiga escapar dos tentáculos dessas intrusões e
obstruções.

\section{a necessidade é a mãe da estratégia}

Esforços para impedir as manifestações externas sempre têm como efeito a
criação do gênio capaz de inventar modos de escape a partir do interior.
A proibição serviu apenas para aguçar a sede americana. A censura à
expressão literária teve como única consequência estimular livros cada
vez mais revolucionários. O que também incidentalmente aumentou o
interesse intelectual e a discussão dos trabalhos proibidos.

No que diz respeito a viagens, com a interferência interminável sobre 
a liberdade de movimento, imposta pelo passaporte e pelo visto, o Estado obriga as
pessoas a exercitarem toda a sua engenhosidade na arte de atravessar o
muro chinês construído ao redor do mundo durante guerra. Naturalmente, isso não
significa que as pessoas estejam morrendo de amores pelo Estado.

Em certa medida, o mesmo se aplica ao casamento. Ninguém
que seja dotado de cérebro pode persistir na crença de que o
casamento é uma obra dos céus, ou
que, uma vez encarnado, deve ter os seus limites estabelecidos por alguém além dos imediatamente envolvidos; que atravessam o processo com o mesmo espírito
de alguém que busca tirar um passaporte ou obter um visto --- para
conseguir um espaço para respirar e proteger a privacidade da
personalidade humana.\footnote{Embora no presente texto, Goldman deixe
  suficientemente claro que o seu casamento com Colton se deu por razões
  pragmáticas, muitos chamaram a atenção para a discrepância entre as
  suas declarações anteriores --- como a de que o casamento seria uma forma
  ainda mais aguda de prostituição --- e a aqui contida: segundo a qual o
  casamento é um assunto do âmbito exclusivamente privado. Há quem veja
  nessa ênfase do casamento como assunto privado (muito embora ela declarasse isso desde o  primeiro
  texto sobre a temática), uma espécie de autojustificativa, como se
  buscasse evitar cair ela mesma na sua crítica anterior: a do
  casamento por interesse, como única alternativa para a manutenção
  econômica (no seu caso, política) da mulher. Seja como for, em 1927,
  segundo o jornal \textit{Toronto Daily Star}, Goldman teria dado uma palestra
  em Toronto defendendo o que na época era chamado de \textit{casamento por
  companheirismo}, uma alternativa bem mais moderada (e um tanto
  reformista) da instituição casamento tal qual havia. Um logo rascunho
  escrito de próprio punho, aproximadamente em 1928, intitulado
  ``Casamento por companheirismo'', confirma o interesse de Goldman na
  temática, pouco depois do seu casamento com Colton.}

\chapterspecial{A luta do feminismo não foi em vão}{A conclusão de Emma Goldman\footnote{Texto originalmente publicado no
  jornal nova"-iorquino \textit{Rochester Times"-Union}, em 1926.}}{}
\markboth{A luta do feminismo}{}

\textit{A carreira bastante diversificada de Emma Goldman foi da
denúncia violenta contra a América e o casamento à sua posição atual, em
Montreal (Canadá), em que na condição de mulher casada, solicita a sua readmissão nos Estados Unidos. Também existe outra fase paradoxal na sua
vida --- quando da antiga posição de militante feminista passou a ser
uma crítica severa do sufrágio feminino. Neste artigo, o terceiro de uma
série de cinco, são apresentadas as suas opiniões sobre esse assunto de
imenso interesse --- escrito exclusivamente para o \textit{\textsc{nea} Service} e o
\textit{Times"-Union.}}

\begin{center}
\adforn{47}
\end{center}

Caso relembremos das profecias radicais das ativistas dos \textit{direitos da
mulher} sobre os milagres que o feminismo viria a realizar, quando a
mulher finalmente tivesse o direito ao sufrágio e igualdade
profissional, é impossível não admitir que os resultados do feminismo
foram tudo menos proporcionais à brava luta empunhada pelas mulheres em
nome da sua emancipação.

Não faz muito tempo, as grandes líderes feministas nos asseguraram
que o seu credo purificaria a política, aboliria a guerra, eliminaria
todos os males sociais e criaria relações completamente novas entre os
sexos. Hoje em dia, nenhuma feminista inteligente se deixaria levar por
uma conversa tão boba. Finalmente aprenderam que abusos de longa
data não podem ser eliminados com o \textit{voto de Minerva}.

E o que é mais importante: elas aprenderam que a emancipação econômica e
social das mulheres está intimamente ligada à luta pela emancipação
humana em geral --- que a total independência do homem, assim como a da
mulher só ocorrerá com a mudança completa de nossa atual estrutura
social e do valor individual e coletivo.

No entanto, é um fato que a luta heroica empunhada pelas mulheres, já há
tantos anos na América e na Europa, não foi em vão. Se ainda lhe é
negada uma remuneração equivalente pelo trabalho que realiza, a mulher
conseguiu provar que é capaz de realizá"-lo muito bem. Não existe
profissão ou ofício, nem mesmo atravessar o Canal da Mancha a
nado,\footnote{Em agosto de 1926, o presente texto foi publicado em
  novembro, Gertrude Ederle se tornou a primeira mulher a atravessar a
  nado o Canal da Mancha.} que seja estranho à mulher.

\section{destreza mais fina}

Enquanto estive na Alemanha, durante a guerra, pude observar que as
mulheres da indústria metalúrgica possuíam uma destreza maior do que os
homens na fabricação de instrumentos delicados. Que as mulheres durante
o cataclismo mundial tiveram a capacidade, e efetivamente
deram conta das tarefas mais difíceis --- enquanto os homens sangravam
nos campos de batalha ---, já não é um aspecto que precise ser
enfatizado.

Sim, as mulheres se saíram muito bem. Ninguém mais ousará
insistir que o seu lugar apropriado é exclusivamente em casa, que deve
desperdiçar sua substância como empregada doméstica ou mercadoria
sexual. Ela rompeu as grades da gaiola dourada e agora se encontra do
lado de fora, no mundo, pronta para assumir a sua parte da
responsabilidade e também para exigir o direito às suas realizações.

As mulheres americanas mais avançadas vêm agindo dessa forma já muito
antes da guerra, mas só agora, depois da guerra, é que as mulheres no
exterior estão conseguindo se recuperar. Existem pouquíssimas
\textit{gretchens}, dependentes, complacentes, obedientes e submissas na
Alemanha de hoje. As mulheres na Inglaterra tampouco estão se valendo de
sigilo e subterfúgios para lidar com a vida e as questões sociais.
Aberta e francamente, elas declaram o seu direito a todo
conhecimento e experiência existentes no mundo.

E mesmo na França, as mulheres, além do direito de amar, sobre o qual
alegam maestria, estão começando a perceber que a vida é mais do que
namoricos fúteis, que há problemas sociais muitos graves e que,
portanto, exigem a atenção das mulheres, assim como a dos homens. Em
outras palavras, em todo o mundo, as mulheres se tornaram profundamente
conscientes da necessidade de fazer a sua parte na grande luta mundial.

\section{vital em todos os domínios}

A mulher atual é, possivelmente, a força mais vital em todos os domínios
do pensamento e do empreendimento humanos.

Se tal condição é efeito da perda de vitalidade da parte de muitos homens, por
conta dos horrores da guerra, é algo que não posso saber. O que eu sei é
apenas que a maioria dos homens da classe média da Europa perdeu completamente o
controle sobre a sua vida. Eles não têm mais fé ou idealismo. Para usar
uma expressão \textit{freudiana}, a maioria dos homens de hoje parece sofrer
de um complexo de inferioridade. Ou se trataria de orgulho ferido por
não mais poder bancar o bravo cavalheiro que protegia a mulher de viver
tão perigosamente quanto ele?

De uma forma ou de outra, o ponto é que, em geral, os homens parecem
estar perdidos, \textit{desempregados}, de certa forma. Eles não sabem o que
fazer com eles mesmos na presença daquelas que antes eram consideradas
inferiores.

\section{viva, ávida e ativa}

Não é isso o que ocorre com as mulheres que conheci na Europa. Elas me
impressionaram pela completa transformação das suas qualidades físicas,
mentais, espirituais e emocionais --- um tipo novo e viril de
feminilidade, muito mais vivo, ávido, ativo e livre do que o dos homens.

Muitos fatores contribuíram para criar o tipo moderno de mulher, sendo o
fator mais vital a solidariedade das mulheres entre si. A necessidade as
ensinou, já na fase inicial da sua luta, que o escravo nunca foi liberto
pelas mãos do seu senhor e que a sua emancipação só poderia ser
conquistada pelo espírito de solidariedade entre os seus companheiros
escravos.

Do mesmo modo, a solidariedade do sexo entre as mulheres tem propiciado,
creio eu, um tremendo ímpeto e encorajamento na luta pela autoafirmação
e pelo direito do seu lugar no mundo --- o direito de ser ela mesma.

\chapter{O elemento sexual da vida\footnote{Rascunho inacabado, do qual faltam muitas
  páginas. A data de 1935 é aproximada. Ao que parece, foi
  concebido como parte de um texto de abordagem mais ampla, a tirar pelo
  título ``Sexualidade, maternidade e controle de natalidade.'' Para
  além da seção aqui traduzida, bastante fragmentada, não restaram
  mais do que duas ou três páginas do documento original. As
  poucas passagens do rascunho excluídas da presente tradução ou
  foram riscadas pela própria Goldman (há uma série de anotações à mão
  no texto, nem sempre legíveis), ou a redação não revisada inviabilizou
  a inteligibilidade do texto.}}\label{sexual}
  \markboth{O elemento sexual}{}

A verdade virá à tona algum dia. Mas, em geral, é a mentira que
persiste. A verdade é nua, direta. Não se apresenta com sorrisos
maliciosos, subterfúgios ou acordos. Veja as mentiras adornadas com seda
e joias. Elas são insinuantes, lisonjeiras e enganosas. Um grande número de pessoas,
incontáveis, ficam deslumbradas pela pompa e autoimportância com que se
transvestem as mentiras e, assim, seguem"-nas, alegremente, sem que percebam
a face irônica sob a máscara cheia de enfeites.

Não é motivo de surpresa, portanto, que a força mais elementar da vida
humana, o sexo, seja ainda hoje degradado e renegado.

As duas instituições que, há séculos, tentam subjugar o sexo, arrancá"-lo
via os métodos mais perversos, são a Igreja e a moralidade --- as
caluniadoras de tudo o que é excelente e saudável na vida. Não obstante,
quanto mais a Igreja e a moralidade tentaram subjugar o sexo e extingui"-lo
das necessidades humanas, mais intensa e devastadoramente o elemento
sexual se afirmou.

É relativamente recente que a verdade sobre o sexo rompeu a rede de
falsidades, ilusões e armadilhas que há tanto tempo assombram a mente do
homem.

Os grandes psicólogos do sexo, Havelock Ellis, Kraft Ebbing, Edward
Carpenter e Freud estavam sozinhos na sua batalha heroica contra as
mentiras sobre o sexo. Atualmente, o número pode ser contado aos
milhares.\footnote{Boa parte desses dois períodos, com os quais o
  parágrafo é iniciado, foram riscados e corrigidos à mão por Goldman.
  Como, porém, a sua caligrafia nessa passagem não está com boa
  legibilidade, optou"-se por manter tal qual datilografado.} E todos
comprovam, para além de qualquer sombra de dúvida, que as fobias, os
terrores, as neuroses e demais transtornos mentais têm a mesma origem
sexual. Mais do que isso: toda a nossa vida, desde as nossas ações mais
simples, depende completamente dos nossos impulsos sexuais. A
dissimulação imposta pela longa repressão moral nos impediu de
reconhecer essa verdade fundamental. Não obstante, na medida em que é
impossível resistir impunemente a uma força natural tão poderosa, nós
nos tornamos todos mais ou menos insanos: alguns completamente, outros
pela expressão das suas aspirações sexuais através de fobias passageiras ou
sonhos monstruosos; outros, ainda, na tentativa de se libertar dessas
aspirações, desperdiçam quase toda a sua existência como se sonhando
acordados, indiferentes à vida cotidiana. Por fim, aqueles que são mais
fortes satisfazem as suas propensões valendo"-se de uma hipocrisia
calculista que lhes permite ocultar-se dos olhos da multidão.

O crescente acúmulo de pesquisas de abordagem freudiana, as
consequências, cada vez mais amplas, da aplicação das doutrinas originais,
fornecem explicações para muitos dos problemas que costumavam nos
confundir e, frequentemente, possibilita"-nos encontrar provas materiais
da teoria da unidade de energia.\footnote{Possivelmente, Goldman esteja
  se referindo aqui à teoria da libido.} Efetivamente, vemos que o
sonho, o sono, o estado normal de consciência e as psicoses formam uma
parte integrante da personalidade humana, que reage às influências
externas via uma ou outra dessas manifestações, de acordo com a fase de
evolução da força no momento. Não é difícil concluir que, dentre as forças que
atuam sobre nós, a mais importante, se não a única, a que quase sempre
encontramos sob os mais variados disfarces, é o impulso sexual.

Precisamos admitir que não é uma possibilidade negar a realidade, porque não gostamos dela, e, assim, torna-se indispensável que
reconheçamos no tão difamado impulso sexual a força psicológica motriz
da humanidade.

%\asterisc

O falecido Prof.\,Dorsey,\footnote{George Amos Dorsey (1868--1931) foi um etnógrafo e indigenista estadunidense.} no seu livro \textit{Why we behave like human beings},\footnote{Em tradução para o português, \textit{Por que nos comportamos como seres humanos}.} chamou a atenção para o que a psicologia diagnosticou
como \textit{complexo de impureza} e nos mostrou o que está por trás desse
puritanismo descarado que anuncia a \textit{pureza} desse homem ou daquela mulher. Ele
também demonstra que a pureza do ignorante, quando adquirida pelo preço
da repressão da sua curiosidade natural, não é segura e nem sã. De outro
lado, continua o professor Dorsey, o estudo da biologia está começando a
derrubar esse \textit{complexo de impureza} e a doutrina profana e
antinatural, iniciada pelos primeiros monges cristãos, de que o impulso
sexual é o sinal de degradação do homem e a fonte da sua energia mais
diabólica. A natureza sempre sabe mais.\label{diabolica}

O sexo é uma função biológica primária a toda forma de vida superior.
Suas características e qualidades vêm de uma linhagem antiga. Seu
impulso é tão real quanto é a força que faz as marés baixarem e
subirem. Possui estrutura e comportamento profundamente suscetíveis. É o
elemento fundamental de toda a vida superior, seus caracteres externos
são um engenhoso estratagema publicitário da natureza, através do qual
ela vende seus produtos e, assim, assegura sua família.

Devemos ao sexo mais do que à poesia; devemos o canto dos pássaros, toda
música cantada e a própria voz, devemos a plumagem que atinge a glória
suprema com as aves"-do"-paraíso, devemos a juba do leão e todas as formas
superiores de vida seja do mundo vegetal ou do animal. Está tecido em
todas as estruturas da vida humana e inserido em todos os costumes. Na
conta do sexo, devemos debitar muitas tolices e estupidezes e algumas
das mais insensatas injustiças que tornam essa coisa que chamamos de
cultura humana o mosaico incrível e variado que é.

É verdade que somos mais esclarecidos do que éramos, mas ainda não
atingimos o estágio em que a simples menção ao sexo não provoque
respostas de reprovação ou insultos. Não são poucas as pessoas que
defendem que o \textit{conhecimento sexual} é algo questionável, ou, na
melhor das hipóteses, algo que deve ser mantido em sigilo no interior
dos livros de medicina; ou, ainda, há quem considere a posse desse
conhecimento uma banalidade característica aos irremediavelmente
desavergonhados. Como resultado, a literatura pseudocientífica acerca do
sexo é derramada sobre as nossas emoções, deixando os fatos
intocados, quando não os apresenta sob disfarces. Muito do que é
produzido não possui fundamentação biológica ou qualquer relação com as
leis da vida que governam o ser humano, não menos do que a qualquer outro ser
vivo. É o medo (às vezes chamado de \textit{reverência}) que nos faz \textit{deixar
o sexo em paz}. São o recato dissimulado e a vergonha estúpida,
disfarçados sob o nome de \textit{decência}, que obrigam os museus a vestir
faunos de mármore e a cobrir \textit{júpiteres} e cupidos de bronze com folhas de
gesso, não raro tortas ou mal cortadas dos lados. Há considerável
infantilidade a envolver a questão do sexo.

O homem é \textit{superior}, os animais são \textit{inferiores} --- sem intelecto
e, obviamente, sem almas. Nós temos. O nosso pai é um pai
\textit{divino}. Nossos corpos são \textit{sagrados}. Consequentemente, a arte, da
escultura de Fídias ao poema \textit{sofomórico},
tende à glorificação cada vez mais inaudita do ser humano: homens e mulheres
mais próximos a deuses e deusas; deuses e deusas como homens e mulheres
glorificados.
%{[}\textit{sophomoric poem}{]}

E assim aconteceu que tudo aquilo de mais comum na natureza, além de
vivo, se tornasse dotado da santidade do céu.

%\asterisc

A fisiologia e a zoologia nos mostram que a raiz da beleza e do sentido
estético reside no elemento sexual --- que não pode e nem deve ser
ignorado. A vida sexual, da forma mais primitiva em diante, busca a
união, a coesão. Em toda parte, há signos de atração: nas plantas,
tem"-se as cores vibrantes e as formas das flores; nos pássaros, a doçura
vitoriosa do seu canto. O pássaro pavilhão da Austrália decora a sua
dança com flores e plumas; o pavão fascina sua parceira; os alces
gemem um para o outro nas florestas; os vaga-lumes acendem suas lanternas
à noite; o ar está cheio de odores místicos a flutuar. Todas as
faculdades e formas da natureza estão dispostas de modo a contribuir com
a expressão dessa grande necessidade de união que explode no mundo
animal. Tudo se transforma em indício, símbolo, lembrança, mensagem,
chamado. Essa faculdade é social. É o começo do panorama da arte.

Penso que não cometeremos um erro ao afirmar que a aceitação livre e
saudável do corpo humano, com todas as suas faculdades, será a chave
mestra da arte do futuro.

%\asterisc

Todos os escritores modernos que tratam da questão do sexo provaram que
a velha noção de que o sexo se inicia na puberdade é falsa. O impulso
sexual, como todos os nossos impulsos, começa no nascimento e termina
com a morte. Diferentemente dos outros animais, porém, ao ser humano não é
permitido interpretar que os seus apetites sexuais exigem a ação, ao
menos não antes que ele atinja um número respeitável de anos (uma causa
verdadeiramente excelente para os distúrbios nervosos dos quais os
demais animais, fiéis aos retornos periódicos da sua excitação sexual,
estão absolutamente livres). Ao que parece, há um despertar da
sexualidade desde os primeiros dias de vida; para Freud e sua escola, as
ideias sexuais ganham existência no momento em que é dado o peito à criança.

Aparentemente, não é só a sensação de satisfação do apetite o que
impele o recém"-nascido a agarrar"-se ao seio da sua mãe ou da ama de
leite. Nessa pequena bola disforme, na qual a gama de sensações ainda
não emergiu na sua totalidade, uma imaginação especial e antecipatória
já complica o prazer de ter o apetite satisfeito por meio de um gozo
mais especializado.

E passará um bom tempo, desde os primeiros anos, em que a imaginação da
criança está voltada para imagens sexuais. Isolada do mundo e sem saber
nada além da vida familiar, embora já dotada de desejos que a
natureza mesma não lhe permitirá satisfazer até um futuro distante, a
criança volta a sua imaginação sexual para os objetos mais próximos, os
únicos que tem em vista. Há, na primeira infância, uma crise
imaginativa que, frequentemente, exerce um efeito distante sobre a vida
sentimental do adulto.

Os psicólogos nos pedem para observar que o desenvolvimento sexual da
criança pode ocorrer em três etapas:

\begin{enumerate}
\item Surgimento do primeiro fenômeno no momento do nascimento;

\item Crise na primeira infância, dos quatro aos sete anos, com a inversão de
sentimentos normais, incesto etc.;

\item Crise da puberdade, entre dez e 14 anos, seguida pelo retorno
das sensações às vias normais.
\end{enumerate}

Essas manifestações psicológicas estão, sem dúvida, intimamente ligadas
às modificações dos órgãos de secreção internos, e, em particular, com o
desenvolvimento sexual, que é apenas uma das especificidades das atividades
glandulares.

Antigamente, todas essas manifestações eram recebidas com profunda
ignorância e estupidez. Repetidas vezes, as mães bateram e ainda batem
nos seus desafortunados filhos, quando veem a menor manifestação do
sexo. Ou simplesmente os aterrorizam, colocando o estigma de depravação nas suas
crianças. Eu poderia citar centenas de exemplos retirados da minha
experiência de enfermeira formada.

A sociedade exige que jovens adultos, homens e mulheres (especialmente
as mulheres), reprimam o impulso sexual por vários anos --- geralmente
por toda a sua vida. A contenção de um desejo tão instintivo não pode
ser bem"-sucedida numa pessoa normal sem consequências diretas na sua
saúde --- todo tipo de distúrbio mental e físico pode resultar daí; e não
é raro que o impulso, forte demais para ser contido, encontre a saída
por alguma via infantil e perversa.

%\asterisc

Ou conforme foi tão habilmente descrito na obra de Jacques Fisher,
\textit{Amor e moralidade}: o desejo sexual frequentemente estoura, como
se fosse um trovão, nas vidas mais ordenadas e respeitosas, que, à
primeira vista, pareceriam as mais imunes. Algumas vezes, sob a
influência de uma causa fisiológica específica (influência da menopausa
ou de alguma alteração nas secreções de um órgão prejudicado pela idade)
e outras de uma causa desconhecida. As regras morais que até aquele
momento governavam toda a vida sexual desaparecem ou então sofrem uma
revolução violenta.

Infelizmente pouquíssimos podem compreender esse tipo de mudança. A eles não foi permitido
ver para além dos atos propriamente ditos. Eles não se dão o trabalho de
descobrir os motivos que conduziram {[}até ali{]}.\footnote{Parágrafo
  riscado no manuscrito datilografado.}

%\asterisc

Max Stirner --- \textit{O único e sua propriedade} --- oferece o retrato
maravilhoso de uma sexualidade faminta.

\begin{quote}
Para onde quer que se olhe, encontraremos vítimas da autonegação. Ali,
na minha frente, está uma garota que talvez já há dez anos vem fazendo
sacrifícios cruéis à sua alma. Sobre o seu corpo cheio de vigor pende
uma cabeça cansada de morte, e as maçãs pálidas do seu rosto denunciam a
sangria lenta da sua juventude. Pobre criança, quantas vezes as paixões
bateram na porta do seu coração e os intensos poderes da juventude
reclamaram os seus direitos? Quando a tua cabeça se revirava no
travesseiro macio, como a natureza desperta tremia pelos teus membros,
como o sangue inchava as tuas veias e fantasias ardentes derramavam o
brilho da volúpia nos seus olhos! Mas aí aparecia o fantasma da alma e
da sua bem"-aventurança eterna. Você ficava aterrorizada, suas mãos se
recolhiam, seus olhos atormentados se voltavam para o alto, você --- rezava. As tempestades da natureza foram silenciadas, a calmaria
deslizou sobre o oceano dos seus desejos. Lentamente, as pálpebras
cansadas caíram por sobre a vida que se apagou debaixo delas, a tensão
foi deixando, imperceptivelmente, os seus membros bem torneados, as
ondas turbulentas secaram no seu coração, as mãos unidas em posição de
oração quedaram, como um peso morto, sobre um peito que já não opunha
resistência; um último suspiro \textit{Ó, querido}, saiu como lamento, e --- a
alma pôde, enfim, encontrar a paz. Adormecias, e despertavas pelas manhãs para
novas batalhas e novas --- orações. Agora, o hábito da renúncia gela o
calor do teu desejo e as rosas da tua juventude estão se tornando
pálidas na --- anemia da sua beatitude. A alma está salva, que o corpo
pereça. Ó, Lais, ó Ninon, quão bem vocês fizeram em desdenhar dessa
pálida virtude! Uma única \textit{grisette} livre contra mil virgens cuja
virtude tornou murchas!
\end{quote}

Não há nada mais patético e cruel do que a vida de uma \textit{solteirona} ou
celibatária que murchou de tanta virtude. Com seus pensamentos e
sentimentos reprimidos, elas se tornam mal"-humoradas, invejosas e
ciumentas. Ressentem"-se da despreocupação e alegria dos jovens. Mas será
que podemos culpá-las por isso? Certamente, não. Elas são vítimas de uma
moral sexual ridícula.

É certo que a repressão sexual levada a cabo por um longo período de
anos mesmo que não seja suficiente para causar uma desordem mental
completa, cria uma série de peculiaridades antissociais e perversões.

Embora alguns moralistas enfatizem que o sexo é uma força dominante em
todas as criaturas vivas --- seja um ser humano, animal ou planta ---,
eles têm para si a ideia de que o sexo é mais forte no macho do que na
fêmea. Também insistem em atribuir ao sexo apenas uma única motivação --- a reprodução. Essas ideias foram exploradas pelos psicólogos sexuais
modernos que, por sua vez, insistem que a diferença no grau de intensidade é
normalmente exagerada para manter as jovens longe do conhecimento das
questões sexuais, com o intuito de sugerir-lhes que há algo mau nesse tipo de assunto. O sexo é uma desgraça para as boas meninas.

O célebre ginecologista, Kisch, nos diz que:

\begin{quote}
O impulso sexual é extremamente poderoso em certos períodos da vida, é
uma força elementar que domina de modo tão absoluto a totalidade do
organismo de uma mulher que não deixa na sua mente lugar algum para
pensamentos acerca da reprodução; pelo contrário, ela deseja
intensamente o ato sexual mesmo quando está com muito medo de engravidar
ou quando não há chance alguma de a gravidez ocorrer.\footnote{Enoch
  Heinrich Kisch. \textit{The Sexual Life of Woman in Its Physiological
  and Hygienic Aspect} (1910).}
\end{quote}  

Já o dr.\,Bloch escreve:

\begin{quote}
Eu mesmo entrevistei um número significativo de mulheres cultas acerca
desse assunto. Sem exceção, elas declararam que a teoria de que a mulher
possui uma sensibilidade sexual menor é equivocada; muitas eram
inclusive da opinião de que a sensibilidade sexual era maior e mais
duradoura na mulher do que no homem.
\end{quote}

dr.\,Bloch também insiste que: 

\begin{quote}
A maioria dos casos de frigidez sexual
nas mulheres é, na verdade, meramente aparente: seja porque, sob o véu
prescrito pela moralidade convencional, por detrás de uma frieza
aparente, há uma sexualidade ardente oculta, seja porque o homem com
quem teve relação sexual não foi bem"-sucedido em despertar a sua
sensibilidade erótica, tão complicada e difícil de ser desperta. Não
obstante, uma vez que ele tenha sucesso em despertá"-la, a
insensibilidade sexual, na maioria dos casos, simplesmente
desaparece.\footnote{Iwan Bloch. \textit{The Sexual Life of Our Time
  in Its Relations to Modern Civilization} (1908).}
\end{quote}

Na verdade, muitas esposas não se atrevem a dar o seu máximo por medo de
que seus maridos as julguem muito agressivas, desprovidas do tipo
correto de feminilidade. A maioria dos homens são criados para acreditar
que as mulheres devem ser possuídas e que jamais devem entregar a si
mesmas ao amor e à paixão com felicidade e alegria.

Isso também impede que homens de uma espécie mais sensível possam se
entregar livremente --- eles temem ultrajar e chocar a sensibilidade e
inocência das suas esposas. Vocês se surpreenderiam com a frequência com que
as esposas se sentem chocadas e ultrajadas.

Laura, esposa do capitão do drama \textit{O pai} de Strindberg, nos
oferece uma chave para compreender os pensamentos e sentimentos de um
grande número de mulheres. Ela diz para o seu marido: ``Quando, primeiramente, eu entrei na sua vida, eu era como uma segunda mãe para você. Eu te amava como meu filho. Mas quando a natureza dos seus
sentimentos mudava e você surgia como o meu amante, eu me envergonhava
e seu abraço era um prazer logo seguido por uma consciência cheia de
remorsos como se meu sangue estivesse envergonhado.'' Essa é a tragédia de muitas mulheres.

%\asterisc

O que é comumente chamado de incompatibilidade de temperamentos é quase
sempre o resultado direto da ausência de harmonia sexual --- a
insatisfação e os atritos que surgem quando a natureza química do sexo
entre marido e mulher falha em os unir harmoniosamente.\label{unir}

A frigidez, em algumas mulheres, é em grande medida um efeito do
sufocamento imposto pelo tabu sexual. Essas mulheres não podem,
ainda que tentem desesperadamente, responder ao desejo sexual do homem.
Na verdade, só o pensamento da relação sexual é, para tais mulheres, uma
tortura. Mesmo que falte ao homem sutileza e que ele imponha
suas necessidades à esposa, não encontrará, com isso, nenhuma satisfação sexual. No
final, ele buscará gratificação em outro lugar. Há uma porcentagem
considerável de homens casados entre a clientela da prostituição. O sexo
é mais poderoso do que todas as decisões. O homem se tornará indiferente
e, por fim, desejará o divórcio ou se ele puder pagar terá a opção de manter uma
amante.

%\asterisc

O sr.\,V.\,F.\,Calverton no seu livro \textit{Sex expression in literature}\footnote{Em tradução para o português, \textit{A expressão sexual na literatura}.} mostra que ``o
melhor método para combater a mitologia popular e antiga de que nossas
atividades e instituições sexuais são originariamente reveladas ao homem por
Deus, de modo que devem permanecer fixas, imutáveis e eternas, é, sem
sombra de dúvida, a execução de um estudo realista sobre a história da
moralidade sexual em todas as diferentes eras culturais.''

Ele prova que remanescentes do culto ao sexo eram comuns até o fim do
século \textsc{xviii} --- as festas fálicas de Dionísio foram mantidas em muitas
províncias da Itália. Para citar as palavras de W.\,J.\,Lawrence, ``quando
a cidade antiga do reino de Nápoles, Isérnia, foi destruída por um
terremoto em 1805, os crentes viram na calamidade um castigo para os
costumes pagãos conservados inescrupulosamente no período do ano do
Festival Báquico.''\footnote{W.\,J.\,Lawrence. ``The Pallus on the Early
  English Stage.'' In: \textit{Psyche and Eros. An International Bi"-Montly
  Journal of Psychanalysis, Psychology and Therapeutic Psychognosis},
  vol.\,\textsc{ii}, n.\,3.}

Depois da conversão ao cristianismo, as raças germânicas mantiveram %\ldots{} 
os seus demônios fálicos da vegetação. Nos tempos antigos, as mulheres
de Israel faziam falos de ouro e prata para serem usados como talismãs
contra a esterilidade; também havia os doces em forma de falo
comidos ao longo de todo o período das trevas com o mesmo propósito em
vista.\footnote{Neste parágrafo, Goldman faz um pequeno resumo de
  algumas das ideias encontradas num outro artigo publicado no mesmo
  volume que o de Lawrence, citado no parágrafo anterior, sendo este:
  ``Sex Fertility and Worship'', cujo autor, anônimo, identifica"-se como
  \textsc{r}.}

Tanto a religião judaica quanto a cristã impuseram a noção de que o
sexo só é admissível caso o objetivo seja o da procriação. Para que fossem
frutíferos, se multiplicassem e reabastecessem a terra, Jeová permitiu
aos judeus ter muitas esposas. A religião cristã, embora não permita
muitas esposas e tenha admitido a gratificação sexual exclusivamente por conta da
finalidade de trazer crianças ao mundo, está perfeitamente consciente do
fato de que seja dentro ou fora do casamento, o sexo continua a existir
sem maior consideração pela procriação. Se muitos religiosos e
puritanos prenderam"-se ao fetiche de que o sexo não deve ser \textit{indulgente}
--- conforme eles denominam essa expressão perfeitamente natural ---,
psicólogos e biólogos modernos já rasgaram os véus de todo absurdo que
envolve o sexo. Bibliotecas inteiras foram preenchidas com trabalhos que
tratam do assunto com entendimento e profundidade e demonstraram que
houve uma percepção inadequada, de um lado, da energia gigantesca que
está por detrás do instinto sexual e, de outro, dos dispositivos
biológicos para a liberação dessa energia através de canais que não são
especificamente sexuais. Provavelmente, é uma verdade absoluta dizer que
o instinto sexual é o instinto criativo, como é igualmente verdadeiro
que qualquer escoadouro que ofereça satisfação emocional pela via do
exercício da criatividade tem a capacidade de neutralizar as
necessidades dos desejos sexuais.

O significado desse ponto de vista deve ser plenamente concretizado
pelos pais e seus filhos, pois as crianças são seus reflexos não apenas
por conta do fator hereditário, mas também pela assimilação das suas
ideias e ideais. Ambos devem entender e aprender a utilizar aquilo que é
essencial para o desenvolvimento e satisfação dos desejos emocionais oriundos
do instinto sexual. É essa relação de liberação de energia, por meio
de atividades recreativas e do desenvolvimento de habilidades, que dá lugar
aos resultados mais satisfatórios.

O espírito criativo não é, porém, um antídoto para o instinto sexual,
antes deve ser considerado uma parte da sua expressão mais vigorosa.
Age de forma a conservar, mas também utiliza o instinto para formas de
satisfação que não têm como objetivo a mera proteção, posto que
conduz a um desenvolvimento superior. Seu alargamento e
aprofundamento se imprimem sobre o caráter inato da pessoa e seu poder
de autodirecionamento e controle. A liberação não sexual de energia,
algumas vezes, é suficiente para compensar as necessidades fundamentais
do desejo sexual e, ainda, para transmutá"-las em autossatisfação e
formas úteis de expressão.

O homem que expressou esse pensamento do modo mais profundo e poético
foi Friedrich Nietzsche. Em ``Moral como antinatureza'',\footnote{Título
  de um dos capítulos da obra \textit{Crepúsculo dos ídolos}.} ele
escreve:

\begin{quote}
Todas as paixões têm uma época em que são unicamente destrutivas,
quando, com o peso da sua estupidez, puxam suas vítimas para baixo; e
têm um período posterior, muito posterior, em que se casam com o
espírito, em que são \textit{espiritualizadas}. Outrora, as pessoas
declaravam guerra contra as próprias paixões, em virtude da sua
estupidez, conspiravam para a sua aniquilação. A formulação mais notável
desta concepção encontra"-se no Novo Testamento, no Sermão da Montanha,
onde, diga"-se de passagem, as coisas não são, de modo algum,
consideradas de um ponto de vista elevado. Por exemplo, com relação à
sexualidade, ali está posto que: ``Se o teu olho te escandalizar,
arranca"-o fora.''
\end{quote}

Felizmente, nenhum cristão age de acordo com esse preceito. Aniquilar as
paixões e desejos unicamente para prevenir a estupidez e resultados
desagradáveis é, para nós, no presente, simplesmente uma forma ainda
mais aguda de estupidez. A Igreja luta contra as paixões,
busca extirpá-las em todos os sentidos; essa é a sua prática, a sua \textit{cura}, a
sua castração. Ela nunca questiona: ``Como espiritualizar, embelezar e
deificar um desejo?'' --- em todas as eras, ela sempre colocou a ênfase
na disciplina para a exterminação (da sensualidade, do orgulho e
ambição). No entanto, atacar as paixões na raiz significa atacar a
própria vida na raiz: a práxis da Igreja é inimiga da vida\ldots{}

Esse truísmo já é reconhecido pelo pensamento, mas um elemento daquela
etapa ainda permaneceu, a saber: que o sexo, embora um fator
extremamente potente, deve ser reprimido nos solteiros.
Felizmente, esse preconceito também está sendo demolido. Em um simpósio
recente sobre \textit{o elemento sexual na vida do adulto solteiro} foram
apresentados fatos claros e surpreendentes sobre o assunto. Para citar
apenas alguns dos argumentos colocados pelo dr.\,Ira Wile na sua contribuição:

\begin{quote}
Os solteiros possuem os mesmos potenciais das pessoas casadas com as
quais estão necessariamente relacionados nos interesses e na
participação em todos os aspectos da vida e, portanto, afetam o
bem"-estar do grupo de casados. A sua vida sexual, de várias formas, é
fundamental para o bem"-estar social, assim como é significativa para o
seu crescimento pessoal e desenvolvimento.

Estudar essa parte da sexualidade é explorar uma das áreas não
mapeadas da nossa civilização na qual o resíduo da ignorância bárbara e
do medo ainda retardam o progresso. Acima das suas profundezas, aparece
uma névoa miasmática e pouco convidativa de dúvida e incerteza. Até
agora, poucos exploraram de modo sistemático o estado dos solteiros.
Consequentemente, o esforço de fazer uma pesquisa nesse campo, abordando
os seus vários ângulos, representa um esforço de mapear fatos, rastrear
influências casuais, determinar a validade e o valor das nossas visões
atuais e estabelecer dados e hipóteses que possam estar a serviço
da interpretação da nossa era. {[}\ldots{}{]}

Há mais de dois mil anos a gratificação oficial para a urgência da procriação
é exclusiva ao estado de casado socialmente reconhecido. O
pressuposto da sociedade é que a proibição autorizada ao desejo por um
parceiro biológico deve ser capaz de adiar a realização desse desejo até que o
rito civil ou religioso exigido o sancione. Se a sociedade espera
que regulamentações externas subjuguem necessidades fisiológicas e
impulsos psicológicos, é o caso de perguntar se essas regulamentações têm
sido efetivas. Aqueles que não estão casados vivem em castidade e
celibato? Se não vivem, em quais aspectos e por quais vias o seu comportamento sexual
difere daquele exibido pelos que lhes são semelhantes em tudo,
exceto por viver em matrimônio?\footnote{Ira S.\,Wile. \textit{The sex
  life of the unmarried adult: an inquiry into and an interpretation of
  current sex practices.} (1934).}
  \end{quote}

A atividade sexual não é um ato isolado; é uma experiência generalizada
que motiva e afeta a personalidade. Das reações mais pessoais
emergem ideias de romance e beleza, exaltação e tranquilidade, devoção e
idealização servil; ou, o sentido de sacrifício, confusão, humilhação,
vergonha, ansiedade, o desejo por autopunição, ou autoaceitação da
fraqueza, fracasso e inadequação. O conceito de \textit{apaixonar"-se} pode
ser aplicado a si mesmo, a membros do mesmo sexo ou do sexo oposto. Há
normalmente uma segunda pessoa que facilita o crescimento, seja dos
impulsos sexuais, ou da própria vida pessoal como um todo. Na mesma
medida em que isso é verdade para aqueles que são solteiros, é para
aqueles que são casados, embora a sociedade torne essa adequação entre homens e mulheres um
desafio, uma propriedade, uma questão.

Havelock Elis, ao se referir à reprodução, salienta que: ``As funções do
sexo no que diz respeito aos aspectos psíquicos e emocionais estão muito
além de qualquer ato de procriação, podem inclusive exclui"-lo
completamente; quando a nossa preocupação está relacionada ao
bem"-estar do ser humano enquanto indivíduo, devemos ampliar a nossa
perspectiva e aprofundar a nossa visão.''\footnote{Havelock Ellis,
  \textit{Studies in the Psychology of Sex: Sex in Relation to Society},
  vol. 3.}

Para interpretar a vida sexual dos solteiros, é preciso reconhecer que
há duas funções do sexo:

Uma, a biológica, que tem a procriação como fim --- envolve um processo cujo
interesse na perpetuação da espécie é muito mais emocional do que
intelectual.

A outra função consiste na promoção do desenvolvimento social através
das relações humanas. Isso envolve o sexo e a atividade erótica, com ou
sem o fim da procriação.

Há duas bases para a energia do impulso sexual --- uma é consciente,
direcionada, guiada, submetida ao controle ético; a outra é
inconsciente, instintiva, impulsiva, responde a estímulos e não está
sujeita à razão.

%\asterisc

Numerosos estudos revelaram que tanto os solteiros quanto os casados
têm experiências sexuais variadas. Dr.\,Klatt descobriu que 18\% das
mulheres tiveram alguma forma ativa de experiência sexual com menos de
18 anos. E que 60,6\%, entre 18 e 22 anos. Entre o grupo seleto do dr.\,Hamilton, 59\% dos homens e 47\% das mulheres fizeram sexo antes do
casamento, e 20\% dos homens e 16\% das mulheres, antes dos 21 anos de
idade. Katherine B.\,Davis, ao estudar a vida sexual de 1200 mulheres
universitárias solteiras, relatou que 61\% admitiram se masturbar, sendo
que 57\% antes dos 15 anos de idade.

Sentimentos e desejos sexuais são experimentados com periodicidade por
70\% do grupo reportado. Os solteiros, não menos do que os
casados, relataram várias formas do fenômeno sexual. É interessante
notar que relações homossexuais foram reportadas mais frequentemente
pelo grupo de solteiros. Isso é significativo até porque, socialmente, a
homossexualidade é considerada mais perigosa entre homens do que entre
mulheres. Boa parte das experiências homossexuais entre os homens finda
quando ganham confiança na sua potência sexual, o que, em geral, ocorre na relação sexual que se segue ao casamento.

O lesbianismo é comum e a sua prevalência é importante, ainda que nem
sempre implique em permanecer solteiro. O grupo de mulheres solteiras
lamentou mais a falta de filhos do que de maridos. Aproximadamente,
38,7\% acredita que a relação sexual é necessária para a saúde mental e
física. Aproximadamente, 20\% se pronunciaram a favor da relação sexual
antes do casamento para homens e mulheres.

Dr.\,Wile pergunta:

\begin{quote}
Os desejos sexuais de adultos solteiros das graduações e
pós"-graduações são, em
geral, menos intensos do que entre os estudantes do ensino médio? Os
jovens mais maduros perdem os seus desejos instintivos primários
quando se tornam adultos? A educação superior é adquirida à custa da
repressão sexual? Isto é absurdo e contrário a todos os fatos. A
atividade sexual se dá de muitas maneiras --- por meio de jogos eróticos
ativos, praticas autoeróticas, práticas homossexuais ou heterossexuais,
ou via a sublimação estética ou vocacional. A sua natureza e intensidade
estão sujeitas às escolhas, julgamentos, padrões e ideais pessoais que
regulam, mas não aniquilam --- temporalmente proibidos ao invés de
permanentemente inibidos.

Se alguém assumir que a vida sexual é e deve ser limitada pela
instituição social do casamento, então é o caso de perguntar o que fazem
os solteiros ou o que eles devem fazer? A questão não é sobre o que eles
podem fazer, posto isso ser idêntico em ambos os grupos. A existência
real e legal do pacto antenupcial atesta o fato de que há efetivamente
uma atividade sexual bastante abrangente na vida de
solteiro.\footnote{Ira S.\,Wile. \textit{The sex life of the unmarried
  adult.}}
\end{quote}

%\asterisc

A dificuldade econômica não é capaz de erradicar desejos heterossexuais,
como tampouco a promulgação de uma lei punitiva pode destruir impulsos
homossexuais. Na verdade, a sociedade reconhece as demandas sexuais, se
não as próprias necessidades sexuais, da parte do grupo dos solteiros,
dada a sua atitude em relação à regulamentação da prostituição, da
perseguição aos homossexuais e às casas de \textit{striptease}.

Em outras palavras, ou os solteiros são levados a métodos artificiais
para a expressão sexual ou eles se entregarão ao desejo que domina as
relações naturais. Na realidade, eles têm se entregado às relações
sexuais ainda que sob tabus impostos há muito tempo, o que se intensificou desde que começaram a ter conhecimento dos
métodos contraceptivos, capazes de prevenir que se traga crianças
indesejadas a esse nosso mundo maravilhoso --- um assunto sobre o qual eu
pretendo falar antes de ir embora dessa cidade.

Adultos solteiros estão lidando com o sexo mais como um fato do que como
teoria. Eles estão aceitando a sua compleição sexual abertamente como um
instrumento de crescimento pessoal e emocional, que dá lugar à
estabilidade social, ao invés de assumirem a hipocrisia de que o sexo é
uma função projetada por algum plano divino com o fim da procriação de
seres puros cujo sentido da vida é o de morrer em castidade para, assim,
atingir a felicidade em um mundo vindouro.

Eles valorizam o sexo como a fonte da vida, pois acreditam que uma vida
sem sexo é uma palhaçada após a maturidade biológica, porque é contrária
à natureza, uma vez que o sexo é também expressão de fatores
psicológicos cuja função é socializar os indivíduos e os conduzir a
abandonar a vida de solteiro em nome de um casamento que esteja em
harmonia com as leis e costumes da sua época.

Todos os jogos eróticos estão geneticamente ligados ao namoro, mas o
namoro está sujeito ao controle social. Por conta disso, a vida sexual
de um jovem adulto solteiro e normal, seja lá o que isso queira dizer, é
a preparação para a perfeição da relação sexual, para a promoção da
felicidade pessoal, e para a adaptação a alguma forma de casamento seja
essa livre e não convencional ou de acordo com os ritos civis ou
religiosos.

É por concordar tão completamente com esse ponto de vista e por saber dos
resultados trágicos e desastrosos da ideia puritana sobre o sexo que
acredito ser imperioso chamar a atenção para a necessidade de se tratar
da questão sexual de modo franco e sem os subterfúgios usualmente
empregados, quando nos preparamos para tratar desse assunto. Em conjunto
com os espíritos mais elevados e livres e o poeta Walt Whitman, eu digo:
``Onde o sexo está faltando, tudo está faltando.'' Livremo"-nos da
humildade dissimulada tão predominante na superficialidade dos bons
modos. Libertemos o sexo da falsidade e da degradação e percebamos logo
de uma vez por todas que o sexo é um fator de importância central para a
saúde e a harmonia na vida e na arte.


