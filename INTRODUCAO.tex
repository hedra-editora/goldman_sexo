\pagestyle{plain}

\chapter{Introdução}

\begin{flushright}
\textsc{mariana lins}
\end{flushright}\bigskip

\noindent\emph{Sobre anarquismo, sexo e casamento} é uma coletânea de
textos --- publicados e não publicados, uma entrevista e um rascunho
inacabado. Nos quase quarenta anos que separam o primeiro texto, escrito em 1896, aos vinte e sete anos, do último, de
aproximadamente 1935, Goldman viajou incansáveis vezes pelos Estados
Unidos, além da Europa, em diversas turnês de conferências --- volta e
meia, vigiadas explicitamente por bandos de policiais, quando não
canceladas pelas autoridades ---; organizou e participou de uma série de
atividades como comícios, greves, levantamento de fundos para presos
políticos, protestos e até mesmo bailes. É conhecida a anedota, relatada
em sua autobiografia \emph{Living my life,} em que ao ser repreendida
por dançar, ou seja, por se permitir uma ``frivolidade'' que poderia
manchar a reputação da causa anarquista, uma Goldman, furiosa, rebateu
com a declaração de que o anarquismo, para ela, significava a
concretização de um ideal: o da liberdade, do direito à
autoexpressão e às coisas belas e radiantes, apesar das prisões e
perseguições políticas e de tudo o mais que há de cruel e terrível neste
nosso mundo. Conforme declara em um dos textos aqui traduzidos,\footnote{Conferir ``Anarquia e a questão do sexo'', p.\,\pageref{ref1}.}
não é correto dizer que a
liberdade só poderá ser alcançada com a implementação do anarquismo.
Afinal, se os pais não forem eles mesmos livres, não se poderá esperar
que a ``nova geração'' ajude a atingir o objetivo ``que é o
estabelecimento de uma sociedade anarquista''.

Também nesses quase quarenta anos, Goldman foi presa um sem-número de
vezes, perdendo a cidadania estadunidense em 1909. Até que, em 1917, sob
o clima de histeria patriótica, com a entrada dos Estados Unidos na
Primeira Guerra, e de paranoia antivermelha, por conta da Revolução
Bolchevique, foi acusada de violar a lei do alistamento seletivo
{[}\emph{Selective Service Act}{]}, promulgada pouco mais de um mês após
a declaração de guerra dos Estados Unidos à Alemanha, e que tornava
obrigatório o alistamento dos homens com idade de vinte um a trinta
anos. Devido aos discursos proferidos contra o alistamento, em junho
daquele ano, e da tentativa de organizar uma liga contra o recrutamento
obrigatório, Goldman foi considerada culpada pelo crime de conspiração,
o que lhe rendeu algo em torno de um ano de prisão e, em seguida, a
deportação à Rússia, em dezembro de 1919. Nascida em 1869 numa província da
Lituânia, então pertencente ao Império Russo, Goldman imigrou
com sua irmã Helene para os Estados Unidos em 1885, com o objetivo de
se reunir com sua outra irmã, Lena, então já residente em Rochester,
Nova York. Posteriormente, as três irmãs foram seguidas por seus pais.\looseness=-1

Quando deportada à Rússia, junto a outros 237
militantes políticos imigrantes, Goldman ainda tinha fé na Revolução
Russa, apesar das notícias chegadas na América da prisão de diversos
anarquistas pelo governo bolchevique. Porém, ao testemunhar em
primeira mão a autocracia do governo, com sua rotina de prisões e
execuções não só de anarquistas, como de diversos revolucionários
destacados e trabalhadores que se opunham ao partido,
Goldman desiludiu-se amargamente --- tentara mesmo apelar diversas vezes às autoridades bolcheviques, como Lênin e Trótski.
Obteve, então, em dezembro de 1921 --- juntamente a Alexander
Berkman ---, autorização para deixar a Rússia, sob o pretexto de
representar o Museu Kropótkin numa conferência em Berlim. A partir daí,
passa por diversos países, como Suécia, Alemanha, França, Espanha e
Inglaterra, e concentra boa parte dos seus esforços na denúncia do
governo bolchevique e na arrecadação de dinheiro para os prisioneiros
políticos. Mesmo antes de Stalin assumir a liderança do partido, Goldman
já apelava às consciências dos intelectuais de todo o mundo para que
encarassem a gravidade das atrocidades políticas que ocorriam
na Rússia sob o pretexto de necessidades revolucionárias. Note-se
que, na época, tal acusação ao Estado Soviético, cantado e
celebrado por diversos dos mais renomados intelectuais e artistas, era
tanto polêmica quanto extremamente audaz. Conforme atesta
em ``Mulheres heroicas da revolução russa'', de 1925:

\begin{quote}
Não há opinião pública na Rússia que não seja a do partido governante, e
os mártires --- homens e mulheres --- da Rússia revolucionária se
transformaram em párias, no sentido mais amplo que pode haver. Eles não
têm nada para os compensar, não podem sequer apelar à consciência do seu
país, pois ela foi paralisada. Inclusive, não apenas a consciência da
Rússia, mas a consciência do mundo como um todo parece silenciada.
{[}\ldots{]} diante das evidências esmagadoras da opressão e perseguição
extremamente cruéis que ocorrem na Rússia, o mundo está silencioso e
insensível.\footnote{Conferir p.\,\pageref{ref2} desta edição.}
\end{quote}

E aqui, é inevitável destacar, da perspectiva atual, a
qualidade dolorosamente profética de uma outra declaração sua, do mesmo texto: ``Os novos autocratas da Rússia desacreditaram os
ideais do socialismo e macularam a honra dos seus grandes expoentes''.

No mesmo ano em que escreve o seu ``Mulheres heroicas da revolução
russa'', Goldman, que então vivia na Inglaterra, casa-se com o escocês
James Colton, um mineiro e anarquista, com o intuito de conseguir a
cidadania britânica e, assim, livrar-se do perigo de ser
extraditada para a União Soviética, além de poder atravessar as
fronteiras com liberdade e segurança. Essa moeda de troca entre
matrimônio e cidadania, ainda que sob condições outras, já havia sido
vivenciada por Goldman no primeiro casamento, posto que até 1922 a
cidadania estadunidense de uma mulher dependia da cidadania do seu pai
ou marido. Em 1887, casara-se com o cidadão norte-americano, nascido na
Ucrânia, Jacob Kersner, numa cerimônia judaica tradicional.
Divorciou-se no mesmo ano para se casar novamente com ele pouco tempo
depois (deprimido, Kersner havia tentado suicídio), separando-se logo em
seguida, algo em torno de três meses. Como não se divorciariam da
segunda vez, Goldman garantiu a cidadania estadunidense até 1909, quando a cidadania
do próprio Kersner foi revogada postumamente --- uma clara
atitude de perseguição política à anarquista, que ao perder
sua cidadania já não podia, dentre outras limitações,
deixar o solo estadunidense, sob o preço de não lhe ser permitido
retornar. Em um dos artigos aqui traduzidos, originalmente publicado em 1926 no
jornal \emph{\textsc{nea} Service} de Chicago, sob o título irônico, dado
pelo periódico, de ``As visões de Emma sobre o amor'', Goldman foi, por
assim dizer, convidada a explicar \emph{como}, após críticas tão
ferrenhas à instituição casamento, pôde vir a se casar com Colton. Ainda
que de modo tergiversado, a anarquista deixa claras as suas razões:

\begin{quote}
O maior infrator da sacralidade da privacidade é o Estado. Desde a
reação desencadeada pela Guerra Mundial, o Estado abandonou a maioria
das suas atividades para se dedicar exclusivamente ao exercício de
controle e sufocamento dos indivíduos. {[}\ldots{]} Ninguém que seja dotado
de cérebro pode persistir na crença de que o casamento é uma obra dos
céus, ou que, uma vez encarnado, deve ter os seus limites estabelecidos
por alguém além dos imediatamente envolvidos; que atravessam o processo
com o mesmo espírito de alguém que busca tirar um passaporte ou obter um
visto --- para conseguir um espaço para respirar e proteger a privacidade
da personalidade humana.\footnote{Conferir p.\,\pageref{ref3} desta edição.}
\end{quote}

No tempo do último texto que compõem a presente coletânea --- o delicioso
rascunho inacabado (talvez melhor designado de conjunto de notas),
intitulado ``O elemento sexual da vida'' ---, Goldman, então com algo em
torno de sessenta e seis anos, vivia em Toronto, sem ter jamais obtido o
direito de retornar aos Estados Unidos, salvo um visto de três meses e
cheio de restrições condicionantes, em 1934. Ao levarmos em conta o
contexto político mundial em foram escritos os textos aqui dispostos,
como a Primeira Guerra, a revolução russa, a ascensão do fascismo
italiano e do nacional-socialismo na Alemanha; eventos históricos
mundiais que, dada a sua condição de russa, judia, anarquista e, por
fim, crítica implacável do puritanismo estadunidense à autocracia
soviética, tornavam-lhe ainda mais vulnerável, e isso dos Estados Unidos
à Rússia e nos mais diferentes círculos; enfim, ao levar tudo isso em
conta, faz-se ainda mais tocante a sensibilidade com que abordou a
condição da mulher, a questão do seu sexo, conforme a sua terminologia,
o que não separava, em absoluto, dos contextos culturais, econômicos e
políticos, embora tampouco reduzisse a questão a essas esferas. Uma
sensibilidade que, conforme os textos aqui atestam, era acompanhada de
uma coragem tão surpreendente, quanto rara, verdadeiramente à altura da
alcunha que lhe foi atribuída: a de ``Suma Sacerdotisa do Anarquismo''.

De um lado, a coragem concretizada na vida e obra de Goldman --- a
coragem que, segundo Hannah Arendt, consiste numa das virtudes políticas
cardeais na medida em que coloca a preocupação com a vida e com o senso
de individualidade como secundários ante a liberdade e a igualdade que
só podem ser conquistadas na esfera política --- revela a herança
indelével dos heróis do movimento populista russo, os mesmos que,
segundo ela, os bolcheviques teriam traído. Uma herança que se apresenta
de múltiplas maneiras: seja no seu destemor, por assim dizer,
``kamikaze'' de ser presa, deportada ou considerada inimiga e pária por
aqueles que supostamente deveriam ser os seus pares (caso, por exemplo,
das feministas sufragistas), seja na sua temerária admissão do emprego
da violência como tática justificável contra a opressão. Não foram raras
as ocasiões em que Goldman defendeu publicamente tentativas bem ou
malsucedidas de assassinato de opressores por parte de rebeldes; caso,
por exemplo, do assassinato do rei Umberto da Itália pelo anarquista
Gaetano Bresci, em julho de 1900, ou da sua expressão de simpatia,
eternizada no artigo ``A tragédia de Búfalo'', pelo assassino do então
presidente dos Estados Unidos, William McKinley, em setembro de 1901 ---
o estadunidense de ascendência polonesa, Leon Czolgosz, que foi preso e
executado no mesmo ano. Goldman, que na época foi acusada pelas
autoridades de ter inspirado o ato com seus discursos --- Czolgosz se
autodesignava anarquista, o que levou a uma onda antianarquista no país
---, não satisfeita com o artigo de 1901, dedicou ainda a edição de
outubro de 1906 do jornal do qual era editora, o \emph{Mother Earth}, à
memória do quinto ano da sua execução. Em realidade, foi, mais de uma
vez, considerada suspeita de participação na organização de atentados,
embora em todas as ocasiões tenha sido liberada por falta de evidências
conclusivas. A mais dramática, certamente, ocorreu em 1891, quando foi
considerada suspeita de cumplicidade na tentativa fracassada de
assassinato de Henry Clay Frick, gerente de uma companhia de aço (a
\emph{Carnegie Steel Company}). O assassinato seria uma forma de
retaliação à chacina de nove trabalhadores em greve por um dos
seguranças contratados pelo o então gerente. O malsucedido assassino,
ninguém menos do que o destacado anarquista Alexander Berkman, foi não
só o grande amigo e companheiro político de Goldman ao longo de toda a
sua vida, como, na época do atentado, era também o seu amante. Apesar
das incansáveis campanhas empreendidas por Goldman, ao longo de anos,
para a libertação de Berkman, ele foi condenado a vinte e dois anos de
prisão, permanecendo em cárcere por quatorze anos, para, posteriormente,
vir a ser, junto a Goldman, um dos 247 imigrantes políticos deportados
dos \textsc{eua}, em 1919. Berkman também era lituano. Diante disso, a declaração
contida em um dos textos aqui dispostos, ``Os aspectos sociais do
controle de natalidade'', é tudo menos retórica, sendo esta: ``O medo de
ir para cadeia pelas próprias ideias é tão forte entre os intelectuais
americanos que é o que faz o movimento ser tão pálido e fraco. Eu não
tenho esse medo. Minha tradição revolucionária diz que aqueles que não
estão dispostos a ir para a cadeia em nome das suas ideias nunca deram
muito valor a elas''.\footnote{Para uma cronologia da sua vida bastante
  completa, ver: \textsc{falk}, Candance (ed.). Emma Goldman: \emph{A Guide to
  Her Life and Documentary Sources}. Alexandria, \textsc{va}: Chadwyck-Healey
  Inc., 1995.} Não por acaso, foi considerada pelo primeiro diretor do
\textsc{fbi}, como ``a mulher mais perigosa da América''.

De outro lado, porém, essa mesma coragem, talvez porque florescida em
meio à chamada primeira revolução sexual e sob os ares do novo mundo,
batia no peito de uma Suma Sacerdotisa que nada tinha de casta. O que
disse de Mary Wollstonecraft, é possível dizer dela mesma, bastando para
isso um simples relance na sua biografia amorosa: Goldman era
``sexualmente faminta''. Leitora voraz de Stirner e Nietzsche e de
``psicólogos do sexo'', como Freud e Havelock Ellis, para citar apenas
algumas das suas inúmeras referências, amava dançar, viver, os amigos, a
alegria, o cuidado, o prazer. A causa para ela não se dava em nome de um
princípio frio e abstrato de liberdade. A sua coragem tinha como
\textit{ânima} o ideal, sempre necessário, de que homens e mulheres podem e
devem ser livres para amarem-se uns aos outros da forma que desejarem,
até porque, somente assim, estarão livres para amar e cuidar das suas
crianças de modo que elas possam desenvolver a plenitude dos seus seres
individuais em harmonia com a coletividade. Pois para Goldman, a
liberdade é acima de tudo doar-se sem reservas, é amar e ser amado. Não
há amor que não seja livre, da mesma forma que não pode haver
emancipação sem amor. Que tenha sido chamada de superficial por militar,
em última instância, em nome de algo tão simples, ao mesmo tempo em que
utópico, porque supostamente impossível, é talvez prova de um
entendimento superficial, ainda que possa ser sofisticado, do que é
verdadeiramente capaz de nos satisfazer e alegrar no breve período em
que somos incendiados pelo sopro da vida. Vide, nesse sentido, com que
simplicidade descreve, em ``O sufrágio feminino'', o problema cuja
solução consiste, se não para todos, certamente para muitos, no mais
caro e profundo anseio:

\begin{quote}
A paz e harmonia entre os sexos e entre os indivíduos como um todo não
dependem necessariamente de um nivelamento superficial dos seres
humanos; como tampouco exigem a eliminação dos traços individuais e das
peculiaridades. O problema que nos confronta atualmente, e que o futuro
próximo precisa resolver, é o de como ser si mesmo e, ainda assim, estar
em unidade com os outros, o de como se sentir profundamente ligado a
todos os seres humanos e, ainda assim, reter as próprias características
individuais.
\end{quote}

Os textos aqui dispostos versam sobre os temas do casamento e do sexo
sob a perspectiva dessa implacável anarquista, o que abrange uma série
de subtemas como o controle de natalidade, o puritanismo
norte-americano, a repressão sexual, o amor livre, o ciúme, a
prostituição, a homossexualidade, a desigualdade entre os sexos, a
maternidade, a emancipação feminina, o movimento sufragista na
Inglaterra e Estados Unidos e a trajetória de uma série de mulheres
extraordinárias como Mary Wollstonecraft, Louise Michel e algumas das
heroínas e mártires do movimento revolucionário russo, dentre as quais
se destaca Maria Spiridonova.

\section{A mulher segundo Emma Goldman }

Não é por acaso, por gosto, inclinação ou por alguma espécie de
``birra'' que, na abordagem de Goldman sobre a questão do seu ``sexo'',
o tema do casamento ocupe lugar central. Pensar a mulher implica
necessariamente pensar sobre o casamento --- e, curiosamente, apenas, por
consequência, sobre a maternidade. E isso, quando o mesmo não se aplica
ao homem. Conforme a história do pensamento ocidental parece atestar,
salvo talvez nas últimas décadas, pensar sobre o homem prescinde da
reflexão sobre o casamento ou a paternidade. A ``necessidade'' da
relação entre os temas do casamento e da condição feminina não se deve,
porém, a alguma provável natureza intrínseca da mulher, a um suposto
conjunto de ``virtudes maritais'' como se oriundas do útero; ou tampouco
a alguma espécie de destinação espiritualmente predeterminada ao amor
incondicional que render-lhe-ia, quando bem-sucedida, o posto máximo de
``rainha do lar''. Para Goldman, o casamento nada tem de natural, na
mesma medida em que nada tem de espiritual ou de comum ao amor. Se
pensar a questão do sexo feminino implica necessariamente pensar sobre o
casamento, isso se dá pelo fato de que o casamento foi, ao longo de
eras, o meio principal, quando não o único, de a mulher alcançar alguma
seguridade material e social, quando não, no melhor dos casos, a
ascensão econômica.

A consequência de tal ``empregabilidade'' mercantil do amor e do próprio
corpo é \emph{trágica}, como atestam os textos a seguir, porque abrange
a totalidade da mulher, não é particular ou acidental, como à primeira
vista se poderia supor. Ao contrário: passa a dizer respeito ao seu
``espírito''. Com a degradação à condição de ``mercadoria sexual'' (cujo
fim seria o de proporcionar prazer sexual ao homem e/ou a procriação),
tudo aquilo que é considerado belo e elevado numa personalidade, como a
honra, a inteligência, a profundidade e mesmo a utilidade, torna-se,
quando \emph{na mulher}, mero acidente de uma condição
preponderantemente ``sexual''; e, portanto, um conjunto de atributos
dispensáveis, quando não indesejáveis e, em casos extremos, \emph{logo}
\emph{extirpáveis}; extirpação, para a qual, em não raros momentos da
história, foram fabricados e utilizados diversos instrumentos de
tortura, que, embora já não sejam vistos sendo usados por aí no seu
sentido \emph{stricto}, continuam, segundo a denúncia reiterada nos
textos aqui presentes, a dominar as mentes e corações no sentido
\emph{lato} --- no caso dos estadunidenses de então, pela via puritana do
protestantismo, herdada dos ingleses. Vide, nesse sentido, o escrito ``A
hipocrisia do puritanismo'', que nos traz à memória os Estados Unidos da
segunda metade do século dezessete, em que Boston e Salem rivalizavam em
crueldade nas perseguições às ``opiniões religiosas não autorizadas'',
perseguições que vitimaram sobretudo mulheres --- lembremos que o
episódio mais afamado de tais perseguições ficou conhecido como o
``julgamento das bruxas de Salem'', hoje tema de filmes de Hollywood.
Episódio extremo, mas do qual Goldman encontrava os ecos na política
estadunidense de então, especialmente encarnados na figura do político
Anthony Comstock (1844--1915), principal arquiteto dos mecanismos de
censura estatais que iam das obras de arte à correspondência privada.
Ora, não há novidade alguma em dizer que as mulheres foram, em geral, os
objetos preferenciais dos instrumentos de tortura, ao menos quando
utilizados em nome da moralidade, sob a escusa de colocá-la de volta no
seu ``eixo''.

Na análise da questão do seu ``sexo'', Goldman, por diversas vezes, faz
atentar para o fato de que, ao longo das eras, as principais qualidades
negociáveis da mulher foram a juventude e os atrativos físicos, uma
negociação milenar (em geral, levada a cabo por homens) que teve por
consequência a redução da mulher a essas qualidades, apesar do curto
tempo em que uma vida, em geral, miserável e infeliz, é capaz de
conservar esse tipo de qualidade --- até porque tanto a juventude, quanto
a beleza que lhe é característica são, essas sim, \emph{por natureza},
passageiras. Vida miserável e infeliz, posto que com o aumento do número
de filhos que, por sua vez, aumenta o trabalho pesado, as noites sem
dormir, as preocupações e, por fim, as brigas com o marido (que também
se vê cada vez mais pressionado do ponto de vista econômico), logo a
esposa se encontra arruinada fisicamente: a sua beleza e a sua saúde a
abandonam. Assim, as qualidades que, então, garantiam à mulher, na
melhor das hipóteses o casamento, não tendiam a ser conservadas quando
atingido o fim para o qual essas qualidades supostamente se destinariam.
Quanto à alma e interioridade, isso seria completamente dispensável,
tanto antes, quanto depois do casamento. Conforme escreve no seu
``Casamento e amor'':

\begin{quote}
Não há necessidade de a mulher saber qualquer coisa sobre o marido,
exceto a sua renda. E o que o homem precisa saber sobre a mulher que não
seja se ela possui uma aparência atraente? Ainda não superamos o mito
teológico de que a mulher não tem alma, que ela é um mero apêndice do
homem, feita da costela apenas para a conveniência do cavaleiro em
questão {[}\ldots{]}.

Talvez a má qualidade da matéria-prima com que a mulher foi feita seja a
responsável pela sua inferioridade. De qualquer forma, a mulher não tem
alma --- o que haveria, então, de se saber sobre ela? Além disso, quanto
menos alma uma mulher tiver, mais adequada estará à condição de esposa,
mais prontamente irá se deixar absorver em seu marido.
\end{quote}

Essa condição ``naturalizada'' e ``espiritualizada'' de mercadoria
sexual foi garantida via a sua escamoteação e santificação sob o manto
da Moralidade. Em ``Vítimas da moralidade'', Goldman é extremamente
direta ao expor a sua compreensão de que a moralidade e religião estão à
serviço das instituições que garantem a opressão econômica e social.
Segundo suas palavras: ``as instituições não poderiam se manter, não
fosse pela religião, que funciona como um escudo, e pela moralidade, que
funciona como uma máscara. Isso explica o interesse que os exploradores
ricos têm tanto na religião, quanto na moralidade.'' Através da
imposição da moralidade pelas instituições religiosas como o único
parâmetro verdadeiro de conduta, os mecanismos de opressão são envoltos
em superstição, o que tem por efeito oferecer àquilo que é violência,
usurpação, sufocamento, perversidade a aparência de sagrado, de amor, de
verdade, de tabu. Dito de modo cru, Goldman denuncia ser esse o papel
desempenhado pela igreja e moralidade tanto no que diz respeito à
instituição casamento, quanto também, de modo mais insidioso, no que diz
respeito à instituição propriedade privada. ``Mesmo que todos saibam que
a Propriedade é um roubo;'' escreve também em ``Vítimas da moralidade'',
``que representa o esforço acumulado de milhões de pessoas que são
desprovidas de propriedades'', ``a Moralidade da Propriedade estabelece
que essa instituição é sagrada. Ai de quem se atrever a questionar a
santidade da propriedade ou pecar contra ela!'' Talvez não seja exagero
dizer que, ainda hoje, talvez até mais do que nunca, a Moralidade da
Propriedade continua a imperar majestosa; e como antes, a imperar mesmo
entre as ``pessoas progressistas'' e os ``trabalhadores com consciência
de classe''; sem que esqueçamos os casos dos socialistas e anarquistas,
que, conforme faz atentar em ``A tragédia da mulher emancipada'', embora
defendam a ideia de que a ``propriedade é um roubo, ficam indignados se
alguém lhes deve algo no valor de meia dúzia de alfinetes.''

Na análise de Goldman, casamento e propriedade são indissociáveis, como
se duas faces de uma mesma moeda. É interessante observar que se,
\emph{de um lado}, Goldman coloca a instituição casamento como
fundamento da propriedade privada e com isso da opressão mesma --- vide,
nesse sentido a declaração contida em ``Casamento'': ``as relações
conjugais {[}são{]} o fundamento da propriedade privada e, portanto, o
fundamento do nosso sistema cruel e desumano'' ---; \emph{de outro lado},
a própria estrutura interna do casamento é explicada via a estrutura da
opressão econômica que possibilita. Se, para a mulher, segundo sua
análise, o casamento seria o ``emprego'' \emph{par excellence}, cujas
funções, de escopo restrito, iriam da procriação indiscriminada de
crianças às atividades de cozinheira e faxineira doméstica (além do
malabarismo para diminuir as despesas, apesar da contínua chegada das
crianças); para o homem, o casamento possibilitaria, no seio da família,
o exercício do domínio que o capitalismo, a outra ``instituição
paternalista'', exerce sobre ele, quando fora do lar: ``O sistema que
força a mulher a vender a sua feminilidade e independência ao melhor
candidato'', escreve em ``Anarquia e a questão do sexo'', ``é um ramo do
mesmo sistema malévolo que dá a poucos o direito de viver da riqueza
produzida por seus companheiros''. O resultado dessa transposição de mão
dupla entre opressão externa e relações privadas não poderia ser mais
infeliz. O que é ilustrado no primeiro parágrafo de ``Casamento'',
quando entoa um lamento ao modo dos coros trágicos:

\begin{quote}
Quanta tristeza, miséria, humilhação; quantas lágrimas e maldições; que
agonia e sofrimento essa palavra tem trazido à humanidade. Do seu
nascimento até a nossa atualidade, homens e mulheres crescem sob o jugo
ferrenho da instituição casamento, de modo que parece não haver nenhum
alívio, nenhuma maneira de escapar dela.
\end{quote}

Goldman que foi operária e que, inclusive, conheceu o seu primeiro
marido na fábrica, ou seja, no ambiente de trabalho, levou em conta, nas
suas análises, as diferenças das condições em que o casamento se
estabelece nas classes média e alta, de um lado, e nas classes
trabalhadoras, de outro, sobretudo porque às mulheres das classes
trabalhadoras nunca foi estranho o trabalho fora do lar, seja na
condição de criada, empregada doméstica, camponesa, datilógrafa,
operária, vendedora etc. Muito antes que as mulheres das classes médias
atinassem para a relação entre emancipação e mercado de trabalho, as
jovens da classe trabalhadora já se encontravam doentes e exaustas da
sua ``independência''. O problema é que nesses casos, a condição de
trabalho, se inevitável, tendia a ser encarada pelas próprias mulheres
como temporária, pronta para ser descartada ante o primeiro pretendente.
Vide o seu diagnóstico em ``A tragédia da mulher emancipada'': ``A
chamada independência que possibilita tão somente o ganho da mera
subsistência não parece tão atraente, tão ideal, a ponto de ser possível
esperar que a mulher venha a sacrificar tudo por ela''. Ora, o que não é
uma novidade hoje, tampouco o era na época: a mulher faz o mesmo
trabalho que o homem e o seu salário é menor --- o que numa condição de
exploração extrema traz consigo, necessariamente, a miséria. Por outro
lado, esse tipo de formação econômico-cultural tornava, segundo essa tão
experiente ativista, ``infinitamente mais difícil'' organizar
politicamente as mulheres do que os homens. Afinal, pergunta em
``Casamento e amor'', para quê lutar contra a exploração do trabalho e
correr riscos desnecessários, se a função suprema da mulher seria a da
maternidade no interior da sacralidade do lar?

Independentemente das variações da instituição casamento nas diferentes
classes, o ponto nerval é que tal instituição ao transformar a mulher,
no melhor dos casos, numa mercadoria sexual que só deveria ser violada
após o casamento, acabou por transformar o seu ideal na mesma coisa que
a sua desgraça. Nos textos aqui dispostos, Goldman repete, à exaustão,
que a única diferença entre a prostituta e a mulher casada é que uma
vende a si mesma ``como escrava privada durante toda a vida, por uma
casa ou um título'', e a outra vende a si mesma ``pelo período de tempo
que deseja''. Vale lembrar que na época de Goldman, embora o divórcio
estivesse crescendo de modo mais do que significativo nas estatísticas
(vide os dados expostos em ``Casamento e amor''), ainda implicava uma
pesada condenação pública à mulher e seus filhos. Pois com ou sem
paixão, com ou sem amor, o casamento era o único meio de subsistência
sancionado social, moral e legalmente à mulher, para o qual, em
contrapartida à sua respeitabilidade de esposa, deveria dar em troca
muito mais do que a prostituta: a sua própria pessoa. ``Quando o
dinheiro, o status social, e a posição são os critérios do amor'',
escreve em ``Causas e possível cura para o ciúme'', ``a prostituição se
apresenta como inevitável, ainda que esteja coberta com o manto da
legitimidade e da moralidade''. Para a anarquista, o casamento não era
nada mais do que uma forma de prostituição sancionada pela Igreja e pelo
Estado. Ou, conforme suas palavras em ``Tráfico de mulheres'': ``para os
moralistas, a prostituição não consiste tanto no fato de que a mulher
venda o seu corpo, mas, ao invés disso, que ela venda o seu corpo fora
do casamento''.

Por ter vivido uma temporada, num hotel, entre prostitutas, além das
experiências, longas e curtas, nas cadeias e presídios, Goldman sabia
muito bem das condições de exploração em que elas viviam, cerceadas
pelos ``cadetes'' (uma subcategoria de cafetão), pelos policiais, pela
clientela, Igreja, opinião pública e o que mais houvesse. Não obstante,
havia também o caso das jovens que, eventualmente, complementavam a
renda recorrendo à prostituição, para o que foi cunhado nos Estados
Unidos da época, o termo \emph{grisette.} Segundo a breve e encantadora
biografia escrita por Elizabeth Souza Lobo, a própria Goldman se valeu
desse ``recurso''. Inspirada por Sônia de \emph{Crime e castigo}, que se
prostitui para atenuar minimamente a miséria da família, Goldman, relata
Lobo, teria recorrido à prostituição com o intuito de conseguir o
dinheiro necessário para compra das armas da tentativa de assassinato,
mencionada acima, de Henry Clay Frick por Alexander Berkman, então seu
companheiro político e amante. \footnote{\textsc{lobo}, Elizabeth S. Emma
  Goldman: A Vida como Revolução. São Paulo: Editora Brasiliense, s/d.}
Talvez, fosse mais preciso dizer que aí, Goldman teria terminado por se
inspirar não só em Sônia como também em Raskólnikov\ldots Anedotas à parte,
nos textos que se seguem, encontrar-se-á exposta a constatação nua e
direta de que a causa da prostituição é a mesma que a do casamento: a
exploração econômica via a questão sexual.

Valendo-se sobretudo de William W. Sanger, Goldman sentencia a
prostituição como \emph{a consequência direta de uma remuneração
desproporcional ao trabalho honesto.} A esmagadora maioria das
prostitutas, segundo os estudos que relata, era formada por mulheres e
garotas da classe trabalhadora. Pois a precariedade econômica retirava
o, então, privilégio ``moral'' de esperar pelo casamento na segurança da
interioridade de um lar minimamente estruturado, ao passo em que
obrigava o lançamento no mundo do trabalho precarizado cujos dentes
foram e são mais afiados no caso da mulher. Conforme escreve em
``Tráfico de mulheres'': ``o sistema industrial não possibilita à
maioria das mulheres outra alternativa que não seja a da prostituição''.
Igualmente fundamentada em estudos e estatísticas, Goldman também
chamará a atenção para a relação diretamente proporcional entre o
aumento da prostituição e o desenvolvimento do capitalismo com sua
sociedade de massa. O que pretende é explicar, o mais didaticamente
possível, que combater a prostituição pela via da moralidade ou através
de leis punitivas, pior do que ser ineficaz e inútil, agrava o problema.
Em vários textos aqui presentes, ela busca demonstrar que a repressão
legislativa e moral dos chamados ``vícios'', invariavelmente, tem como
resultado o desvio do mal para ``vias secretas'', que, escondido,
multiplica, sob a proteção do silêncio, os perigos para a comunidade.
Caso, conforme apresenta em ``A hipocrisia do puritanismo'', das
``cruzadas'' contra o consumo e venda de bebidas alcoólicas, contra os
jogos de azar e a prostituição. Posto o resultado ser sempre o mesmo:
``Os jogos de azar continuam aumentando, os bares estão realizando
negócios às escondidas, a prostituição está no auge, e o sistema
coordenado por cafetões e cadetes se intensificou''. Em outras palavras,
a aprovação e execução de leis punitivas têm por efeito tornar ainda
mais promissores os vícios que supostamente estariam sendo combatidos.
Naturalmente, alguém lucra com isso. Em ``Tráfico de mulheres'', Goldman
relata que com a extinção dos bordeis, como estratégia legal para o
combate à prostituição, o resultado foi o aumento da corrupção policial
via o suborno, o lançamento das jovens à violência das ruas e das
delegacias e, o conseguinte, aparecimento dos cadetes. Ela também
destaca a ironia de ser a Igreja da Trindade, a, então, maior
proprietária dos imóveis do mais famoso centro de prostituição de Nova
York --- o que atualiza a sua tese da correlação entre religião e
prostituição, já que segundo os estudos que apresenta, a prostituição
tem origem comprovadamente religiosa, o que não exclui a religião
cristã, apesar da sua Virgem Maria. Outro caso mencionado por ela, desse
tipo de ocultamento moral que empurra o diabo para mais fundo no
sistema, diz respeito às doenças venéreas --- ainda hoje um tabu. Como
disse do seu tempo, poderia ter dito do nosso: a ``cegueira deliberada''
imposta pela moralidade implica recusar a platitude de que ``o
verdadeiro método de prevenção é aquele que deixa claro para todos que
`doenças venéreas não são uma coisa misteriosa ou terrível, não são um
castigo pelos pecados da carne, alguma espécie de mal do qual se deva
ter vergonha {[}\ldots{]}; mas, sim, que são doenças comuns que podem ser
tratadas e curadas'''. Certamente, teremos muito a ganhar, no caso de
refletirmos seriamente sobre esse postulado, aparentemente, tão simples,
acerca do ``verdadeiro método de prevenção'', quando a despeito dos
avanços na medicina, as doenças venéreas não param de se multiplicar.

O efeito mais pernicioso da Moralidade sobre as mulheres --- mais
pernicioso porque primeiro (no sentido de elementar, originário) ---, diz
respeito à repressão sexual. Seguindo as pistas de Freud e dos demais
``psicólogos do sexo'', para Goldman dentre todas as forças que atuam
sobre nós, seres humanos, o impulso sexual se não a única, é a mais
importante. Conforme escreve em ``O elemento sexual da vida'', o sexo é
a ``função biológica primária'' de toda forma de ``vida superior'', é a
força que faz as suas marés baixarem e subirem, de modo que a ele,
``devemos mais do que à poesia'': do canto dos pássaros à música, da
plumagem das aves-do-paraíso à juba do leão; das formas superiores de
vida do mundo vegetal e animal à própria cultura com seus costumes não
raro tolos, insensatos e injustos; tudo isso, escreve Goldman, que torna
``essa coisa que chamamos de cultura humana o mosaico incrível e variado
que é'' --- devemos debitar na conta do sexo. Em outro momento dessas
suas anotações, irá afirmar que a raiz mesma da beleza e do nosso
sentido estético reside no elemento sexual, do que imediatamente conclui
ser também o elemento sexual a raiz da nossa sociabilidade como um todo.
Amparada pelo discurso psicanalítico da época, segundo o qual a pulsão
de vida seria biologicamente determinada no sentido de buscar sempre, e
cada vez mais, agregar a substância viva dispersa em partículas (o que
tenderia a tornar a vida cada mais complexa, variada e, no nosso caso,
multicultural); Goldman irá compreender a sexualidade para além da busca
pelo gozo propriamente dito, também como socialização e criatividade. O
``instinto sexual é o instinto criativo'', postula; e é por espalhar
``signos de atração'' por toda parte, por expressar, em todo lugar e em
toda instante, ``essa grande necessidade de união'', transformando tudo
em ``indício, símbolo, lembrança, mensagem, chamado''; que essa
``faculdade'', conforme escreve, ``é social'' e ``o começo do panorama
da arte''. Numa sentença: ``A natureza sabe sempre mais'' --- e é a ela
que devemos nos voltar, de modo a nos livrarmos da ``doutrina profana e
antinatural, iniciada pelos primeiros monges cristãos, de que o impulso
sexual é o sinal de degradação do homem e a fonte da sua energia mais
diabólica''.

A sua crítica às instituições da moralidade e da religião extrapola,
portanto, os limites da denúncia do papel que exercem na escamoteação da
opressão social e econômica; tais instituições atacam a vida na sua
própria ``raiz'': o elemento sexual. Nas trilhas de Nietzsche, Goldman
irá compreender a moralidade e religião como antinaturais, como
caluniadoras e sufocadoras da vida. Vide, nesse sentido, a oposição que
estabelece, em ``A hipocrisia do puritanismo'', entre vida e a concepção
puritana de vida. Por ser sempre movida pela criação que tem por fim a
união, a vida, escreve, ``para além da arte, para além do esteticismo'',
``é representação da beleza em milhares de variações; é, de fato, o
panorama gigante da eterna mudança''. Já a ``concepção de vida'' do
puritanismo, derivada da ``ideia calvinista de que a vida é uma maldição
imposta ao homem pela ira de Deus'', diz respeito, ao contrário, a uma
vida, idealmente, ``fixa e imóvel'' --- ideal que impõe à vida mesma, ao
ser humano animado por ela, a necessidade de penitências constantes, o
repúdio a ``todo impulso natural e saudável'', o abandono de toda
``alegria e beleza''. A ``atividade sexual'' não era, portanto, para
essa ativista política, algo como ``um ato isolado''; mas sim ``uma
experiência generalizada que motiva e afeta a personalidade'' (``O
elemento sexual da vida''). Daí a sua famosa refutação a Kropótkin,
quando ele criticou o excesso de espaço que ela, em geral, dava à
discussão sobre o sexo, uma vez que, para ele, a coisa seria resolvida,
por assim dizer, por uma via mais racional: ``quando ela {[}a mulher{]}
for intelectualmente igual ao homem e compartilhar dos seus ideais
sociais, ela será tão livre quanto ele'' --- ter-lhe-ia dito este que, na
época, já era um reverenciado anarquista. Segundo o relato na sua
autobiografia \emph{Living my life}, a sua resposta não poderia ser mais
surpreendente. Depois de ouvir em ``crescente agitação'', a explanação
vagarosa e arrastada de Kropótkin, Goldman respondeu-lhe: ``Concordo,
camarada, que quando eu atinja a sua idade, a questão do sexo já não
tenha importância para mim. Mas \emph{agora}, trata-se de um fator de
crucial importância para milhares, milhões até, de jovens''. Ainda
segundo ela, um Kropótkin que, em geral, não lhe era simpático, depois
dessa, deu-se por vencido. Anos depois a esse encontro, Goldman parece
ir ainda mais longe; por exemplo, quando afirma, em ``Causas e cura
possível para o ciúme'', ser autoevidente que o ``radical completo'' ---
já que, segundo, ela ``há muitos radicais meia boca'' --- deve aplicar o
conhecimento da centralidade e importância do sexo às suas ``relações
sexuais e amorosas''. Note-se que, aqui, ela traz para a pauta da
necessidade de adequação entre teoria e prática --- temática central à
tradição revolucionária a que reivindicava o pertencimento --- a questão
da sexualidade, do amor: o radical \emph{completo} deve aplicar esse
conhecimento que, em primeiro lugar, diz respeito às suas relações mais
íntimas. Vide também quando, em ``A hipocrisia do puritanismo'', ela
parece inverter a equação de Kropótkin acima colocada, ao sugerir que é
mais provável que a mulher atinja a igualdade intelectual para com o
homem, quando \emph{antes} for livre para vivenciar as suas paixões e
sexualidade. Para fundamentar esse ponto, faz uso de Freud, quem,
segundo ela, acreditava ``que a inferioridade intelectual de tantas
mulheres está relacionada à inibição do pensamento imposta sobre elas
com o fim da repressão sexual''. O que, por sua vez, ecoa às palavras de
Mary Wollstonecraft, citadas pela própria Goldman no seu ensaio sobre a
pensadora também disposto na presente coletânea. Segundo Wollstonecraft,
na referida passagem:

\begin{quote}
Regular a paixão nem sempre é uma atitude sábia. Ao contrário, talvez
seja justamente essa uma das razões pela qual os homens têm um
julgamento superior e maior coragem do que as mulheres: eles dão livre
curso à sua grande paixão e, por perderem a si mesmos com mais
frequência, ampliam as suas ideias.
\end{quote}

Sob as bênçãos da Igreja e do Estado, a instituição casamento foi
imposta como a única válvula de escape legalizada e socialmente aceita
contra o \emph{pernicioso} despertar sexual da mulher. Na ausência de
outras ``opções'', ter-se-ia, de um lado, a abstinência sexual --- caso
das, popularmente, conhecidas como ``solteironas'' --- e, de outro, a
prostituição. Pois antes da opressão econômica como causa do casamento e
da prostituição, Goldman parece alocar a repressão sexual. Como se a
redução da mulher à condição de mercadoria sexual exigisse,
\emph{antes}, para essa redução mesma, a repressão sexual. Segundo suas
palavras, em ``Tráfico de mulheres'': ``Seria tanto parcial, quanto
extremamente superficial considerar que o fator econômico é a única
causa da prostituição''. É fato conhecido que embora o ``sexo'' tenha
sido o principal atributo negociável da mulher, ao longo de eras, sob a
exigência da Moralidade, (ao menos no que diz respeito ao universo
judaico-cristão), a mulher foi privada de qualquer espécie de
``treinamento'' ou mesmo de qualquer conhecimento sobre o ofício que
deveria necessariamente desempenhar. ``Por mais estranho que pareça'',
escreve em ``Casamento e amor'', é permitido à mulher saber ``muito
menos sobre a sua função como esposa e mãe do que a um artesão comum
sobre o seu ofício.'' Note-se aqui a charada através da qual a mulher
foi subjugada naquilo que, para Goldman é o mais fundamental: o elemento
sexual. Pois, ao mesmo tempo em que era incutido na mulher, desde a
infância, que o casamento seria o seu objetivo final --- e isso de modo
que ``todo o seu treinamento e educação'' eram ``dirigidos com vistas a
esse fim'' ---, o sexo, paradoxalmente, era-lhe um tema-tabu, impuro e
imoral, a ponto de ser uma indecência a simples menção à temática. Sem
saber nada da ``função mais importante a ser exercida na sua vida'',
conclui, mais uma vez, em ``Tráfico de mulheres'', é natural que uma
mulher não soubesse ``cuidar de si mesma'', o que a tornava ``uma presa
fácil da prostituição ou'' --- o que é ainda uma realidade, conforme
comprovam as nossas estatísticas em muito defasadas ---, uma presa ``de
\emph{qualquer outra forma de relacionamento que a degrade à posição de
objeto para uma gratificação meramente sexual}'' .

No caso das jovens, muitas vezes ainda crianças, cuja necessidade
econômica privou do lar e de qualquer conforto, obrigando ao trabalho
precoce nas fábricas e, portanto, a uma rotina diária ao lado de homens
de todas as idades, nada mais natural que, em algum momento, terminassem
por se entregar à primeira experiência sexual. Para Goldman, às jovens
das classes trabalhadoras era possível uma expressão mais normal dos
seus instintos físicos e, com isso, do amor. ``Os rapazes e moças do
povo não são moldados de modo tão inflexível pelos fatores externos'',
diagnostica em ``Vítimas da moralidade'', ``e frequentemente se lançam
ao chamado do amor e da paixão, independentemente dos costumes e
tradição''. O problema é que a perda da virgindade ``sem a sanção da
Igreja'', em conjunto com a precariedade econômica e social, não raro se
convertiam, para essas jovens, num ``primeiro passo em direção à
prostituição''. Curiosamente, porém, no que diz respeito às jovens
mulheres oriundas das classes e famílias mais estruturadas, a situação é
descrita por Goldman como ainda pior. Pois o ``privilégio'' de ter a
sexualidade ``protegida'' na interioridade do lar paterno implicava que
a jovem em questão só poderia exercer a sua sexualidade quando
encontrasse um rapaz que não apenas estivesse disposto a casar-se com
ela, como também que fosse dotado do montante de dinheiro considerado
suficiente para sustentar a vindoura prole. Até que conseguisse tal
montante, isso poderia custar ao jovem casal, a espera de muitos e
cansativos anos para a sua primeira relação sexual; muito embora, os
custos fossem aí consideravelmente desiguais --- o que não é uma novidade
para ninguém. Aos homens, mesmo se comprometidos, era socialmente
permitido o exercício da sexualidade, o que tornava a prostituição,
nesse contexto, uma instituição praticamente necessária à instituição
casamento. No que diz respeito à jovem noiva, a ela só caberia subjugar
a sua saúde, vida, paixão e desejo até que o ``bom'' partido em questão
estivesse financeiramente apto a tomá-la como esposa. Daí o que
convencionou-se chamar de ``padrão duplo da moralidade'' --- que, então,
se desdobrava em formas de educar tão completamente distintas, em
hábitos e costumes condizentes a mundos tão profundamente separados, que
homens e mulheres teriam sido transformados em seres, praticamente,
alienígenas um ao outro, a ponto de a união amorosa entre ambos estar,
como se por princípio, fadada ao fracasso.

Entre seres estranhos um ao outro, \emph{moralmente} divergentes um do
outro no que diz respeito à sexualidade e ao amor, o desencontro sexual
e afetivo não poderia ser mais absoluto. Mesmo na interioridade
legalizada do casamento e do lar, dificilmente, a mulher (especialmente
as oriundas das classes médias) poderia encontrar o prazer sexual, em
geral, por ter internalizado o tabu nela mesma, embora também houvesse o
caso de que o medo de desagradar o parceiro com um comportamento julgado
inadequado a uma esposa fosse a razão da sua, por assim dizer,
inacessibilidade ao prazer sexual. De acordo com o seu diagnóstico em
``Casamento e amor'', não era raro que uma esposa, dada a sua educação e
criação, se sentisse verdadeiramente ``chocada, rechaçada, indignada
para além de qualquer medida com o mais natural e saudável dos seus
instintos, o sexo''. É nessa linha que compreenderá como ``efeito do
sufocamento imposto pelo tabu sexual'' tanto o mito de que a mulher
possui um interesse sexual menor do que o do homem quanto o problema,
ainda hoje alarmante, da frigidez sexual entre mulheres sexualmente
ativas. De outro lado --- como se, aparentemente, invertendo o vetor ---,
ela escreverá em ``O elemento sexual da vida'', que a
``incompatibilidade de temperamentos'' que torna tão infelizes os
casamentos, especialmente com o passar dos anos, é ``resultado direto da
ausência de harmonia sexual''; ou, em outras palavras: ``a insatisfação
e os atritos surgem, quando a natureza química do sexo entre marido e
mulher falha em os unir harmoniosamente'' --- uma observação
interessante, inclusive para analisarmos os nossos próprios
relacionamentos amorosos e sexuais. Até porque, segundo postula em
``Casamento'': ``onde não há igualdade não pode haver harmonia''.

Através da sua análise implacável do casamento, Goldman coloca o seu
leitor (ou, no seu tempo, ouvinte) diante da história da ``diferença de
posição e de privilégios'' no que diz respeito a homens e mulheres ---
uma diferença que, como não é novidade para ninguém, continua a ser
``existente ainda hoje''. Em face de tal ``afresco'' de infelicidade
conjugal universal, não é de admirar que tenha repetidamente negado,
indo contra muitos radicais do seu tempo, ser possível reformar, de modo
satisfatório, essa instituição --- fosse através da melhoria das leis
matrimoniais, como, por exemplo, com a flexibilização do divórcio, fosse
através da escolha de se casar livremente, isto é, sem o consentimento
da igreja e da lei. Pois o problema, para ela, é que a instituição
casamento é em si mesma degradante: ``Não importa o quanto mude, sempre
dá ao homem direito e poder sobre sua esposa, não apenas sobre o seu
corpo, mas também sobre suas ações, seus desejos; na verdade, sobre a
sua vida como um todo'' --- escreve no seu ``Casamento''. Daí que acuse
os radicais de fazerem ``concessões aos padrões morais do nosso tempo'',
posto, que apesar de toda a sua crítica à Igreja, ao Estado e à
propriedade privada continuavam a vivenciar as relações sexuais e
amorosas sob os parâmetros do monopólio sexual; o que seria prova tanto
de ``falta de coragem'', quanto de ausência de ``energia para desafiar a
opinião pública e viver a prática em sua própria vida''. Em ``A nova
mulher'', Goldman também critica os radicais ao constatar que, em geral,
eles não desejavam que suas esposas se tornassem elas mesmas radicais,
já que à mulher a posição mais confortável --- mesmo para um homem
supostamente emancipado de preconceitos --- é a de alguém a ser
protegido. E nesse ponto, vale mais uma vez retornar a ``Casamento'' e
citar uma das passagens em que a semelhança com a atualidade, mais de
cem anos depois, é por demais flagrante:

\begin{quote}
Vocês matraqueiam sobre a igualdade dos sexos na sociedade do futuro,
mas pensam que é um mal necessário que a mulher deva sofrer no presente.
Vocês dizem que a mulher é inferior e mais fraca, e ao invés de
auxiliá-la a se tornar mais forte, ajudam a mantê-la numa posição
degradante. Vocês exigem de nós exclusividade, mas amam a variedade de
parceiras e aproveitam essa variedade sempre que têm a chance.
\end{quote}

É fato que, anos depois, quando, em 1925, casou-se com Bolton, Goldman
se viu em apuros ante este tipo de declaração implacavelmente contra o
casamento; caso também, por exemplo, da entrevista de 1897 (``O que há
na anarquia para as mulheres?''), quando à pergunta do jornalista sobre
se teria intenção de casar, responde de modo taxativo: ``Não; eu não
acredito em casamento no que diz respeito aos outros e, certamente, não
sou de pregar uma coisa e fazer outra''. Se as razões oferecidas por
ela, no mencionado artigo de 1926, ``As visões de Emma sobre o amor'',
em que foi convidada a explicar essa ``auto-heresia'', foram ou não
suficientes --- dadas inclusive as suas condições de existência na época
que, vale repetir, incluíam além da origem judaica, o estigma de exilada
política dos Estados Unidos e de ``foragida'' da União Soviética
(estigmas potencializados pela sua incessante militância política) ---, é
algo que aqui não será problematizado. Pois o ponto verdadeiramente
digno de relevância é o de que, segundo sua análise, há, na instituição
casamento, a convergência de uma série de mecanismos de opressão, que se
confundem uns com os outros, na medida em que são indissociáveis. E mais
do que isso: a própria base ``sentimental'' deste modo supostamente
ideal de relacionamento amoroso-sexual seria, na verdade, o ``maior dos
males da nossa vida amorosa mutilada''.

Conforme desenvolve em ``Causas e cura possível para o ciúme'', o
monopólio sexual sobre o qual se fundamenta o casamento --- uma clara
derivação da ``Moralidade da Propriedade'' --- terminou por envenenar a
nossa forma mesma de amar, uma vez que o ciúme passou a se apresentar
como algo ``natural'' ao amor. O monopólio sexual, ``transmitido de
geração a geração como um direito sagrado e a base da perfeição da
família e do lar'', terminou por transformar o ``objeto'' do amor numa
espécie de propriedade privada, em meio a outras propriedades privadas
de outras naturezas. Nesse sentido, é que a anarquista irá conceber o
ciúme como uma espécie de ``arma'' sentimental ``para a proteção desse
direito de propriedade''. ``Arma'', porque o ciúme entra em cena
justamente quando, com ou sem motivos, o indivíduo sente alguma ameaça
ao seu monopólio sexual encarnado no seu parceiro ou parceira; ``arma'',
porque implica ``revolver os órgãos vitais'' daquele quem supostamente
se ama (e de si mesmo) ante o menor indício de desejo por uma outra
pessoa. Descrito por Goldman, como um misto de inveja, fanatismo, posse,
vontade obstinada de punição e sobretudo vaidade ferida, o ciúme, em
nada se relaciona com a ``angústia'' oriunda de ``um amor perdido'' ou
do ``fim de um caso de amor''; como tampouco é resultado do amor. Ao
contrário, para ela, o ciúme é ``o próprio reverso do entendimento, da
simpatia e dos sentimentos generosos''. É verdadeiramente surpreende a
sua compreensão de que, na maioria dos casos, a virulência do ciúme é
tanto maior, quanto menor for o amor e a paixão. ``O aspecto grotesco
desse assunto todo'', escreve também em ``Causas e cura possível para o
ciúme'', ``é que homens e mulheres normalmente se tornam violentamente
ciumentos com pessoas que, na realidade, não lhes importam muito''. Que
``a maioria das pessoas'' continue a viver perto uma da outra, embora
tenham, há muito, ``cessado de viver uma com a outra'' --- esse, e não o
amor, é, para Goldman, o solo ``fértil'' para a atividade do ciúme.

Certamente, um dos ensinamentos mais comoventes do presente livro é o
truísmo de que numa relação amorosa não pode haver algo como
conquistadores e conquistados, dominadores e dominados, pois o amor é em
si mesmo livre e ``não pode viver em outra atmosfera''. ``Amor livre?''
--- pergunta em ``Casamento e amor'' --- ``Como se o amor pudesse não ser
livre!'' Não há dinheiro que possa comprar o amor, não há força que seja
capaz de subjugá-lo, não há lei ou punição que possa arrancá-lo, uma vez
que tenha criado raízes. É interessante observar que Goldman traz para a
relação amorosa mais íntima, um tipo de radicalidade que, em certo
sentido, constituiu o cerne do espírito revolucionário populista, para o
qual essa nossa noção moderna de liberdade --- que, em alguma medida,
opõe-se à igualdade --- nada mais é do que egoísmo e mesquinhez. Ao invés
da liberdade como autossoberania e direito à propriedade privada, os
populistas a compreenderam como a necessidade de se estar disposto ao
autossacrifício em nome da liberdade mesma, que uma vez que diga
respeito ao todo, não pode estar limitada a algo tão cerceado quanto a
propriedade privada. Vide, nesse sentido, a sua definição do ``amor'',
ante o questionamento do jornalista na entrevista ``O que há na anarquia
para as mulheres?'': amar é o ``desejo irresistível de fazer bem à
pessoa, até diante do sacrifício de desejos pessoais''; como se nessa
doação mesma estivesse a possibilidade de finalmente nos transfigurarmos
em nós mesmos. Ou ainda, o que anos depois, irá afirmar em ``A tragédia
da mulher emancipada'': ``Uma concepção verdadeira da relação entre os
sexos {[}\ldots{]} conhecerá apenas uma grande coisa: doar-se sem limites,
a fim de encontrar um eu mais rico, profundo, melhor''. Que isso só
possa acontecer em relação a uma única pessoa, ao longo de toda uma
vida, ou mesmo em relação a uma única pessoa por vez não faz sob a
concepção tal qual exposta por Goldman, sentido algum. Pois o amor e a
sexualidade, assim como a liberdade, a criatividade e a sociabilidade,
encontram, pela sua própria natureza, expressões variadas, múltiplas e
mutáveis. Cada ``caso de amor'', escreve em ``Causas e cura possível
para o ciúme'', é ``independente, diferente de qualquer outro'', está
profundamente relacionado com ``as características físicas e psíquicas''
dos envolvidos. Daí a sua pergunta retórica ao jornalista, na entrevista
supracitada: se uma pessoa encontrar as ``mesmas características que lhe
fascinam em diferentes pessoas'', ``o que poderia lhe impedir de amar
essas mesmas características em diferentes pessoas?'' Aqui, porém, não
há receita: a própria natureza da relação amorosa impõe que seja um
assunto concernente, de modo exclusivo, aos envolvidos. Que tenhamos
compreendido a mais alta realização do amor a partir do ideal do
monopólio sexual encarnado na instituição casamento revela, para
Goldman, o ``nosso estado atual de pigmeus''.

Se de um lado a exigência da radicalidade revolucionária é trazida para
o âmbito das relações amorosas privadas e mais íntimas, de outro a sua
defesa do ``amor livre'' ultrapassa essa esfera mesma, a das relações
amorosas privadas, da satisfação exclusivamente sexual --- o que, talvez,
vá de encontro ao que, à primeira vista, a expressão ``amor livre''
possa sugerir. Não podemos esquecer, em nenhum momento, que estamos aqui
diante de uma \emph{radical}. Para Goldman, o amor é a \emph{ânima} que
move todo aquele que anseia pela liberdade, que o faz arder e se lançar,
como disse sobre as mulheres heroicas da Revolução Russa, aos ``feitos
mais ousados'' ​​e seguir ``para a morte ou para a Sibéria'' ``com um
sorriso nos lábios''. E aqui, vale, mais uma vez, remetermo-nos ao texto
``Casamento e amor'', no qual, \emph{apesar do título}, o amor não é
restrito ao âmbito das relações sexuais --- antes o contrário. Pois o
amor, segundo escreve, é o ``arauto da esperança, da alegria, do
êxtase''; é a força ``que \emph{desafia todas as leis}, todas as
convenções''; é o ``\emph{mais poderoso modelador do destino humano}'';
é ``o único alicerce criativo capaz de fazer ascender {[}\ldots{]} a um
novo mundo'' --- e isso, seja digno de nota, quer ``dure apenas um breve
lapso de tempo ou por toda eternidade''. Em Goldman, sexualidade, gozo,
compaixão, cuidado, delicadeza, rebeldia, amor, liberdade, coragem,
ousadia, luta, criatividade, arte, revolução são potencialidades que se
trespassam. No seu comovente ensaio sobre Mary Wollstonecraft, isso é
ainda mais evidente. Pois ela define o ``verdadeiro rebelde'', como o
``verdadeiro pioneiro'', como aquele que \emph{cria ``}ao invés de se
submeter a uma forma {[}\ldots{]} estabelecida'', como aquele que é
consumido ``pelo fogo da compaixão e simpatia por todo sofrimento e por
todos os seus camaradas''; e muitas vezes, caso de Mary Wollstonecraft,
como aquele que, para o seu próprio desespero, é portador de um
``coração ingovernável'' e de uma sexualidade, não raro, ``faminta'' ---
dado ser a tragédia e a penalidade que acompanham todo grande espírito,
a impossibilidade de receber o amor que a sua alma anseia e que, como se
por transbordamento, está o tempo todo a doar.\footnote{Alguns
  estudiosos observaram que no comovente ensaio não publicado sobre Mary
  Wollstonecraft, Goldman estava, na verdade, também a falar de si
  mesma. E nesse ponto, como forma de ilustração e irresistível anedota,
  vale dá uma olhada numa das cartas escritas por Goldman ao médico e
  anarquista Ben Reitman. Segundo os seus biógrafos e ela mesma, o
  relacionamento longo e tumultuado foi marcado por uma paixão intensa,
  de um lado, e, de outro, pelo seu ciúme implacável por Reitman que
  tinha, abertamente, vários casos amorosos --- comportamento que se
  punha de acordo com as ideias mesmas pregadas por Goldman. Esse evento
  significativo da vida da escritora e ativista é um elemento
  interessante de se ter como pano de fundo não só durante a leitura do
  ensaio sobre Mary Wollstonecraft, mas também, por exemplo, do texto
  ``Causas e possível cura para o ciúme'', disposto na presente
  coletânea e já mencionado nesta introdução e que foi escrito em meio a
  uma das inúmeras idas e vindas atribuladas com Reitman. Eis, então,
  uma passagem da referida carta, tanto sensual, quanto atormentada,
  retirada da biografia escrita por Lobo: ``Vagabundo querido. Que
  pequenino você é, tão ingênuo, tão que nem uma criança. Realmente não
  deveria me zangar com você. Se ao menos não tivesse despertado a
  mulher em mim, a selvagem mulher primitiva, que anseia pelo amor e
  carinho do homem acima de tudo o mais no mundo. Tenho um grande e
  profundo instinto materno por você, meu bebê querido, esse instinto
  tem sido o lado redentor de nossa relação\ldots{} Mas meu amor maternal é
  apenas uma parte de meu ser, as outras 99 partes são da mulher, a
  intensa, apaixonada e selvagem mulher, a quem você deu vida como
  nenhum outro antes de você. Acho que aí se encontra a chave de nosso
  sofrimento e também de nossa grande felicidade, por mais rara que
  seja. Amo você com uma loucura que não conhece limites, nem desculpas,
  nem rivais, nem paciência, nem lógica. Quero seu amor, sua paixão, sua
  devoção, seu carinho. Quero ser o centro de seus pensamentos, de sua
  vida, de cada um de seus segundos. Qualquer coisa que afasta você de
  mim, mesmo que por um momento, me deixa louca e toma a minha vida um
  perfeito inferno. É porque amo você tanto que anseio pelo seu carinho.
  Quero realmente que cuide de mim. Ninguém jamais cuidou, coma sabe.
  Sempre tomei conta dos outras e fiz tudo par eles. Nunca quis que
  ninguém tomasse conta de mim. Mas com você desejo isso, oh tão
  intensamente \ldots Estive pensando que há algo mais profundo no fato de
  uma mulher se agarrar ao homem que ama. É o sentimento confortador de
  segurança, de se ter alguma pessoa que encontra prazer em fazer você
  feliz e não julga nenhum esforço por você difícil demais. Nunca senti
  falta disso, nunca me importei com isso, nunca imaginei que pudesse
  vir a ter necessidade disso, até que você entrou na minha vida, até
  que despertou esse lado da minha natureza, esse lado da minha
  psicologia. Olhe, meu precioso, você deu vida a uma força que não sabe
  coma manejar, como enfrentar, daí os conflitos, daí a falta de
  harmonia e paz.'' (\textsc{goldman} \emph{apud} \textsc{lobo}. Op. cit., pp. 51--2).}

Ora, aqui, finalmente, parece surgir a pergunta: mas e a maternidade?
Mesmo se não para todas as mulheres, certamente para muitas, a
maternidade é sentida e concebida como a mais plena realização da doação
de si, do amor. Segundo Goldman, mais do que isso, tratar-se-ia de ``um
desejo inato'', porque natural e fisiológico à mulher. Assim, cabe
também a pergunta sobre se não estariam os filhos melhores protegidos na
interioridade do lar sob os parâmetros da instituição casamento, no seio
estruturado da família. Nesse ponto, a crítica de Goldman não poderia
ser mais implacável, já que, para ela, o casamento não teria feito nada
além de ``desonrar, ultrajar e corromper essa realização''. É verdade
que, diferentemente do seu tempo, os filhos nascidos fora do casamento
já não são, em geral, coroados com o epíteto de ``Bastardo''; como
também é verdade que, diferentemente do seu tempo em que o controle de
natalidade era uma espécie de tabu, hoje gozamos não só de informações,
como do acesso a muitos dos métodos contraceptivos gratuitamente, além
de uma liberdade sexual incomparavelmente maior. Não obstante, parece
também ser verdade que apesar da extrema relevância dessas conquistas, a
``maternidade livre'', pela qual advogou sob o preço da sua liberdade
individual e a daqueles quem amava, não está por isso mais próxima seja
dentro ou fora do casamento tradicional --- ao menos não na maioria dos
casos. Conforme sentencia do modo cru e direto que lhe é característico,
no seu ``A nova mulher'': ``A beleza da maternidade, sobre a qual os
poetas cantaram e escreveram, é uma farsa, e não pode ocorrer enquanto
não tivermos liberdade --- econômica''.

É interessante observar nos textos que se seguem que, embora de modo não
sistemático, Goldman analisa as diferentes condições econômicas e
sociais em que uma mulher se torna mãe. Para o caso das mães casadas
que, além do trabalho doméstico, exerciam o trabalho remunerado fora do
lar ou das chamadas ``mães solteiras'' que tinham de prover sozinhas os
seus rebentos, a situação de exploração, tal qual descreve, não poderia
ser considerada mais amena. Em ``Casamento e amor'', ela se remete a
estatísticas que então demonstravam que 10\% das trabalhadoras
assalariadas de Nova York eram casadas, para, com isso, denunciar a
dupla exploração à qual essas mulheres estavam submetidas, já que
assomado ao trabalho do ambiente doméstico, teriam ``de continuar
trabalhando na condição de mão-de-obra pior paga do mundo''. Já ao
tratar do caso das ``mães solteiras'' que, segundo ``Os aspectos sociais
do controle de natalidade'', estariam a lotar as ``lojas, fábricas e
indústrias'', numa palavra, ``todos os lugares, não por escolha, mas por
necessidade econômica'', apesar de serem continuamente ignoradas pelos
moralistas no seu louvor à maternidade; Goldman se concentra na denúncia
do aborto clandestino. Num contexto em que meras informações sobre o
controle de natalidade eram expressamente proibidas por lei, às mulheres
que mantivessem uma vida sexual minimamente ativa fora da instituição
casamento --- o que tornava absolutamente impossível o sustento de uma
prole numerosa ---, o aborto clandestino, via de regra, se apresentava,
em face de uma gravidez acidental, como a única alternativa minimamente
viável. Em ``A hipocrisia do puritanismo'', ela chama atenção para as
estatísticas concernentes ao aborto nos Estados Unidos de então, que
demostrariam a proporção de 17 abortos a cada 100 gestações --- uma
proporção que apesar de gritante deixava transparecer apenas a ponta do
iceberg do problema, dado que somente os casos mais drásticos vinham à
tona, quando o aborto mal feito conduzia ao óbito ou a problemas de
saúde demasiado graves para serem ignorados. Não é preciso dizer que
esse tipo de consideração permanece extremamente atual, se não no que
diz respeito especificamente aos Estados Unidos, onde o aborto é na
atualidade legalizado, certamente ao mundo como um todo. Segundo dados
da Organização Mundial de Saúde, por exemplo, dos 55 milhões de abortos
induzidos ocorridos no mundo entre os anos de 2010 a 2014, pelo menos
45\% foram realizados em condições precárias. O que remete, mais uma
vez, à triste obviedade declarada em ``Os aspectos sociais do controle
de natalidade'': ``Em segredo e com pressa, milhares de mulheres são
sacrificadas por causa de abortos realizados por médicos charlatões e
parteiras ignorantes. No entanto, os poetas e os políticos louvam a
maternidade. Não houve crime maior contra a mulher do que esse''.

No que diz respeito às mães com melhores salários, isto é, à minoria das
mulheres que havia tido acesso ao raro privilégio de escolher a
profissão e o ofício, de modo a ter o sustento garantido via o exercício
das suas faculdades superiores, Goldman traz à luz a dificuldade de
conciliar uma carreira profissional bem-sucedida e a maternidade, drama
mais atual do que nunca, não obstante na sua época os mecanismos de
opressão para essa situação específica fossem ainda mais claros e
brutais. Vide, nesse sentido, o caso mencionado também em ``Os aspectos
sociais do controle de natalidade''. Segundo coloca, o Conselho de
Educação havia então recentemente aventado a proposta de proibir
professoras que se tornassem mães de continuar com seu trabalho de
magistério. Goldman chama atenção para o fato de que independentemente
de tal ``proposta'' ter sido dissuadida pela repercussão negativa na
opinião pública, uma trabalhadora intelectual ao se tornar mãe ``perde,
ano após ano'', fatalmente, ``a sua posição'' --- fatalidade também hoje
mais atual do que nunca e, paradoxalmente, sobretudo presente nos mais
altos círculos intelectuais e científicos, nos quais a emancipação de
tais mecanismos de opressão e de perpetuação da desigualdade entre os
``sexos'' deveria, por princípio, ter sido mais completa.

Por outro lado, segundo o seu diagnóstico cruel em ``O sufrágio
feminino'' --- este, sim, talvez ultrapassado ---, para Goldman, em geral,
faltaria às mulheres a ``força necessária para competir com o homem'';
certamente, não devido à sua natureza, mas ao próprio ``treinamento''
recebido pelo seu ``sexo'' ao longo das eras. Ao considerar a mulher
intelectual e economicamente emancipada, a anarquista postula estarmos
diante de uma pessoa ``frequentemente forçada a exaurir toda a sua
energia, a esgotar toda a sua vitalidade e sobrecarregar cada um de seus
nervos de modo a atingir o valor de mercado''. Não obstante, e o que
também é uma platitude hoje em dia, pouquíssimas mulheres são, conforme
escreve, bem-sucedidas nessa ``empreitada'', ``pois é um fato que
professoras, médicas, advogadas, arquitetas e engenheiras não são
tratadas com a mesma confiança que os seus colegas homens, como tampouco
recebem a mesma remuneração''. Daí que a tão aclamada emancipação
feminina, conforme denuncia em ``A tragédia da mulher emancipada'',
tenha, na prática, visto no ``amor'' e na ``alegria da maternidade''
empecilhos à sua realização, o que terminou por transformar a mulher
emancipada ideal numa espécie de ``profissional autômata'', que ``sente
profundamente a ausência da essência da vida''. Esta é a nova tragédia
da mulher, atualizada pelos novos tempos. Assim, se fosse o caso de
estabelecer uma competição de desgraça ante as ``opções'' então cabíveis
a uma mulher respeitável, a coisa não se resolveria de modo tão fácil.
Pois, de um lado, na maioria dos textos que se seguem, há a sugestão de
que a mãe, quando exclusivamente posta sob o pedestal de ``rainha do
lar'', é a mais escravizada, já que tal posição implica, como se por
necessidade, estar apartada dos assuntos do mundo, quando deveria ser
uma igual ao homem perante os assuntos do mundo --- \emph{deveria} ser
uma igual, porque é isso ``o que ela de fato é na realidade''; em ``A
tragédia da mulher emancipada'', porém, Goldman irá rememorar uma
observação sua acerca da maior compatibilidade ``entre a mãe à moda
antiga, a dona de casa, sempre atenta à felicidade dos seus filhos e ao
conforto daqueles que ama, e a mulher verdadeiramente nova, do que entre
esta última e a sua irmã emancipada com a qual nos deparamos
comumente''. Tal declaração é inexoravelmente polêmica, posto que
contrapõe a ``nova mulher'' --- tema que lhe foi caro em certo período da
juventude (como o texto ``A nova mulher'' aqui presente atesta) --- às
feministas, mulheres emancipadas do seu tempo. Como não poderia ser
diferente, essa comparação lhe valeu, como ela mesma relata, também em
``A tragédia da mulher emancipada'', a condenação ``à condição de pagã,
apta ao empalhamento'' pelas ``discípulas dessa emancipação pura e
simples''. Isso porque, segundo se defende:

\begin{quote}
O fervor cego delas não permitiu que elas vissem que a minha comparação
entre o antigo e o novo foi meramente para provar que boa parte das
nossas avós tinham mais sangue nas veias, muito mais humor e sagacidade,
e certamente uma naturalidade, amabilidade e simplicidade muito maiores
do que a mulher profissional em emancipação que preenche as
universidades, salas de estudo e escritórios vários. Isso não significa
que eu queira retornar ao passado, como tampouco condenar a mulher de
volta aos seus domínios do passado, limitados à cozinha e ao berçário.
\end{quote}

O aspecto central ressaltado através das suas múltiplas análises é que
fosse qual fosse a condição da mãe em questão, o resultado era quase
sempre o mesmo: logo se encontrava ``incapacitada de cuidar dos filhos
até mesmo no sentido mais elementar''. Segundo declara em algumas
passagens dos textos a seguir, poucas mães sabem efetivamente como
cuidar dos seus filhos. Via de regra, a educação dada pela mãe aos seus
rebentos é moldada através da imitação da educação recebida da sua
própria mãe; e ``uma mãe assim educada'', escreve justamente em ``A nova
mulher'', ``não pode ter qualquer ideia do verdadeiro conhecimento sobre
como educar os filhos, isto é, sobre a profissão de educar os filhos e,
desse modo, sob este sistema, ela nunca educa {[}\ldots{]} como deveria''.
Ora, é de se pressupor aqui certa contradição com a declaração que viria
a fazer futuramente, mencionada no parágrafo acima, de que a mãe à moda
antiga estaria mais próxima da nova mulher do que a mulher emancipada
que então se encontrava no seu tempo. A questão talvez se resolva ao
considerarmos que, para Goldman, as feministas, semelhante às
religiosamente devotas mães à moda antiga, também eram adoradoras de
fetiches, também eram escravas de deuses (ainda que secularizados), não
obstante diferentemente das primeiras haviam relegado a capacidade de
amar e se doar ao segundo ou terceiro plano de suas vidas e, com isso,
perdido, paulatinamente, a força e a coragem que o amor, a entrega e a
doação de si exigem para que possam ser e vir a ser continuamente.
Independentemente, porém, de a amável mãe à moda antiga estar ou não
estar mais próxima da nova mulher, a sua condição de escrava do marido e
dos filhos tornava-lhe também irremediavelmente inapta à ``profissão de
educar filhos ``, posto que, para a anarquista, e o que não é difícil
concordar, a condição de escrava é ``incompatível com ser uma boa mãe''.
Uma vez que a repressão sexual, social e econômica não só moldou a
instituição casamento, como também compõe a sua própria substância ---,
nada mais lógico que ao invés de proteger as crianças, em geral, as
vulnerabilizasse. Não é uma novidade para ninguém que, em parte
considerável dos casos, as crianças originadas no interior dessa
instituição vêm ao mundo ``em meio ao ódio e brigas domésticas'' (``O
que há na anarquia para as mulheres?'').

Entre os anos de 1914 a 1916, a anarquista desempenhou papel de grande
destaque no Movimento de controle de natalidade nos Estados Unidos, o
que lhe possibilitou, por conta do apelo exercido pela temática --- de
audiência incomparavelmente maior do que a suscitada pela pauta
anarquista ---, exercitar técnicas de desobediência civil com o objetivo
de derrubar a lei considerada injusta. Na presente coletânea, estão
dispostos os dois textos sobre o Movimento de controle de natalidade
publicados por Goldman, nos quais deixa clara a importância da ação
direta nesse caso específico, que segundo ela teve como efeito não só
educar as massas sobre a importância da pauta, como inclusive juízes. Em
``Novamente o movimento do controle de natalidade'' --- publicado apenas
seis meses depois de ``Os aspectos sociais do controle de natalidade, em
novembro de 1916 --- , Goldman relata a história, então recente, de um
juiz que havia se posicionado, em pleno julgamento, de modo favorável ao
controle de natalidade, ao se deparar com o caso de uma mãe levada ao
tribunal, pela segunda vez, sob a acusação de roubar comida para a sua
prole miserável que não parava de crescer. ``Com certeza, valeu a pena
ir para a prisão já que, com isso, foi possível ensinar a um juiz a
importância do controle de natalidade para as massas'' --- escreve no
referido texto. A técnica de ação direta não-violenta aí utilizada não
poderia ser mais simples: consistia tão somente em estar disposto a ir
para a cadeia ao se posicionar publicamente contra uma lei já
compreendida por boa parte da opinião pública como injusta e opressiva.
Conforme mencionado anteriormente, na primeira metade do século \textsc{xx}, a
mera circulação de informação sobre métodos contraceptivos era
legalmente proibida nos Estados Unidos. O movimento de controle de
natalidade ao concentrar as suas atividades na distribuição gratuita de
informações à população através de panfletos e comícios, implicava de
modo necessário, dada a força mesma da lei, num sem-número de prisões e
detenções dos seus militantes e ``simpatizantes''. Só no que diz
respeito a Goldman é praticamente impossível fazer as contas do número
das suas detenções e prisões nos quase três anos em que esteve à frente
do movimento. E a ideia era justamente essa: demonstrar a obsolescência
da lei através de uma série de prisões sem o menor sentido. Nos dois
textos aqui presentes --- que, podem inclusive ser considerados
documentos históricos, Goldman denuncia o Departamento de polícia de
Nova York por plantar provas e dar falsos testemunhos contra os
ativistas do controle de natalidade, um tipo de ``prática'' que, de todo
modo, não era, segundo ela, ``uma novidade no nosso Departamento de
Polícia''. De outro lado, há quem defenda que a dedicação obstinada de
Goldman ao Movimento de Controle de Natalidade --- que pelo menos ao
longo do tempo em que ocupou o foco central da sua militância --- foi
sobretudo uma estratégia de ganhar audiência para as ideias anarquistas
e para os métodos de ação direta, dado que passado o período em que
desempenhou papel central, distanciou-se completamente do movimento, sem
quaisquer explicações. Fosse um meio ou fim em si mesmo, é inegável que
a pauta do movimento de controle de natalidade se coadunava, em muitos
aspectos, com as suas concepções anarquistas, e também, talvez não seja
um erro dizer, com a sua própria vida, uma vez que abriu mão, apesar do
desejo reiterado de mais de um dos companheiros que teve ao longo da
vida, de ser ela mesma mãe.

Ao condenar e reduzir a mulher à função de ``incubadora'' de crianças, a
instituição casamento rendeu-lhe ainda, como um efeito colateral
praticamente necessário, uma saúde arruinada, sobretudo no que diz
respeito ao seu sistema reprodutivo: ``Ademais'', escreve em ``Os
aspectos sociais do controle de natalidade'', ``é consenso entre os
médicos mais sérios que a reprodução constante da parte das mulheres
resulta no que em termos leigos é chamado `problemas femininos'''. Na
sua defesa do controle de natalidade, Goldman inicia chamando a atenção
para as constatações científicas e pesquisas sobre a temática e o
direito das mulheres de terem acesso a essas informações e pesquisas que
diziam respeito diretamente às suas próprias vidas, aos abusos sofridos
contra os seus corpos. Ela reproduz, por exemplo, a informação simples
--- embora, na época, inacessível a boa parte das mulheres ---, de que
para uma reabilitação fisiológica completa seriam necessários intervalos
de três a cinco anos entre as gestações; reabilitação, inclusive, que
teria efeito na sua disposição física e psicológica para ``cuidar melhor
das crianças já existentes''. Também em ``Os aspectos sociais do
controle de natalidade'', ela toca na questão ainda hoje polêmica, ainda
hoje tabu --- sobretudo em países com problemas sérios de desigualdade
social e, por consequência, de alimentação, saneamento básico etc. ---,
sobre até que ponto a gestação por parte de mulheres subnutridas e
doentes daria origem a crianças com deficiências físicas e/ou mentais.
Um ``organismo sobrecarregado e subnutrido não pode reproduzir prole
saudável'' --- postula, sem titubear, em ``Os aspectos sociais do
controle de natalidade''. Se tal questão é \emph{ainda} polêmica, ela
não o é por ter sido descreditada pela ciência, mas porque encará-la
implica admitir a urgência de políticas públicas que garantam, ao menos,
a seguridade alimentar das gestantes pobres. No que diz respeito ao
Brasil, por exemplo, recentemente, em 2004, a fortificação obrigatória
das farinhas de trigo e milho com ácido fólico, garantiu uma redução
significativa nas estatísticas de certos defeitos congênitos em
recém-nascidos e anemia graves nas suas mães gestantes, muito embora
diversas pesquisas também apontem que no que diz respeito às camadas
mais pobres, a medida não foi suficiente para liquidar esse problema tão
grave e tão visceralmente injusto.

``Mas por qual razão a mulher exaure o seu organismo numa gravidez
perpétua?'' --- faz a pergunta que, por sua vez, remete ao cerne da sua
concepção, dado que a sua resposta é: porque a instituição casamento
está à serviço da opressão econômica e social. Certamente, naquele
tempo, o capitalismo e o militarismo --- militarismo sem o qual, diz
Goldman, o capitalismo não pode se realizar --- necessitavam de uma
``raça numerosa'', de uma massa excedente, o que os economistas chamavam
de ``margem de trabalho'': ``sob nenhuma circunstância'', escreve ainda
em ``Os aspectos sociais do controle de natalidade'', ``deve a margem de
trabalho ser diminuída, caso contrário a sagrada instituição conhecida
como civilização capitalista será prejudicada''. Por detrás da neblina
lançada pela idealização, moralização e santificação da maternidade, o
destino da mulher --- dito mais uma vez de modo cru --- teria sido o de
ser coagida a participar ``do crime de trazer ao mundo crianças
infelizes apenas para serem moídas e transformadas em pó pelas rodas do
capitalismo e dilaceradas em pedaços nas trincheiras e campos de
batalha.''. E muito embora, escreve num parágrafo comovente do texto
supracitado, a mulher tenha cumprido ``o seu dever mil vezes maior do
que o de qualquer soldado em campo de batalha'' --- que é o dever de dar
a vida ao invés de tirá-la ---; diferentemente dos chamados defensores da
pátria, com suas hierarquias e patentes, ela nunca recebeu, para além
dos louvores compostos de palavras ocas, qualquer retribuição financeira
ou reconhecimento social pela tarefa em nome da qual teria sido
sacrificada, individualmente, ao longo de toda sua vida, e
coletivamente, ao longo das eras.

É mesmo surpreendente que os problemas levantados por Goldman na pauta
do controle de natalidade pareçam ainda terrivelmente atuais. Pois a
``prole'', conforme constatou, continua a dificultar que os pais
adentrem organizações se não revolucionárias, pelo menos, civis ou
trabalhistas, que tenham tempo, disposição física e coragem de, por
exemplo, lutar ativamente por direitos, participar de greves, arriscar
empregos e, em casos extremos, a vida. Um impedimento que parece ainda
mais dramático no contexto atual, posto que o abandono da luta coletiva
contra a opressão política e econômica no que diz respeito às pesadas
responsabilidades e demandas que envolvem a criação dos filhos, já não
precisa sequer contar com uma prole numerosa. Por outro lado, e o que é
ainda mais grave, o capitalismo especulativo, com sua economia cada vez
mais automatizada, necessita, cada vez menos, das massas ou, segundo a
terminologia da autora, da ``raça numerosa'' para fazer girar as suas
rodas e as do militarismo. Em outras palavras, na nossa atualidade, em
que o excesso populacional atingiu limites inauditos, em que a
automatização \emph{hi-tec} seja do capitalismo ou do militarismo parece
já não ter limites, a massa como margem de trabalho se torna cada vez
mais dispensável. Nesse sentido, é inevitável suspeitar, ao ler as
considerações aqui dispostas sobre o movimento do controle de
natalidade, que há na pauta que ela estabelece aspectos urgentes ainda
hoje --- sobretudo em países periféricos como o nosso. Não é por não
querer assumir responsabilidades ou por não ter a capacidade de amar os
vindouros filhos, como então acusavam os moralistas, que a mulher
moderna deve buscar evitar a gravidez --- o que não raro a leva a
recorrer ao mais drástico dos métodos, o aborto. Antes, seria o caso de
afirmar o contrário. ``Com a guerra econômica a destruir todo o entorno,
com os conflitos, a miséria, o crime, as doenças e a insanidade postas
bem diante dos seus olhos, com inúmeras crianças pequenas trituradas e
destruídas'', questiona dolorosamente em ``Vítimas da moralidade'',
``como poderia uma mulher autoconsciente e consciente da própria
humanidade vir a se tornar mãe?'' É por possuir um senso de
responsabilidade e humanidade, completamente diferente do das suas avós
que a mulher moderna se vê obrigada a impor sobre si mesma a renúncia à
glória da maternidade. Somente assim não terá ``de dar à luz numa
atmosfera em que se respira apenas destruição e morte''. Vide nesse
sentido, a comovente conclusão de ``Vítimas da moralidade'':

\begin{quote}
Através da sua consciência renascida como unidade, como personalidade,
como construtora da raça humana, ela {[}a mulher{]} se tornará mãe
apenas se desejar o filho e se puder dar à criança, mesmo antes de seu
nascimento, tudo o que sua natureza intelecto sejam capazes de alcançar:
harmonia, saúde, conforto, beleza e, acima de tudo, compreensão,
reverência e amor, que é o único solo verdadeiramente fértil para uma
nova vida, para um novo ser.
\end{quote}

Antes de adentrar os escritos a seguir, é importante ter em mente que a
análise da questão do ``sexo'' tal qual desenvolvida por Goldman, é
volta e meia, dura e irônica para com as próprias mulheres e, o que é
pior, especialmente, para com as feministas. Em ``O sufrágio feminino'',
título que indica a principal pauta do movimento feminista da época, a
anarquista chega ao paroxismo de concordar com a misoginia de Nietzsche,
ao escrever logo na primeira página: ``A máxima memorável de Nietzsche,
`Quando você encontrar uma mulher, leve o chicote', é considerada muito
brutal e, no entanto, Nietzsche expressou numa frase a atitude da mulher
para com os seus deuses.'' Antes de corrermos o risco de, no susto,
condená-la à condição de herege da causa da mulher, como, conforme
mencionado acima, fizeram muitas feministas do seu tempo, tentemos
compreender o motivo da sua irritação e, conseguinte, virulência.
Goldman parece se valer de Nietzsche para fazer atentar que a mulher,
independentemente do homem, já traz consigo o seu próprio chicote para
chicotear-se em submissão voluntária ante os ``deuses'' que, podem se
apresentar sob ``diferentes formas e substâncias'', no caso do seu
tempo, sob a nova roupagem do sufrágio feminino. Dito mais precisamente,
Goldman compreende a subordinação à igreja, à guerra, à família, ao lar,
ao Estado e, naquele momento, à causa do sufrágio, como prostração ante
o mesmo deus da opressão nas suas diferentes facetas. Ela é extremamente
dura com as mulheres, ao atribuir-lhes um fanatismo ainda maior do que o
dos homens; e isso a ponto de afirmar que ``a religião já teria deixado
de ser um fator relevante na vida das pessoas, não fosse pelo apoio que
recebe da mulher''. Dito resumidamente, a mulher, segundo Emma Goldman,
terminou por se tornar, de um modo geral, uma ``adoradora de fetiches''
e, portanto, sem necessitar da intervenção direta dos homens, oferece,
de bom grado, a si mesma em sacrifício para o benefício das
``divindades'' que a condenam à ``vida de um ser inferior''. Mesmo que
``os seus ídolos possam mudar'', escreve a nietzschianamente impiedosa,
a mulher ``está sempre de joelhos, sempre levantando as mãos, sempre
cega para o fato de que seu deus tem pés de barro''.

A crítica de Goldman às sufragistas se confunde com luta de classe: de
modo mais ou menos direto, ela denuncia o elitismo ralé que
caracterizava o espírito dessas ``revolucionárias''. Em especial no que
diz respeito às sufragistas estadunidenses, Goldman nos oferece o
retrato de uma típica mulher branca da classe média cuja autoproclamada
emancipação, de tão medíocre, só tornava mais lamentável a sua condição
de submissa. Nos dois textos aqui dispostos sobre a temática, tal
retrato é expresso de diferentes maneiras. Primeiro, na condenação
aberta à estreiteza intelectual revelada no otimismo das sufragistas,
para quem o sufrágio feminino abriria, ironiza, as portas de toda
``vida, felicidade, alegria, liberdade, independência.'' De acordo com o
seu diagnóstico, o problema é que a ``devoção cega'' ao sufrágio impediu
que as autodeclaradas ``emancipadas'' enxergassem a obviedade de que o
direito ao voto ``é apenas um meio de fortalecer a onipotência dos
mesmos deuses'' a quem servem ``desde os tempos imemoriais''.

Através de relatos da experiência de diferentes países e Estados
norte-americanos em que o sufrágio feminino (ou mesmo a atuação direta
da mulher na política) já havia sido legalizado, Goldman traz à luz a
sua irrelevância no que diz respeito à melhoria das condições de vida da
classe trabalhadora; inclusive, em alguns casos, como o da Austrália ---
país então apontado, em conjunto com a Nova Zelândia, como a ``Meca do
sufrágio igualitário'' --- , o contrário é que teria acontecido. Pois na
Austrália, informa-nos, \emph{apesar} do sufrágio igualitário, foram
aprovadas leis trabalhistas mais rigorosas em detrimento dos
trabalhadores, a ponto de greves não aprovadas por algum ``comitê
arbitrário'' passarem a ser enquadradas como um ``crime do mesmo porte
que o de traição''. O caso do sufrágio feminino então recém-instaurado
no Colorado é também bastante significativo. Segundo relata, o
governador Davis Waite, que teria justamente facilitado a promulgação do
sufrágio feminino durante o seu governo, foi derrotado por um candidato
reacionário na eleição seguinte --- derrota que contou, portanto, com boa
parte dos votos das mesmas mulheres que ele havia contribuído na
``emancipação''. Ela também faz questão de frisar que alianças políticas
entre as feministas sufragistas e as mulheres da classe trabalhadora,
quando havia, eram exceção. Em realidade, o direito ao voto pelo qual
militavam tanto as expoentes das sufragistas inglesas, quanto das
norte-americanas não abrangia todas as classes. A luta das
``emancipadas'', nesse caso, era pelo direito ao voto das mulheres
proprietárias --- e, portanto, em geral, brancas, casadas e/ou herdeiras.
Na condição de ``mulheres inteligentes e liberais'', contesta também em
``O sufrágio feminino'', elas já deveriam ter percebido ``que se o voto
é uma arma, os deserdados precisam dele mais do que a classe econômica
superior, e que esta classe já desfruta de bastante poder em virtude da
sua superioridade econômica mesma.'' Diferentemente das heroínas russas
cuja linhagem Goldman reivindicava como sua, as sufragistas inglesas e
norte-americanas, herdeiras das puritanas, não teriam o mínimo
comprometimento com a verdadeira igualdade: ``Quão pouco significa a
igualdade para elas em comparação com as mulheres russas que passam pelo
inferno em nome do seu ideal!'' Pertencente ao que há de mais valioso na
tradição revolucionária, Goldman sabia muito bem, o que não poucos de
nós, habitantes dos tristes trópicos, estamos descobrindo só agora, cem
anos depois, em face do supostamente disruptivo esfacelamento da
democracia nessa nossa estranha entrada na década de vinte do século
\textsc{xxi}: que ``todo milímetro a mais de direitos só foi conquistado através
de lutas constantes, uma luta interminável por autoafirmação e jamais
pelo sufrágio.''

Em ``O camaleônico sufrágio feminino'', escrito no contexto da Primeira
Grande Guerra, a denúncia às sufragistas vai além. O mal-estar não
poderia ser maior. Segundo denuncia, as sufragistas inglesas, imitadas,
logo em seguida, pelas estadunidenses, teriam oferecido como moeda de
troca do direito ao voto (e dos cargos), o apoio até então insuspeito à
entrada dos seus respectivos países na guerra --- apoio insuspeito porque
a causa sufragista havia desde sempre caminhado lado a lado a movimentos
pacifistas. A coisa fica ainda mais grave, quando, no que diz respeito
às sufragistas estadunidenses, Goldman sugere, sem grandes preâmbulos,
que elas estariam se valendo dos seus atributos sexuais, desta vez,
deliberadamente, para o incremento do seu recentíssimo ativismo
patriótico cujo fito então urgente era o de promover o alistamento; o
que, no mínimo, implicava a contradição de reduzir-se à condição de
mercadoria sexual para, com isso, garantir os seus direitos cívicos ---
contradição que numa versão mais grotesca seja talvez atualmente
ilustrada pelas patriotas norte-americanas excessivamente malhadas e
maquiadas a exibir os seus fuzis em nome da pátria e da família. Seja
como for, provoca Goldman, pela via do argumento, a mulher reduzida, ao
longo das eras, à condição de mercadoria sexual, já havia tido provas e
mais provas de que não poderia convencer os homens --- ao menos a maioria
deles ---, de absolutamente nada: ``Não, não foi nenhum argumento, razão
ou humanitarismo que o Partido Sufragista prometeu ao governo; e sim, o
poder da atração sexual, o apelo vulgar, persuasivo e envolvente da
mulher liberada a serviço da glória do seu país''.

Justa ou injustamente, a avaliação de Goldman do movimento sufragista
coloca o seu leitor ante um quadro de oligofrênica confusão entre
moralidade, reacionarismo, política e desejo de vanguarda; como se a
mulher, nesse momento inicial, tivesse confundido as limitações a que
foi submetida ao longo das eras com o ideal mesmo de poder político; ou
nas palavras irônicas da anarquista: como se acreditasse que com a
política fosse acontecer o mesmo que com ela: que bastaria ``afagar a
besta para que ela se tornasse tão gentil quanto um cordeiro doce e
puro''. Obviamente, a sua oposição à causa do sufrágio feminino, como
trata de deixar claro, não estava em nada relacionada à compreensão
cretina de que a mulher não tem capacidade psicológica ou intelectual
para a façanha do ``voto consciente''. A oposição é, aqui, muito mais
radical. O que a ela combate em ambos os textos, e de modo central, é a
substância mesma que, então, animava a causa das feministas sufragistas:
a ``concepção absurda'', ``de que a mulher será bem-sucedida naquilo em
que o homem falhou''. Goldman desprezava absolutamente a, então,
bandeira das sufragistas de que a mulher seria algo como que dotada do
poder ``purificar'' a política; para ela uma tal bandeira não passava de
uma confirmação flagrante de que a feminista emancipada sufragista de
ocasião continuava a se colocar no mesmo pedestal de mártir das deidades
--- o pedestal contra o qual o seu feminismo supostamente estaria a se
opor. Eis o truísmo: se a mulher é um ser humano, ela está ``sujeita a
todas as loucuras e erros humanos''. E, nesse ponto, a perspicácia de
Goldman não poderia ser mais afiada: ``Presumir que a mulher triunfará
em purificar algo que não é passível de purificação, é creditar-lhe
poderes sobrenaturais. Uma vez que a tragédia da mulher tem sido a de
ser olhada como ou um anjo ou um demônio, a sua verdadeira salvação
reside em ser colocada sobre a terra'' (``O sufrágio feminino'').

Aí não há nada ultrapassado. Certamente, seria um exercício proveitoso
imaginar o que a anarquista pensaria das nossas pautas
político-eleitorais feministas de agora, que ante a escalada de um
autoritarismo armado até os dentes, propõe como alternativa de luta mais
louvável que se vote na candidatura de mulheres --- candidaturas, é justo
dizer, hoje não raro, de mulheres oriundas das classes trabalhadoras.
Anos depois ao assumir uma postura menos bélica contra as feministas, a
própria Goldman admitiu o inevitável: que a luta do feminismo não foi em
vão --- vide, nesse sentido, o artigo da presente coletânea que leva esse
nome. De todo modo, é quase necessário que, após a leitura dos textos
aqui dispostos, salte aos olhos que a pauta feminista eleitoral do
momento consiste, em alguma medida, numa releitura da mesma ``concepção
absurda'' que Goldman identificou nas sufragistas; que, misturando e
atualizando vocabulários, poderia ser sintetizada algo assim: sob a
pauta feminista brasileira eleitoral do momento subjaz a compreensão de
que um número maior de mulheres em cargos políticos será capaz de algo
como ``purificar'' a política da ``falocracia''. Ocupação de mulheres em
cargos públicos que, é inevitável ponderar, encontra a sua legitimidade
inquestionável no próprio valor da representatividade e pluralidade tão
caras à democracia e, quiçá, ainda que num registro outro, também ao
anarquismo. Em se optando, porém, por fazer o exercício de imaginar o
que a ``Suma Sacerdotisa do Anarquismo'' pensaria da nossa
\emph{avant-garde} ``antifalocrata'', talvez concluíssemos ser mesmo o
caso de considerar o exemplo, por ela, repetidamente, trazido à memória,
das heroínas revolucionárias do ``mais sombrio de todos os países, dado
o seu despotismo absoluto, a Rússia''. Ou ainda, o exemplo da lendária
anarquista Louise Michel que lutou nas barricadas da Comuna de Paris,
lutou ao longo de toda a sua vida, e a quem Goldman --- que não só
conhecia, como amava ---, dedicou um dos mais comoventes escritos
dispostos na presente coletânea. Louise Michel que possuía uma verve
revolucionária tão completa, a ponto de ter sido reivindicada pelos
ativistas pioneiros dos direitos dos homossexuais como membro póstumo da
sua comunidade; sob o argumento de que as características que compunham
a sua personalidade extraordinária seriam incompatíveis com as
qualidades típicas a uma mulher e, portanto, prova suficiente da sua
condição de ``uraniana'' --- termo utilizado nos Estados Unidos da época
para indicar uma mulher em corpo de homem e vice-versa. O texto ``Louise
Michel, uma refutação'', aqui traduzido, consiste numa carta aberta em
resposta a essa tese defendida por Karl von Levetzow no seu ensaio sobre
Louise Michel, publicado logo após sua morte, em 1905. Embora Goldman
demonstre grande respeito por esse intelectual e escritor, ela se vê na
obrigação de se opor à sua concepção antiquada e opressiva da mulher, do
que seria supostamente compatível ou não com o conceito de
``feminilidade'' naturalizado na mulher, via mecanismos de tortura mais
ou menos sofisticados.

Que Louise Michel não fosse, segundo Goldman, uma uraniania, é nesse
sentido um detalhe. O ponto é que tanto nessa carta aberta, quanto no
ensaio sobre Mary Wollstonecraft, e no texto acusatório contra o
tratamento dispensado pelo regime soviético às heroínas russas,
deparamo-nos com a compreensão de que as mulheres se tornam
politicamente iguais aos homens, não através do voto, mas sim do seu
maravilhoso heroísmo, da sua coragem, capacidade, força de vontade e
perseverança na luta pela liberdade. Conforme postula taxativamente no
seu belicosíssimo ``O camaleônico sufrágio feminino'': ``Não há
esperança que a mulher, com o seu direito de votar'' --- e, aqui,
poderíamos, acrescentar, como tampouco com o seu direito de se
candidatar e se eleger de modo proporcional aos homens ---, ``possa em
algum momento purificar a política''. Afinal, continua do modo preciso,
curto e brutal que lhe é característico:

\begin{quote}
A corrupção política não tem nada a ver com a moralidade ou com a
frouxidão moral das várias personalidades da política. Sua causa é
totalmente material. A política é o reflexo do mundo empresarial e
industrial, cujos lemas são: ``Tomar é melhor do que dar''; ``compre
barato e venda caro''; ``uma mão suja lava a outra''.
\end{quote}

Para que haja a Nova Mulher é preciso haver também o ``Novo Homem''. E
daí que ``um dos grandes erros da nova mulher ideal'' seja ``o de imitar
o homem'' (``A nova mulher''). Imitação sob a qual subjaz a concepção de
que o homem já é, inclusive, superior à nova mulher, posto que a
ascensão da mulher à condição de ``nova'' dependeria de, via a imitação,
tornar-se igual ao homem e, portanto, ``tornar-se masculina''; isto é,
tomar como seus os atributos da ``masculinidade'' então vigente ---
masculinidade cuja confusão entre liberdade e opressão no interior das
relações mais íntimas é um dos elementos essenciais. Daí aquela polêmica
declaração sua de que a mulher verdadeiramente nova está mais próxima da
``mãe à moda antiga'', sempre atenta ao conforto e felicidade dos seus,
do que da mulher emancipada do seu tempo --- e quiçá também do nosso.
Conforme identifica em ``O sufrágio feminino'', o ``infortúnio da mulher
não é o de que ela é incapaz de realizar o trabalho de um homem, mas o
de que ela está desperdiçando a sua vitalidade na tentativa de
superá-lo''. Numa palavra, é como se as mulheres emancipadas tivessem
confundido a emancipação mesma com a condição do homem, de modo que para
ser emancipada só lhe coubesse como alternativa ser igual ao homem ou
até melhor do que ele. Para Goldman, porém, a ordem dos fatores, nesse
caso, altera o produto, posto que o ponto que ela, através da sua
crítica, busca o tempo todo fundamentar é o de que somente em liberdade,
em todas as coisas, homens e mulheres podem ser iguais. É a liberdade
que é o solo da igualdade, e não a imitação e, conseguinte, igualação
àquele que, sob a perspectiva da mulher, oprime. Assim, escreve ela em
``A tragédia da mulher emancipada'': ``Independentemente de todas as
linhas que demarcam artificialmente os respectivos direitos dos homens e
das mulheres, defendo que há um ponto em que todas essas diferenças
podem se encontrar, de modo a se transformar em um todo perfeito''. Para
a anarquista é preciso ser universal o suficiente para não transformar
os atributos culturalmente consagrados a apenas um dos sexos, no caso o
masculino, nos únicos atributos adequados a um ser humano
verdadeiramente livre ou em luta pela verdadeira liberdade, como também,
por outro lado, é preciso ser universal o suficiente para não culpar um
único sexo, no caso também o masculino, por toda a opressão.

Um outro grande erro, aparentemente contraditório ao de imitar o homem,
é que a mulher supostamente emancipada ou supostamente em vias de
emancipação tenha passado a ver no mesmo homem que imita o seu opositor,
algo como o seu arqui-inimigo imemorial --- o que teve como efeito
tentativas deliberadas e reiteradas de bani-lo da sua vida afetiva.
``Naturalmente'', o que ela disse ter sido reconhecido por Mary
Wollstonecraft há mais de duzentos anos, já não pode ser, a nós mulheres
do século \textsc{xxi}, digno de surpresa: ``que o homem tem sido um tirano há
tanto tempo que ele se ressente de qualquer violação do seu domínio''. A
questão é que se a luta efetivamente se der em nome da causa da
emancipação mesma não é possível abrir mão da convergência entre a
liberdade da mulher e a liberdade do homem. Como tampouco é possível, em
nome do amor, abrir mão da ``verdade'' de ``que um homem profundo e
sensível'' não difere ``de modo considerável de uma mulher profunda e
sensível''; dado que ele ``também busca a beleza e o amor, a harmonia e
o entendimento.'' Na passagem possivelmente mais tocante a toda e
qualquer mulher cuja vida consista, em maior ou menor medida, na busca e
na luta pela a sua liberdade e autorrealização, Goldman, com a
naturalidade de uma árvore que produz fruto, oferece o conselho
paradoxal da necessidade de a mulher ``se emancipar da emancipação, se
ela realmente deseja ser livre''. Posto que, conforme nos esclarece mais
uma vez: a ``emancipação, tal como entendida pela maioria dos seus
seguidores e representantes, é muito estreita para deixar espaço para o
amor e o êxtase sem limites'' (``A tragédia da mulher emancipada''). E
uma vez que, para ela, vale repetir, não pode haver emancipação sem amor
--- posto que o amor constitui a própria substância da liberdade e
vice-versa ---, não há emancipação no que até aqui foi aclamado como
emancipação feminina. Nesse ponto, remetamo-nos novamente à entrevista
supracitada, ``O que há na anarquia para as mulheres?'': ``Uma aliança
deve ser formada {[}\ldots{]} não como a de agora, para dar à mulher um
sustento e uma casa, mas, sim, porque o amor existe, e esse estado das
coisas só pode ser suscitado com uma revolução interna, em suma, com a
anarquia.''

A verdadeira emancipação implica necessariamente e, em primeiro lugar,
libertar-se dos ``tiranos internos''. Pois a ``emancipação meramente
externa'', postula em ``A tragédia da mulher emancipada'', ``fez da
mulher moderna um ser artificial, que lembra {[}\ldots{]} qualquer coisa,
exceto as formas que seriam alcançadas através da expressão de suas
qualidades internas''. Para que haja a adequação entre as qualidades
internas e a forma exterior --- de modo a não se cair num simulacro de
liberdade ---, a mulher deve, em primeiro lugar, libertar-se na sua
própria fonte, o elemento sexual. Ao se concentrar na ``independência
das tiranias externas'', sem dar a devida ênfase aos ``tiranos
internos'', a mulher então considerada emancipada, diz Goldman, deu
provas de não ter compreendido ``verdadeiramente o significado da
emancipação''. ``A maior deficiência da emancipação de hoje'',
diagnostica, ``é a sua rigidez artificial e a sua respeitabilidade
estrita, que criam um vazio na alma da mulher que não lhe permite beber
na fonte da vida''.

Conforme colocado no início da presente introdução, a consequência da
redução da mulher à condição de mercadoria sexual que, por sua vez, tem
como causa originária a repressão sexual, não é particular ou acidental,
antes, o contrário: reatualizada (ainda que supostamente atenuada) ao
longo das eras, passou a dizer respeito ao seu ``espírito''. Em diversos
dos textos que se seguem, de modo mais ou menos direto, há a sugestão de
que essa ``impotência'' e mutilação na capacidade de satisfação erótica
usual à condição feminina --- seja na figura da esposa ou da solteirona
(no caso da prostituta, devido à variabilidade sexual concernente à
profissão, ao menos neste aspecto, estaria em relativa vantagem) ---;
terminou por dar origem a certa espécie de ressentimento, como se
tipicamente feminino, como se uma espécie de efeito colateral necessário
à essência da ``respeitabilidade'' da mulher. ``A instituição
casamento'', escreve em ``Casamento e amor'', ``transforma a mulher num
parasita, num ser absolutamente dependente. Incapacita-a para a luta
pela vida, aniquila a sua consciência social, paralisa a sua imaginação,
e depois impõe a sua proteção benevolente, que na realidade é uma
armadilha, uma paródia do real caráter humano''. Na medida em que a
atividade sexual é algo que diz respeito à personalidade como um todo,
em que se constitui como a fonte mesma da criação, alegria,
sociabilidade e amor, nada mais natural que ante a ``repressão sexual
levada a cabo por um longo período de anos'', a consequência fosse como
ela mesmo diz, a transformação do ``real caráter humano'' numa
``paródia''. Ou ainda, segundo suas observações no esboço intitulado ``O
elemento sexual da vida'': ``a contenção de um desejo tão instintivo não
pode ser bem-sucedida numa pessoa normal sem consequências diretas na
sua saúde''; quando não ``uma desordem mental completa'', ``todo tipo de
distúrbio mental e físico'' pode vir a resultar da repressão. Daí aos
demais animais estarem completamente livres dos distúrbios nervosos que
tanto nos dominam, quanto impelem as nossas mais desastrosas ações.

Segundo Emma Goldman, a mulher verdadeiramente emancipada, mais do que
uma intelectual ou artista, é uma mulher ardente. A autorrealização
jamais poderia dizer respeito exclusivamente à esfera profissional,
posto que há a dimensão natural e afetiva que, acima de tudo, constitui
a vida no seu sentido mais próprio. Sob a perspectiva oferecida pela
anarquista, o conhecimento está subordinado à vida e não o contrário.
Como mencionado anteriormente, em inúmeras passagens, encontramos a
sugestão de que o desenvolvimento intelectual e artístico simplesmente
não pode prescindir da vivência e entrega às paixões e à sensualidade. É
verdade que em ``O elemento sexual da vida'', uma Goldman já na casa dos
sessenta anos irá admitir que o exercício da criatividade, em muitos
casos, é suficiente para transfigurar, via a sublimação, o gozo
propriamente dito, o ponto culminante da autorrealização passível de ser
atingida pelo corpo. ``A liberação não-sexual de energia, algumas vezes,
é suficiente para compensar as necessidades fundamentais do desejo
sexual e, ainda, para transmutá-las em autossatisfação e formas úteis de
expressão'' --- escreve. De todo modo, isso não significa que o instinto
criativo deva ser visto como uma espécie de ``antídoto'' contra o
instinto sexual --- até porque, segundo ela, trata-se essencialmente da
mesma força. Seja como for, boa parte das mulheres exemplares invocadas,
rememoradas e homenageadas nas páginas que se seguem, tiveram a sua
biografia marcada, segundo o relato de Goldman, pelo desenvolvimento do
seu talento criativo --- quer nas artes, na militância política, ciências
ou humanidades ---; mas também pela ânsia eternamente frustrada de
realização da sua sensualidade e afetividade nos braços do amor. Ainda
hoje, no caso da mulher, a possibilidade de uma autorrealização sensível
e intelectual, profissional e afetiva, é uma rara possibilidade, quando
comparada com a de um homem --- que também, de todo modo, é rara. ``Quão
maior seja o desenvolvimento mental de uma mulher'', postula mais uma
vez em ``A tragédia da mulher emancipada'', ``menor a possibilidade de
que ela venha a encontrar um companheiro''. Em outras palavras, a certas
mulheres de ``mentalidade extraordinária'', aquelas admiradas por
Goldman, é tanto impossível ter como companheiro um homem que veja nela
nada além de um mero ``sexo'', desprovido de personalidade e
profundidade; como igualmente é-lhe um companheiro impossível, aquele
que para além da sua intelectualidade e do seu gênio, falha em despertar
a sua natureza de mulher, o seu desejo sexual. Mary Wollstonecraft, dada
a riqueza e idealidade mesma do seu ser, teria encontrado nessa
contradição a sua fatalidade: ``Vida sem amor para um caráter como o de
Mary é inconcebível, e foi a sua busca e ânsia pelo amor que a lançou
contra as rochas da inconsistência e do desespero'' --- escreve, num
momento de grande lirismo. E nesse ponto vale lembrar ainda o que
escreveu no seu ensaio sobre Louise Michel, ao explicar o porquê de ela
escolher viver entre amigas mulheres, apesar de não ser homossexual.
Embora a passagem seja longa, vale a pena citá-la na sua integralidade;
é sempre uma esperança digna a de que um homem sensível possa finalmente
brotar do eterno Adão.

\begin{quote}
A razão para isso, no caso das mulheres, é que elas encontram uma
compreensão mais profunda entre os membros de seu próprio sexo do que
com os homens do seu tempo. O problema é que o homem moderno ainda se
assemelha demais ao seu antepassado Adão, não diferindo muito, na sua
atitude em relação à mulher, de um homem mediano qualquer. Por outro
lado, a mulher moderna já não se satisfaz com um homem que seja tão
somente seu amante; ela quer compreensão, camaradagem, quer ser tratada
como um ser humano, e não como um objeto de gratificação sexual. Uma vez
que ela nem sempre pode encontrar isso no homem, ela se volta para suas
irmãs. É precisamente porque não há o elemento sexual entre elas que
elas podem compreender melhor uma a outra. Em outras palavras, ao invés
de se sentir atraída por suas amigas mulheres por conta de tendências
homossexuais, Louise se atraía por elas justamente porque era uma mulher
e precisava da companhia de mulheres.
\end{quote}

\section{Considerações finais}

Sob a luz dessa leitura \emph{econômica} da ``espiritualidade''
supostamente ideal a uma mulher da primeira metade do século \textsc{xx} --- a de
ser jovem e dócil como um cordeiro pronta para ter abatida a sua
personalidade ---, é curioso pensar na atualidade com os seus infinitos
recursos artificiais e intervenções cirúrgicas que trazem a promessa de
uma eterna aparência de juventude assomada a um formato de corpo
``sexualmente desejável''; promessas cuja realização é, mesmo hoje, mais
urgente aos corpos femininos. A pergunta que se impõe, a partir dessa
perspectiva trazida por Goldman, e da qual talvez não seja desejável
escapar, é a sobre até que ponto, nós mulheres, superamos e até que
ponto nos afogamos ainda mais nessa condição de mercadoria sexual. Mesmo
que seja o caso de considerarmos que, atualmente, temos, por suposto, a
opção de ser uma mercadoria sexual, por assim dizer, financeiramente
emancipada e sexualmente ``livre'' --- a liberdade talvez tenha de estar
aí, nesse caso, sempre entre aspas, até porque, segundo um número
considerável de pesquisas mais do que atuais, o orgasmo é, em geral,
desconhecido para algo em torno de 50\% das mulheres. Que haja liberdade
sexual que não venha acompanhada de orgasmo é, convenhamos, algo, no
mínimo incongruente e, por certo, incompleto e insatisfatório. Embora
seja um tanto triste admitir, a pergunta que fica é a sobre até que
ponto Mary Wollstonecraft estaria anda hoje certa, ao enfatizar, segundo
o relato de Goldman, que ``a própria mulher é um obstáculo ao progresso
humano, porque insiste em ser um objeto sexual ao invés de uma
personalidade, uma força criativa na vida''.

Também parece ser uma herança dessa condição de mera mercadoria sexual
que, ainda hoje cause certa estranheza que mulheres ``de certa idade''
se relacionem com homens mais jovens ou que mulheres pertencentes a
esferas sociais e econômicas mais altas relacionem-se com homens
pertencentes a esferas sociais e econômicas mais baixas, o que,
especialmente no caso de um país como o nosso, envolve a questão da
raça. Talvez não seja exagero dizer que apesar das tantas e tão radicais
mudanças ocorridas, nas últimas décadas, no campo da moral sexual e da
compreensão da questão do ``gênero'', é como se o amor ainda não se
encaixasse muito bem nas relações entre homens mais jovens, menos ricos
e escolarizados e mulheres mais velhas, mais ricas e escolarizadas;
muito embora, o mesmo não possa ser dito, no caso oposto --- a relação
erótica entre professores universitários e suas alunas, por exemplo,
praticamente uma instituição (silenciosa) erigida nos bastidores das
instituições de ensino superior, parece ser prova disso. Por maiores e
mais radicais que tenham sido as desconstruções e novas construções de
gênero, a mulher continua a ser o sexo associado a alguma espécie de
amor universal e incondicional que é, por sua vez, um desdobramento
afetivo da sua condição de mercadoria sexual --- e, portanto, não o amor
universal e incondicional mesmo. Afinal, apesar desse amor do qual a
mulher seria supostamente o reservatório, ela continua a possuir, ao
menos sob o ponto de vista da heteronormatividade (para utilizar aqui um
termo a nós contemporâneo), um leque por demais restrito de sujeitos
dignos do seu amor presumidamente inato.

E, como disse Nietzsche, citando os Vedas: ``Há muitas auroras que não
brilharam ainda.''

%Vale do Capão, 14 de outubro de 2020
